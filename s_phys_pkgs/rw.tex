\section{RW Basic binary I/O utilities}
\label{sec:pkg:rw}
The {\tt rw} package provides a very rudimentary binary I/O capability
for quickly writing {\it single record} direct-access Fortran binary files. 
It is primarily used for writing diagnostic output.

\subsection{Introduction}
Package {\tt rw} is an interface to the more general {\tt mdsio} package.
The {\tt rw} package can be used to write or read direct-access Fortran 
binary files for two-dimensional XY and three-dimensional XYZ arrays. 
The arrays are assumed to have been decalred according to the standard
MITgcm two-dimensional or the-dimensional floating poit array type 
(see figure \ref{fig:pkg:rw:standarddeclaration}).

\begin{figure}
\begin{verbatim}
C     Example of declaring a standard two dimensional "long" floating
C     point type array (the _RL macro is usually mapped to 64-bit 
C     floats in most configurations)
      _RL anArray(1-OLx:sNx+OLx,1-OLy:sNy+OLy,nSx,nSy)
\end{verbatim}
\caption{An example of the fixed form Fortran declaration for a 
standard MITgcm two-dimensional array type.  }
\label{fig:pkg:rw:standarddeclaration}
\end{figure}

Each call to an {\tt rw} read or write routine will read (or write) to the 
first record of a file. To write direct access Fortran files with 
multiple records use the package {\tt mdsio} (see section 
\ref{sec:pkg:mdsio}). To write self-describing files that contain
embedded information describing the variables being written and
the spatial and temporal locations of those variables use the 
package {\tt mnc} (see section \ref{sec:pkg:mnc}) which produces 
\htlink{netCDF}{http://www.unidata.ucar.edu/packages/netcdf} 
\cite{rew:97} based output.

\subsection{Key subroutines, parameters and files}
\label{sec:pkg:rw:implementation_synopsis}
The {\tt rw} package has 
\subsection{Package Reference}
