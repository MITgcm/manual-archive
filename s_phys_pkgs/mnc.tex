% $Header: /u/gcmpack/manual/s_phys_pkgs/Attic/mnc.tex,v 1.5 2004/02/12 03:35:05 edhill Exp $
% $Name:  $

\section{NetCDF I/O Integration}
\label{sec:pkg:mnc}

The \texttt{mnc} package is a set of convenience routines written to
expedite the process of creating, appending, and reading NetCDF files.
NetCDF is an increasingly popular self-describing file format
\cite{rew:97} intended primarily for scientific data sets.  An
extensive collection of NetCDF reference papers, user guides,
software, FAQs, and other information can be obtained from UCAR's web
site at:
\begin{rawhtml} <A href="http://www.unidata.ucar.edu/packages/netcdf/"> \end{rawhtml}
\begin{verbatim}
http://www.unidata.ucar.edu/packages/netcdf/
\end{verbatim}
\begin{rawhtml} </A> \end{rawhtml}


\subsection{Introduction}

The \texttt{mnc} package is a two-level convenience library (or
``wrapper'') for most of the NetCDF Fortran API.  Its purpose is to
streamline the user interface to NetCDF by maintaining internal
relations (``look-up tables'') keyed with strings (or ``names'') and
entities such as NetCDF files, variables, and attributes.

The two levels of the \texttt{mnc} package are:
\begin{description}

\item[Upper level] \ 
  
  The upper level contains information about two kinds of
  associations:
  \begin{description}
  \item[Grid type] is lookup table indexed with a grid type name.
    Each grid type name is associated with a number of dimensions, the
    dimension sizes (one of which may be unlimited), and starting and
    ending index arrays.  The intent is to store all the necessary
    size and shape information for the Fortran arrays containing
    MITgcm--style ``tile'' variables (that is, a central region
    surrounded by a variably-sized ``halo'' or exchange region as
    shown in Figures \ref{fig:communication_primitives} and
    \ref{fig:tiling-strategy}).
  
  \item[Variable type] is a lookup table indexed by a variable type
    name.  For each name, the table contains a reference to a grid
    type for the variable and the names and values of various
    attributes.
  \end{description}
  
  Within the upper level, these associations are not permanently tied
  to any particular NetCDF file.  This allows the information to be
  re-used over multiple file reads and writes.

\item[Lower level] \ 
  
  In the lower (or internal) level, associations are stored for NetCDF
  files and many of the entities that they contain including
  dimensions, variables, and global attributes.  All associations are
  on a per-file basis.  Thus, each entity is tied to a unique NetCDF
  file and will be created or destroyed when files are, respectively,
  opened or closed.

\end{description}


\subsection{Using MNC}

\subsubsection{``Grid--Types'' and ``Variable--Types''}

As a convenience for users, the MNC package includes numerous routines
to aid in the writing of data to NetCDF format.  Probably the biggest
convenience is the use of pre-defined ``grid types'' and ``variable
types''.  These ``types'' are simply look-up tables that store
dimensions, indicies, attributes, and other information that can all
be retrieved using a single character string.

The ``grid types'' are a way of mapping variables within MITgcm to
NetCDF arrays.  Within MITgcm, most spatial variables are defined
using two-- or three--dimensional arrays with ``overlap'' regions (see
Figures \ref{fig:communication_primitives}, a possible vertical index,
and \ref{fig:tiling-strategy}) and tile indicies such as the following
``U'' velocity:
\begin{verbatim}
      _RL  uVel (1-OLx:sNx+OLx,1-OLy:sNy+OLy,Nr,nSx,nSy)
\end{verbatim}
as defined in \filelink{model/inc/DYNVARS.h}{model-inc-DYNVARS.h}

The grid type is a character string that encodes the presence and
types associated with the four possible dimensions.  The character
string follows the format
\begin{center}
  \texttt{H0\_H1\_H2\_\_V\_\_T}
\end{center}
where the terms \textit{H0,H1,H2,V,T} can be almost any combination of
the following:
\begin{center}
  \begin{tabular}[h]{|ccc|c|c|}\hline
    \multicolumn{3}{|c|}{Horizontal} & Vertical &  \\
    Location & Direction & Halo & Location & Time  \\\hline
    H0  &  H1  &  H2  &  V  &  T  \\\hline
    \texttt{-} & xy & Hn & \texttt{-} & \texttt{-} \\
    U  &  x  &  Hy  &  i  &  t  \\
    V  &  y  &      &  c  &     \\
    Cen  &   &      &     &     \\
    Cor  &   &      &     &     \\\hline
  \end{tabular}
\end{center}
A example list of all pre-defined combinations is contained in the
file
\begin{center}
  \texttt{pkg/mnc/pre-defined\_grids.txt}.
\end{center}


\subsubsection{An Example}

Writing variables to NetCDF files can be accomplished in as few as two
function calls.  The first function call defines a variable type,
associates it with a name (character string), and provides additional
information about the indicies for \texttt{bi,bj} dimensions.  The
second function call will write variable at, if necessary, the
current time level within the model.

Examples of the initialization calls can be found in the file 
\begin{center}
  \filelink{model/src/initialise\_fixed.F}{model-src-initialise_fixed.F}
\end{center}
where these four function calls: {\footnotesize
\begin{verbatim}
  C     Create MNC definitions for DYNVARS.h variables
        CALL MNC_CW_ADD_VNAME(myThid, 'iter', '-_-_--__-__t', 0,0)
        CALL MNC_CW_ADD_VATTR_TEXT(myThid,'iter',1,
       &     'long_name','iteration_count')
        CALL MNC_CW_ADD_VNAME(myThid, 'U', 'U_xy_Hn__C__t', 4,5)
        CALL MNC_CW_ADD_VATTR_TEXT(myThid,'U',1,'units','m/s')
\end{verbatim} }
{\noindent initialize two \texttt{VNAME}s and add one attribute to each.}
    
The two variables are subsequently written at specific time steps
within 
\begin{center}
  \filelink{model/src/write\_state.F}{model-src-write_state.F}
\end{center}
using the function calls: {\footnotesize
\begin{verbatim}
  C     Write the DYNVARS.h variables using the MNC package
        mnc_iter = myIter
        CALL MNC_CW_RL_W_R(myThid,'state',0,0,'iter',-1,mnc_iter)
        CALL MNC_CW_RL_W_D(myThid,'state',0,0,'U', 0, uVel)
\end{verbatim} }


\subsection{Key subroutines, parameters and files}

All of the variables used to implement the lookup tables are described
in \filelink{model/src/write\_state.F}{model-src-write_state.F}



\subsection{Package Reference}

