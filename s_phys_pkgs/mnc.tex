% $Header: /u/gcmpack/manual/s_phys_pkgs/Attic/mnc.tex,v 1.9 2004/04/06 16:48:33 edhill Exp $
% $Name:  $

\section{NetCDF I/O Integration: MNC}
\label{sec:pkg:mnc}

The \texttt{mnc} package is a set of convenience routines written to
expedite the process of creating, appending, and reading NetCDF files.
NetCDF is an increasingly popular self-describing file format
\cite{rew:97} intended primarily for scientific data sets.  An
extensive collection of NetCDF reference papers, user guides,
software, FAQs, and other information can be obtained from UCAR's web
site at:
\begin{rawhtml} <A href="http://www.unidata.ucar.edu/packages/netcdf/"> \end{rawhtml}
\begin{verbatim}
http://www.unidata.ucar.edu/packages/netcdf/
\end{verbatim}
\begin{rawhtml} </A> \end{rawhtml}


\subsection{Introduction}

The \texttt{mnc} package is a two-level convenience library (or
``wrapper'') for most of the NetCDF Fortran API.  Its purpose is to
streamline the user interface to NetCDF by maintaining internal
relations (look-up tables) keyed with strings (or names) and entities
such as NetCDF files, variables, and attributes.

The two levels of the \texttt{mnc} package are:
\begin{description}

\item[Upper level] \ 
  
  The upper level contains information about two kinds of
  associations:
  \begin{description}
  \item[grid type] is lookup table indexed with a grid type name.
    Each grid type name is associated with a number of dimensions, the
    dimension sizes (one of which may be unlimited), and starting and
    ending index arrays.  The intent is to store all the necessary
    size and shape information for the Fortran arrays containing
    MITgcm--style ``tile'' variables (that is, a central region
    surrounded by a variably-sized ``halo'' or exchange region as
    shown in Figures \ref{fig:communication_primitives} and
    \ref{fig:tiling-strategy}).
  
  \item[variable type] is a lookup table indexed by a variable type
    name.  For each name, the table contains a reference to a grid
    type for the variable and the names and values of various
    attributes.
  \end{description}
  
  Within the upper level, these associations are not permanently tied
  to any particular NetCDF file.  This allows the information to be
  re-used over multiple file reads and writes.

\item[Lower level] \ 
  
  In the lower (or internal) level, associations are stored for NetCDF
  files and many of the entities that they contain including
  dimensions, variables, and global attributes.  All associations are
  on a per-file basis.  Thus, each entity is tied to a unique NetCDF
  file and will be created or destroyed when files are, respectively,
  opened or closed.

\end{description}


\subsection{Using MNC}

\subsubsection{Grid--Types and Variable--Types}

As a convenience for users, the MNC package includes numerous routines
to aid in the writing of data to NetCDF format.  Probably the biggest
convenience is the use of pre-defined ``grid types'' and ``variable
types''.  These ``types'' are simply look-up tables that store
dimensions, indicies, attributes, and other information that can all
be retrieved using a single character string.

The ``grid types'' are a way of mapping variables within MITgcm to
NetCDF arrays.  Within MITgcm, most spatial variables are defined
using two-- or three--dimensional arrays with ``overlap'' regions (see
Figures \ref{fig:communication_primitives}, a possible vertical index,
and \ref{fig:tiling-strategy}) and tile indicies such as the following
``U'' velocity:
\begin{verbatim}
      _RL  uVel (1-OLx:sNx+OLx,1-OLy:sNy+OLy,Nr,nSx,nSy)
\end{verbatim}
as defined in \filelink{model/inc/DYNVARS.h}{model-inc-DYNVARS.h}

The grid type is a character string that encodes the presence and
types associated with the four possible dimensions.  The character
string follows the format
\begin{center}
  \texttt{H0\_H1\_H2\_\_V\_\_T}
\end{center}
where the terms \textit{H0}, \textit{H1}, \textit{H2}, \textit{V},
\textit{T} can be almost any combination of the following:
\begin{center}
  \begin{tabular}[h]{|ccc|c|c|}\hline
    \multicolumn{3}{|c|}{Horizontal} & Vertical & Time \\
    \textbf{H0}: location & \textbf{H1}: dimensions & \textbf{H2}: halo 
          & \textbf{V}: location & \textbf{T}: level  \\\hline
    \texttt{-} & xy & Hn & \texttt{-} & \texttt{-} \\
    U  &  x  &  Hy  &  i  &  t  \\
    V  &  y  &      &  c  &     \\
    Cen  &   &      &     &     \\
    Cor  &   &      &     &     \\\hline
  \end{tabular}
\end{center}
A example list of all pre-defined combinations is contained in the
file
\begin{center}
  \texttt{pkg/mnc/pre-defined\_grids.txt}.
\end{center}

The variable type is an association between a variable type name and the
following items:
\begin{center}
  \begin{tabular}[h]{|l|l|}\hline
    \textbf{Item}  & \textbf{Purpose}  \\\hline
    grid type  &  defines the in-memory arrangement  \\
    \texttt{bi,bj} dimensions  &  tiling indices, if present  \\\hline
  \end{tabular}
\end{center}
and is used by the \texttt{mnc\_cw\_*\_[R|W]} subroutines for reading
and writing variables.


\subsubsection{An Example}

Writing variables to NetCDF files can be accomplished in as few as two
function calls.  The first function call defines a variable type,
associates it with a name (character string), and provides additional
information about the indicies for the tile (\texttt{bi},\texttt{bj})
dimensions.  The second function call will write the data at, if
necessary, the current time level within the model.

Examples of the initialization calls can be found in the file 
\filelink{model/src/ini\_mnc\_io.F}{model-src-ini_mnc_io.F}
where these function calls:
{\footnotesize
\begin{verbatim}
C     Create MNC definitions for DYNVARS.h variables
      CALL MNC_CW_ADD_VNAME('iter', '-_-_--__-__t', 0,0, myThid)
      CALL MNC_CW_ADD_VATTR_TEXT('iter',1,
     &     'long_name','iteration_count', myThid)

      CALL MNC_CW_ADD_VNAME('model_time', '-_-_--__-__t', 0,0, myThid)
      CALL MNC_CW_ADD_VATTR_TEXT('model_time',1,
     &     'long_name','Model Time', myThid)
      CALL MNC_CW_ADD_VATTR_TEXT('model_time',1,'units','s', myThid)

      CALL MNC_CW_ADD_VNAME('U', 'U_xy_Hn__C__t', 4,5, myThid)
      CALL MNC_CW_ADD_VATTR_TEXT('U',1,'units','m/s', myThid)
      CALL MNC_CW_ADD_VATTR_TEXT('U',1,
     &     'coordinates','XU YU RC iter', myThid)

      CALL MNC_CW_ADD_VNAME('T', 'Cen_xy_Hn__C__t', 4,5, myThid)
      CALL MNC_CW_ADD_VATTR_TEXT('T',1,'units','degC', myThid)
      CALL MNC_CW_ADD_VATTR_TEXT('T',1,'long_name',
     &     'potential_temperature', myThid)
      CALL MNC_CW_ADD_VATTR_TEXT('T',1,
     &     'coordinates','XC YC RC iter', myThid)
\end{verbatim}
}
{\noindent initialize four \texttt{VNAME}s and add one or more NetCDF
  attributes to each.}
    
The four variables defined above are subsequently written at specific
time steps within
\filelink{model/src/write\_state.F}{model-src-write_state.F}
using the function calls:
{\footnotesize
\begin{verbatim}
C       Write dynvars using the MNC package
        CALL MNC_CW_SET_UDIM('state', -1, myThid)
        CALL MNC_CW_RL_W('I','state',0,0,'iter', myIter, myThid)
        CALL MNC_CW_SET_UDIM('state', 0, myThid)
        CALL MNC_CW_RL_W('D','state',0,0,'model_time',myTime, myThid)
        CALL MNC_CW_RL_W('D','state',0,0,'U', uVel, myThid)
        CALL MNC_CW_RL_W('D','state',0,0,'T', theta, myThid)
\end{verbatim}
}


\subsubsection{Parameters}

All the MNC parameters are contained within a file named
\texttt{data.mnc}.  If this file does not exist, then the MNC package
will interpret that as an indication that it is not to be used.  If
the \texttt{data.mnc} does exist, then it may contain the following
parameters:

\begin{center}
  {\footnotesize
    \begin{tabular}[htb]{|l|l|l|l|}\hline
      &  &  &  \\
      \textbf{Name}  &  \textbf{Type}  &  
      \textbf{Default}  &  \textbf{Description}  \\\hline
      &  &  &  \\
      \texttt{useMNC}  &  Logical  & \texttt{.FALSE.}  &  
      \textbf{overall MNC ON/OFF switch}  \\
      \texttt{mnc\_echo\_gvtypes}  &  Logical  & \texttt{.FALSE.}  &  
      echo pre-defined ``types'' to STDOUT?   \\
      \texttt{mnc\_use\_outdir}  &  Logical  & \texttt{.FALSE.}  &  
      create a directory for output?  \\
      \texttt{mnc\_outdir\_str}  &  String  & \texttt{'mnc\_'}  &  
      output directory name \\
      \texttt{mnc\_outdir\_date}  &  Logical  & \texttt{.FALSE.}  &  
      embed date in output directory name?  \\
      \texttt{mnc\_pickup\_write}  &  Logical  & \texttt{.FALSE.}  &  
      use MNC to write (create) pickup files?  \\
      \texttt{mnc\_pickup\_read}  &  Logical  & \texttt{.FALSE.}  &  
      use MNC to read pickup files?  \\
      \texttt{mnc\_use\_indir}  &  Logical  & \texttt{.FALSE.}  &  
      use a directory (path) for input?  \\
      \texttt{mnc\_indir\_str}  &  String  & \texttt{''}  &  
      input directory (or path) name  \\
      \texttt{mnc\_use\_for\_mon}  &  Logical  & \texttt{.FALSE.}  &  
      write \texttt{monitor} output using MNC?  \\\hline
    \end{tabular}
  }
\end{center}

%\subsection{Package Reference}

