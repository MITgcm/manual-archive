\section{Fizhi: High-end Atmospheric Physics}
\label{sec:pkg:fizhi}
\begin{rawhtml}
<!-- CMIREDIR:package_fizhi: -->
\end{rawhtml}
%   EPSF.TEX macro file:
%   Written by Tomas Rokicki of Radical Eye Software, 29 Mar 1989.
%   Revised by Don Knuth, 3 Jan 1990.
%   Revised by Tomas Rokicki to accept bounding boxes with no
%      space after the colon, 18 Jul 1990.
%
%   TeX macros to include an Encapsulated PostScript graphic.
%   Works by finding the bounding box comment,
%   calculating the correct scale values, and inserting a vbox
%   of the appropriate size at the current position in the TeX document.
%
%   To use with the center environment of LaTeX, preface the \epsffile
%   call with a \leavevmode.  (LaTeX should probably supply this itself
%   for the center environment.)
%
%   To use, simply say
%   \input epsf           % somewhere early on in your TeX file
%   \epsfbox{filename.ps} % where you want to insert a vbox for a figure
%
%   Alternatively, you can type
%
%   \epsfbox[0 0 30 50]{filename.ps} % to supply your own BB
%
%   which will not read in the file, and will instead use the bounding
%   box you specify.
%
%   The effect will be to typeset the figure as a TeX box, at the
%   point of your \epsfbox command. By default, the graphic will have its
%   `natural' width (namely the width of its bounding box, as described
%   in filename.ps). The TeX box will have depth zero.
%
%   You can enlarge or reduce the figure by saying
%     \epsfxsize=<dimen> \epsfbox{filename.ps}
%   (or
%     \epsfysize=<dimen> \epsfbox{filename.ps})
%   instead. Then the width of the TeX box will be \epsfxsize and its
%   height will be scaled proportionately (or the height will be
%   \epsfysize and its width will be scaled proportiontally).  The
%   width (and height) is restored to zero after each use.
%
%   A more general facility for sizing is available by defining the
%   \epsfsize macro.    Normally you can redefine this macro
%   to do almost anything.  The first parameter is the natural x size of
%   the PostScript graphic, the second parameter is the natural y size
%   of the PostScript graphic.  It must return the xsize to use, or 0 if
%   natural scaling is to be used.  Common uses include:
%
%      \epsfxsize  % just leave the old value alone
%      0pt         % use the natural sizes
%      #1          % use the natural sizes
%      \hsize      % scale to full width
%      0.5#1       % scale to 50% of natural size
%      \ifnum#1>\hsize\hsize\else#1\fi  % smaller of natural, hsize
%
%   If you want TeX to report the size of the figure (as a message
%   on your terminal when it processes each figure), say `\epsfverbosetrue'.
%
\newread\epsffilein    % file to \read
\newif\ifepsffileok    % continue looking for the bounding box?
\newif\ifepsfbbfound   % success?
\newif\ifepsfverbose   % report what you're making?
\newdimen\epsfxsize    % horizontal size after scaling
\newdimen\epsfysize    % vertical size after scaling
\newdimen\epsftsize    % horizontal size before scaling
\newdimen\epsfrsize    % vertical size before scaling
\newdimen\epsftmp      % register for arithmetic manipulation
\newdimen\pspoints     % conversion factor
%
\pspoints=1bp          % Adobe points are `big'
\epsfxsize=0pt         % Default value, means `use natural size'
\epsfysize=0pt         % ditto
%
\def\epsfbox#1{\global\def\epsfllx{72}\global\def\epsflly{72}%
   \global\def\epsfurx{540}\global\def\epsfury{720}%
   \def\lbracket{[}\def\testit{#1}\ifx\testit\lbracket
   \let\next=\epsfgetlitbb\else\let\next=\epsfnormal\fi\next{#1}}%
%
\def\epsfgetlitbb#1#2 #3 #4 #5]#6{\epsfgrab #2 #3 #4 #5 .\\%
   \epsfsetgraph{#6}}%
%
\def\epsfnormal#1{\epsfgetbb{#1}\epsfsetgraph{#1}}%
%
\def\epsfgetbb#1{%
%
%   The first thing we need to do is to open the
%   PostScript file, if possible.
%
\openin\epsffilein=#1
\ifeof\epsffilein\errmessage{I couldn't open #1, will ignore it}\else
%
%   Okay, we got it. Now we'll scan lines until we find one that doesn't
%   start with %. We're looking for the bounding box comment.
%
   {\epsffileoktrue \chardef\other=12
    \def\do##1{\catcode`##1=\other}\dospecials \catcode`\ =10
    \loop
       \read\epsffilein to \epsffileline
       \ifeof\epsffilein\epsffileokfalse\else
%
%   We check to see if the first character is a % sign;
%   if not, we stop reading (unless the line was entirely blank);
%   if so, we look further and stop only if the line begins with
%   `%%BoundingBox:'.
%
          \expandafter\epsfaux\epsffileline:. \\%
       \fi
   \ifepsffileok\repeat
   \ifepsfbbfound\else
    \ifepsfverbose\message{No bounding box comment in #1; using defaults}\fi\fi
   }\closein\epsffilein\fi}%
%
%   Now we have to calculate the scale and offset values to use.
%   First we compute the natural sizes.
%
\def\epsfclipstring{}% do we clip or not?  If so,
\def\epsfclipon{\def\epsfclipstring{ clip}}%
\def\epsfclipoff{\def\epsfclipstring{}}%
%
\def\epsfsetgraph#1{%
   \epsfrsize=\epsfury\pspoints
   \advance\epsfrsize by-\epsflly\pspoints
   \epsftsize=\epsfurx\pspoints
   \advance\epsftsize by-\epsfllx\pspoints
%
%   If `epsfxsize' is 0, we default to the natural size of the picture.
%   Otherwise we scale the graph to be \epsfxsize wide.
%
   \epsfxsize\epsfsize\epsftsize\epsfrsize
   \ifnum\epsfxsize=0 \ifnum\epsfysize=0
      \epsfxsize=\epsftsize \epsfysize=\epsfrsize
      \epsfrsize=0pt
%
%   We have a sticky problem here:  TeX doesn't do floating point arithmetic!
%   Our goal is to compute y = rx/t. The following loop does this reasonably
%   fast, with an error of at most about 16 sp (about 1/4000 pt).
% 
     \else\epsftmp=\epsftsize \divide\epsftmp\epsfrsize
       \epsfxsize=\epsfysize \multiply\epsfxsize\epsftmp
       \multiply\epsftmp\epsfrsize \advance\epsftsize-\epsftmp
       \epsftmp=\epsfysize
       \loop \advance\epsftsize\epsftsize \divide\epsftmp 2
       \ifnum\epsftmp>0
          \ifnum\epsftsize<\epsfrsize\else
             \advance\epsftsize-\epsfrsize \advance\epsfxsize\epsftmp \fi
       \repeat
       \epsfrsize=0pt
     \fi
   \else \ifnum\epsfysize=0
     \epsftmp=\epsfrsize \divide\epsftmp\epsftsize
     \epsfysize=\epsfxsize \multiply\epsfysize\epsftmp   
     \multiply\epsftmp\epsftsize \advance\epsfrsize-\epsftmp
     \epsftmp=\epsfxsize
     \loop \advance\epsfrsize\epsfrsize \divide\epsftmp 2
     \ifnum\epsftmp>0
        \ifnum\epsfrsize<\epsftsize\else
           \advance\epsfrsize-\epsftsize \advance\epsfysize\epsftmp \fi
     \repeat
     \epsfrsize=0pt
    \else
     \epsfrsize=\epsfysize
    \fi
   \fi
%
%  Finally, we make the vbox and stick in a \special that dvips can parse.
%
   \ifepsfverbose\message{#1: width=\the\epsfxsize, height=\the\epsfysize}\fi
   \epsftmp=10\epsfxsize \divide\epsftmp\pspoints
   \vbox to\epsfysize{\vfil\hbox to\epsfxsize{%
      \ifnum\epsfrsize=0\relax
        \special{PSfile=#1 llx=\epsfllx\space lly=\epsflly\space
            urx=\epsfurx\space ury=\epsfury\space rwi=\number\epsftmp
            \epsfclipstring}%
      \else
        \epsfrsize=10\epsfysize \divide\epsfrsize\pspoints
        \special{PSfile=#1 llx=\epsfllx\space lly=\epsflly\space
            urx=\epsfurx\space ury=\epsfury\space rwi=\number\epsftmp\space
            rhi=\number\epsfrsize \epsfclipstring}%
      \fi
      \hfil}}%
\global\epsfxsize=0pt\global\epsfysize=0pt}%
%
%   We still need to define the tricky \epsfaux macro. This requires
%   a couple of magic constants for comparison purposes.
%
{\catcode`\%=12 \global\let\epsfpercent=%\global\def\epsfbblit{%BoundingBox}}%
%
%   So we're ready to check for `%BoundingBox:' and to grab the
%   values if they are found.
%
\long\def\epsfaux#1#2:#3\\{\ifx#1\epsfpercent
   \def\testit{#2}\ifx\testit\epsfbblit
      \epsfgrab #3 . . . \\%
      \epsffileokfalse
      \global\epsfbbfoundtrue
   \fi\else\ifx#1\par\else\epsffileokfalse\fi\fi}%
%
%   Here we grab the values and stuff them in the appropriate definitions.
%
\def\epsfempty{}%
\def\epsfgrab #1 #2 #3 #4 #5\\{%
\global\def\epsfllx{#1}\ifx\epsfllx\epsfempty
      \epsfgrab #2 #3 #4 #5 .\\\else
   \global\def\epsflly{#2}%
   \global\def\epsfurx{#3}\global\def\epsfury{#4}\fi}%
%
%   We default the epsfsize macro.
%
\def\epsfsize#1#2{\epsfxsize}
%
%   Finally, another definition for compatibility with older macros.
%
\let\epsffile=\epsfbox



\subsection{Introduction}
The fizhi (high-end atmospheric physics) package includes a collection of state-of-the-art
physical parameterizations for atmospheric radiation, cumulus convection, atmospheric
boundary layer turbulence, and land surface processes.

% *************************************************************************
% *************************************************************************
 
\subsection{Equations}

\subsubsection{Moist Convective Processes}

\paragraph{Sub-grid and Large-scale Convection}
\label{sec:fizhi:mc}

Sub-grid scale cumulus convection is parameterized using the Relaxed Arakawa
Schubert (RAS) scheme of Moorthi and Suarez (1992), which is a linearized Arakawa Schubert
type scheme.  RAS predicts the mass flux from an ensemble of clouds.  Each subensemble is identified
by its entrainment rate and level of neutral bouyancy which are determined by the grid-scale properties.

The thermodynamic variables that are used in RAS to describe the grid scale vertical profile are
the dry static energy, $s=c_pT +gz$, and the moist static energy, $h=c_p T + gz + Lq$. 
The conceptual model behind RAS depicts each subensemble as a rising plume cloud, entraining 
mass from the environment during ascent, and detraining all cloud air at the level of neutral 
buoyancy. RAS assumes that the normalized cloud mass flux, $\eta$, normalized by the cloud base 
mass flux, is a linear function of height, expressed as:
\[
\pp{\eta(z)}{z} = \lambda \hspace{0.4cm}or\hspace{0.4cm} \pp{\eta(P^{\kappa})}{P^{\kappa}} = 
-{c_p \over {g}}\theta\lambda
\]
where we have used the hydrostatic equation written in the form:
\[
\pp{z}{P^{\kappa}} = -{c_p \over {g}}\theta
\]

The entrainment parameter, $\lambda$, characterizes a particular subensemble based on its
detrainment level, and is obtained by assuming that the level of detrainment is the level of neutral
buoyancy, ie., the level at which the moist static energy of the cloud, $h_c$, is equal 
to the saturation moist static energy of the environment, $h^*$.  Following Moorthi and Suarez (1992),
$\lambda$ may be written as
\[
\lambda = { {h_B - h^*_D} \over { {c_p \over g} {\int_{P_D}^{P_B}\theta(h^*_D-h)dP^{\kappa}}} } ,
\]

where the subscript $B$ refers to cloud base, and the subscript $D$ refers to the detrainment level.


The convective instability is measured in terms of the cloud work function $A$, defined as the
rate of change of cumulus kinetic energy. The cloud work function is 
related to the buoyancy, or the difference
between the moist static energy in the cloud and in the environment:
\[
A = \int_{P_D}^{P_B} { {\eta \over {1 + \gamma} } 
\left[ {{h_c-h^*} \over {P^{\kappa}}} \right] dP^{\kappa}}
\]

where $\gamma$ is ${L \over {c_p}}\pp{q^*}{T}$ obtained from the Claussius Clapeyron equation,
and the subscript $c$ refers to the value inside the cloud.


To determine the cloud base mass flux, the rate of change of $A$ in time {\em due to dissipation by 
the clouds} is assumed to approximately balance the rate of change of $A$ {\em due to the generation 
by the large scale}. This is the quasi-equilibrium assumption, and results in an expression for $m_B$:
\[
m_B = {{- \left.{dA \over dt} \right|_{ls}} \over K}
\]

where $K$ is the cloud kernel, defined as the rate of change of the cloud work function per
unit cloud base mass flux, and is currently obtained by analytically differentiating the 
expression for $A$ in time.
The rate of change of $A$ due to the generation by the large scale can be written as the
difference between the current $A(t+\Delta t)$ and its equillibrated value after the previous 
convective time step 
$A(t)$, divided by the time step. $A(t)$ is approximated as some critical $A_{crit}$,
computed by Lord (1982) from $in situ$ observations.


The predicted convective mass fluxes are used to solve grid-scale temperature
and moisture budget equations to determine the impact of convection on the large scale fields of
temperature (through latent heating and compensating subsidence) and moisture (through
precipitation and detrainment):
\[
\left.{\pp{\theta}{t}}\right|_{c} = \alpha { m_B \over {c_p P^{\kappa}}} \eta \pp{s}{p}
\]
and
\[
\left.{\pp{q}{t}}\right|_{c} = \alpha { m_B \over {L}} \eta (\pp{h}{p}-\pp{s}{p})
\]
where $\theta = {T \over P^{\kappa}}$, $P = (p/p_0)$, and $\alpha$ is the relaxation parameter.

As an approximation to a full interaction between the different allowable subensembles,
many clouds are simulated frequently, each modifying the large scale environment some fraction
$\alpha$ of the total adjustment. The parameterization thereby ``relaxes'' the large scale environment
towards equillibrium.  

In addition to the RAS cumulus convection scheme, the fizhi package employs a
Kessler-type scheme for the re-evaporation of falling rain (Sud and Molod, 1988), which
correspondingly adjusts the temperature assuming $h$ is conserved. RAS in its current
formulation assumes that all cloud water is deposited into the detrainment level as rain.
All of the rain is available for re-evaporation, which begins in the level below detrainment. 
The scheme accounts for some microphysics such as
the rainfall intensity, the drop size distribution, as well as the temperature, 
pressure and relative humidity of the surrounding air.  The fraction of the moisture deficit 
in any model layer into which the rain may re-evaporate is controlled by a free parameter,
which allows for a relatively efficient re-evaporation of liquid precipitate and larger rainout
for frozen precipitation.

Due to the increased vertical resolution near the surface, the lowest model 
layers are averaged to provide a 50 mb thick sub-cloud layer for RAS.  Each time RAS is
invoked (every ten simulated minutes), 
a number of randomly chosen subensembles are checked for the possibility 
of convection, from just above cloud base to 10 mb.  

Supersaturation or large-scale precipitation is initiated in the fizhi package whenever 
the relative humidity in any grid-box exceeds a critical value, currently 100 \%.
The large-scale precipitation re-evaporates during descent to partially saturate 
lower layers in a process identical to the re-evaporation of convective rain. 

 
\paragraph{Cloud Formation}
\label{sec:fizhi:clouds}

Convective and large-scale cloud fractons which are used for cloud-radiative interactions are determined
diagnostically as part of the cumulus and large-scale parameterizations.
Convective cloud fractions produced by RAS are proportional to the 
detrained liquid water amount given by

\[
F_{RAS} = \min\left[ {l_{RAS}\over l_c}, 1.0 \right]
\]

where $l_c$ is an assigned critical value equal to $1.25$ g/kg.
A memory is associated with convective clouds defined by:

\[
F_{RAS}^n = \min\left[ F_{RAS} + (1-{\Delta t_{RAS}\over\tau})F_{RAS}^{n-1}, 1.0 \right]
\]

where $F_{RAS}$ is the instantanious cloud fraction and $F_{RAS}^{n-1}$ is the cloud fraction
from the previous RAS timestep.  The memory coefficient is computed using a RAS cloud timescale,
$\tau$, equal to 1 hour.  RAS cloud fractions are cleared when they fall below 5 \%.

Large-scale cloudiness is defined, following Slingo and Ritter (1985), as a function of relative
humidity:

\[
F_{LS} = \min\left[ { \left( {RH-RH_c \over 1-RH_c} \right) }^2, 1.0 \right]
\]

where

\bqa
RH_c & = & 1-s(1-s)(2-\sqrt{3}+2\sqrt{3} \, s)r \nonumber \\
   s & = & p/p_{surf} \nonumber \\
   r & = & \left( {1.0-RH_{min} \over \alpha} \right) \nonumber \\
RH_{min} & = & 0.75 \nonumber \\
\alpha & = & 0.573285 \nonumber  .
\eqa

These cloud fractions are suppressed, however, in regions where the convective
sub-cloud layer is conditionally unstable.  The functional form of $RH_c$ is shown in
Figure (\ref{fig:fizhi:rhcrit}).

\begin{figure*}[htbp]
  \vspace{0.4in}
  \centerline{  \epsfysize=4.0in  \epsfbox{part6/rhcrit.ps}}
  \vspace{0.4in}
  \caption  [Critical Relative Humidity for Clouds.] 
            {Critical Relative Humidity for Clouds.} 
  \label{fig:fizhi:rhcrit}
\end{figure*}

The total cloud fraction in a grid box is determined by the larger of the two cloud fractions:

\[
F_{CLD} = \max \left[ F_{RAS},F_{LS} \right] .
\]

Finally, cloud fractions are time-averaged between calls to the radiation packages.


\subsubsection{Radiation}

The parameterization of radiative heating in the fizhi package includes effects 
from both shortwave and longwave processes.
Radiative fluxes are calculated at each
model edge-level in both up and down directions.
The heating rates/cooling rates are then obtained 
from the vertical divergence of the net radiative fluxes.

The net flux is
\[
F = F^\uparrow - F^\downarrow
\]
where $F$ is the net flux, $F^\uparrow$ is the upward flux and $F^\downarrow$ is
the downward flux.

The heating rate due to the divergence of the radiative flux is given by
\[
\pp{\rho c_p T}{t} = - \pp{F}{z}
\]
or
\[
\pp{T}{t} = \frac{g}{c_p \pi} \pp{F}{\sigma}
\]
where $g$ is the accelation due to gravity
and $c_p$ is the heat capacity of air at constant pressure.
  
The time tendency for Longwave
Radiation is updated every 3 hours.  The time tendency for Shortwave Radiation is updated once
every three hours assuming a normalized incident solar radiation, and subsequently modified at
every model time step by the true incident radiation.  
The solar constant value used in the package is equal to 1365 $W/m^2$
and a $CO_2$ mixing ratio of 330 ppm. 
For the ozone mixing ratio, monthly mean zonally averaged 
climatological values specified as a function
of latitude and height (Rosenfield, et al., 1987) are linearly interpolated to the current time.


\paragraph{Shortwave Radiation}

The shortwave radiation package used in the package computes solar radiative 
heating due to the absoption by water vapor, ozone, carbon dioxide, oxygen,
clouds, and aerosols and due to the
scattering by clouds, aerosols, and gases.
The shortwave radiative processes are described by 
Chou (1990,1992). This shortwave package
uses the Delta-Eddington approximation to compute the
bulk scattering properties of a single layer following King and Harshvardhan (JAS, 1986).
The transmittance and reflectance of diffuse radiation
follow the procedures of Sagan and Pollock (JGR, 1967) and Lacis and Hansen (JAS, 1974).

Highly accurate heating rate calculations are obtained through the use
of an optimal grouping strategy of spectral bands.  By grouping the UV and visible regions
as indicated in Table \ref{tab:fizhi:solar2}, the Rayleigh scattering and the ozone absorption of solar radiation
can be accurately computed in the ultraviolet region and the photosynthetically
active radiation (PAR) region.
The computation of solar flux in the infrared region is performed with a broadband
parameterization using the spectrum regions shown in Table \ref{tab:fizhi:solar1}.
The solar radiation algorithm used in the fizhi package can be applied not only for climate studies but
also for studies on the photolysis in the upper atmosphere and the photosynthesis in the biosphere.

\begin{table}[htb]
\begin{center}
{\bf UV and Visible Spectral Regions} \\
\vspace{0.1in}
\begin{tabular}{|c|c|c|} 
\hline
Region & Band & Wavelength (micron) \\ \hline
\hline
UV-C   &  1.  &  .175 - .225  \\
       &  2.  &  .225 - .245  \\
       &      &  .260 - .280  \\
       &  3.  &  .245 - .260  \\ \hline
UV-B   &  4.  &  .280 - .295  \\
       &  5.  &  .295 - .310  \\
       &  6.  &  .310 - .320  \\ \hline
UV-A   &  7.  &  .320 - .400  \\ \hline
PAR    &  8.  &  .400 - .700  \\
\hline
\end{tabular}
\end{center}
\caption{UV and Visible Spectral Regions used in shortwave radiation package.}
\label{tab:fizhi:solar2}
\end{table}

\begin{table}[htb]
\begin{center}
{\bf Infrared Spectral Regions} \\
\vspace{0.1in}
\begin{tabular}{|c|c|c|} 
\hline
Band & Wavenumber(cm$^{-1}$) & Wavelength (micron) \\ \hline
\hline
1  &    1000-4400    &    2.27-10.0 \\
2  &    4400-8200    &    1.22-2.27 \\
3  &    8200-14300   &    0.70-1.22 \\
\hline
\end{tabular}
\end{center}
\caption{Infrared Spectral Regions used in shortwave radiation package.}
\label{tab:fizhi:solar1}
\end{table}

Within the shortwave radiation package, 
both ice and liquid cloud particles are allowed to co-exist in any of the model layers. 
Two sets of cloud parameters are used, one for ice paticles and the other for liquid particles.
Cloud parameters are defined as the cloud optical thickness and the effective cloud particle size.
In the fizhi package, the effective radius for water droplets is given as 10 microns,
while 65 microns is used for ice particles.  The absorption due to aerosols is currently
set to zero.

To simplify calculations in a cloudy atmosphere, clouds are
grouped into low ($p>700$ mb), middle (700 mb $\ge p > 400$ mb), and high ($p < 400$ mb) cloud regions. 
Within each of the three regions, clouds are assumed maximally
overlapped, and the cloud cover of the group is the maximum
cloud cover of all the layers in the group.  The optical thickness
of a given layer is then scaled for both the direct (as a function of the
solar zenith angle) and diffuse beam radiation 
so that the grouped layer reflectance is the same as the original reflectance.
The solar flux is computed for each of the eight cloud realizations possible 
(see Figure \ref{fig:fizhi:cloud}) within this
low/middle/high classification, and appropriately averaged to produce the net solar flux.

\begin{figure*}[htbp]
  \vspace{0.4in}
  \centerline{  \epsfysize=4.0in  %\epsfbox{part6/rhcrit.ps}
             }
  \vspace{0.4in}
  \caption  {Low-Middle-High Cloud Configurations} 
  \label{fig:fizhi:cloud}
\end{figure*}


\paragraph{Longwave Radiation}

The longwave radiation package used in the fizhi package is thoroughly described by Chou and Suarez (1994).
As described in that document, IR fluxes are computed due to absorption by water vapor, carbon
dioxide, and ozone.  The spectral bands together with their absorbers and parameterization methods,
configured for the fizhi package, are shown in Table \ref{tab:fizhi:longwave}.


\begin{table}[htb]
\begin{center}
{\bf IR Spectral Bands} \\
\vspace{0.1in}
\begin{tabular}{|c|c|l|c| } 
\hline
Band & Spectral Range (cm$^{-1}$) & Absorber & Method \\ \hline
\hline
1   & 0-340      & H$_2$O line      & T \\ \hline
2   & 340-540    & H$_2$O line      & T \\ \hline
3a  & 540-620    & H$_2$O line      & K \\ 
3b  & 620-720    & H$_2$O continuum & S \\ 
3b  & 720-800    & CO$_2$           & T \\ \hline 
4   & 800-980    & H$_2$O line      & K \\ 
    &            & H$_2$O continuum & S \\ \hline 
    &            & H$_2$O line      & K \\ 
5   & 980-1100   & H$_2$O continuum & S \\ 
    &            & O$_3$            & T \\ \hline 
6   & 1100-1380  & H$_2$O line      & K \\ 
    &            & H$_2$O continuum & S \\ \hline
7   & 1380-1900  & H$_2$O line      & T \\ \hline 
8   & 1900-3000  & H$_2$O line      & K \\ \hline 
\hline
\multicolumn{4}{|l|}{ \quad K: {\em k}-distribution method with linear pressure scaling } \\
\multicolumn{4}{|l|}{ \quad T: Table look-up with temperature and pressure scaling } \\
\multicolumn{4}{|l|}{ \quad S: One-parameter temperature scaling } \\
\hline
\end{tabular}
\end{center}
\vspace{0.1in}
\caption{IR Spectral Bands, Absorbers, and Parameterization Method (from Chou and Suarez, 1994)}
\label{tab:fizhi:longwave}
\end{table}


The longwave radiation package accurately computes cooling rates for the middle and 
lower atmosphere from 0.01 mb to the surface.  Errors are $<$ 0.4 C day$^{-1}$ in cooling
rates and $<$ 1\% in fluxes.  From Chou and Suarez, it is estimated that the total effect of 
neglecting all minor absorption bands and the effects of minor infrared absorbers such as
nitrous oxide (N$_2$O), methane (CH$_4$), and the chlorofluorocarbons (CFCs), is an underestimate
of $\approx$ 5 W/m$^2$ in the downward flux at the surface and an overestimate of $\approx$ 3 W/m$^2$
in the upward flux at the top of the atmosphere.

Similar to the procedure used in the shortwave radiation package, clouds are grouped into
three regions catagorized as low/middle/high.
The net clear line-of-site probability $(P)$ between any two levels, $p_1$ and $p_2 \quad (p_2 > p_1)$,  
assuming randomly overlapped cloud groups, is simply the product of the probabilities within each group:

\[ P_{net} = P_{low} \times P_{mid} \times P_{hi} . \]

Since all clouds within a group are assumed maximally overlapped, the clear line-of-site probability within
a group is given by:

\[ P_{group} = 1 - F_{max} , \]

where $F_{max}$ is the maximum cloud fraction encountered between $p_1$ and $p_2$ within that group.
For groups and/or levels outside the range of $p_1$ and $p_2$, a clear line-of-site probability equal to 1 is
assigned.


\paragraph{Cloud-Radiation Interaction}
\label{sec:fizhi:radcloud}

The cloud fractions and diagnosed cloud liquid water produced by moist processes 
within the fizhi package are used in the radiation packages to produce cloud-radiative forcing.
The cloud optical thickness associated with large-scale cloudiness is made
proportional to the diagnosed large-scale liquid water, $\ell$, detrained due to super-saturation.
Two values are used corresponding to cloud ice particles and water droplets.
The range of optical thickness for these clouds is given as

\[ 0.0002 \le \tau_{ice} (mb^{-1}) \le 0.002  \quad\mbox{for}\quad  0 \le \ell \le 2 \quad\mbox{mg/kg} , \]
\[ 0.02 \le \tau_{h_2o} (mb^{-1}) \le 0.2  \quad\mbox{for}\quad  0 \le \ell \le 10 \quad\mbox{mg/kg} . \]

The partitioning, $\alpha$,  between ice particles and water droplets is achieved through a linear scaling
in temperature:

\[ 0 \le \alpha \le 1 \quad\mbox{for}\quad  233.15 \le T \le 253.15 . \]

The resulting optical depth associated with large-scale cloudiness is given as

\[ \tau_{LS} = \alpha \tau_{h_2o} + (1-\alpha)\tau_{ice} . \]

The optical thickness associated with sub-grid scale convective clouds produced by RAS is given as

\[ \tau_{RAS} = 0.16 \quad mb^{-1} . \]

The total optical depth in a given model layer is computed as a weighted average between
the large-scale and sub-grid scale optical depths, normalized by the total cloud fraction in the
layer:

\[ \tau = \left( {F_{RAS} \,\,\, \tau_{RAS} + F_{LS} \,\,\, \tau_{LS} \over F_{RAS}+F_{LS} } \right) \Delta p, \]

where $F_{RAS}$ and $F_{LS}$ are the time-averaged cloud fractions associated with RAS and large-scale
processes described in Section \ref{sec:fizhi:clouds}.
The optical thickness for the longwave radiative feedback is assumed to be 75 $\%$ of these values.

The entire Moist Convective Processes Module is called with a frequency of 10 minutes. 
The cloud fraction values are time-averaged over the period between Radiation calls (every 3
hours).  Therefore, in a time-averaged sense, both convective and large-scale 
cloudiness can exist in a given grid-box.  

\subsubsection{Turbulence}
Turbulence is parameterized in the fizhi package to account for its contribution to the
vertical exchange of heat, moisture, and momentum.  
The turbulence scheme is invoked every 30 minutes, and employs a backward-implicit iterative 
time scheme with an internal time step of 5 minutes.
The tendencies of atmospheric state variables due to turbulent diffusion are calculated using
the diffusion equations:

\[
{\pp{u}{t}}_{turb} = {\pp{}{z} }{(- \overline{u^{\prime}w^{\prime}})}
 = {\pp{}{z} }{(K_m \pp{u}{z})}
\]
\[
{\pp{v}{t}}_{turb} = {\pp{}{z} }{(- \overline{v^{\prime}w^{\prime}})}
 = {\pp{}{z} }{(K_m \pp{v}{z})}
\]
\[
{\pp{T}{t}} = P^{\kappa}{\pp{\theta}{t}}_{turb} = 
P^{\kappa}{\pp{}{z} }{(- \overline{w^{\prime}\theta^{\prime}})}
 = P^{\kappa}{\pp{}{z} }{(K_h \pp{\theta_v}{z})}
\]
\[
{\pp{q}{t}}_{turb} = {\pp{}{z} }{(- \overline{w^{\prime}q^{\prime}})}
 = {\pp{}{z} }{(K_h \pp{q}{z})}
\]

Within the atmosphere, the time evolution
of second turbulent moments is explicitly modeled by representing the third moments in terms of 
the first and second moments.  This approach is known as a second-order closure modeling.
To simplify and streamline the computation of the second moments, the level 2.5 assumption
of Mellor and Yamada (1974) and Yamada (1977) is employed, in which only the turbulent 
kinetic energy (TKE),

\[ {\h}{q^2}={\overline{{u^{\prime}}^2}}+{\overline{{v^{\prime}}^2}}+{\overline{{w^{\prime}}^2}}, \]

is solved prognostically and the other second moments are solved diagnostically.
The prognostic equation for TKE allows the scheme to simulate 
some of the transient and diffusive effects in the turbulence. The TKE budget equation
is solved numerically using an implicit backward computation of the terms linear in $q^2$
and is written:

\[
{\dd{}{t} ({{\h} q^2})} - { \pp{}{z} ({ {5 \over 3} {{\lambda}_1} q { \pp {}{z} 
({\h}q^2)} })} =
{- \overline{{u^{\prime}}{w^{\prime}}} { \pp{U}{z} }} - {\overline{{v^{\prime}}{w^{\prime}}} 
{ \pp{V}{z} }} + {{g \over {\Theta_0}}{\overline{{w^{\prime}}{{{\theta}_v}^{\prime}}}} } 
- { q^3 \over {{\Lambda} _1} }
\]

where $q$ is the turbulent velocity, ${u^{\prime}}$, ${v^{\prime}}$, ${w^{\prime}}$ and 
${{\theta}^{\prime}}$ are the fluctuating parts of the velocity components and potential 
temperature, $U$ and $V$ are the mean velocity components, ${\Theta_0}^{-1}$ is the
coefficient of thermal expansion, and ${{\lambda}_1}$ and ${{\Lambda} _1}$ are constant
multiples of the master length scale, $\ell$, which is designed to be a characteristic measure
of the vertical structure of the turbulent layers.

The first term on the left-hand side represents the time rate of change of TKE, and
the second term is a representation of the triple correlation, or turbulent
transport term. The first three terms on the right-hand side represent the sources of
TKE due to shear and bouyancy, and the last term on the right hand side is the dissipation
of TKE.

In the level 2.5 approach, the vertical fluxes of the scalars $\theta_v$ and $q$ and the
wind components $u$ and $v$ are expressed in terms of the diffusion coefficients $K_h$ and
$K_m$, respectively.  In the statisically realizable level 2.5 turbulence scheme of Helfand
and Labraga (1988), these diffusion coefficients are expressed as

\[
K_h 
 = \left\{ \begin{array}{l@{\quad\mbox{for}\quad}l} q \, \ell \, S_H(G_M,G_H) \, & \mbox{decaying turbulence}
\\ { q^2 \over {q_e} } \, \ell \, S_{H}(G_{M_e},G_{H_e}) \, & \mbox{growing turbulence} \end{array} \right.
\]

and

\[
K_m
 = \left\{ \begin{array}{l@{\quad\mbox{for}\quad}l} q \, \ell \, S_M(G_M,G_H) \, & \mbox{decaying turbulence}                
\\ { q^2 \over {q_e} } \, \ell \, S_{M}(G_{M_e},G_{H_e}) \, & \mbox{growing turbulence} \end{array} \right.
\]

where the subscript $e$ refers to the value under conditions of local equillibrium
(obtained from the Level 2.0 Model), $\ell$ is the master length scale related to the 
vertical structure of the atmosphere,
and $S_M$ and $S_H$ are functions of $G_H$ and $G_M$, the dimensionless buoyancy and
wind shear parameters, respectively.
Both $G_H$ and $G_M$, and their equilibrium values $G_{H_e}$ and $G_{M_e}$,
are functions of the Richardson number:

\[
{\bf RI} = { { {g \over \theta_v} \pp {\theta_v}{z} } \over { (\pp{u}{z})^2 + (\pp{v}{z})^2 } }
 =  {  {c_p \pp{\theta_v}{z} \pp{P^ \kappa}{z} } \over { (\pp{u}{z})^2 + (\pp{v}{z})^2 } } .
\]

Negative values indicate unstable buoyancy and shear, small positive values ($<0.2$)
indicate dominantly unstable shear, and large positive values indicate dominantly stable
stratification.

Turbulent eddy diffusion coefficients of momentum, heat and moisture in the surface layer,
which corresponds to the lowest GCM level (see \ref{tab:fizhi:sigma}),
are calculated using stability-dependant functions based on Monin-Obukhov theory:
\[
{K_m} (surface) = C_u \times u_* = C_D W_s
\]
and
\[
{K_h} (surface) =  C_t \times u_* = C_H W_s
\]
where $u_*=C_uW_s$ is the surface friction velocity,
$C_D$ is termed the surface drag coefficient, $C_H$ the heat transfer coefficient, 
and $W_s$ is the magnitude of the surface layer wind.

$C_u$ is the dimensionless exchange coefficient for momentum from the surface layer
similarity functions: 
\[
{C_u} = {u_* \over W_s} = { k \over \psi_{m} }
\]
where k is the Von Karman constant and $\psi_m$ is the surface layer non-dimensional 
wind shear given by
\[
\psi_{m} = {\int_{\zeta_{0}}^{\zeta} {\phi_{m} \over \zeta} d \zeta} .
\]
Here $\zeta$ is the non-dimensional stability parameter, and
$\phi_m$ is the similarity function of $\zeta$ which expresses the stability dependance of
the momentum gradient.  The functional form of $\phi_m$ is specified differently for stable and unstable 
layers.

$C_t$ is the dimensionless exchange coefficient for heat and 
moisture from the surface layer similarity functions:
\[
{C_t} = -{( {\overline{w^{\prime}\theta^{\prime}}}) \over {u_* \Delta \theta }} =
-{( {\overline{w^{\prime}q^{\prime}}}) \over {u_* \Delta q }} =
{ k \over { (\psi_{h} + \psi_{g}) } }
\]
where $\psi_h$ is the surface layer non-dimensional temperature gradient given by
\[
\psi_{h} = {\int_{\zeta_{0}}^{\zeta} {\phi_{h} \over \zeta} d \zeta} .
\]
Here $\phi_h$ is the similarity function of $\zeta$, which expresses the stability dependance of
the temperature and moisture gradients, and is specified differently for stable and unstable
layers according to Helfand and Schubert, 1995.

$\psi_g$ is the non-dimensional temperature or moisture gradient in the viscous sublayer, 
which is the mosstly laminar region between the surface and the tops of the roughness 
elements, in which temperature and moisture gradients can be quite large.
Based on Yaglom and Kader (1974):
\[
\psi_{g} = { 0.55 (Pr^{2/3} - 0.2) \over \nu^{1/2} }
(h_{0}u_{*} - h_{0_{ref}}u_{*_{ref}})^{1/2}
\]
where Pr is the Prandtl number for air, $\nu$ is the molecular viscosity, $z_{0}$ is the 
surface roughness length, and the subscript {\em ref} refers to a reference value.
$h_{0} = 30z_{0}$ with a maximum value over land of 0.01
 
The surface roughness length over oceans is is a function of the surface-stress velocity,
\[
{z_0} = c_1u^3_* + c_2u^2_* + c_3u_* + c_4 + {c_5 \over {u_*}}
\]
where the constants are chosen to interpolate between the reciprocal relation of
Kondo(1975) for weak winds, and the piecewise linear relation of Large and Pond(1981)
for moderate to large winds.  Roughness lengths over land are specified
from the climatology of Dorman and Sellers (1989).

For an unstable surface layer, the stability functions, chosen to interpolate between the
condition of small values of $\beta$ and the convective limit, are the KEYPS function 
(Panofsky, 1973) for momentum, and its generalization for heat and moisture:  
\[
{\phi_m}^4 - 18 \zeta {\phi_m}^3 = 1 \hspace{1cm} ; \hspace{1cm} 
{\phi_h}^2 - 18 \zeta {\phi_h}^3 = 1 \hspace{1cm} .
\]
The function for heat and moisture assures non-vanishing heat and moisture fluxes as the wind 
speed approaches zero. 

For a stable surface layer, the stability functions are the observationally 
based functions of Clarke (1970),  slightly modified for
the momemtum flux:  
\[
{\phi_m} = { { 1 + 5 {{\zeta}_1} } \over { 1 + 0.00794 {{\zeta}_1}
(1+ 5 {{\zeta}_1}) } } \hspace{1cm} ; \hspace{1cm}
{\phi_h} = { { 1 + 5 {{\zeta}_1} } \over { 1 + 0.00794 {\zeta}
(1+ 5 {{\zeta}_1}) } } .
\]
The moisture flux also depends on a specified evapotranspiration
coefficient, set to unity over oceans and dependant on the climatological ground wetness over
land.  

Once all the diffusion coefficients are calculated, the diffusion equations are solved numerically
using an implicit backward operator.

\paragraph{Atmospheric Boundary Layer}

The depth of the atmospheric boundary layer (ABL) is diagnosed by the parameterization as the
level at which the turbulent kinetic energy is reduced to a tenth of its maximum near surface value.
The vertical structure of the ABL is explicitly resolved by the lowest few (3-8) model layers.

\paragraph{Surface Energy Budget}

The ground temperature equation is solved as part of the turbulence package
using a backward implicit time differencing scheme:
\[
C_g\pp{T_g}{t} = R_{sw} - R_{lw} + Q_{ice} - H - LE
\]
where $R_{sw}$ is the net surface downward shortwave radiative flux and $R_{lw}$ is the
net surface upward longwave radiative flux. 

$H$ is the upward sensible heat flux, given by:
\[
{H} =  P^{\kappa}\rho c_{p} C_{H} W_s (\theta_{surface} - \theta_{NLAY})
\hspace{1cm}where: \hspace{.2cm}C_H = C_u C_t
\]
where $\rho$ = the atmospheric density at the surface, $c_{p}$ is the specific
heat of air at constant pressure, and $\theta$ represents the potential temperature
of the surface and of the lowest $\sigma$-level, respectively.
 
The upward latent heat flux, $LE$, is given by
\[
{LE} =  \rho \beta L C_{H} W_s (q_{surface} - q_{NLAY})
\hspace{1cm}where: \hspace{.2cm}C_H = C_u C_t
\]
where $\beta$ is the fraction of the potential evapotranspiration actually evaporated,
L is the latent heat of evaporation, and $q_{surface}$ and $q_{NLAY}$ are the specific
humidity of the surface and of the lowest $\sigma$-level, respectively.

The heat conduction through sea ice, $Q_{ice}$, is given by
\[
{Q_{ice}} = {C_{ti} \over {H_i}} (T_i-T_g)
\]
where $C_{ti}$ is the thermal conductivity of ice, $H_i$ is the ice thickness, assumed to
be $3 \hspace{.1cm} m$ where sea ice is present, $T_i$ is 273 degrees Kelvin, and $T_g$ is the 
surface temperature of the ice.

$C_g$ is the total heat capacity of the ground, obtained by solving a heat diffusion equation
for the penetration of the diurnal cycle into the ground (Blackadar, 1977), and is given by:
\[
C_g = \sqrt{ {\lambda C_s \over 2\omega} } = \sqrt{(0.386 + 0.536W + 0.15W^2)2\times10^{-3}
{86400 \over 2 \pi} } \, \, .
\]
Here, the thermal conductivity, $\lambda$, is equal to $2\times10^{-3}$ ${ly\over{ sec}}
{cm \over {^oK}}$,    
the angular velocity of the earth, $\omega$, is written as $86400$ $sec/day$ divided
by $2 \pi$ $radians/  
day$, and the expression for $C_s$, the heat capacity per unit volume at the surface,
is a function of the ground wetness, $W$.

\subsubsection{Land Surface Processes}

\paragraph{Surface Type}
The fizhi package surface Types are designated using the Koster-Suarez (1992) mosaic
philosophy which allows multiple ``tiles'', or multiple surface types, in any one
grid cell. The Koster-Suarez Land Surface Model (LSM) surface type classifications
are shown in Table \ref{tab:fizhi:surftype}. The surface types and the percent of the grid
cell occupied by any surface type were derived from the surface classification of
Defries and Townshend (1994), and information about the location of permanent
ice was obtained from the classifications of Dorman and Sellers (1989).
The surface type for the \txt GCM grid is shown in Figure \ref{fig:fizhi:surftype}.
The determination of the land or sea category of surface type was made from NCAR's
10 minute by 10 minute Navy topography 
dataset, which includes information about the percentage of water-cover at any point.
The data were averaged to the model's \fxf and \txt grid resolutions,
and any grid-box whose averaged water percentage was $\geq 60 \%$ was
defined as a water point. The \fxf grid Land-Water designation was further modified
subjectively to ensure sufficient representation from small but isolated land and water regions.
 
\begin{table}
\begin{center}
{\bf Surface Type Designation} \\
\vspace{0.1in}
\begin{tabular}{ |c|l| }
\hline
Type & Vegetation Designation \\ \hline
\hline
  1 & Broadleaf Evergreen Trees \\ \hline
  2 & Broadleaf Deciduous Trees \\ \hline
  3 & Needleleaf Trees \\ \hline
  4 & Ground Cover \\ \hline   
  5 & Broadleaf Shrubs \\ \hline
  6 & Dwarf Trees (Tundra) \\ \hline
  7 & Bare Soil \\ \hline
  8 & Desert (Bright) \\ \hline
  9 & Glacier \\ \hline
 10 & Desert (Dark) \\ \hline
100 & Ocean \\ \hline
\end{tabular}
\end{center}
\caption{Surface type designations used to compute surface roughness (over land) 
and surface albedo.}
\label{tab:fizhi:surftype}
\end{table}
 
 
\begin{figure*}[htbp]
  \centerline{  \epsfysize=7in  \epsfbox{part6/surftypes.ps}}
  \vspace{0.3in}
  \caption  {Surface Type Compinations at \txt resolution.}
  \label{fig:fizhi:surftype}
\end{figure*}

\begin{figure*}[htbp]
  \centerline{  \epsfysize=7in  \epsfbox{part6/surftypes.descrip.ps}}
  \vspace{0.3in}
  \caption  {Surface Type Descriptions.}
  \label{fig:fizhi:surftype.desc}
\end{figure*}


\paragraph{Surface Roughness}
The surface roughness length over oceans is computed iteratively with the wind
stress by the surface layer parameterization (Helfand and Schubert, 1991).
It employs an interpolation between the functions of Large and Pond (1981)
for high winds and of Kondo (1975) for weak winds.


\paragraph{Albedo}
The surface albedo computation, described in Koster and Suarez (1991),
employs the ``two stream'' approximation used in Sellers' (1987) Simple Biosphere (SiB)
Model which distinguishes between the direct and diffuse albedos in the visible
and in the near infra-red spectral ranges. The albedos are functions of the observed
leaf area index (a description of the relative orientation of the leaves to the
sun), the greenness fraction, the vegetation type, and the solar zenith angle.
Modifications are made to account for the presence of snow, and its depth relative
to the height of the vegetation elements.

\subsubsection{Gravity Wave Drag}
The fizhi package employs the gravity wave drag scheme of Zhou et al. (1996).
This scheme is a modified version of Vernekar et al. (1992),
which was based on Alpert et al. (1988) and Helfand et al. (1987).  
In this version, the gravity wave stress at the surface is
based on that derived by Pierrehumbert (1986) and is given by:

\bq
|\vec{\tau}_{sfc}| = {\rho U^3\over{N \ell^*}} \left(F_r^2 \over{1+F_r^2}\right) \, \, ,
\eq

where $F_r = N h /U$ is the Froude number, $N$ is the {\em Brunt - V\"{a}is\"{a}l\"{a}} frequency, $U$ is the 
surface wind speed, $h$ is the standard deviation of the sub-grid scale orography,
and $\ell^*$ is the wavelength of the monochromatic gravity wave in the direction of the low-level wind.
A modification introduced by Zhou et al. allows for the momentum flux to
escape through the top of the model, although this effect is small for the current 70-level model.  
The subgrid scale standard deviation is defined by $h$, and is not allowed to exceed 400 m. 

The effects of using this scheme within a GCM are shown in Takacs and Suarez (1996).
Experiments using the gravity wave drag parameterization yielded significant and
beneficial impacts on both the time-mean flow and the transient statistics of the
a GCM climatology, and have eliminated most of the worst dynamically driven biases 
in the a GCM simulation. 
An examination of the angular momentum budget during climate runs indicates that the 
resulting gravity wave torque is similar to the data-driven torque produced by a data 
assimilation which was performed without gravity
wave drag.  It was shown that the inclusion of gravity wave drag results in 
large changes in both the mean flow and in eddy fluxes.
The result is a more
accurate simulation of surface stress (through a reduction in the surface wind strength), 
of mountain torque (through a redistribution of mean sea-level pressure), and of momentum
convergence (through a reduction in the flux of westerly momentum by transient flow eddies).  


\subsubsection{Boundary Conditions and other Input Data}

Required fields which are not explicitly predicted or diagnosed during model execution must
either be prescribed internally or obtained from external data sets.  In the fizhi package these
fields include:  sea surface temperature, sea ice estent, surface geopotential variance, 
vegetation index, and the radiation-related background levels of: ozone, carbon dioxide, 
and stratospheric moisture.

Boundary condition data sets are available at the model's \fxf and \txt 
resolutions for either climatological or yearly varying conditions. 
Any frequency of boundary condition data can be used in the fizhi package; 
however, the current selection of data is summarized in Table \ref{tab:fizhi:bcdata}\@.
The time mean values are interpolated during each model timestep to the 
current time. Future model versions will incorporate boundary conditions at
higher spatial \mbox{($1^\circ$ x $1^\circ$)} resolutions.

\begin{table}[htb]
\begin{center}
{\bf Fizhi Input Datasets} \\
\vspace{0.1in}
\begin{tabular}{|l|c|r|} \hline
\multicolumn{1}{|c}{Variable} & \multicolumn{1}{|c}{Frequency} & \multicolumn{1}{|c|}{Years} \\ \hline\hline
Sea Ice Extent & monthly & 1979-current, climatology \\ \hline
Sea Ice Extent & weekly  & 1982-current, climatology \\ \hline
Sea Surface Temperature & monthly & 1979-current, climatology \\ \hline
Sea Surface Temperature & weekly & 1982-current, climatology \\ \hline
Zonally Averaged Upper-Level Moisture & monthly  & climatology \\ \hline
Zonally Averaged Ozone Concentration & monthly  & climatology \\ \hline
\end{tabular}
\end{center}
\caption{Boundary conditions and other input data used in the fizhi package.  Also noted are the
current years and frequencies available.}
\label{tab:fizhi:bcdata}
\end{table}


\paragraph{Topography and Topography Variance}

Surface geopotential heights are provided from an averaging of the Navy 10 minute
by 10 minute dataset supplied by the National Center for Atmospheric Research (NCAR) to the
model's grid resolution. The original topography is first rotated to the proper grid-orientation
which is being run, and then  
averages the data to the model resolution.  
The averaged topography is then passed through a Lanczos (1966) filter in both dimensions 
which removes the smallest
scales while inhibiting Gibbs phenomena.  

In one dimension, we may define a cyclic function in $x$ as:
\begin{equation}
f(x) = {a_0 \over 2} + \sum_{k=1}^N \left( a_k \cos(kx) + b_k \sin(kx) \right)
\label{eq:fizhi:filt}
\end{equation}
where $N = { {\rm IM} \over 2 }$ and ${\rm IM}$ is the total number of points in the $x$ direction.
Defining $\Delta x = { 2 \pi \over {\rm IM}}$, we may define the average of $f(x)$ over a 
$2 \Delta x$ region as:

\begin{equation}
\overline {f(x)} = {1 \over {2 \Delta x}} \int_{x-\Delta x}^{x+\Delta x} f(x^{\prime}) dx^{\prime}
\label{eq:fizhi:fave1}
\end{equation}

Using equation (\ref{eq:fizhi:filt}) in equation (\ref{eq:fizhi:fave1}) and integrating, we may write:
 
\begin{equation}
\overline {f(x)} = {a_0 \over 2} + {1 \over {2 \Delta x}}
\sum_{k=1}^N \left [
\left. a_k { \sin(kx^{\prime}) \over k } \right /_{x-\Delta x}^{x+\Delta x} -
\left. b_k { \cos(kx^{\prime}) \over k } \right /_{x-\Delta x}^{x+\Delta x} 
\right]
\end{equation}
or

\begin{equation}
\overline {f(x)} = {a_0 \over 2} + \sum_{k=1}^N {\sin(k \Delta x) \over {k \Delta x}}
\left( a_k \cos(kx) + b_k \sin(kx) \right)
\label{eq:fizhi:fave2}
\end{equation}

Thus, the Fourier wave amplitudes are simply modified by the Lanczos filter response
function ${\sin(k\Delta x) \over {k \Delta x}}$.  This may be compared with an $mth$-order 
Shapiro (1970) filter response function, defined as $1-\sin^m({k \Delta x \over 2})$,
shown in Figure \ref{fig:fizhi:lanczos}.
It should be noted that negative values in the topography resulting from
the filtering procedure are {\em not} filled.

\begin{figure*}[htbp]
  \centerline{  \epsfysize=7.0in  \epsfbox{part6/lanczos.ps}}
  \caption{ \label{fig:fizhi:lanczos} Comparison between the Lanczos and $mth$-order Shapiro filter 
  response functions for $m$ = 2, 4, and 8. }
\end{figure*}

The standard deviation of the subgrid-scale topography
is computed from a modified version of the the Navy 10 minute by 10 minute dataset.
The 10 minute by 10 minute topography is passed through a wavelet
filter in both dimensions which removes the scale smaller than 20 minutes.
The topography is then averaged to $1^\circ x 1^\circ$ grid resolution, and then
re-interpolated back to the 10 minute by 10 minute resolution. 
The sub-grid scale variance is constructed based on this smoothed dataset.


\paragraph{Upper Level Moisture}
The fizhi package uses climatological water vapor data above 100 mb from the Stratospheric Aerosol and Gas 
Experiment (SAGE) as input into the model's radiation packages.  The SAGE data is archived
as monthly zonal means at 5$^\circ$ latitudinal resolution.  The data is interpolated to the
model's grid location and current time, and blended with the GCM's moisture data.  Below 300 mb,
the model's moisture data is used.  Above 100 mb, the SAGE data is used.  Between 100 and 300 mb,
a linear interpolation (in pressure) is performed using the data from SAGE and the GCM. 

\subsection{Key subroutines, parameters and files}

\subsection{Dos and donts}

\subsection{Fizhi Reference}
