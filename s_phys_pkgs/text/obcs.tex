\subsection{OBCS: Open boundary conditions for regional modeling}

\label{sec:pkg:obcs}
\begin{rawhtml}
<!-- CMIREDIR:package_obcs: -->
\end{rawhtml}

Authors: 
Alistair Adcroft, Patrick Heimbach, Samar Katiwala, Martin Losch

\subsubsection{Introduction
\label{sec:pkg:obcs:intro}}



%----------------------------------------------------------------------

\subsubsection{OBCS configuration and compiling
\label{sec:pkg:obcs:comp}}

As with all MITgcm packages, OBCS can be turned on or off 
at compile time
%
\begin{itemize}
%
\item
using the \texttt{packages.conf} file by adding \texttt{obcs} to it,
%
\item
or using \texttt{genmake2} adding
\texttt{-enable=obcs} or \texttt{-disable=obcs} switches
%
\item
\textit{Required packages and CPP options:} \\
%
To alternatives are available for prescribing open boundary values,
which differ in the way how OB's are treated in time:
A simple time-management (e.g. constant in time, or cyclic with
fixed fequency) is provided through 
S/R \texttt{obcs\_external\_fields\_load}.
More sophisticated ``real-time'' (i.e. calendar time) management is
available through \texttt{obcs\_prescribe\_read}. 
The latter case requires
packages \texttt{cal} and \texttt{exf} to be enabled.
%
\end{itemize}
(see also Section \ref{sec:buildingCode}).

Parts of the OBCS code can be enabled or disabled at compile time
via CPP preprocessor flags. These options are set in
\texttt{OBCS\_OPTIONS.h}. Table \ref{tab:pkg:obcs:cpp} summarizes them.

\begin{table}[!ht]
\centering
  \label{tab:pkg:obcs:cpp}
  {\footnotesize
    \begin{tabular}{|l|l|}
      \hline 
      \textbf{CPP option}  &  \textbf{Description}  \\
      \hline \hline
        \texttt{ALLOW\_OBCS\_NORTH} & 
          enable Northern OB \\
        \texttt{ALLOW\_OBCS\_SOUTH} & 
          enable Southern OB \\
        \texttt{ALLOW\_OBCS\_EAST} & 
          enable Eastern OB \\
        \texttt{ALLOW\_OBCS\_WEST} & 
          enable Western OB \\
      \hline
        \texttt{ALLOW\_OBCS\_PRESCRIBE} & 
          enable code for prescribing OB's \\
        \texttt{ALLOW\_OBCS\_SPONGE} & 
          enable sponge layer code \\
        \texttt{ALLOW\_OBCS\_BALANCE} & 
          enable code for balancing transports through OB's \\
        \texttt{ALLOW\_ORLANSKI} & 
          enable Orlanski radiation conditions at OB's \\
      \hline
    \end{tabular}
  }
  \caption{~}
\end{table}


%----------------------------------------------------------------------

\subsubsection{Run-time parameters
\label{sec:pkg:obcs:runtime}}

Run-time parameters are set in files 
\texttt{data.pkg}, \texttt{data.obcs}, and \texttt{data.exf} 
if ``real-time'' prescription is requested 
(i.e. package \texttt{exf} enabled).
These parameter files are read in S/R
\texttt{packages\_readparms.F}, \texttt{obcs\_readparms.F}, and
\texttt{exf\_readparms.F}, respectively.
Run-time parameters may be broken into 3 categories:
(i) switching on/off the package at runtime,
(ii) OBCS package flags and parameters,
(iii) additional timing flags in \texttt{data.exf}, if selected.

\paragraph{Enabling the package}
~ \\
%
The OBCS package is switched on at runtime by setting
\texttt{useOBCS = .TRUE.} in \texttt{data.pkg}.

\paragraph{Package flags and parameters}
~ \\
%
Table \ref{tab:pkg:obcs:runtime_flags} summarizes the
runtime flags that are set in \texttt{data.obcs}, and
their default values.

\begin{table}[!ht]
\centering
  {\footnotesize
    \begin{tabular}{|l|c|l|}
      \hline 
      \textbf{Flag/parameter} & \textbf{default} &  \textbf{Description}  \\
      \hline \hline
         \multicolumn{3}{|c|}{\textit{basic flags \& parameters} } \\
         \hline
        OB\_Jnorth & 0 & 
           Nx-vector of J-indices (w.r.t. Ny) of Northern OB
           at each I-position (w.r.t. Nx) \\
        OB\_Jsouth & 0 & 
           Nx-vector of J-indices (w.r.t. Ny) of Southern OB
           at each I-position (w.r.t. Nx) \\
        OB\_Ieast & 0 & 
           Ny-vector of I-indices (w.r.t. Nx) of Eastern OB
           at each J-position (w.r.t. Ny) \\
        OB\_Iwest & 0 & 
           Ny-vector of I-indices (w.r.t. Nx) of Western OB
           at each J-position (w.r.t. Ny) \\
        useOBCSprescribe & \texttt{.FALSE.} & 
           ~ \\
        useOBCSsponge & \texttt{.FALSE.} & 
           ~ \\
        useOBCSbalance & \texttt{.FALSE.} & 
           ~ \\
        OB\textbf{X}\textbf{y}File & ~ & 
           file name of OB field \\
        ~ & ~ & 
           \textbf{X}: \textbf{N}(orth), \textbf{S}(outh), 
                       \textbf{E}(ast), \textbf{W}(est) \\
        ~ & ~ & 
           \textbf{y}: \textbf{t}(emperature), \textbf{s}(salinity), 
           \textbf{u}(-velocity), \textbf{v}(-velocity) \\
      \hline
      \multicolumn{3}{|c|}{\textit{Orlanski parameters} } \\
      \hline
        cvelTimeScale & 2000 sec & 
           averaging period for phase speed \\
        CMAX & 0.45 m/s & 
           maximum allowable phase speed-CFL for AB-II \\
        CFIX & 0.8 m/s & 
           fixed boundary phase speed \\
        useFixedCEast & .FALSE. & 
           ~ \\
        useFixedCWest & .FALSE. & 
           ~ \\
      \hline
      \multicolumn{3}{|c|}{\textit{Sponge-layer parameters} } \\
      \hline
        spongeThickness & 0 & 
           sponge layer thickness (in \# grid points) \\
        Urelaxobcsinner & 0 sec & 
           relaxation time scale at the 
           innermost sponge layer point of a meridional OB \\
        Vrelaxobcsinner & 0 sec & 
           relaxation time scale at the 
           innermost sponge layer point of a zonal OB \\
        Urelaxobcsbound & 0 sec & 
           relaxation time scale at the 
           outermost sponge layer point of a meridional OB \\
        Vrelaxobcsbound & 0 sec & 
           relaxation time scale at the 
           outermost sponge layer point of a zonal OB \\
         \hline
      \hline
    \end{tabular}
  }
  \caption{pkg OBCS run-time parameters}
  \label{tab:pkg:obcs:runtime_flags}
\end{table}



%----------------------------------------------------------------------

\subsubsection{Defining open boundary positions
\label{sec:pkg:obcs:defining}}

There are four open boundaries (OBs), a 
Northern, Southern, Eastern, and Western.
All OB locations are specified by their absolute
meridional (Northern/Southern) or zonal (Eastern/Western) indices.
Thus, for each zonal position $i=1,\ldots,Nx$ a meridional index
$j$ specifies the Northern/Southern OB position,
and for each meridional position $j=1,\ldots,Ny$, a zonal index
$i$ specifies the Eastern/Western OB position.
For Northern/Southern OB this defines an $Nx$-dimensional
``row'' array $\tt OB\_Jnorth(Ny)$ / $\tt OB\_Jsouth(Ny)$,
and an $Ny$-dimenisonal 
``column'' array $\tt OB\_Ieast(Nx)$ / $\tt OB\_Iwest(Nx)$
Positions determined in this way allows Northern/Southern
OBs to be at variable $j$ (or $y$) positions, and Eastern/Western
OBs at variable $i$ (or $x$) positions.
Here, indices refer to tracer points on the C-grid.
A zero (0) element in $\tt OB\_I\ldots$, $\tt OB\_J\ldots$
means there is no corresponding OB in that column/row.
For a Northern/Southern OB, the OB V point is to the South/North.
For an Eastern/Western OB, the OB U point is to the West/East.

\begin{verbatim}
 For example
     OB_Jnorth(3)=34  means that:
          T( 3 ,34) is a an OB point
          U(3:4,34) is a an OB point
          V( 4 ,34) is a an OB point
 while
     OB_Jsouth(3)=1  means that:
          T( 3 ,1) is a an OB point
          U(3:4,1) is a an OB point
          V( 4 ,2) is a an OB point
\end{verbatim}

For convenience, negative values for Jnorth/Ieast refer to
points relative to the Northern/Eastern edges of the model
eg. $\tt OB\_Jnorth(3)=-1$  means that the point $\tt (3,Ny)$ 
is a northern OB.

\noindent
\textsf{Add special comments for case \#define NONLIN\_FRSURF,
see obcs\_ini\_fixed.F}

%----------------------------------------------------------------------

\subsubsection{Equations and key routines
\label{sec:pkg:obcs:equations}}

\paragraph{OBCS\_READPARMS:} ~ \\
Set OB positions through arrays
{\tt OB\_Jnorth(Ny), OB\_Jsouth(Ny), OB\_Ieast(Nx), OB\_Iwest(Nx)},
and runtime flags (see Table \ref{tab:pkg:obcs:runtime_flags}).

\paragraph{OBCS\_CALC:} ~ \\
%
Top-level routine for filling values to be applied at OB for 
$T,S,U,V,\eta$ into corresponding 
``slice'' arrays $(x,z)$, $(y,z)$ for each OB:
$\tt OB[N/S/E/W][t/s/u/v]$; e.g. for salinity array at
Southern OB, array name is $\tt OBSt$.
Values filled are either
%
\begin{itemize}
%
\item
constant vertical $T,S$ profiles as specified in file
{\tt data} ({\tt tRef(Nr), sRef(Nr)}) with zero velocities $U,V$,
%
\item
$T,S,U,V$ values determined via Orlanski radiation conditions
(see below),
%
\item
prescribed time-constant or time-varying fields (see below).
%
\end{itemize}


\paragraph{ORLANSKI} ~ \\
%
Orlanski radiation conditions \citep{orl:76}

\paragraph{OBCS\_PRESCRIBE\_READ} Setting OB fields and updates \\
%
~

\paragraph{OBCS\_BALANCE} ~ \\
%
~

\paragraph{OBCS\_APPLY\_*:} ~ \\
~

\paragraph{OBCS\_SPONGE} Setting sponge layer characteristics \\
%
~

\paragraph{OB's with nonlinear free surface} ~ \\
%
~


%----------------------------------------------------------------------

\subsubsection{Flow chart
\label{sec:pkg:obcs:flowchart}}


{\footnotesize
\begin{verbatim}

C     !CALLING SEQUENCE:
c ...

\end{verbatim}
}

%----------------------------------------------------------------------

\subsubsection{OBCS diagnostics
\label{sec:pkg:obcs:diagnostics}}

Diagnostics output is available via the diagnostics package
(see Section \ref{sec:pkg:diagnostics}).
Available output fields are summarized in 
Table \ref{tab:pkg:obcs:diagnostics}.

\begin{table}[!ht]
\centering
\label{tab:pkg:obcs:diagnostics}
{\footnotesize
\begin{verbatim}
------------------------------------------------------
 <-Name->|Levs|grid|<--  Units   -->|<- Tile (max=80c)
------------------------------------------------------

\end{verbatim}
}
\caption{~}
\end{table}

%----------------------------------------------------------------------

\subsubsection{Reference experiments}



%----------------------------------------------------------------------

\subsubsection{References}

\subsubsection{Experiments and tutorials that use obcs}
\label{sec:pkg:obcs:experiments}

\begin{itemize}
\item{Ocean experiment in exp4 verification directory. }
\end{itemize}

