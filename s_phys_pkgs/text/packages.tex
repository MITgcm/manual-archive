% $Header: /u/gcmpack/manual/s_phys_pkgs/text/packages.tex,v 1.8 2010/08/30 23:09:21 jmc Exp $
% $Name:  $

\section{Using MITgcm Packages}
\label{sec:pkg:using}
\begin{rawhtml}
<!-- CMIREDIR:package_using: -->
\end{rawhtml}

The set of packages that will be used within a partiucular model can
be configured using a combination of both ``compile--time'' and
``run--time'' options.  Compile--time options are those used to select
which packages will be ``compiled in'' or implemented within the
program.  Packages excluded at compile time are completely absent from
the executable program(s) and thus cannot be later activated by any
set of subsequent run--time options.

\subsection{Package Inclusion/Exclusion}

There are numerous ways that one can specify compile--time package
inclusion or exclusion and they are all implemented by the
\texttt{genmake2} program which was previously described in Section
\ref{sec:buildingCode}.  The options are as follows:
\begin{enumerate}
\item Setting the \texttt{genamake2} options \texttt{--enable PKG}
  and/or \texttt{--disable PKG} specifies inclusion or exclusion.
  This method is intended as a convenient way to perform a single
  (perhaps for a quick test) compilation.
  
\item By creating a text file with the name \texttt{packages.conf} in
  either the local build directory or the \texttt{-mods=DIR}
  directory, one can specify a list of packages (one package per line,
  with '\texttt{\#}' as the comment character) to be included.  Since
  the \texttt{packages.conf} file can be saved, this is the preferred
  method for setting and recording (for future reference) the package
  configuration.
  
\item For convenience, a list of ``standard'' package groups is
  contained in the \texttt{pkg/pkg\_groups} file.  By selecting one of
  the package group names in the \texttt{packages.conf} file, one
  automatically obtains all packages in that group.

\item By default (that is, if a \texttt{packages.conf} file is not
  found), the \texttt{genmake2} program will use the
  package group default ``\texttt{default\_pkg\_list}'' as defined 
  in \texttt{pkg/pkg\_groups} file.

\item To help prevent users from creating unusable package groups, the
  \texttt{genmake2} program will parse the contents of the
  \texttt{pkg/pkg\_depend} file to determine:
  \begin{itemize}
  \item whether any two requested packages cannot be simultaneously
    included (\textit{eg.} \textit{seaice} and \textit{thsice} are
    mutually exclusive),
  \item whether additional packages must be included in order to
    satisfy package dependencies (\textit{eg.} \textit{rw} depends
    upon functionality within the \textit{mdsio} package), and
  \item whether the set of all requested packages is compatible with
    the dependencies (and producing an error if they aren't).
  \end{itemize}
  Thus, as a result of the dependencies, additional packages may be
  added to those originally requested.

\end{enumerate}


\subsection{Package Activation}

For run--time package control, MITgcm uses flags set through a
\texttt{data.pkg} file.  While some packages (\textit{eg.}
\texttt{debug}, \texttt{mnc}, \texttt{exch2}) may have their own usage
conventions, most follow a simple flag naming convention of the form:
\begin{verbatim}
  usePackageName=.TRUE.
\end{verbatim}
where the \texttt{usePackageName} variable can activate or disable the
package at runtime.  As mentioned previously, packages must be
included in order to be activated.  Generally, such mistakes will be
detected and reported as errors by the code.  However, users should
still be aware of the dependency.


\subsection{Package Coding Standards}

The following sections describe how to modify and/or create new MITgcm
packages.

\subsubsection{Packages are Not Libraries}

To a beginner, the MITgcm packages may resemble libraries as used in
myriad software projects.  While future versions are likely to
implement packages as libraries (perhaps using FORTRAN90/95 syntax)
the current packages (FORTRAN77) are \textbf{not} based upon any
concept of libraries.

\subsubsection{File Inclusion Rules}

Instead, packages should be viewed only as directories containing
``sets of source files'' that are built using some simple mechanisms
provided by \texttt{genmake2}.  Conceptually, the build process adds
files as they are found and proceeds according to the following rules:
\begin{enumerate}
\item \texttt{genmake2} locates a ``core'' or main set of source files
  (the \texttt{-standarddirs} option sets these locations and the
  default value contains the directories \texttt{eesupp} and
  \texttt{model}).
  
\item \texttt{genmake2} then finds additional source files by
  inspecting the contents of each of the package directories:
  \begin{enumerate}
  \item As the new files are found, they are added to a list of source
    files.

  \item If there is a file name ``collision'' (that is, if one of the
    files in a package has the same name as one of the files
    previously encountered) then the file within the newer (more
    recently visited) package will superseed (or ``hide'') any
    previous file(s) with the same name.
    
  \item Packages are visited (and thus files discovered) {\it in the
      order that the packages are enabled} within \texttt{genmake2}.
    Thus, the files in \texttt{PackB} may superseed the files in
    \texttt{PackA} if \texttt{PackA} is enabled before \texttt{PackB}.
    Thus, package ordering can be significant!  For this reason,
    \texttt{genmake2} honors the order in which packages are
    specified.
  \end{enumerate}
\end{enumerate}

These rules were adopted since they provide a relatively simple means
for rapidly including (or ``hiding'') existing files with modified
versions.

\subsubsection{Conditional Compilation and \texttt{PACKAGES\_CONFIG.h}}

Given that packages are simply groups of files that may be added or
removed to form a whole, one may wonder how linking (that is, FORTRAN
symbol resolution) is handled.  This is the second way that
\texttt{genmake2} supports the concept of packages.  Basically,
\texttt{genmake2} creates a \texttt{Makefile} that, in turn, is able
to create a file called \texttt{PACKAGES\_CONFIG.h} that contains a set
of C pre-processor (or ``CPP'') directives such as:
\begin{verbatim}
   #undef  ALLOW_KPP
   #undef  ALLOW_LAND
   ...
   #define ALLOW_GENERIC_ADVDIFF
   #define ALLOW_MDSIO
   ...
\end{verbatim}
These CPP symbols are then used throughout the code to conditionally
isolate variable definitions, function calls, or any other code that
depends upon the presence or absence of any particular package.

An example illustrating the use of these defines is:
\begin{verbatim}
   #ifdef ALLOW_GMREDI
         IF (useGMRedi) CALL GMREDI_CALC_DIFF(
        I        bi,bj,iMin,iMax,jMin,jMax,K,
        I        maskUp,
        O        KappaRT,KappaRS,
        I        myThid)
   #endif
\end{verbatim}
which is included from the file
\filelink{calc\_diffusivity.F}{model-src-calc_diffusivity.F}
and shows how both the compile--time \texttt{ALLOW\_GMREDI} flag and the
run--time \texttt{useGMRedi} are nested.

There are some benefits to using the technique described here.  The
first is that code snippets or subroutines associated with packages
can be placed or called from almost anywhere else within the code.
The second benefit is related to memory footprint and performance.
Since unused code can be removed, there is no performance penalty due
to unnecessary memory allocation, unused function calls, or extra
run-time \texttt{IF (...)} conditions.  The major problems with this
approach are the potentially difficult-to-read and difficult-to-debug
code caused by an overuse of CPP statements.  So while it can be done,
developers should exerecise some discipline and avoid unnecesarily
``smearing'' their package implementation details across numerous
files.


\subsubsection{Package Startup or Boot Sequence}

Calls to package routines within the core code timestepping loop can
vary.  However, all packages should follow a required "boot" sequence
outlined here:

{\footnotesize
\begin{verbatim}
    1. S/R PACKAGES_BOOT()
            :
        CALL OPEN_COPY_DATA_FILE( 'data.pkg', 'PACKAGES_BOOT', ... )
 

    2. S/R PACKAGES_READPARMS()
            :
        #ifdef ALLOW_${PKG}
          if ( use${Pkg} )
     &       CALL ${PKG}_READPARMS( retCode )
        #endif

    3. S/R PACKAGES_INIT_FIXED()
            :
        #ifdef ALLOW_${PKG}
          if ( use${Pkg} )
     &       CALL ${PKG}_INIT_FIXED( retCode )
        #endif

    4. S/R PACKAGES_CHECK()
            :
        #ifdef ALLOW_${PKG}
          if ( use${Pkg} )
     &       CALL ${PKG}_CHECK( retCode )
        #else
          if ( use${Pkg} )
     &       CALL PACKAGES_CHECK_ERROR('${PKG}')
        #endif

    5. S/R PACKAGES_INIT_VARIABLES()
            :
        #ifdef ALLOW_${PKG}
          if ( use${Pkg} )
     &       CALL ${PKG}_INIT_VARIA( )
        #endif

     6. S/R DO_THE_MODEL_IO

        #ifdef ALLOW_${PKG}
          if ( use${Pkg} )
     &       CALL ${PKG}_OUTPUT( )
        #endif

     7. S/R PACKAGES_WRITE_PICKUP()

        #ifdef ALLOW_${PKG}
          if ( use${Pkg} )
     &       CALL ${PKG}_WRITE_PICKUP( )
        #endif\end{verbatim}
}


\subsubsection{Adding a package to PARAMS.h and  packages\_boot()}

An MITgcm package directory contains all the code needed for that package apart
from one variable for each package. This variable is the {\it use\$\{Pkg\} } 
flag. This flag, which is of type logical, {\bf must} be declared in the 
shared header file {\it PARAMS.h} in the {\it PARM\_PACKAGES} block. This 
convention is used to support a single runtime control file {\it data.pkg} 
which is read by the startup routine {\it packages\_boot()} and that sets a 
flag controlling the runtime use of a package. This routine needs to be able to 
read the flags for packages that were not built at compile time. Therefore
when adding a new package, in addition to creating the per-package directory
in the {\it pkg/} subdirectory a developer should add a {\it use\$\{Pkg\} }
flag to {\it PARAMS.h} and a {\it use\$\{Pkg\} } entry to the 
{\it packages\_boot()} {\it PACKAGES} namelist.
The only other package specific code that should appear outside the individual 
package directory are calls to the specific package API.


