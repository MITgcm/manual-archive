
\begin{table}[!ht]
\caption{Run-time parameters and default values
\label{tab:pkg:seaice:runtimeparms}}
{\tiny
%\hspace*{-1.5in}
\begin{tabular}{|llp{5cm}c|}
\hline
  & & & \\
  \textbf{Name}  &  \textbf{Default value}  
    &  \textbf{Description}   &  \textbf{Reference}  \\
  & & & \\
\hline \hline
   SEAICEwriteState    &                     T
    &   write sea ice state to file 
    &  %---ref---
    \\
   SEAICEuseDYNAMICS   &                     T
    &   use dynamics 
    &  %---ref---
    \\
   SEAICEuseJFNK       &                     F
    &   use the JFNK-solver
    &  %---ref---
    \\
    SEAICEuseTEM       &                     F
    & use truncated ellipse method
    &  %---ref---
    \\
    SEAICEuseStrImpCpl &                     F
    & use strength implicit coupling in LSR/JFNK
    &  %---ref---
    \\
    SEAICEuseMetricTerms &                   T
    & use metric terms in dynamics
    &  %---ref---
    \\
    SEAICEuseEVPpickup  &                    T
    & use EVP pickups
    &  %---ref---
    \\
    SEAICEuseFluxForm   &                    F
    & use flux form for 2nd central difference advection scheme
    &  %---ref---
    \\
    SEAICErestoreUnderIce &                  F
    & enable restoring to climatology under ice
    &  %---ref---
    \\
    useHB87stressCoupling &                  F
    & turn on ice-ocean stress coupling following \citet{hibler87}
    &  %---ref---
    \\
    usePW79thermodynamics &                T
    & flag to turn off zero-layer-thermodynamics for testing
    &  %---ref---
    \\
    SEAICEadvHeff/Area/Snow/Salt &      T
    & flag to turn off advection of scalar state variables
    &  %---ref---
    \\
    SEAICEuseFlooding   &                   T
    & use flood-freeze algorithm
    &  %---ref---
    \\
    SEAICE\_no\_slip      &                   F
    & switch between free-slip and no-slip boundary conditions 
    &  %---ref---
    \\
   SEAICE\_deltaTtherm  &                   dTracerLev(1)
    &   thermodynamic timestep 
    &  %---ref---
    \\
   SEAICE\_deltaTdyn    &                   dTracerLev(1)
    &   dynamic timestep 
    &  %---ref---
    \\
   SEAICE\_deltaTevp    &                   0
    &   EVP sub-cycling time step, values $>$ 0 turn on EVP
    &  
    \\
   SEAICEuseEVPstar    & F & use modified EVP* instead of EVP & %---ref---
    \\
   SEAICEuseEVPrev     & F & use yet another variation on EVP* & %---ref---
    \\
   SEAICEnEVPstarSteps & UNSET & number of modified EVP* iteration & %---ref---
    \\
   SEAICE\_evpAlpha & UNSET & EVP* parameter & %---ref---
    \\
   SEAICE\_evpBeta & UNSET & EVP* parameter & %---ref---
    \\
   SEAICE\_elasticParm & $\frac{1}{3}$ 
     & EVP paramter $E_0$ & \\
   SEAICE\_evpTauRelax & $\Delta{t}_{EVP}E_0$ 
     & relaxation time scale $T$ for EVP waves & \\
%   SEAICE_evpDampC & & & \\
    SEAICEnewtonIterMax & 10
    & maximum number of JFNK-Newton iterations
    &  %---ref---
    \\
    SEAICEkrylovIterMax & 10
    & maximum number of JFNK-Krylov iterations
    &  %---ref---
    \\
    SEAICE\_JFNK\_lsIter & (off)
    & start line search after ``lsIter'' Newton iterations
    &  %---ref---
    \\
%     SEAICE\_JFNK\_tolIter    & 100
%     & number of Newton iterations after which the
%     the tolerance is relaxed again
%     &  %---ref---
%     \\
    JFNKgamma\_nonlin        & 1.0E-05
    & non-linear tolerance parameter for JFNK solver
    &  %---ref---
    \\
    JFNKgamma\_lin\_min/max  & 0.10/0.99
    & tolerance parameters for linear JFNK solver
    &  %---ref---
    \\
    JFNKres\_tFac            & UNSET
    & tolerance parameter for FGMRES residual
    &  %---ref---
    \\
    SEAICE\_JFNKepsilon      & 1.0E-06
    & step size for the FD-Jacobian-times-vector 
    &  %---ref---
    \\
    SEAICE\_dumpFreq     &                   dumpFreq
    &   dump frequency
    &  %---ref---
    \\
   SEAICE\_taveFreq     &                   taveFreq
    &   time-averaging frequency 
    &  %---ref---
    \\
   SEAICE\_dump\_mdsio   &                     T
    &   write snap-shot   using MDSIO 
    &  %---ref---
    \\
   SEAICE\_tave\_mdsio   &                     T
    &   write TimeAverage using MDSIO 
    &  %---ref---
    \\
   SEAICE\_dump\_mnc     &                     F
    &   write snap-shot   using MNC 
    &  %---ref---
    \\
   SEAICE\_tave\_mnc     &                     F
    &   write TimeAverage using MNC 
    &  %---ref---
    \\
   SEAICE\_initialHEFF  &                   0.00000E+00
    &   initial sea-ice thickness 
    &  %---ref---
    \\
   SEAICE\_drag         &                   2.00000E-03
    &   air-ice drag coefficient 
    &  %---ref---
    \\
   OCEAN\_drag          &                   1.00000E-03
    &   air-ocean drag coefficient 
    &  %---ref---
    \\
   SEAICE\_waterDrag    &                   5.50000E+00
    &   water-ice drag 
    &  %---ref---
    \\
   SEAICE\_dryIceAlb    &                   7.50000E-01
    &   winter albedo 
    &  %---ref---
    \\
   SEAICE\_wetIceAlb    &                   6.60000E-01
    &   summer albedo 
    &  %---ref---
    \\
   SEAICE\_drySnowAlb   &                   8.40000E-01
    &   dry snow albedo 
    &  %---ref---
    \\
   SEAICE\_wetSnowAlb   &                   7.00000E-01
    &   wet snow albedo 
    &  %---ref---
    \\
   SEAICE\_waterAlbedo  &                   1.00000E-01
    &   water albedo 
    &  %---ref---
    \\
   SEAICE\_strength     &                   2.75000E+04
    &   sea-ice strength $P^{*}$
    &  %---ref---
    \\
   SEAICE\_cStar       &                    20.0000E+00
    &   sea-ice strength paramter $C^{*}$ 
    &  %---ref---
    \\
    SEAICE\_rhoAir      & 1.3 (or \code{exf} value)
    & density of air (kg/m$^3$)
    & %---ref--- 
    \\
    SEAICE\_cpAir       & 1004 (or \code{exf} value)
    & specific heat of air (J/kg/K)
    & %---ref--- 
    \\
    SEAICE\_lhEvap      & 2,500,000 (or \code{exf} value)
    & latent heat of evaporation % for water (J/kg)
    & %---ref--- 
    \\
    SEAICE\_lhFusion    &   334,000 (or \code{exf} value) 
    & latent heat of fusion %for ice and snow (J/kg)   
    & %---ref---
    \\
    SEAICE\_lhSublim    & 2,834,000 
    & latent heat of sublimation  %for ice and snow (J/kg)
    & %---ref---
    \\
    SEAICE\_dalton      & 1.75E-03  
    & sensible heat transfer coefficient
    & %---ref--- 
    \\
   SEAICE\_iceConduct   &                   2.16560E+00
    &   sea-ice conductivity 
    &  %---ref---
    \\
   SEAICE\_snowConduct  &                   3.10000E-01
    &   snow conductivity 
    &  %---ref---
    \\
   SEAICE\_emissivity   &                   5.50000E-08
    &   Stefan-Boltzman 
    &  %---ref---
    \\
   SEAICE\_snowThick    &                   1.50000E-01
    &   cutoff snow thickness 
    &  %---ref---
    \\
   SEAICE\_shortwave    &                   3.00000E-01
    &   penetration shortwave radiation 
    &  %---ref---
    \\
   SEAICE\_freeze       &                  -1.96000E+00
    &   freezing temp. of sea water 
    &  %---ref---
    \\
    SEAICE\_saltFrac    &                   0.0
    &   salinity newly formed ice (fraction of ocean surface salinity)
    &  %---ref---
    \\
%     SEAICE\_gamma\_t    &                   UNSET
%     &   restoring time scale for basal freezing and melting
%     &  %---ref---
%     \\
%     SEAICE\_gamma\_t\_frz &                 UNSET
%     &   restoring time scale for basal freezing
%     &  %---ref---
%     \\
%     SEAICE\_mcPheePiston  &                  UNSET
%     &  ocean-ice turbulent flux "piston velocity" (m/s) that sets melt
%     efficiency. 
%     &  %---ref---
%     \\
%     SEAICE\_mcPheeTaper   &                  UNSET
%     & tapering down of turbulent flux term with ice concentration.
%     &  %---ref---
%     \\
    SEAICE\_frazilFrac    &                   0.0
    &  Fraction of surface level negative heat content anomalies
    (relative to the local freezing point) 
    &  %---ref---
    \\
    SEAICEstressFactor  &                  1.00000E+00
    &   scaling factor for ice-ocean stress
    &  %---ref---
    \\
    Heff/Area/HsnowFile/Hsalt & UNSET
    & initial fields for variables HEFF/AREA/HSNOW/HSALT
    &  %---ref---
    \\
    LSR\_ERROR           &                   1.00000E-04
    &   sets accuracy of LSR solver 
    &  %---ref---
    \\
    DIFF1               &                   0.0
    &   parameter used in advect.F 
    &  %---ref---
    \\
   HO                  &                   5.00000E-01
    &   demarcation ice thickness (AKA lead closing paramter $h_0$)
    &  %---ref---
    \\
   MAX\_HEFF            &                   1.00000E+01
    &   maximum ice thickness 
    &  %---ref---
    \\
   MIN\_ATEMP           &                  -5.00000E+01
    &   minimum air temperature 
    &  %---ref---
    \\
   MIN\_LWDOWN          &                   6.00000E+01
    &   minimum downward longwave 
    &  %---ref---
    \\
   MAX\_TICE            &                   3.00000E+01
    &   maximum ice temperature 
    &  %---ref---
    \\
   MIN\_TICE            &                  -5.00000E+01
    &   minimum ice temperature 
    &  %---ref---
    \\
   IMAX\_TICE           &                        10
    &   iterations for ice heat budget 
    &  %---ref---
    \\
   SEAICE\_EPS          &                   1.00000E-10
    &   reduce derivative singularities 
    &  %---ref---
    \\
   SEAICE\_area\_reg    &                   1.00000E-5
    &   minimum concentration to regularize ice thickness
    &  %---ref---
    \\
   SEAICE\_hice\_reg    &                   0.05 m
    &   minimum ice thickness for regularization
    &  %---ref---
    \\
\hline
\end{tabular}
}
\end{table}

%%% Local Variables: 
%%% mode: latex
%%% TeX-master: "../../manual"
%%% End: 
