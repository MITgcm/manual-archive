\subsection{Land package}
\label{sec:pkg:land}
\begin{rawhtml}
<!-- CMIREDIR:package_land: -->
\end{rawhtml}

This package provides a simple land model
based on Rong Zhang [e-mail:roz@gfdl.noaa.gov] 2 layers model
(see documentation below).

It is primarily implemented for AIM (\_v23) atmospheric physics
but could be adapted to work with a different atmospheric physics.
Two subroutines ({\it aim\_aim2land.F} {\it aim\_land2aim.F}
in {\it pkg/aim\_v23}) are used as interface with AIM physics. 

Number of layers is a parameter ({\it land\_nLev} in {\it LAND\_SIZE.h})
and can be changed. 

%---------------------------------------------------------------------

% \documentclass[12pt,thmsa]{article}

% \begin{document}

\begin{center}
{\bf Note on Land Model}\\
date: June 1999\\
author: Rong Zhang\\
\end{center}

% \baselineskip19pt

This is a simple 2-layer land model. The top layer depth $z1=0.1m$, the
second layer depth $z2=4m$.

Let $T_{g1},T_{g2}$ be the temperature of each layer, $W_{1,}W_{2}$ be the
soil moisture of each layer. The field capacity $f_{1,}$ $f_{2}$ are the
maximum water amount in each layer, so $W_{i}$ is the ratio of available
water to field capacity. $f_{i}=\gamma z_{i},\gamma =0.24$ is the field
capapcity per meter soil$,$ so $f_{1}=0.024m,$ $f_{2}=0.96m.$

The land temperature is determined by total surface downward heat flux $F,$

\begin{equation}
z_{1}C_{1}\frac{dT_{g1}}{dt}=F-\lambda \frac{T_{g1}-T_{g2}}{(z_{1}+z_{2})/2}
\end{equation}

\begin{center}
\begin{equation}
z_{2}C_{2}\frac{dT_{g2}}{dt}=\lambda \frac{T_{g1}-T_{g2}}{(z_{1}+z_{2})/2}
\end{equation}
\end{center}

here $C_{1},C_{2}$ are the heat capacity of each layer , $\lambda $ is the
thermal conductivity, $\lambda =0.42Wm^{-1}K^{-1}.$

\begin{center}
\bigskip 
\begin{equation}
C_{1}=C_{w}W_{1}\gamma +C_{s}
\end{equation}

\begin{equation}
C_{2}=C_{w}W_{2}\gamma +C_{s}
\end{equation}
\end{center}

$C_{w},C_{s}$ are the heat capacity of water and dry soil respectively. $%
C_{w}=4.2\times 10^{6}Jm^{-3}K^{-1},C_{s}=1.13\times 10^{6}Jm^{-3}K^{-1}.$

\bigskip

The soil moisture is determined by precipitation $P(m/s)$,surface
evaporation $E(m/s)$ and runoff $R(m/s).$

\begin{equation}
\frac{dW_{1}}{dt}=\frac{P-E-R}{f_{1}}+\frac{W_{2}-W_{1}}{\tau }
\end{equation}

$\tau =2$ $days$ is the time constant for diffusion of moisture between
layers.

\begin{equation}
\frac{dW_{2}}{dt}=\frac{f_{1}}{f_{2}}\frac{W_{1}-W_{2}}{\tau }
\end{equation}

In the code, $R=0$ gives better result, $W_{1},W_{2}$ are set to be within
[0, 1]. If $W_{1}$ is greater than 1, then let $\delta W_{1}=W_{1}-1,W_{1}=1$
and $W_{2}=W_{2}+p\delta W_{1}\frac{f_{1}}{f_{2}}$, i.e. the runoff of top
layer is put into second layer. $p=0.5$ is the fraction of top layer runoff
that is put into second layer.

The time step is 1 hour, it takes several years to reach equalibrium offline.

\begin{center}
\bigskip
\end{center}

\textbf{References}

Hansen J. et al. Efficient three-dimensional global models for climate
studies: models I and II. \emph{Monthly Weather Review}, vol.111, no.4, pp.
609-62, 1983

% \end{document}
