\subsection{Atmospheric Intermediate Physics: AIM}
\label{sec:pkg:aim}
\begin{rawhtml}
<!-- CMIREDIR:package_aim: -->
\end{rawhtml}

Note:
 The folowing document below describes the \texttt{aim\_v23} package
 that is based on the version v23 of the SPEEDY code (\cite{molteni:03}).

\subsubsection{Key subroutines, parameters and files}
\label{sec:pkg:aim:implementation}

\subsubsection{AIM Diagnostics}
\label{sec:pkg:aim:diagnostics}

{\footnotesize
\begin{verbatim}

------------------------------------------------------------------------
<-Name->|Levs|<-parsing code->|<--  Units   -->|<- Tile (max=80c) 
------------------------------------------------------------------------
DIABT   |  5 |SM      ML      |K/s             |Pot. Temp.  Tendency (Mass-Weighted) from Diabatic Processes
DIABQ   |  5 |SM      ML      |g/kg/s          |Spec.Humid. Tendency (Mass-Weighted) from Diabatic Processes
RADSW   |  5 |SM      ML      |K/s             |Temperature Tendency due to Shortwave Radiation (TT_RSW)
RADLW   |  5 |SM      ML      |K/s             |Temperature Tendency due to Longwave  Radiation (TT_RLW)
DTCONV  |  5 |SM      MR      |K/s             |Temperature Tendency due to Convection (TT_CNV)
TURBT   |  5 |SM      ML      |K/s             |Temperature Tendency due to Turbulence in PBL (TT_PBL)
DTLS    |  5 |SM      ML      |K/s             |Temperature Tendency due to Large-scale condens. (TT_LSC)
DQCONV  |  5 |SM      MR      |g/kg/s          |Spec. Humidity Tendency due to Convection (QT_CNV)
TURBQ   |  5 |SM      ML      |g/kg/s          |Spec. Humidity Tendency due to Turbulence in PBL (QT_PBL)
DQLS    |  5 |SM      ML      |g/kg/s          |Spec. Humidity Tendency due to Large-Scale Condens. (QT_LSC)
TSR     |  1 |SM P    U1      |W/m^2           |Top-of-atm. net Shortwave Radiation (+=dw)
OLR     |  1 |SM P    U1      |W/m^2           |Outgoing Longwave  Radiation (+=up)
RADSWG  |  1 |SM P    L1      |W/m^2           |Net Shortwave Radiation at the Ground (+=dw)
RADLWG  |  1 |SM      L1      |W/m^2           |Net Longwave  Radiation at the Ground (+=up)
HFLUX   |  1 |SM      L1      |W/m^2           |Sensible Heat Flux (+=up)
EVAP    |  1 |SM      L1      |g/m^2/s         |Surface Evaporation (g/m2/s)
PRECON  |  1 |SM P    L1      |g/m^2/s         |Convective  Precipitation (g/m2/s)
PRECLS  |  1 |SM      M1      |g/m^2/s         |Large Scale Precipitation (g/m2/s)
CLDFRC  |  1 |SM P    M1      |0-1             |Total Cloud Fraction (0-1)
CLDPRS  |  1 |SM PC167M1      |0-1             |Cloud Top Pressure (normalized)
CLDMAS  |  5 |SM P    LL      |kg/m^2/s        |Cloud-base Mass Flux  (kg/m^2/s)
DRAG    |  5 |SM P    LL      |kg/m^2/s        |Surface Drag Coefficient (kg/m^2/s)
WINDS   |  1 |SM P    L1      |m/s             |Surface Wind Speed  (m/s)
TS      |  1 |SM      L1      |K               |near Surface Air Temperature  (K)
QS      |  1 |SM P    L1      |g/kg            |near Surface Specific Humidity  (g/kg)
ENPREC  |  1 |SM      M1      |W/m^2           |Energy flux associated with precip. (snow, rain Temp)
ALBVISDF|  1 |SM P    L1      |0-1             |Surface Albedo (Visible band) (0-1)
DWNLWG  |  1 |SM P    L1      |W/m^2           |Downward Component of Longwave Flux at the Ground (+=dw)
SWCLR   |  5 |SM      ML      |K/s             |Clear Sky Temp. Tendency due to Shortwave Radiation
LWCLR   |  5 |SM      ML      |K/s             |Clear Sky Temp. Tendency due to Longwave  Radiation
TSRCLR  |  1 |SM P    U1      |W/m^2           |Clear Sky Top-of-atm. net Shortwave Radiation (+=dw)
OLRCLR  |  1 |SM P    U1      |W/m^2           |Clear Sky Outgoing Longwave  Radiation  (+=up)
SWGCLR  |  1 |SM P    L1      |W/m^2           |Clear Sky Net Shortwave Radiation at the Ground (+=dw)
LWGCLR  |  1 |SM      L1      |W/m^2           |Clear Sky Net Longwave  Radiation at the Ground (+=up)
UFLUX   |  1 |UM   184L1      |N/m^2           |Zonal Wind Surface Stress  (N/m^2)
VFLUX   |  1 |VM   183L1      |N/m^2           |Meridional Wind Surface Stress  (N/m^2)
DTSIMPL |  1 |SM P    L1      |K               |Surf. Temp Change after 1 implicit time step
\end{verbatim}
}

