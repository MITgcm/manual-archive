% \documentclass[12pt]{article}
% \usepackage{amssymb}

%%%%%%%%%%%%%%%%%%%%%%%%%%%%%%%%%%%%%%%%%%%
%%  \usepackage{graphics}


% \oddsidemargin -4mm \evensidemargin 0mm
% \textwidth 165mm
% \textheight 230mm
% \topmargin -2mm \headsep -2mm
% \renewcommand{\baselinestretch}{1.5}
% \begin{document}


\def\deg{$^o$}
%%%--------------------------------------%%%
\subsection{THSICE: The Thermodynamic Sea Ice Package}
\label{sec:pkg:thsice}
\begin{rawhtml}
<!-- CMIREDIR:package_thsice: -->
\end{rawhtml}

{\bf Important note:}
This document has been written by Stephanie Dutkiewicz
and describes an earlier implementation of the sea-ice package.
This needs to be updated to reflect the recent changes (JMC).

\noindent
This thermodynamic ice model is based on the 3-layer model by Winton (2000).
and the energy-conserving LANL CICE model (Bitz and Lipscomb, 1999).
The model considers two equally thick ice layers; the upper layer has
a variable specific heat resulting from brine pockets,
the lower layer has a fixed heat capacity. A zero heat capacity snow
layer lies above the ice. Heat fluxes at the top and bottom
surfaces are used to calculate the change in ice and snow layer
thickness. Grid cells of the ocean model are 
either fully covered in ice or are open water. There is
a provision to parametrize ice fraction (and leads) in this package.
Modifications are discussed in small font following the
subroutine descriptions.

%%%%%%%%%%%%%%%%%%%%%%%%%%%%%%%%%%%%%%%%%%%%%%%%%%%%%%%%%%%%%%

\vspace{1cm}

\noindent
The ice model is called from {\it thermodynamics.F}, subroutine
{\it ice\_forcing.F} is called in place of {\it external\_forcing\_surf.F}.

%%%%%%%%%%%%%%%%%%%%%%%%%%%%%%%%%%%%%%%%%%%%%%%%%%%%%%%%%%%%%%

\vspace{1cm}
\noindent
{\bf \underline{subroutine ICE\_FORCING}}

\noindent
In {\it ice\_forcing.F}, we calculate the freezing potential of the
ocean model surface layer of water:
\[
  {\bf frzmlt} = (T_f - SST) \frac{c_{sw} \rho_{sw} \Delta z}{\Delta t}
\]
where $c_{sw}$ is seawater heat capacity, 
$\rho_{sw}$ is the seawater density, $\Delta z$
is the ocean model upper layer thickness and $\Delta t$ is the model (tracer)
timestep. The freezing temperature, $T_f=\mu S$ is a function of the
salinity.


1) Provided there is no ice present and {\bf frzmlt} is less than 0,
   the surface tendencies of wind, heat and freshwater are calculated
   as usual (ie. as in {\it external\_forcing\_surf.F}).

2) If there is ice present in the grid cell
   we call the main ice model routine {\it ice\_therm.F} (see below).
   Output from this routine gives net heat and freshwater flux 
   affecting the top of the ocean.

Subroutine {\it ice\_forcing.F} uses these values to find the 
sea surface tendencies
in grid cells. When there is ice present,  
the surface stress tendencies are
set to zero; the ice model is purely thermodynamic and the
effect of ice motion on the sea-surface is not examined.

Relaxation of surface $T$ and $S$ is only allowed equatorward
of {\bf relaxlat} (see {\bf DATA.ICE below}), and no relaxation is
allowed under the ice at any latitude.

\noindent
{\tiny (Note that there is provision for allowing grid cells to have both
open water and seaice; if {\bf compact} is between  0 and 1)}

%%%%%%%%%%%%%%%%%%%%%%%%%%%%%%%%%%%%%%%%%%%%%%%%%%%%%%%%%%%%%%
\vspace{1cm}
\noindent
{\bf {\underline{ subroutine ICE\_FREEZE}}}

This routine is called from {\it thermodynamics.F}
after the new temperature calculation, {\it calc\_gt.F}, 
but before {\it calc\_gs.F}.
In {\it ice\_freeze.F}, any ocean upper layer grid cell
with no ice cover, but with temperature below freezing,
$T_f=\mu S$ has ice initialized.
We calculate {\bf frzmlt} from all the grid cells in
the water column that have a temperature less than
freezing. In this routine, any water below the surface
that is below freezing is set to $T_f$.
A call to
{\it ice\_start.F} is made if {\bf frzmlt} $>0$, 
and salinity tendancy is updated for brine release.

\noindent
{\tiny (There is a provision for fractional ice:
In the case where the grid cell has less ice coverage than
{\bf icemaskmax} we allow {\it ice\_start.F} to be called).}

%%%%%%%%%%%%%%%%%%%%%%%%%%%%%%%%%%%%%%%%%%%%%%%%%%%%%%%%%%%%%%%%%%%

\vspace{1cm}
\noindent
{\bf {\underline{ subroutine ICE\_START}}}

\noindent
The energy available from freezing
the sea surface is brought into this routine as {\bf esurp}.
The enthalpy of the 2 layers of any new ice is calculated as:
\begin{eqnarray}
q_1 & = & -c_{i}*T_f + L_i \nonumber \\
q_2 & = & -c_{f}T_{mlt}+ c_{i}(T_{mlt}-T{f}) + L_i(1-\frac{T_{mlt}}{T_f} 
\nonumber \\
\end{eqnarray}
where  $c_f$ is specific heat of liquid fresh water, $c_i$ is the
specific heat of fresh ice, $L_i$ is latent heat of freezing, 
$\rho_i$ is density of ice and
$T_{mlt}$ is melting temperature of ice with salinity of 1.
The height of a new layer of ice is
\[
  h_{i new} = \frac{{\bf esurp} \Delta t}{qi_{0av}}
\]
where $qi_{0av}=-\frac{\rho_i}{2} (q_1+q_2)$.

The surface skin temperature $T_s$ and ice temperatures
$T_1$, $T_2$ and the sea surface temperature are set at $T_f$.

\noindent
{\tiny ( There is provision for fractional ice:
new ice is formed over open water; the first freezing in the cell
must have a height of {\bf himin0}; this determines the ice
fraction {\bf compact}. If there is already ice in the grid cell,
the new ice must have the same height and the new ice fraction
is 
\[
i_f=(1-\hat{i_f}) \frac{h_{i new}}{h_i}
\]
where $\hat{i_f}$ is ice fraction from previous timestep
and $h_i$ is current ice height. Snow is redistributed 
over the new ice fraction. The ice fraction is
not allowed to become larger than {\bf iceMaskmax} and
if the ice height is above {\bf hihig} then freezing energy
comes from the full grid cell,  ice growth does not occur
under orginal ice due to freezing water.
}
%%%%%%%%%%%%%%%%%%%%%%%%%%%%%%%%%%%%%%%%%%%%%%%%%%%%%%%%%%%%%%%%%%%

\vspace{1cm}
\noindent
{\bf {\underline{subroutine ICE\_THERM}}}

\noindent
The main subroutine of this package is {\it ice\_therm.F} where the
ice temperatures are calculated and the changes in ice and snow
thicknesses are determined. Output provides the net heat and fresh 
water fluxes that force the top layer of the ocean model.

If the current ice height is less than {\bf himin} then
the ice layer is set to zero and the ocean model upper layer temperature
is allowed to drop lower than its freezing temperature; and atmospheric
fluxes are allowed to effect the grid cell.
If the ice height is greater than  {\bf himin} we proceed with
the ice model calculation.

We follow the procedure
of Winton (1999) -- see equations 3 to 21 -- to calculate
the surface and internal ice temperatures. 
The surface temperature is found from the balance of the
flux at the surface $F_s$, the shortwave heat flux absorbed by the ice, 
{\bf fswint}, and
the upward conduction of heat through the snow and/or ice, $F_u$.
We linearize $F_s$ about the surface temperature, $\hat{T_s}$, 
at the previous timestep (where \mbox{}$\hat{ }$ indicates the value at
the  previous timestep):
\[
F_s (T_s) = F_s(\hat{T_s}) + \frac{\partial F_s(\hat{T_s)}}{\partial T_s}
(T_s-\hat{T_s})
\]
where, 
\[
F_s  =  F_{sensible}+F_{latent}+F_{longwave}^{down}+F_{longwave}^{up}+ (1-
\alpha) F_{shortwave}
\]
and
\[
 \frac{d F_s}{dT} = \frac{d F_{sensible}}{dT} + \frac{d F_{latent}}{dT}
+\frac{d F_{longwave}^{up}}{dT}.
\]
$F_s$ and $\frac{d F_s}{dT}$ are currently calculated from the {\bf BULKF} 
package described separately, but could also be provided by an atmospheric
model. The surface albedo is calculated from the ice height and/or 
surface temperature (see below, {\it srf\_albedo.F}) and the 
shortwave flux absorbed in the ice is
\[
{\bf fswint} = (1-e^{\kappa_i h_i})(1-\alpha) F_{shortwave}
\]
where $\kappa_i$ is bulk extinction coefficient.

The conductive flux to the surface is
\[
F_u=K_{1/2}(T_1-T_s)
\]
where $K_{1/2}$ is the effective conductive coupling of the snow-ice
layer between the surface and the mid-point of the upper layer of ice
$
K_{1/2}=\frac{4 K_i K_s}{K_s h_i + 4 K_i h_s}
$.
$K_i$ and $K_s$ are constant thermal conductivities of seaice and snow.

From the above equations we can develop a system of equations to
find the skin surface temperature, $T_s$ and the two ice layer
temperatures (see Winton, 1999, for details). We solve these
equations iteratively until the change in $T_s$ is small.
When the surface temperature is greater then
the melting temperature of the surface, the temperatures are
recalculated setting $T_s$ to 0.  The enthalpy
of the ice layers are calculated in order to keep track of the energy in the
ice model. Enthalpy is defined, here, as the energy required to melt a
unit mass of seaice with temperature $T$.
For the upper layer (1) with brine pockets  and
the lower fresh layer (2):
\begin{eqnarray}
q_1 & = & - c_f T_f + c_i (T_f-T)+ L_{i}(1-\frac{T_f}{T})
\nonumber \\
q_2 & = & -c_i T+L_i \nonumber
\end{eqnarray}
where $c_f$ is specific heat of liquid fresh water, $c_i$ is the
specific heat of fresh ice, and $L_i$ is latent heat of melting fresh ice.



From the new ice temperatures, we can calculate
the energy flux at the surface available for melting (if $T_s$=0)
and the energy at the ocean-ice interface for either melting or freezing.
\begin{eqnarray}
E_{top} &  =  & (F_s- K_{1/2}(T_s-T_1) ) \Delta t
\nonumber \\
E_{bot} &= & (\frac{4K_i(T_2-T_f)}{h_i}-F_b) \Delta t
\nonumber
\end{eqnarray}
where $F_b$ is the heat flux at the ice bottom due to the sea surface
temperature variations from freezing.
If $T_{sst}$ is above freezing, $F_b=c_{sw} \rho_{sw} 
\gamma (T_{sst}-T_f)u^{*}$, $\gamma$ is the heat transfer coefficient
and $u^{*}=QQ$ is frictional velocity between ice 
and water. If $T_{sst}$ is below freezing,
$F_b=(T_f - T_{sst})c_f \rho_f \Delta z /\Delta t$ and set $T_{sst}$
to $T_f$. We also
include the energy from lower layers that drop below freezing,
and set those layers to $T_f$.

If $E_{top}>0$ we melt snow from the surface, if all the snow is melted
and there is energy left, we melt the ice. If the ice is all gone
and there is still energy left, we apply the left over energy to 
heating the ocean model upper layer (See Winton, 1999, equations 27-29).
Similarly if $E_{bot}>0$ we melt ice from the bottom. If all the ice
is melted, the snow is melted (with energy from the ocean model upper layer
if necessary). If $E_{bot}<0$ we grow ice at the bottom
\[
\Delta h_i = \frac{-E_{bot}}{(q_{bot} \rho_i)}
\]
where $q_{bot}=-c_{i} T_f + L_i$ is the enthalpy of the new ice,
The enthalpy of the second ice layer, $q_2$ needs to be modified:
\[
q_2 = \frac{ \hat{h_i}/2 \hat{q_2} + \Delta h_i q_{bot} }
        {\hat{h_i}/{2}+\Delta h_i}
\]

If there is a ice layer and the overlying air temperature is
below 0$^o$C then any precipitation, $P$ joins the snow layer:
\[
\Delta h_s  = -P \frac{\rho_f}{\rho_s} \Delta t, 
\]
$\rho_f$ and $\rho_s$ are the fresh water and snow densities.
Any evaporation, similarly, removes snow or ice from the surface.
We also calculate the snow age here, in case it is needed for
the surface albedo calculation (see {\it srf\_albedo.F} below).

For practical reasons we limit the ice growth to {\bf hilim}
and snow is limited to {\bf hslim}. We converts any
ice and/or snow above these limits back to water, maintaining the salt
balance. Note however, that heat is not conserved in this
conversion; sea surface temperatures below the ice are not
recalculated.

If the snow/ice interface is below the waterline, snow is converted
to ice (see Winton, 1999, equations 35 and 36). The subroutine
{\it new\_layers\_winton.F}, described below, repartitions the ice into
equal thickness layers while conserving energy.

The subroutine {\it ice\_therm.F} now calculates the heat and fresh
water fluxes affecting the ocean model surface layer. The heat flux:
\[
q_{net}= {\bf fswocn} - F_{b} - \frac{{\bf esurp}}{\Delta t}
\]
is composed of the shortwave flux that has passed through the
ice layer and is absorbed by the water, {\bf fswocn}$=QQ$,
the ocean flux to the ice $F_b$,
and the surplus energy left over from the melting, {\bf esurp}.
The fresh water flux is determined from the amount of
fresh water and salt in the ice/snow system before and after the
timestep.

\noindent
{\tiny (There is a provision for fractional ice:
If ice height is above {\bf hihig} then all energy from freezing at
sea surface is used only in the open water aparts of the cell (ie.
$F_b$ will only have the conduction term).
The melt energy is partitioned by {\bf frac\_energy} between melting
ice height and ice extent. However, once ice height drops below
{\bf himon0} then all energy melts ice extent.}

%%%%%%%%%%%%%%%%%%%%%%%%%%%%%%%%%%%%%%%%%%%%%%%%%%%%%%%%%%%%%%%
\vspace{1cm}

\noindent
{\bf {\underline{subroutine SFC\_ALBEDO} } }

\noindent
The routine {\it ice\_therm.F} calls this routine to determine
the surface albedo. There are two calculations provided here:

\noindent
{\bf 1)} from LANL CICE model
\[ \alpha = f_s \alpha_s + (1-f_s) (\alpha_{i_{min}}
         + (\alpha_{i_{max}}- \alpha_{i_{min}}) (1-e^{-h_i/h_{\alpha}}))
\]
where $f_s$ is 1 if there is snow, 0 if not; the snow albedo, 
$\alpha_s$ has two values
depending on whether $T_s<0$ or not; $\alpha_{i_{min}}$ and 
$\alpha_{i_{max}}$ are ice albedos for thin melting ice, and
thick bare ice respectively, and $h_{\alpha}$ is a scale
height.

\noindent
{\bf 2)} From GISS model (Hansen et al 1983)
\[
 \alpha = \alpha_i e^{-h_s/h_a} + \alpha_s (1-e^{-h_s/h_a})
\]
where $\alpha_i$ is a constant albedo for bare ice, $h_a$
is a scale height and $\alpha_s$ is a variable snow albedo.
\[
\alpha_s = \alpha_1 + \alpha_2 e^{-\lambda_a a_s}
\]
where $\alpha_1$ is a constant, $\alpha_2$ depends on $T_s$,
$a_s$ is the snow age, and $\lambda_a$ is a scale frequency.
The snow age is calculated in {\it ice\_therm.F} and is given
in equation 41 in Hansen et al (1983).

%%%%%%%%%%%%%%%%%%%%%%%%%%%%%%%%%%%%%%%%%%%%%%%%%%%%%%%%%%%%%%%

\vspace{1cm}

\noindent
{\bf {\underline{subroutine NEW\_LAYERS\_WINTON}}}

\noindent
The subroutine
{\it new\_layers\_winton.F} repartitions the ice into
equal thickness layers while conserving energy. We pass
to this subroutine, the ice layer enthalpies after
melting/growth and the new height of the ice layers.
The ending layer height should be half the sum of the
new ice heights from {\it ice\_therm.F}. The enthalpies
of the ice layers are adjusted accordingly to maintain
total energy in the ice model. If layer 2 height is
greater than layer 1 height then layer 2 gives ice to
layer 1 and:
\[
q_1=f_1 \hat{q_1} + (1-f1) \hat{q_2}
\]
where $f_1$ is the fraction of the new to old upper layer heights.
$T_1$ will therefore also have changed.
Similarly for when ice layer height 2 is less than
layer 1 height, except here we need to to be careful
that the new $T_2$ does not fall below the melting temperature.

%%%%%%%%%%%%%%%%%%%%%%%%%%%%%%%%%%%%%%%%%%%%%%%%%%%%%%%%%%%%%%%

\vspace{1cm}

\noindent
{\bf {\underline{Initializing subroutines}}}

\noindent
{\it ice\_init.F}:
Set ice variables to zero, or reads in pickup information
from {\bf pickup.ic} (which was written out in {\it checkpoint.F})

\noindent
{\it ice\_readparms.F}:
Reads {\bf data.ice}

%%%%%%%%%%%%%%%%%%%%%%%%%%%%%%%%%%%%%%%%%%%%%%%%%%%%%%%%%%%%%%%

\vspace{1cm}

\noindent
{\bf {\underline{Diagnostic subroutines}}}

\noindent
{\it ice\_ave.F}:
Keeps track of means of the ice variables

\noindent
{\it ice\_diags.F}:
Finds averages and writes out diagnostics

%%%%%%%%%%%%%%%%%%%%%%%%%%%%%%%%%%%%%%%%%%%%%%%%%%%%%%%%%%%%%%%%
\vspace{1cm}

\noindent
{\bf {\underline{Common Blocks}}}

\noindent
{\it ICE.h}: Ice Varibles, also 
{\bf relaxlat} and {\bf startIceModel}

\noindent
{\it ICE\_DIAGS.h}: matrices for diagnostics: averages of fields
from {\it ice\_diags.F}

\noindent
{\it BULKF\_ICE\_CONSTANTS.h} (in {\bf BULKF} package): 
all the parameters need by the ice model

%%%%%%%%%%%%%%%%%%%%%%%%%%%%%%%%%%%%%%%%%%%%%%%%%%%%%%%%%%%%%%%%%%
\vspace{1cm}

\noindent
{\bf {\underline{Input file DATA.ICE}}}

\noindent
Here we need to set {\bf StartIceModel}: which is 1 if the
model starts from no ice; and 0 if there is a pickup file
with the ice matrices ({\bf pickup.ic}) which is read
in {\it ice\_init.F} and written out in {\it checkpoint.F}.
The parameter {\bf relaxlat} defines the latitude poleward
of which there is no relaxing of surface $T$ or $S$ to
observations. This avoids the relaxation forcing the ice
model at these high latitudes.

\noindent
({\tiny Note: {\bf hicemin} is set to 0 here. If the
provision for allowing grid cells to have both
open water and seaice is ever implemented, this would
be greater than 0})

%%%%%%%%%%%%%%%%%%%%%%%%%%%%%%%%%%%%%%%%%%%%%%%%%%%%%%%%%%%%
\vspace{1cm}

\noindent
{\bf {\underline{Important Notes}}}

\noindent
{\bf 1)} heat fluxes have different signs in the ocean and ice
models.

\noindent
{\bf 2)} {\bf StartIceModel} must be changed in {\bf data.ice}:
1 (if starting from no ice), 0 (if using pickup.ic file).

%%%%%%%%%%%%%%%%%%%%%%%%%%%%%%%%%%%%%%%%%%%%%%%%%%%%%%%%%%%%%%

\vspace{1cm}

\noindent 
{\bf {\underline{References}}}

\noindent
Bitz, C.M. and W.H. Lipscombe, 1999: An Energy-Conserving
Thermodynamic Model of Sea Ice.
{\it Journal of Geophysical Research}, 104, 15,669 -- 15,677.

\vspace{.2cm}

\noindent
Hansen, J., G. Russell, D. Rind, P. Stone, A. Lacis, S. Lebedeff,
R. Ruedy and L.Travis, 1983: Efficient Three-Dimensional
Global Models for Climate Studies: Models I and II.
{\it Monthly Weather Review}, 111, 609 -- 662.

\vspace{.2cm}

\noindent
Hunke, E.C and W.H. Lipscomb, circa 2001: CICE: the Los Alamos
Sea Ice Model Documentation and Software User's Manual.
LACC-98-16v.2.\\
(note: this documentation is no longer available as CICE has progressed
to a very different version 3)


\vspace{.2cm}

\noindent
Winton, M, 2000: A reformulated Three-layer Sea Ice Model.
{\it Journal of Atmospheric and Ocean Technology}, 17, 525 -- 531.



%%%%%%%%%%%%%%%%%%%%%%%%%%%%%%%%%%%%%%%%%%%
% \end{document}
