\subsection{KPP: Nonlocal K-Profile Parameterization for 
Vertical Mixing}

\label{sec:pkg:kpp}
\begin{rawhtml}
<!-- CMIREDIR:package_kpp: -->
\end{rawhtml}

Authors: Dimitris Menemenlis and Patrick Heimbach

\subsubsection{Introduction
\label{sec:pkg:kpp:intro}}

The nonlocal K-Profile Parameterization (KPP) scheme
of \cite{lar-eta:94} unifies the treatment of a variety of 
unresolved processes involved in vertical mixing. 
To consider it as one mixing scheme is, in the view of the authors,
somewhat misleading since it consists of several entities
to deal with distinct mixing processes in the ocean's surface 
boundary layer, and the interior:
%
\begin{enumerate}
%
\item
mixing in the interior is goverened by
shear instability (modeled as function of the local gradient
Richardson number), internal wave activity (assumed constant),
and double-diffusion (not implemented here).
%
\item
a boundary layer depth $h$ or \texttt{hbl} is determined
at each grid point, based on a critical value of turbulent
processes parameterized by a bulk Richardson number;
%
\item
mixing is strongly enhanced in the boundary layer under the
stabilizing or destabilizing influence of surface forcing 
(buoyancy and momentum) enabling boundary layer properties
to penetrate well into the thermocline; 
mixing is represented through a polynomial profile whose 
coefficients are determined subject to several contraints;
%
\item
the boundary-layer profile is made to agree with similarity
theory of turbulence and is matched, in the asymptotic sense
(function and derivative agree at the boundary),
to the interior thus fixing the polynomial coefficients; 
matching allows for some fraction of the boundary layer mixing
to affect the interior, and vice versa;
%
\item
a ``non-local'' term $\hat{\gamma}$ or \texttt{ghat}
which is independent of the vertical property gradient further 
enhances mixing where the water column is unstable
%
\end{enumerate}
%
The scheme has been extensively compared to observations
(see e.g. \cite{lar-eta:97}) and is now coomon in many
ocean models.

The following sections will describe the KPP package
configuration and compiling (\ref{sec:pkg:kpp:comp}),
the settings and choices of runtime parameters
(\ref{sec:pkg:kpp:runtime}),
more detailed description of equations to which these
parameters relate (\ref{sec:pkg:kpp:equations}),
and key subroutines where they are used (\ref{sec:pkg:kpp:subroutines}),
and diagnostics output of KPP-derived diffusivities, viscosities
and boundary-layer/mixed-layer depths.

%----------------------------------------------------------------------

\subsubsection{KPP configuration and compiling
\label{sec:pkg:kpp:comp}}

As with all MITgcm packages, KPP can be turned on or off at compile time
%
\begin{itemize}
%
\item
using the \texttt{packages.conf} file by adding \texttt{kpp} to it,
%
\item
or using \texttt{genmake2} adding
\texttt{-enable=kpp} or \texttt{-disable=kpp} switches
%
\end{itemize}
(see Section \ref{sect:buildingCode}).

Parts of the KPP code can be enabled or disabled at compile time
via CPP preprocessor flags. These options are set in
\texttt{KPP\_OPTIONS.h}. Table \ref{tab:pkg:kpp:cpp} summarizes them.

\begin{table}[h!]
\centering
  \label{tab:pkg:kpp:cpp}
  {\footnotesize
    \begin{tabular}{|l|l|}
      \hline 
      \textbf{CPP option}  &  \textbf{Description}  \\
      \hline \hline
        \texttt{\_KPP\_RL} & 
          ~ \\
        \texttt{FRUGAL\_KPP} & 
          ~ \\
        \texttt{KPP\_SMOOTH\_SHSQ} & 
          ~ \\
        \texttt{KPP\_SMOOTH\_DVSQ} & 
          ~ \\
        \texttt{KPP\_SMOOTH\_DENS} & 
          ~ \\
        \texttt{KPP\_SMOOTH\_VISC} & 
          ~ \\
        \texttt{KPP\_SMOOTH\_DIFF} & 
          ~ \\
        \texttt{KPP\_ESTIMATE\_UREF} & 
          ~ \\
        \texttt{INCLUDE\_DIAGNOSTICS\_INTERFACE\_CODE} & 
          ~ \\
        \texttt{KPP\_GHAT} & 
          ~ \\
        \texttt{EXCLUDE\_KPP\_SHEAR\_MIX} & 
          ~ \\
      \hline
    \end{tabular}
  }
  \caption{~}
\end{table}


%----------------------------------------------------------------------

\subsubsection{Run-time parameters
\label{sec:pkg:kpp:runtime}}

Run-time parameters are set in files 
\texttt{data.pkg} and \texttt{data.kpp}
which are read in \texttt{kpp\_readparms.F}.
Run-time parameters may be broken into 3 categories:
(i) switching on/off the package at runtime,
(ii) required MITgcm flags,
(iii) package flags and parameters.

\paragraph{Enabling the package}
~ \\
%
The KPP package is switched on at runtime by setting
\texttt{useKPP = .TRUE.} in \texttt{data.pkg}.

\paragraph{Required MITgcm flags}
~ \\
%
The following flags/parameters of the MITgcm dynamical
kernel need to be set in conjunction with KPP:

\begin{tabular}{ll}
\texttt{implicitViscosity = .TRUE.} & enable implicit vertical viscosity \\
\texttt{implicitDiffusion = .TRUE.} & enable implicit vertical diffusion \\
\end{tabular}


\paragraph{Package flags and parameters}
~ \\
%
Table \ref{tab:pkg:kpp:runtime_flags} summarizes the
runtime flags that are set in \texttt{data.pkg}, and
their default values.

\begin{table}[h!]
\centering
  \label{tab:pkg:kpp:runtime_flags}
  {\footnotesize
    \begin{tabular}{|l|c|l|}
      \hline 
      \textbf{Flag/parameter} & \textbf{default} &  \textbf{Description}  \\
      \hline \hline
         \multicolumn{3}{|c|}{\textit{I/O related parameters} } \\
         \hline
        kpp\_freq & \texttt{deltaTClock} & 
           Recomputation frequency for KPP fields \\
        kpp\_dumpFreq & \texttt{dumpFreq} & 
           Dump frequency of KPP field snapshots \\
        kpp\_taveFreq & \texttt{taveFreq} & 
           Averaging and dump frequency of KPP fields \\
        KPPmixingMaps & \texttt{.FALSE.} & 
           include KPP diagnostic maps in STDOUT \\
        KPPwriteState & \texttt{.FALSE.} & 
           write KPP state to file \\
        KPP\_ghatUseTotalDiffus & \texttt{.FALSE.} & 
           if \texttt{.T.} compute non-local term using total vertical diffusivity \\
        ~ & ~ &
           if \texttt{.F.} use KPP vertical diffusivity \\
      \hline
      \multicolumn{3}{|c|}{\textit{Genral KPP parameters} } \\
      \hline
        minKPPhbl & \texttt{delRc(1)} & 
           Minimum boundary layer depth \\
        epsilon & 0.1 & 
           nondimensional extent of the surface layer \\
        vonk & 0.4 & 
           von Karman constant \\
        dB\_dz & 5.2E-5 1/s$^2$ & 
           maximum dB/dz in mixed layer hMix \\
        concs & 98.96 &
           ~ \\
        concv & 1.8 &
           ~ \\
      \hline
      \multicolumn{3}{|c|}{\textit{Boundary layer parameters (S/R \texttt{bldepth})} } \\
      \hline
        Ricr & 0.3 & 
           critical bulk Richardson number \\
        cekman & 0.7 & 
           coefficient for Ekman depth \\
        cmonob & 1.0 & 
           coefficient for Monin-Obukhov depth \\
        concv & 1.8 & 
           ratio of interior to  entrainment depth buoyancy frequency \\
        hbf & 1.0 & 
           fraction of depth to which absorbed solar radiation contributes \\
        ~ & ~ &
           to surface buoyancy forcing \\
        Vtc & \texttt{~} & 
           non-dim. coeff. for velocity scale of turbulant velocity shear \\
        ~ & ~ &
           ( = function of concv,concs,epsilon,vonk,Ricr) \\
      \hline
      \multicolumn{3}{|c|}{\textit{Boundary layer mixing parameters (S/R \texttt{blmix})} } \\
      \hline
        cstar & 10. & 
           proportionality coefficient for nonlocal transport \\
        cg & ~ & 
           non-dimensional coefficient for counter-gradient term \\
        ~ & ~ &
           ( = function of cstar,vonk,concs,epsilon) \\
      \hline
      \multicolumn{3}{|c|}{\textit{Interior mixing parameters (S/R \texttt{Ri\_iwmix})} } \\
      \hline
        Riinfty & 0.7 & 
           gradient Richardson number limit for shear instability \\
        BVDQcon & -0.2E-4 1/s$^2$ & 
           Brunt-V\"ai\"sal\"a squared \\
        difm0 & 0.005 m$^2$/s & 
           viscosity max. due to shear instability \\
        difs0 & 0.005 m$^2$/s & 
           tracer diffusivity max. due to shear instability \\
        dift0 & 0.005 m$^2$/s & 
           heat diffusivity max. due to shear instability \\
        difmcon & 0.1 & 
           viscosity due to convective instability \\
        difscon & 0.1 & 
           tracer diffusivity due to convective instability \\
        diftcon & 0.1 & 
           heat diffusivity due to convective instability \\
      \hline
      \multicolumn{3}{|c|}{\textit{Double-diffusive mixing parameters (S/R \texttt{ddmix})} } \\
      \hline
        Rrho0 & not used & 
           limit for double diffusive density ratio \\
        dsfmax & not used & 
           maximum diffusivity in case of salt fingering \\
         \hline
      \hline
    \end{tabular}
  }
  \caption{~}
\end{table}



%----------------------------------------------------------------------

\subsubsection{Equations
\label{sec:pkg:kpp:equations}}

We restrict ourselves to writing out only the essential equations
that relate to main processes and parameters mentioned above.
We closely follow the notation of \cite{lar-eta:94}.

\paragraph{Mixing in the boundary layer} ~ \\
%
~

The vertical fluxes $\overline{wx}$
of momentum and tracer properties $X$
is composed of a gradient-flux term (proportional to
the vertical property divergence $\partial_z X$), and
a ``nonlocal'' term $\gamma_x$ that enhances the
gradient-flux mixing coefficient $K_x$
%
\begin{equation}
\overline{wx}(d) \, = \, -K_x \left(
\frac{\partial X}{\partial z} \, - \, \gamma_x \right)
\end{equation}

\begin{itemize}
%
\item
\textit{Boundary layer mixing profile} \\
%
It is expressed as the product of the boundary layer depth $h$,
a depth-dependent turbulent velocity scale $w_x(\sigma)$ and a
non-dimensional shape function $G(\sigma)$
%
\begin{equation}
K_x(\sigma) \, = \, h \, w_x(\sigma) \, G(\sigma)
\end{equation}
%
with dimensionless vertical coordinate $\sigma = d/h$.
For details of $ w_x(\sigma)$ and $G(\sigma)$ we refer to
\cite{lar-eta:94}.

%
\item
\textit{Nonlocal mixing term} \\
%
The nonlocal transport term $\gamma$ is nonzero only for
tracers in unstable (convective) forcing conditions.
Thus, depending on the  stability parameter $\zeta = d/L$ 
(with depth $d$, Monin-Obukhov length scale $L$)
it has the following form:
%
\begin{eqnarray}
\begin{array}{cl}
\gamma_x \, = \, 0 & \zeta \, \ge \, 0 \\
~ & ~ \\
\left.
\begin{array}{c}
\gamma_m \, = \, 0 \\
 ~ \\
\gamma_s \, = \, C_s 
\frac{\overline{w s_0}}{w_s(\sigma) h} \\
 ~ \\
\gamma_{\theta} \, = \, C_s
\frac{\overline{w \theta_0}+\overline{w \theta_R}}{w_s(\sigma) h} \\
\end{array}
\right\} 
&
\zeta \, < \, 0 \\
\end{array}
\end{eqnarray}

\end{itemize}


\paragraph{Mixing in the interior} ~ \\
%
~

\paragraph{Implicit time integration} ~ \\
%
~

%----------------------------------------------------------------------

\subsubsection{Key subroutines
\label{sec:pkg:kpp:subroutines}}

\paragraph{kpp\_calc:} Top-level routine. \\
~

\paragraph{kpp\_mix:} Intermediate-level routine \\
~

\paragraph{ri\_iwmix:} ~ \\
%
Compute interior viscosity and diffusivity coefficients due to
%
\begin{itemize}
%
\item
shear instability (dependent on a local gradient Richardson number),
%
\item
to background internal wave activity, and
%
\item
to static instability (local Richardson number < 0).
%
\end{itemize}


\paragraph{bldepth:} ~ \\
%
The oceanic planetary boundary layer depth, \texttt{hbl}, is determined as
the shallowest depth where the bulk Richardson number is
equal to the critical value, \texttt{Ricr}.

Bulk Richardson numbers are evaluated by computing velocity and
buoyancy differences between values at zgrid(kl) < 0 and surface
reference values.
In this configuration, the reference values are equal to the
values in the surface layer.
When using a very fine vertical grid, these values should be
computed as the vertical average of velocity and buoyancy from
the surface down to epsilon*zgrid(kl).

When the bulk Richardson number at k exceeds Ricr, hbl is
linearly interpolated between grid levels zgrid(k) and zgrid(k-1).

The water column and the surface forcing are diagnosed for
stable/ustable forcing conditions, and where hbl is relative
to grid points (caseA), so that conditional branches can be
avoided in later subroutines.

\paragraph{blmix:} ~ \\
%
Compute boundary layer mixing coefficients.
Mixing coefficients within boundary layer depend on surface
forcing and the magnitude and gradient of interior mixing below
the boundary layer ("matching").
%
\begin{enumerate}
%
\item
compute velocity scales at hbl
%
\item
find the interior viscosities and derivatives at hbl
%
\item
compute turbulent velocity scales on the interfaces
%
\item
compute the dimensionless shape functions at the interfaces
%
\item
compute boundary layer diffusivities at the interfaces
%
\item
compute nonlocal transport term
%
\item
find diffusivities at kbl-1 grid level
%
\end{enumerate}

\paragraph{kpp\_calc\_diff\_t/\_s, kpp\_calc\_visc:} ~  \\
%
Add contribution to net diffusivity/viscosity from 
KPP diffusivity/viscosity.

\paragraph{kpp\_transport\_t/\_s/\_ptr:} ~ \\
%
Add non local KPP transport term (ghat) to diffusive
temperature/salinity/passive tracer flux.
The nonlocal transport term is nonzero only for scalars
in unstable (convective) forcing conditions. 

\paragraph{Flow chart:} ~ \\
%
{\footnotesize
\begin{verbatim}

C     !CALLING SEQUENCE:
c ...
c  kpp_calc (TOP LEVEL ROUTINE)
c  |
c  |-- statekpp: o compute all EOS/density-related arrays
c  |             o uses S/R FIND_ALPHA, FIND_BETA, FIND_RHO
c  |
c  |-- kppmix
c  |   |--- ri_iwmix (compute interior mixing coefficients due to constant
c  |   |              internal wave activity, static instability, 
c  |   |              and local shear instability).
c  |   |
c  |   |--- bldepth (diagnose boundary layer depth)
c  |   |
c  |   |--- blmix (compute boundary layer diffusivities)
c  |   |
c  |   |--- enhance (enhance diffusivity at interface kbl - 1)
c  |   o
c  |
c  |-- swfrac
c  o

\end{verbatim}
}

%----------------------------------------------------------------------

\subsubsection{KPP diagnostics
\label{sec:pkg:kpp:diagnostics}}

Diagnostics output is available via the diagnostics package
(see Section \ref{sec:pkg:diagnostics}).
Available output fields are summarized in 
Table \ref{tab:pkg:kpp:diagnostics}.

\begin{table}[h!]
\centering
\label{tab:pkg:kpp:diagnostics}
{\footnotesize
\begin{verbatim}
------------------------------------------------------
 <-Name->|Levs|grid|<--  Units   -->|<- Tile (max=80c)
------------------------------------------------------
 KPPviscA| 23 |SM  |m^2/s           |KPP vertical eddy viscosity coefficient
 KPPdiffS| 23 |SM  |m^2/s           |Vertical diffusion coefficient for salt & tracers
 KPPdiffT| 23 |SM  |m^2/s           |Vertical diffusion coefficient for heat
 KPPghat | 23 |SM  |s/m^2           |Nonlocal transport coefficient
 KPPhbl  |  1 |SM  |m               |KPP boundary layer depth, bulk Ri criterion
 KPPmld  |  1 |SM  |m               |Mixed layer depth, dT=.8degC density criterion
 KPPfrac |  1 |SM  |                |Short-wave flux fraction penetrating mixing layer
\end{verbatim}
}
\caption{~}
\end{table}

%----------------------------------------------------------------------

\subsubsection{Reference experiments}

lab\_sea:

natl\_box:

%----------------------------------------------------------------------

\subsubsection{References}

