\section{EXF: The external forcing package
\label{sec:pkg:exf}}
\begin{rawhtml}
<!-- CMIREDIR:sectionexf: -->
\end{rawhtml}


\subsection{Introduction
\label{sec:pkg:exf:intro}}

The external forcing package, in conjunction with the
calendar package (cal), enables the handling of real-time
(or ``model-time'') forcing
fields of differing temporal forcing patterns.
It comprises climatological restoring and relaxation.
Bulk formulae are implemented to convert atmospheric fields
to surface fluxes.
An interpolation routine provides on-the-fly interpolation of
forcing fields an arbitrary grid onto the model grid.

CPP options enable or disable different aspects of the package
(Section \ref{sec:pkg:exf:config}).
Runtime options, flags, filenames and field-related dates/times are
set in \texttt{data.exf} and \texttt{data.exf\_clim}
(Section \ref{sec:pkg:exf:runtime}).
A description of key subroutines is given in Section
\ref{sec:pkg:exf:subroutines}.
Input fields, units and sign conventions are summarized in
Section \ref{sec:pkg:exf:fields_units}, and available diagnostics
output is listed in Section \ref{sec:pkg:exf:fields_diagnostics}.

%----------------------------------------------------------------------

\subsection{EXF configuration \& compiling
\label{sec:pkg:exf:config}}

As with all MITgcm packages, EXF can be turned on or off at compile time
using the \texttt{packages.conf} file or the \texttt{genmake2}
\texttt{-enable=exf} or \texttt{-disable=exf} switches.

Parts of the exf code can be enabled or disabled at compile time
via CPP preprocessor flags. These options are set in either
\texttt{EXF\_OPTIONS.h} or in \texttt{ECCO\_CPPOPTIONS.h}.
Table \ref{tab:pkg:exf:cpp} summarizes these options.

\begin{table}[b!]
  \label{tab:pkg:exf:cpp}
  {\footnotesize
    \begin{tabular}{|l|l|}
      \hline
      \textbf{CPP option}  &  \textbf{Description}  \\
      \hline
        \texttt{EXF\_VERBOSE} & 
          verbose mode (recommended only for testing) \\
        \texttt{ALLOW\_ATM\_TEMP} & 
          compute heat/freshwater fluxes from atmos. state input \\
        \texttt{ALLOW\_ATM\_WIND} & 
          compute wind stress from wind speed input\\
        \texttt{ALLOW\_BULKFORMULAE} & 
          is used if either ALLOW\_ATM\_TEMP or ALLOW\_ATM\_WIND is enabled \\
        \texttt{EXF\_READ\_EVAP} & read evaporation instead of computing it \\
        \texttt{ALLOW\_RUNOFF} & read time-constant river/glacier run-off field \\
        \texttt{ALLOW\_DOWNWARD\_RADIATION} & compute net from downward or downward from net radiation \\
        \texttt{USE\_EXF\_INTERPOLATION} & enable on-the-fly bilinear or bicubic interpolation of input fields \\
      \hline
        \texttt{ALLOW\_CLIMTEMP\_RELAXATION} & 
          relaxation to 3-D potential temperature field \\
        \texttt{ALLOW\_CLIMSALT\_RELAXATION} & 
          relaxation to 3-D salinity field \\
        \texttt{ALLOW\_CLIMSST\_RELAXATION} &
          relaxation to 2-D SST relaxation \\
        \texttt{ALLOW\_CLIMSSS\_RELAXATION} &
          relaxation to 2-D SSS relaxation  \\
      \hline
        \texttt{SHORTWAVE\_HEATING} & in \texttt{CPP\_OPTIONS.h}: enable shortwave radiation \\
        \texttt{ATMOSPHERIC\_LOADING} &  in \texttt{CPP\_OPTIONS.h}: enable surface pressure forcing \\
      \hline
    \end{tabular}
  }
  \caption{~}
\end{table}


%----------------------------------------------------------------------

\subsection{EXF runtime parameters
\label{sec:pkg:exf:runtime}}

%----------------------------------------------------------------------

\subsection{EXF fields and units
\label{sec:pkg:exf:fields_units}}

The following list is taken from the header file \texttt{exf\_fields.h}.

{\footnotesize
\begin{verbatim}



c     ustress   :: Zonal surface wind stress in N/m^2
c                  > 0 for increase in uVel, which is west to
c                      east for cartesian and spherical polar grids
c                  Typical range: -0.5 < ustress < 0.5
c                  Southwest C-grid U point
c                  Input field
c
c     vstress   :: Meridional surface wind stress in N/m^2
c                  > 0 for increase in vVel, which is south to
c                      north for cartesian and spherical polar grids
c                  Typical range: -0.5 < vstress < 0.5
c                  Southwest C-grid V point
c                  Input field
c
c     hflux     :: Net upward surface heat flux in W/m^2 
c                  excluding shortwave (on input)
c                  hflux = latent + sensible + lwflux
c                  > 0 for decrease in theta (ocean cooling)
c                  Typical range: -250 < hflux < 600
c                  Southwest C-grid tracer point
c                  Input field
c
c     sflux     :: Net upward freshwater flux in m/s
c                  sflux = evap - precip - runoff
c                  > 0 for increase in salt (ocean salinity)
c                  Typical range: -1e-7 < sflux < 1e-7
c                  Southwest C-grid tracer point
c                  Input field
c
c     swflux    :: Net upward shortwave radiation in W/m^2
c                  swflux = - ( swdown - ice and snow absorption - reflected )
c                  > 0 for decrease in theta (ocean cooling)
c                  Typical range: -350 < swflux < 0
c                  Southwest C-grid tracer point
c                  Input field
c
c     uwind     :: Surface (10-m) zonal wind velocity in m/s
c                  > 0 for increase in uVel, which is west to
c                      east for cartesian and spherical polar grids
c                  Typical range: -10 < uwind < 10
c                  Southwest C-grid U point
c                  Input or input/output field
c
c     vwind     :: Surface (10-m) meridional wind velocity in m/s
c                  > 0 for increase in vVel, which is south to
c                      north for cartesian and spherical polar grids
c                  Typical range: -10 < vwind < 10
c                  Southwest C-grid V point
c                  Input or input/output field
c
c     atemp     :: Surface (2-m) air temperature in deg K
c                  Typical range: 200 < atemp < 300
c                  Southwest C-grid tracer point
c                  Input or input/output field
c
c     aqh       :: Surface (2m) specific humidity in kg/kg
c                  Typical range: 0 < aqh < 0.02
c                  Southwest C-grid tracer point
c                  Input or input/output field
c
c     lwflux    :: Net upward longwave radiation in W/m^2
c                  lwflux = - ( lwdown - ice and snow absorption - emitted )
c                  > 0 for decrease in theta (ocean cooling)
c                  Typical range: -20 < lwflux < 170
c                  Southwest C-grid tracer point
c                  Input field
c
c     evap      :: Evaporation in m/s
c                  > 0 for increase in salt (ocean salinity)
c                  Typical range: 0 < evap < 2.5e-7
c                  Southwest C-grid tracer point
c                  Input, input/output, or output field
c
c     precip    :: Precipitation in m/s
c                  > 0 for decrease in salt (ocean salinity)
c                  Typical range: 0 < precip < 5e-7
c                  Southwest C-grid tracer point
c                  Input or input/output field
c
c     runoff    :: River and glacier runoff in m/s
c                  > 0 for decrease in salt (ocean salinity)
c                  Typical range: 0 < runoff < ????
c                  Southwest C-grid tracer point
c                  Input or input/output field
c                  !!! WATCH OUT: Default exf_inscal_runoff !!!
c                  !!! in exf_readparms.F is not 1.0        !!!
c
c     swdown    :: Downward shortwave radiation in W/m^2
c                  > 0 for increase in theta (ocean warming)
c                  Typical range: 0 < swdown < 450
c                  Southwest C-grid tracer point
c                  Input/output field
c
c     lwdown    :: Downward longwave radiation in W/m^2
c                  > 0 for increase in theta (ocean warming)
c                  Typical range: 50 < lwdown < 450
c                  Southwest C-grid tracer point
c                  Input/output field
c
c     apressure :: Atmospheric pressure field in N/m^2
c                  > 0 for ????
c                  Typical range: ???? < apressure < ????
c                  Southwest C-grid tracer point
c                  Input field
C
C
c     NOTES:
c     ======
c
c     Input and output units and sign conventions can be customized
c     using variables exf_inscal_* and exf_outscal_*, which are set
c     by exf_readparms.F
c
c     Output fields fu, fv, Qnet, Qsw, and EmPmR are
c     defined in FFIELDS.h
c
c     #ifndef SHORTWAVE_HEATING, hflux includes shortwave,
c     that is, hflux = latent + sensible + lwflux +swflux
c
c     If (EXFwindOnBgrid .EQ. .TRUE.), uwind and vwind are
c     defined on northeast B-grid U and V points, respectively.
c
c     Arrays *0 and *1 below are used for temporal interpolation.
\end{verbatim}
}

%----------------------------------------------------------------------

\subsection{Key subroutines
\label{sec:pkg:exf:subroutines}}

%----------------------------------------------------------------------

\subsection{EXF diagnostics
\label{sec:pkg:exf:diagnostics}}

Diagnostics output is available via the diagnostics package
(see Section \ref{sec:pkg:diagnostics}).
Available output fields are summarized in 
Table \ref{tab:pkg:exf:diagnostics}.

\begin{table}
\label{tab:pkg:exf:diagnostics}
\caption{~}
{\footnotesize
\begin{verbatim}
------------------------------------------------------
 <-Name->|Levs|grid|<--  Units   -->|<- Tile (max=80c)
------------------------------------------------------
 EXFlwdn |  1 | SM |W/m^2           |Downward longwave radiation, >0 increases theta
 EXFswdn |  1 | SM |W/m^2           |Downward shortwave radiation, >0 increases theta
 EXFqnet |  1 | SM |W/m^2           |Net upward heat flux (turb+rad), >0 decreases theta
 EXFtaux |  1 | SU |N/m^2           |zonal surface wind stress, >0 increases uVel
 EXFtauy |  1 | SV |N/m^2           |meridional surface wind stress, >0 increases vVel
 EXFuwind|  1 | SM |m/s             |zonal 10-m wind speed, >0 increases uVel
 EXFvwind|  1 | SM |m/s             |meridional 10-m wind speed, >0 increases uVel
 EXFatemp|  1 | SM |degK            |surface (2-m) air temperature
 EXFaqh  |  1 | SM |kg/kg           |surface (2-m) specific humidity
 EXFevap |  1 | SM |m/s             |evaporation, > 0 increases salinity
 EXFpreci|  1 | SM |m/s             |evaporation, > 0 decreases salinity
 EXFempmr|  1 | SM |m/s             |net upward freshwater flux, > 0 increases salinity
 EXFpress|  1 | SM |N/m^2           |atmospheric pressure field
\end{verbatim}
}
\end{table}

%----------------------------------------------------------------------

\subsection{Reference experiments}

%----------------------------------------------------------------------

\subsection{References}
