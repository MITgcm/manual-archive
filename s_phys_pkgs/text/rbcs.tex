\subsection {RBCS Package} 
\label{sec:pkg:rbcs}
\begin{rawhtml}
<!-- CMIREDIR:package_rbcs: -->
\end{rawhtml}

\subsubsection {Introduction}

A package which provides the flexibility
to relax fields (temperature, salinity, ptracers)
in any 3-D location:
so could be used as a sponge layer, or as a
"source" anywhere in the domain.

\noindent
For a tracer ($T$) at every grid point the tendency is modified so that:
\[
\frac{dT}{dt}=\frac{dT}{dt} - \frac{M_{rbc}}{\tau_T} (T-T_{rbc})
\]

\noindent
where $M_{rbc}$ is a 3-D mask (no time dependence) with
values between 0 and 1. Where $M_{rbc}$ is 1, relaxing timescale
is $1/\tau_T$. Where it is 0 there is no relaxing.
The value relaxed to is a 3-D (potentially varying in
time) field given by $T_{rbc}$. 

A seperate mask can be used for T,S and ptracers and
each of these
can be relaxed or not and can have its own timescale
$\tau_T$. These are set in data.rbcs (see below).


\subsubsection {Key subroutines and parameters}

The only compile-time parameter you are likely to have to change is in {RBCS.h},
the number of masks, PARAMETER(maskLEN = 3 ), see below.

The runtime parameters are set in {\it data.rbcs}:

\vspace{.5cm}
\noindent
Set in {RBCS\_PARM01}:\\
$\bullet$ {\bf rbcsForcingPeriod}: time interval between forcing fields
(in seconds), zero means constant-in-time forcing.\\
$\bullet$ {\bf rbcsForcingCycle}: repeat cycle of forcing fields (in seconds),
zero means non-cyclic forcing.\\
$\bullet$  {\bf rbcsForcingOffset}: time offset of forcing fields
(in seconds, default 0); this is relative to time averages starting at
$t=0$, i.e., the first forcing record/file is placed at
${\rm rbcsForcingOffset+rbcsForcingPeriod}/2$; see below for examples.\\
$\bullet$  {\bf rbcsSingleTimeFiles}: true or false (default false),
if true, forcing fields are given 1 file per rbcsForcingPeriod.\\
$\bullet$  {\bf deltaTrbcs}: time step used to compute the iteration numbers
for rbcsSingleTimeFiles=T.\\
$\bullet$  {\bf rbcsIter0}: shift in iteration numbers used to label files if
rbcsSingleTimeFiles=T (default 0, see below for examples).\\
$\bullet$  {\bf useRBCtemp}: true or false (default false)\\
$\bullet$  {\bf useRBCsalt}: true or false (default false)\\
$\bullet$  {\bf useRBCptracers}: true or false (default false), must be using
ptracers to set true\\
$\bullet$  {\bf tauRelaxT}: timescale in seconds of relaxing
in temperature ($\tau_T$ in equation above). 
Where mask is 1, relax rate will be
1/tauRelaxT. Default is 1.\\
$\bullet$  {\bf tauRelaxS}: same for salinity.\\
$\bullet$  {\bf relaxMaskFile(irbc)}: filename of 3-D file
with mask ($M_{rbc}$ in equation above. 
Need a file for each irbc. 1=temperature,
2=salinity, 3=ptracer01, 4=ptracer02 etc. If the mask numbers
end (see maskLEN) are less than the number tracers, then
relaxMaskFile(maskLEN) is used for all remaining ptracers.\\
$\bullet$  {\bf relaxTFile}: name of file where temperatures
that need to be relaxed to ($T_{rbc}$ in equation above)
are stored.  The file must contain 3-D records to match the model domain.
If rbcsSingleTimeFiles=F, it must have one record for each forcing period.
If T, there must be a separate file for each period and a 10-digit iteration
number is appended to the file name (see Table~\ref{tab:pkg:rbcs:timing} 
and examples below).\\
$\bullet$  {\bf relaxSFile}: same for salinity.\\

\vspace{.5cm}
\noindent
Set in {RBCS\_PARM02} for each of the ptracers (iTrc):\\
$\bullet$ {\bf useRBCptrnum(iTrc)}: true or false (default
is false).\\
$\bullet$ {\bf tauRelaxPTR(iTrc)}: relax timescale.\\
$\bullet$ {\bf relaxPtracerFile(iTrc)}: file with relax
fields.\\


\subsubsection{Timing of relaxation forcing fields}

For constant-in-time relaxation, set rbcsForcingPeriod=0.
For time-varying relaxation, Table~\ref{tab:pkg:rbcs:timing} illustrates the
relation between model time and forcing fields (either records in
one big file or, for rbcsSingleTimeFiles=T, individual files labeled with an
iteration number).  With rbcsSingleTimeFiles=T, this is the same as in the
offline package, except that the forcing offset is in seconds.
\newcommand{\dtr}{\Delta t_{\text{rbcs}}}%
\begin{table}
    \centering
    \begin{tabular}{|l|l|l|c|}
      \hline 
      &
      \multicolumn{2}{|c|}{rbcsSingleTimeFiles = T} &
      F \\
      &
      \textbf{$c=0$} &
      \textbf{$c\ne0$} &
      \textbf{$c\ne0$}
      \\ \hline
      \textbf{model time} &
      \textbf{file number} &
      \textbf{file number} &
      \textbf{record} \\
      \hline \hline
        $t_0 -     p/2$ & $i_0$            & $i_0 +   c/\dtr$ & $c/p$ \\ \hline
        $t_0 +     p/2$ & $i_0 +   p/\dtr$ & $i_0 +   p/\dtr$ & $1$ \\ \hline
        $t_0 + p + p/2$ & $i_0 + 2 p/\dtr$ & $i_0 + 2 p/\dtr$ & $2$ \\ \hline
        \dots & \dots & \dots & \dots \\ \hline
        $t_0 + c - p/2$ & \dots            & $i_0 +   c/\dtr$ & $c/p$ \\ \hline
        \dots & \dots & \dots & \dots \\ \hline
    \end{tabular}
    \qquad
    \begin{tabular}{c@{${}={}$}l}
        \multicolumn{2}{l}{} \\[4ex]
        \multicolumn{2}{l}{where} \\[1ex]
        $p$    & rbcsForcingPeriod \\
        $c$    & rbcsForcingCycle \\
        $t_0$  & rbcsForcingOffset \\
        $i_0$  & rbcsIter0 \\
        $\dtr$ & deltaTrbcs \\
    \end{tabular}\\[3ex]
    \caption{Timing of relaxation forcing fields.}
    \label{tab:pkg:rbcs:timing}
\end{table}


\subsubsection{Example 1: forcing with time averages starting at $t=0$}

\paragraph{Cyclic data in a single file.}  Set rbcsSingleTimeFiles=F and
rbcsForcingOffset=0, and the model will start by interpolating the last and first
records of rbcs data, placed at $-p/2$ and $p/2$, resp., as appropriate for fields
averaged over the time intervals $[-p, 0]$ and $[0, p]$.

\paragraph{Non-cyclic data, multiple files.}  Set rbcsForcingCycle=0 and
rbcsSingleTimeFiles=T.  With rbcsForcingOffset=0, rbcsIter0=0 and
deltaTrbcs=rbcsForcingPeriod, the model would then start by interpolating data from
files relax*File.0000000000.data and relax*File.0000000001.data, \dots,
again placed at $-p/2$ and $p/2$.


\subsubsection{Example 2: forcing with snapshots starting at $t=0$}

\paragraph{Cyclic data in a single file.}  Set rbcsSingleTimeFiles=F and
rbcsForcingOffset=$-p/2$, and the model will start forcing with the first
record at $t=0$.

\paragraph{Non-cyclic data, multiple files.}  Set rbcsForcingCycle=0 and
rbcsSingleTimeFiles=T.  In this case, it is more natural to set
rbcsForcingOffset=$+p/2$.
With rbcsIter0=0 and deltaTrbcs=rbcsForcingPeriod, the model would then start
with data from files relax*File.0000000000.data at $t=0$.
It would then proceed to interpolate between this file and files
relax*File.0000000001.data at $t={}$rbcsForcingPeriod.


\subsubsection{Do's and Don'ts}

\subsubsection{Reference Material}

\subsubsection{Experiments and tutorials that use rbcs}
\label{sec:pkg:rbcs:experiments}

In the directory \code{verifcation}, the following experiments use
\code{rbcs}:
\begin{itemize}
\item \code{exp4}: box with 4 open boundaries, simulating flow over a
  Gaussian bump based on \citet{adcroft:97}.
\end{itemize}



%%% \end{itemize}

