\subsection {RBCS Package} 
\label{sec:pkg:rbcs}
\begin{rawhtml}
<!-- CMIREDIR:package_rbcs: -->
\end{rawhtml}

\subsubsection {Introduction}

A package which provides the flexibility
to relax fields (temperature, salinity, ptracers)
in any 3-D location:
so could be used as a sponge layer, or as a
"source" anywhere in the domain.

\noindent
For a tracer ($T$) at every grid point the tendency is modified so that:
\[
\frac{dT}{dt}=\frac{dT}{dt} - \frac{M_{rbc}}{\tau_T} (T-T_{rbc})
\]

\noindent
where $M_{rbc}$ is a 3-D mask (no time dependence) with
values between 0 and 1. Where $M_{rbc}$ is 1, relaxing timescale
is $1/\tau_T$. Where it is 0 there is no relaxing.
The value relaxed to is a 3-D (potentially varying in
time) field given by $T_{rbc}$. 

A seperate mask can be used for T,S and ptracers and
each of these
can be relaxed or not and can have its own timescale
$\tau_T$. These are set in data.rbcs (see below).

\subsubsection {Key subroutines and parameters}

The only change need in the code might be in {RBCS.H}, for
PARAMETER(maskLEN = 3 ), if you need more than 3
masks (see below).

\vspace{.5cm}

\noindent
There are runtime parameters
set in {\it data.rbcs}:\\
These runtime options include\\
Set in {RBCS\_PARM01}:\\
$\bullet$ Parameters to set the timing for periodic fields to
relax to are to
be loaded are: {\it rbcs\_ForcingPeriod}, {\it rbcs\_ForcingCycle}.
The former is how often to load, the latter is how often to cycle
through those fields (eg. period couple be monthly and cycle one year).
rbcs\_ForcingCycle=0 meaning no periodic forcing, and the relax field
is only read in at the beginning of the run and kept constant
the rest of the run. Default is 0.
\\
$\bullet$  {\bf rbcsIniter}: if you want to offset rbcs forcing
timing. Default is nIter0.\\
$\bullet$  {\bf useRBCtemp}: true or false (default false)\\
$\bullet$  {\bf useRBCsalt}: true or false (default false)\\
$\bullet$  {\bf useRBCptracers}: true or false (default false), must be using
ptracers to set true\\
$\bullet$  {\bf tauRelaxT}: timescale in seconds of relaxing
in temperature ($\tau_T$ in equation above). 
Where mask is 1, relax rate will be
1/tauRelaxT. Default is 1.
$\bullet$  {\bf tauRelaxS}: same for salinity.
$\bullet$  {\bf relaxMaskFile(irbc)}: filename of 3-D file
with mask ($M_{rbc}$ in equation above. 
Need a file for each irbc. 1=temperature,
2=salinity, 3=ptracer01, 4=ptracer02 etc. If the mask numbers
end (see maskLEN) are less than the number tracers, then
relaxMaskFile(maskLEN) is used for all remaining ptracers.\\
$\bullet$  {\bf relaxTFile}: name of file where temperatures
that need to be realxed to ($T_{rbc}$ in equation above)
are stored. Need 3-D fields to
match model domain, and as many entries as given by
rbcsForcingPeriod and rbcsForcingCycle.\\
$\bullet$  {\bf relaxSFile}: same for salinity.\\

\vspace{.5cm}
\noindent
Set in {RBCS\_PARM02} for each of the ptracers (iTrc):\\
$\bullet$ {\bf useRBCptrnum(iTrc)}: true or false (default
is false).\\
$\bullet$ {\bf tauRelaxPTR(iTrc)}: relax timescale.\\
$\bullet$ {\bf relaxPtracerFile(iTrc)}: file with relax
fields.\\


\subsubsection{Do's and Don'ts}

\subsubsection{Reference Material}

\subsubsection{Experiments and tutorials that use rbcs}
\label{sec:pkg:rbcs:experiments}



%%% \end{itemize}

