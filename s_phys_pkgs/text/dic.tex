\subsection {DIC Package} 
\label{sec:pkg:dic}
\begin{rawhtml}
<!-- CMIREDIR:package_dic: -->
\end{rawhtml}

\subsubsection {Introduction}
This is one of the biogeochemical packages handled from the
pkg gchem. The main purpose of this package is to consider
the cycling of carbon in the ocean. It also looks at the
cycling of phosphorous and potentially oxygen and iron. 
There are four standard tracers
$DIC$, $ALK$, $PO4$, $DOP$ and also possibly $O2$ and $Fe$. 
The air-sea exchange
of CO$_2$ and O$_2$ are handled as in the OCMIP experiments
(reference). The export of biological matter is computed
as a function of available light and PO$_4$ (and Fe). This export is
remineralized at depth according to a Martin curve (again,
this is the same as in the OCMIP experiments). There is
also a representation of the carbonate flux handled as in
the OCMIP experiments. The air-sea exchange on CO$_2$
is affected by temperature, salinity and the pH of the
surface waters. The pH is determined following the
method of Follows et al. 
For more details of the equations see section 
\ref{sect:eg-biogeochem_tutorial}.

\subsubsection {Key subroutines and parameters}

\noindent
{{\bf INITIALIZATION}} \\
{\it DIC\_ABIOTIC.h} contains the common block for the 
parameters and fields needed to calculate the air-sea
flux of $CO_2$ and $O_2$. The fixed parameters are set in
{\it dic\_abiotic\_param} which is called from {\it gchem\_init\_fixed.F}.
The parameters needed for the biotic part of the calculations
are initialized in {\it dic\_biotic\_param} and are stored
in {\it DIC\_BIOTIC.h}. The first guess of pH is calculated
in {\it dic\_surfforcing\_init.F}.

\vspace{.5cm}

\noindent
{{\bf LOADING FIELDS}}\\
The air-sea exchange of $CO_2$ and $O_2$ need wind, atmospheric
pressure (although the current version has this hardwired to 1),
and sea-ice coverage. The calculation of pH needs silica fields.
These fields are read in in {\it dic\_fields\_load.F}. These
fields are initialized to zero in {\it dic\_ini\_forcing.F}.
The fields for interpolating are in common block in 
{\it DIC\_LOAD.h}.

\vspace{.5cm}

\noindent
{{\bf FORCING}}\\
The tracers are advected-diffused in {\it ptracers\_integrate.F}.
The updated tracers are passed to {\it dic\_biotic\_forcing.F}
where the effects of the air-sea exchange and biological
activity and remineralization are calculated and the tracers
are updated for a second time. Below we discuss the 
subroutines called from {\it dic\_biotic\_forcing.F}.


Air-sea exchange of $CO_2$ is calculated in {\it dic\_surfforcing}.
Air-Sea Exchange of $CO_2$ depends on T,S and pH. The determination
of pH is done in {\it carbon\_chem.F}. There are three subroutines
in this file: {\it carbon\_coeffs} which determines the coefficients
for the carbon chemistry equations; {\it calc\_pco2} which calculates
the pH using a Newton-Raphson method; and {\it calc\_pco2\_approx}
which uses the much more efficient method of Follows et al.
The latter is hard-wired into this package, the former is kept
here for completeness.

Biological productivity is determined following
Dutkiewicz et al. (2005) and is calculated in {\it bio\_export.F}
The light in each latitude band is calculate in {\it insol.F},
unless using one of the flags listed below.
The formation of hard tissue (carbonate) is linked to
the biological productivity and has an effect on the
alkalinity - the flux of carbonate is calculated in
{\it car\_flux.F}, unless using the flag listed below
for the Friis et al (2006) scheme. The flux of phosphate to depth where
it instantly remineralized is calculated in {\it phos\_flux.F}.

The dilution or concentration of  carbon and alkalinity by
the addition  or subtraction of freshwater is important to 
their surface patterns. These "virtual" fluxes can be calculated 
by the model in several ways.
The older scheme is done following OCMIP protocols (see
more in Dutkiewicz et al 2005), in the subroutines
{\it dic\_surfforcing.F} and {\it alk\_surfforcing.F}.
To use this you need to set in GCHEM\_OPTIONS.h:\\
\#define ALLOW\_OLD\_VIRTUALFLUX\\
But this can also be done by the ptracers pkg if this
is undefined. You will then need to set the concentration
of the tracer in rainwater and potentially a reference
tracer value in data.ptracer
(PTRACERS\_EvPrRn, and PTRACERS\_ref respectively).

Oxygen air-sea exchange is calculated in {\it o2\_surfforcing.F}.

Iron chemistry (the amount of free iron) is taken care of in
{\it fe\_chem.F}.
 
\vspace{.5cm}

\noindent
{{\bf DIAGNOSTICS}}\\
Averages of air-sea exchanges, biological productivity,
carbonate activity and pH are calculated. These are
initialized to zero in {\it dic\_biotic\_init} and
are stored in common block in {\it DIC\_BIOTIC.h}.

\vspace{.5cm}

\noindent
{{\bf COMPILE TIME FLAGS}}\\
These are set in GCHEM\_OPTIONS.h: \\

DIC\_BIOTIC: needs to be set for dic to work properly
(should be fixed sometime).\\
ALLOW\_O2: include the tracer oxygen.\\
ALLOW\_FE: include the tracer iron. Note you will need an
iron dust file set in data.gchem in this case.\\
MINFE: limit the iron, assuming precpitation of any
excess free iron.\\
CAR\_DISS: use the calcium carbonate scheme of Friis et al 2006.\\
ALLOW\_OLD\_VIRTUALFLUX: use the old OCMIP style virtual flux
for alklinity adn carbon (rather than doing it through pkg/ptracers).
\\
READ\_PAR: read the light (photosynthetically available
radiation) from a file set in data.gchem.\\
USE\_QSW: use the numbers from QSW to be the PAR. Note that
a file for Qsw must be supplied in data, or Qsw must be
supplied by an atmospheric model.\\
If the above two flags are not set, the model calculates
PAR in insol.F as a function of latitude and year day.\\
USE\_QSW\_UNDERICE: if using a sea ice model, or if the
Qsw variable has the seaice fraction already taken into
account, this flag must be set.\\
\\
AD\_SAFE: will use a tanh function instead of a
max function - this is better if using the adjoint\\
DIC\_NO\_NEG: will include some failsafes in case any
of the variables become negative. (This is advicable).
ALLOW\_DIC\_COST: was used for calculating cost function 
(but hasn't been updated or maintained, so not sure if it works still)



\subsubsection{Do's and Don'ts}

This package must be run with both ptracers and gchem enabled.
It is set up for at least 4 tracers, but there is the provision for
oxygen and iron. Note the flags above.

\subsubsection{Reference Material}

Dutkiewicz. S., A. Sokolov, J. Scott and P. Stone, 2005:
A Three-Dimensional Ocean-Seaice-Carbon Cycle Model and its Coupling
to a Two-Dimensional Atmospheric Model: Uses in Climate Change Studies,
Report 122, Joint Program of the Science and Policy of Global Change,
M.I.T., Cambridge, MA.
\\

Follows, M., T. Ito and S. Dutkiewicz, 2006:
A Compact and Accurate Carbonate Chemistry Solver for Ocean
Biogeochemistry Models. {\it Ocean Modeling}, 12, 290-301.
\\

Friis, K.,  R. Najjar, M.J. Follows, and S. Dutkiewicz, 2006:
Possible overestimation of shallow-depth calcium carbonate
dissolution in the ocean,
{\it Global Biogeochemical Cycles}, 20, GB4019, doi:10.1029/2006GB002727.
\\


\subsubsection{Experiments and tutorials that use dic}
\label{sec:pkg:dic:experiments}

\begin{itemize}
\item{Global Ocean tutorial, in tutorial\_global\_oce\_biogeo verification directory, 
described in section \ref{sect:eg-biogeochem_tutorial} }
\end{itemize}

