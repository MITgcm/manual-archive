\subsection {DIC Package} 
\label{sec:pkg:dic}
\begin{rawhtml}
<!-- CMIREDIR:package_dic: -->
\end{rawhtml}

\subsubsection {Introduction}
This is one of the biogeochemical packages handled from the
pkg gchem. The main purpose of this package is to consider
the cycling of carbon in the ocean. It also looks at the
cycling of phosphorous and oxygen. There are five tracers
$DIC$, $ALK$, $PO4$, $DOP$ and $O2$. The air-sea exchange
of CO$_2$ and O$_2$ are handled as in the OCMIP experiments
(reference). The export of biological matter is computed
as a function of available light and PO$_4$. This export is
remineralized at depth according to a Martin curve (again,
this is the same as in the OCMIP experiments). There is
also a representation of the carbonate flux handled as in
the OCMIP experiments. The air-sea exchange on CO$_2$
is affected by temperature, salinity and the pH of the
surface waters. The pH is determined following the
method of Follows et al. 

\subsubsection {Key subroutines and parameters}

\noindent
{{\bf INITIALIZATION}} \\
{\it DIC\_ABIOTIC.h} contains the common block for the 
parameters and fields needed to calculate the air-sea
flux of $CO_2$ and $O_2$. The fixed parameters are set in
{\it dic\_abiotic\_param} which is called from {\it gchem\_init\_fixed.F}.
The parameters needed for the biotic part of the calculations
are initialized in {\it dic\_biotic\_param} and are stored
in {\it DIC\_BIOTIC.h}. The first guess of pH is calculated
in {\it dic\_surfforcing\_init.F}.

\vspace{.5cm}

\noindent
{{\bf LOADING FIELDS}}\\
The air-sea exchange of $CO_2$ and $O_2$ need wind, atmospheric
pressure (although the current version has this hardwired to 1),
and sea-ice coverage. The calculation of pH needs silica fields.
These fields are read in in {\it dic\_fields\_load.F}. These
fields are initialized to zero in {\it dic\_ini\_forcing.F}.
The fields for interpolating are in common block in 
{\it DIC\_LOAD.h}.

\vspace{.5cm}

\noindent
{{\bf FORCING}}\\
The tracers are advected-diffused in {\it ptracers\_integrate.F}.
The updated tracers are passed to {\it dic\_biotic\_forcing.F}
where the effects of the air-sea exchange and biological
activity and remineralization are calculated and the tracers
are updated for a second time. Below we discuss the 
subroutines called from {\it dic\_biotic\_forcing.F}.


Air-sea exchange of $CO_2$ is calculated in {\it dic\_surfforcing}.
Air-Sea Exchange of $CO_2$ depends on T,S and pH. The determination
of pH is done in {\it carbon\_chem.F}. There are three subroutines
in this file: {\it carbon\_coeffs} which determines the coefficients
for the carbon chemistry equations; {\it calc\_pco2} which calculates
the pH using a Newton-Raphson method; and {\it calc\_pco2\_approx}
which uses the much more efficient method of Follows et al.
The latter is hard-wired into this package, the former is kept
here for completeness.

Biological productivity is determined following
McKinely et al. and is calculated in {\it bio\_export.F}
The light in each latitude band is calculate in {\it insol.F}.
The formation of hard tissue (carbonate) is linked to
the biological productivity and has an effect on the
alkalinity - the flux of carbonate is calculated in
{\it car\_flux.F}. The flux of phosphate to depth where
it instantly remineralized is calculated in {\it phos\_flux.F}.

Alkalinity tendency comes from changes to the salinity from
addition/subtraction of freshwater in the surface. This
is handled in {\it alk\_surfforcing.F}.

Oxygen air-sea exchange is calculated in {\it o2\_surfforcing.F}.
 
\vspace{.5cm}

\noindent
{{\bf DIAGNOSTICS}}\\
Averages of air-sea exchanges, biological productivity,
carbonate activity and pH are calculated. These are
initialized to zero in {\it dic\_biotic\_init} and
are stored in common block in {\it DIC\_BIOTIC.h}.

\subsubsection{Do's and Don'ts}

This package must be run with both ptracers and gchem enabled.
It is set up for 5 tracers, but there is the provision of
a 6th tracer (iron) that is not discussed here.

\subsubsection{Reference Material}

\subsubsection{Experiments and tutorials that use dic}
\label{sec:pkg:dic:experiments}

\begin{itemize}
\item{Global Ocean tutorial, in tutorial\_global\_oce\_biogeo verification directory, 
described in section \ref{sect:eg-biogeochem_tutorial} }
\end{itemize}

