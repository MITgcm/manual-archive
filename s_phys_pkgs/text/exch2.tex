% $Header: /u/gcmpack/manual/s_phys_pkgs/text/exch2.tex,v 1.5 2004/01/29 21:03:53 afe Exp $
% $Name:  $

%%  * Introduction
%%    o what it does, citations (refs go into mitgcm_manual.bib, 
%%      preferably in alphabetic order)
%%    o Equations 
%%  * Key subroutines and parameters
%%  * Reference material (auto generated from Protex and structured comments)
%%    o automatically inserted at \section{Reference} 


\section{exch2: Extended Cubed Sphere Exchange}
\label{sec:exch2}


\subsection{Introduction}

The exch2 package is an extension to the original cubed sphere exchanges
to allow more flexible domain decomposition and parallelization.  Cube faces
(subdomains) may be divided into whatever number of tiles that divide evenly
into the grid point dimensions of the subdomain.  Furthermore, the individual
tiles may be run on separate processors in different combinations,
and whether exchanges between particular tiles occur between different
processors is determined at runtime.

The exchange parameters are declared in {\em W2\_EXCH2\_TOPOLOGY.h} and 
assigned in {\em w2\_e2setup.F}, both in the 
{\em pkg/exch2} directory.  The validity of the cube topology depends
on the {\em SIZE.h} file as detailed below.  Both files are generated by 
Matlab scripts and
should not be edited.  The default files provided in the release set up
a cube sphere arrangement of six tiles, one per subdomain, each with 32x32 grid
points, running on a single processor.  

\subsection{Key Variables}

The descriptions of the variables are divided up into scalars,
one-dimensional arrays indexed to the tile number, and two-dimensional
arrays indexed to tile number and neighboring tile.  This division
actually reflects  the functionality of these variables, not just the
whim of some FORTRAN enthusiast.

\subsubsection{Scalars}

The number of tiles in a particular topology is set with the parameter
{\em NTILES}, and the maximum number of neighbors of any tiles by 
{\em MAX\_NEIGHBOURS}.  These parameters are used for defining the size of
the various one and two dimensional arrays that store tile parameters
indexed to the tile number.

The scalar parameters {\em exch2\_domain\_nxt} and 
{\em exch2\_domain\_nyt} express the number of tiles in the x and y global
indices.  For example, the default setup of six tiles has 
{\em exch2\_domain\_nxt=6} and {\em exch2\_domain\_nyt=1}.  A topology of
twenty-four square (in gridpoints) tiles, four (2x2) per subdomain, will
have {\em exch2\_domain\_nxt=12} and {\em exch2\_domain\_nyt=2}.  Note 
that these parameters express the tile layout to allow global data files that
are tile-layout-neutral and have no bearing on the internal storage of the
arrays.  The tiles are internally stored in a range from {\em 1,bi} (in the
x axis) and y-axis variable {\em bj} is generally ignored within the package.

\subsubsection{One-Dimensional Arrays}

The following arrays are of size {\em NTILES}, are indexed to the tile number, 
and the indices are omitted in their descriptions.

The arrays {\em exch2\_tnx} and {\em exch2\_tny} 
express the x and y dimensions of each tile.  At present for each tile
{\em exch2\_tnx = sNx} 
and {\em exch2\_tny = sNy}, as assigned in {\em SIZE.h}.  Future releases of 
MITgcm are to allow varying tile sizes.

The location of the tiles' Cartesian origin within a subdomain are determined 
by the arrays {\em exch2\_tbasex} and {\em exch2\_tbasey}.  These variables
are used to relate the location of the edges of the tiles to each other.  As 
an example, in the default six-tile topology (the degenerate case) 
each index in these arrays are 
set to 0.  The twenty-four, 32x32 cube face case discussed above will have
values of 0 or 16, depending on the quadrant the tile falls within the 
subdomain.  {\em exch2\_myFace} contains the number of the 
cubeface/subdomain of each tile, numbered 1-6 in the case of the standard
cube topology.  

The arrays {\em exch2\_txglobalo} and {\em exch2\_txglobalo} are similar to
{\em exch2\_tbasex} and {\em exch2\_tbasey}, but locate the tiles within
the global address space, similar to that used by global files.  

The arrays {\em exch2\_isWedge}, {\em exch2\_isEedge}, {\em exch2\_isSedge}, 
and {\em exch2\_isNedge} are set to 1 if the indexed tile lies on the edge
of a subdomain, 0 if not.  The values are used within the topology generator
to determine the orientation of neighboring tiles and to indicate whether 
a tile lies on the corner of a subdomain.  The latter case indicates 
special exchange and numerical handling for the singularities at the eight 
corners of the cube.  {\em exch2\_isNedge} contains a count of how many
neighboring tiles each tile has, and is used for setting bounds for looping
over neighboring tiles.  {\em exch2\_tProc} holds the process rank of each tile,
and is used in interprocess communication.

\subsubsection{Two-Dimensional Arrays}

The following arrays are all of size {\em MAX\_NEIGHBOURS}x{\em NTILES} and
describe the orientations between the the tiles.

The array {\em exch2\_neighbourId(a,T)} holds the tile number $T_{n}$ for each tile 
{\em T}'s neighbor tile {\em a}, and {\em exch2\_opposingSend\_record(a,T)} holds 
the index c in {\em exch2\_neighbourId(b,$T_{n}$)} that holds the tile number T.
In other words,

\begin{verbatim}   
exch2_neighbourId( exch2_opposingSend_record(a,T), exch2_neighbourId(a,T) ) = T
\end{verbatim}

{\em exch2\_neighbourId(exch2\_opposingSend\_record(a,T),exch2\_neighbourId(a,T))=T}.
This is to provide a backreference from the neighbor tiles.


//

\begin{verbatim}


C      exch2_neighbourId :: Tile number for each neighbour entry.        
C      exch2_opposingSend_record :: Record for entry in target tile send 
C                                :: list that has this tile and face     
C                                :: as its target.                       
C      exch2_pi          :: X index row of target to source permutation 
C                        :: matrix for each neighbour entry.            
C      exch2_pj          :: Y index row of target to source permutation 
C                        :: matrix for each neighbour entry.            
C      exch2_oi          :: X index element of target to source 
C                        :: offset vector for cell-centered quantities  
C                        :: of each neighbor entry.                     
C      exch2_oj          :: Y index element of target to source 
C                        :: offset vector for cell-centered quantities  
C                        :: of each neighbor entry.                     
C      exch2_oi_f        :: X index element of target to source 
C                        :: offset vector for face quantities           
C                        :: of each neighbor entry.                     
C      exch2_oj_f        :: Y index element of target to source 
C                        :: offset vector for face quantities           
C                        :: of each neighbor entry.                     
\end{verbatim}




\subsection{Key Routines}



\subsection{References}
