% $Header: /u/gcmpack/manual/s_phys_pkgs/text/seaice.tex,v 1.22 2015/01/21 18:18:22 mlosch Exp $
% $Name:  $

%%EH3  Copied from "MITgcm/pkg/seaice/seaice_description.tex"
%%EH3  which was written by Dimitris M.


\subsection{SEAICE Package}
\label{sec:pkg:seaice}
\begin{rawhtml}
<!-- CMIREDIR:package_seaice: -->
\end{rawhtml}

Authors: Martin Losch, Dimitris Menemenlis, An Nguyen, Jean-Michel Campin,
Patrick Heimbach, Chris Hill and Jinlun Zhang

%----------------------------------------------------------------------
\subsubsection{Introduction
\label{sec:pkg:seaice:intro}}


Package ``seaice'' provides a dynamic and thermodynamic interactive
sea-ice model. 

CPP options enable or disable different aspects of the package
(Section \ref{sec:pkg:seaice:config}).
Run-Time options, flags, filenames and field-related dates/times are
set in \code{data.seaice}
(Section \ref{sec:pkg:seaice:runtime}).
A description of key subroutines is given in Section
\ref{sec:pkg:seaice:subroutines}.
Input fields, units and sign conventions are summarized in
Section \ref{sec:pkg:seaice:fields_units}, and available diagnostics
output is listed in Section \ref{sec:pkg:seaice:diagnostics}.

%----------------------------------------------------------------------

\subsubsection{SEAICE configuration, compiling \& running}

\paragraph{Compile-time options
\label{sec:pkg:seaice:config}}
~

As with all MITgcm packages, SEAICE can be turned on or off at compile time
%
\begin{itemize}
%
\item
using the \code{packages.conf} file by adding \code{seaice} to it,
%
\item
or using \code{genmake2} adding
\code{-enable=seaice} or \code{-disable=seaice} switches
%
\item
\textit{required packages and CPP options}: \\
SEAICE requires the external forcing package \code{exf} to be enabled;
no additional CPP options are required.
%
\end{itemize}
(see Section \ref{sec:buildingCode}).

Parts of the SEAICE code can be enabled or disabled at compile time
via CPP preprocessor flags. These options are set in 
\code{SEAICE\_OPTIONS.h}.
Table \ref{tab:pkg:seaice:cpp} summarizes the most important ones.

\begin{table}[!ht]
\centering
  \label{tab:pkg:seaice:cpp}
  {\footnotesize
    \begin{tabular}{|l|p{10cm}|}
      \hline 
      \textbf{CPP option}  &  \textbf{Description}  \\
      \hline \hline
        \code{SEAICE\_DEBUG} & 
          Enhance STDOUT for debugging \\
        \code{SEAICE\_ALLOW\_DYNAMICS} & 
          sea-ice dynamics code \\
        \code{SEAICE\_CGRID} & 
          LSR solver on C-grid (rather than original B-grid) \\
        \code{SEAICE\_ALLOW\_EVP} & 
          enable use of EVP rheology solver \\
        \code{SEAICE\_ALLOW\_JFNK} & 
          enable use of JFNK rheology solver \\
        \code{SEAICE\_EXTERNAL\_FLUXES} & 
          use EXF-computed fluxes as starting point \\
        \code{SEAICE\_ZETA\_SMOOTHREG} & 
          use differentialable regularization for viscosities \\
        \code{SEAICE\_VARIABLE\_FREEZING\_POINT} & 
          enable linear dependence of the freezing point on salinity
          (by default undefined)\\
        \code{ALLOW\_SEAICE\_FLOODING} & 
          enable snow to ice conversion for submerged sea-ice \\
        \code{SEAICE\_VARIABLE\_SALINITY} & 
          enable sea-ice with variable salinity (by default undefined) \\
        \code{SEAICE\_SITRACER} & 
          enable sea-ice tracer package (by default undefined) \\
        \code{SEAICE\_BICE\_STRESS} &
          B-grid only for backward compatiblity: turn on ice-stress on
          ocean\\
        \code{EXPLICIT\_SSH\_SLOPE} &
          B-grid only for backward compatiblity: use ETAN for tilt
          computations rather than geostrophic velocities \\
      \hline
    \end{tabular}
  }
  \caption{Some of the most relevant CPP preprocessor flags in the
    \code{seaice}-package.} 
\end{table}

%----------------------------------------------------------------------

\subsubsection{Run-time parameters
\label{sec:pkg:seaice:runtime}}

Run-time parameters (see Table~\ref{tab:pkg:seaice:runtimeparms}) are set
in files \code{data.pkg} (read in \code{packages\_readparms.F}), and
\code{data.seaice} (read in \code{seaice\_readparms.F}).

\paragraph{Enabling the package}
~ \\
%
A package is switched on/off at run-time by setting
(e.g. for SEAICE) \code{useSEAICE = .TRUE.} in \code{data.pkg}.

\paragraph{General flags and parameters}
~ \\
%
Table~\ref{tab:pkg:seaice:runtimeparms} lists most run-time parameters.
\newpage

\begin{table}
{\small
%\hspace*{-1.5in}
\begin{tabular}{|lllc|}
\hline
  & & & \\
  \textbf{Name}  &  \textbf{Default value}  
    &  \textbf{Description}   &  \textbf{Reference}  \\
  & & & \\
\hline \hline
   SEAICEwriteState    &                     T
    &   write sea ice state to file 
    &  %---ref---
    \\
   SEAICEuseDYNAMICS   &                     T
    &   use dynamics 
    &  %---ref---
    \\
   LAD                 &                         2
    &   time stepping scheme 
    &  %---ref---
    \\
   IMAX\_TICE           &                        10
    &   iterations for ice heat budget 
    &  %---ref---
    \\
   SEAICE\_deltaTtherm  &                   3.60000E+03
    &   thermodynamic timestep 
    &  %---ref---
    \\
   SEAICE\_deltaTdyn    &                   3.60000E+03
    &   dynamic timestep 
    &  %---ref---
    \\
   SEAICE\_dumpFreq     &                   0.00000E+00
    &   dump frequency 
    &  %---ref---
    \\
   SEAICE\_taveFreq     &                   3.60000E+04
    &   time-averaging frequency 
    &  %---ref---
    \\
   SEAICE\_dump\_mdsio   &                     T
    &   write snap-shot   using MDSIO 
    &  %---ref---
    \\
   SEAICE\_tave\_mdsio   &                     T
    &   write TimeAverage using MDSIO 
    &  %---ref---
    \\
   SEAICE\_dump\_mnc     &                     F
    &   write snap-shot   using MNC 
    &  %---ref---
    \\
   SEAICE\_tave\_mnc     &                     F
    &   write TimeAverage using MNC 
    &  %---ref---
    \\
   SEAICE\_initialHEFF  &                   1.00000E+00
    &   initial sea-ice thickness 
    &  %---ref---
    \\
   SEAICE\_drag         &                   2.00000E-03
    &   air-ice drag coefficient 
    &  %---ref---
    \\
   OCEAN\_drag          &                   1.00000E-03
    &   air-ocean drag coefficient 
    &  %---ref---
    \\
   SEAICE\_waterDrag    &                   5.50000E+00
    &   water-ice drag 
    &  %---ref---
    \\
   SEAICE\_dryIceAlb    &                   7.50000E-01
    &   winter albedo 
    &  %---ref---
    \\
   SEAICE\_wetIceAlb    &                   6.60000E-01
    &   summer albedo 
    &  %---ref---
    \\
   SEAICE\_drySnowAlb   &                   8.40000E-01
    &   dry snow albedo 
    &  %---ref---
    \\
   SEAICE\_wetSnowAlb   &                   7.00000E-01
    &   wet snow albedo 
    &  %---ref---
    \\
   SEAICE\_waterAlbedo  &                   1.00000E-01
    &   water albedo 
    &  %---ref---
    \\
   SEAICE\_strength     &                   2.75000E+04
    &   sea-ice strength Pstar 
    &  %---ref---
    \\
   SEAICE\_sensHeat     &                   2.28400E+00
    &   sensible heat transfer 
    &  %---ref---
    \\
   SEAICE\_latentWater  &                   5.68750E+03
    &   latent heat transfer for water 
    &  %---ref---
    \\
   SEAICE\_latentIce    &                   6.44740E+03
    &   latent heat transfer for ice 
    &  %---ref---
    \\
   SEAICE\_iceConduct   &                   2.16560E+00
    &   sea-ice conductivity 
    &  %---ref---
    \\
   SEAICE\_snowConduct  &                   3.10000E-01
    &   snow conductivity 
    &  %---ref---
    \\
   SEAICE\_emissivity   &                   5.50000E-08
    &   Stefan-Boltzman 
    &  %---ref---
    \\
   SEAICE\_snowThick    &                   1.50000E-01
    &   cutoff snow thickness 
    &  %---ref---
    \\
   SEAICE\_shortwave    &                   3.00000E-01
    &   penetration shortwave radiation 
    &  %---ref---
    \\
   SEAICE\_freeze       &                  -1.96000E+00
    &   freezing temp. of sea water 
    &  %---ref---
    \\
   LSR\_ERROR           &                   1.00000E-12
    &   sets accuracy of LSR solver 
    &  %---ref---
    \\
   DIFF1               &                   4.00000E-03
    &   parameter used in advect.F 
    &  %---ref---
    \\
   A22                 &                   1.50000E-01
    &   parameter used in growth.F 
    &  %---ref---
    \\
   HO                  &                   5.00000E-01
    &   demarcation ice thickness 
    &  %---ref---
    \\
   MAX\_HEFF            &                   1.00000E+01
    &   maximum ice thickness 
    &  %---ref---
    \\
   MIN\_ATEMP           &                  -5.00000E+01
    &   minimum air temperature 
    &  %---ref---
    \\
   MIN\_LWDOWN          &                   6.00000E+01
    &   minimum downward longwave 
    &  %---ref---
    \\
   MAX\_TICE            &                   3.00000E+01
    &   maximum ice temperature 
    &  %---ref---
    \\
   MIN\_TICE            &                  -5.00000E+01
    &   minimum ice temperature 
    &  %---ref---
    \\
   SEAICE\_EPS          &                   1.00000E-10
    &   reduce derivative singularities 
    &  %---ref---
    \\
\hline
\end{tabular}
}
\end{table}



\paragraph{Input fields and units\label{sec:pkg:seaice:fields_units}}
\begin{description}
\item[\code{HeffFile}:] Initial sea ice thickness averaged over grid cell
  in meters; initializes variable \code{HEFF};
\item[\code{AreaFile}:] Initial fractional sea ice cover, range $[0,1]$;
  initializes variable \code{AREA};
\item[\code{HsnowFile}:] Initial snow thickness on sea ice averaged
  over grid cell in meters; initializes variable \code{HSNOW};
\item[\code{HsaltFile}:] Initial salinity of sea ice averaged over grid
  cell in g/m$^2$; initializes variable \code{HSALT};
\end{description}

%----------------------------------------------------------------------
\subsubsection{Description
\label{sec:pkg:seaice:descr}}

[TO BE CONTINUED/MODIFIED]

% Sea-ice model thermodynamics are based on Hibler
% \cite{hib80}, that is, a 2-category model that simulates ice thickness
% and concentration.  Snow is simulated as per Zhang et al.
% \cite{zha98a}.  Although recent years have seen an increased use of
% multi-category thickness distribution sea-ice models for climate
% studies, the Hibler 2-category ice model is still the most widely used
% model and has resulted in realistic simulation of sea-ice variability
% on regional and global scales.  Being less complicated, compared to
% multi-category models, the 2-category model permits easier application
% of adjoint model optimization methods.

% Note, however, that the Hibler 2-category model and its variants use a
% so-called zero-layer thermodynamic model to estimate ice growth and
% decay.  The zero-layer thermodynamic model assumes that ice does not
% store heat and, therefore, tends to exaggerate the seasonal
% variability in ice thickness.  This exaggeration can be significantly
% reduced by using Semtner's \cite{sem76} three-layer thermodynamic
% model that permits heat storage in ice.  Recently, the three-layer
% thermodynamic model has been reformulated by Winton \cite{win00}.  The
% reformulation improves model physics by representing the brine content
% of the upper ice with a variable heat capacity.  It also improves
% model numerics and consumes less computer time and memory.  The Winton
% sea-ice thermodynamics have been ported to the MIT GCM; they currently
% reside under pkg/thsice. The package pkg/thsice is fully
% compatible with pkg/seaice and with pkg/exf. When turned on togeter
% with pkg/seaice, the zero-layer thermodynamics are replaced by the by
% Winton thermodynamics

The MITgcm sea ice model (MITgcm/sim) is based on a variant of the
viscous-plastic (VP) dynamic-thermodynamic sea ice model \citep{zhang97}
first introduced by \citet{hib79, hib80}. In order to adapt this model
to the requirements of coupled ice-ocean state estimation, many
important aspects of the original code have been modified and
improved \citep{losch10:_mitsim}:
\begin{itemize}
\item the code has been rewritten for an Arakawa C-grid, both B- and
  C-grid variants are available; the C-grid code allows for no-slip
  and free-slip lateral boundary conditions;
\item three different solution methods for solving the nonlinear
  momentum equations have been adopted: LSOR \citep{zhang97}, EVP
  \citep{hun97}, JFNK \citep{lemieux10,losch14:_jfnk};
\item ice-ocean stress can be formulated as in \citet{hibler87} or as in
  \citet{cam08}; 
\item ice variables are advected by sophisticated, conservative
  advection schemes with flux limiting;
\item growth and melt parameterizations have been refined and extended
  in order to allow for more stable automatic differentiation of the code.
\end{itemize}
The sea ice model is tightly coupled to the ocean compontent of the
MITgcm.  Heat, fresh water fluxes and surface stresses are computed
from the atmospheric state and -- by default -- modified by the ice
model at every time step.

The ice dynamics models that are most widely used for large-scale
climate studies are the viscous-plastic (VP) model \citep{hib79}, the
cavitating fluid (CF) model \citep{fla92}, and the
elastic-viscous-plastic (EVP) model \citep{hun97}.  Compared to the VP
model, the CF model does not allow ice shear in calculating ice
motion, stress, and deformation.  EVP models approximate VP by adding
an elastic term to the equations for easier adaptation to parallel
computers.  Because of its higher accuracy in plastic solution and
relatively simpler formulation, compared to the EVP model, we decided
to use the VP model as the default dynamic component of our ice
model. To do this we extended the line successive over relaxation
(LSOR) method of \citet{zhang97} for use in a parallel
configuration. An EVP model and a free-drift implemtation can be
selected with runtime flags.

\paragraph{Compatibility with ice-thermodynamics package \code{thsice}\label{sec:pkg:seaice:thsice}}~\\
%
Note, that by default the \code{seaice}-package includes the orginial
so-called zero-layer thermodynamics following \citet{hib80} with a
snow cover as in \citet{zha98a}. The zero-layer thermodynamic model
assumes that ice does not store heat and, therefore, tends to
exaggerate the seasonal variability in ice thickness.  This
exaggeration can be significantly reduced by using
\citeauthor{sem76}'s~[\citeyear{sem76}] three-layer thermodynamic
model that permits heat storage in ice. Recently, the three-layer thermodynamic model has been reformulated by
\citet{win00}.  The reformulation improves model physics by
representing the brine content of the upper ice with a variable heat
capacity.  It also improves model numerics and consumes less computer
time and memory. 

The Winton sea-ice thermodynamics have been ported to the MIT GCM;
they currently reside under \code{pkg/thsice}.  The package
\code{thsice} is described in section~\ref{sec:pkg:thsice}; it is
fully compatible with the packages \code{seaice} and \code{exf}. When
turned on together with \code{seaice}, the zero-layer thermodynamics
are replaced by the Winton thermodynamics. In order to use the
\code{seaice}-package with the thermodynamics of \code{thsice},
compile both packages and turn both package on in \code{data.pkg}; see
an example in \code{global\_ocean.cs32x15/input.icedyn}. Note, that
once \code{thsice} is turned on, the variables and diagnostics
associated to the default thermodynamics are meaningless, and the
diagnostics of \code{thsice} have to be used instead.

\paragraph{Surface forcing\label{sec:pkg:seaice:surfaceforcing}}~\\
%
The sea ice model requires the following input fields: 10-m winds, 2-m
air temperature and specific humidity, downward longwave and shortwave
radiations, precipitation, evaporation, and river and glacier runoff.
The sea ice model also requires surface temperature from the ocean
model and the top level horizontal velocity.  Output fields are
surface wind stress, evaporation minus precipitation minus runoff, net
surface heat flux, and net shortwave flux.  The sea-ice model is
global: in ice-free regions bulk formulae are used to estimate oceanic
forcing from the atmospheric fields.

\paragraph{Dynamics\label{sec:pkg:seaice:dynamics}}~\\
%
\newcommand{\vek}[1]{\ensuremath{\vec{\mathbf{#1}}}}
\newcommand{\vtau}{\vek{\mathbf{\tau}}}
The momentum equation of the sea-ice model is
\begin{equation}
  \label{eq:momseaice}
  m \frac{D\vek{u}}{Dt} = -mf\vek{k}\times\vek{u} + \vtau_{air} +
  \vtau_{ocean} - m \nabla{\phi(0)} + \vek{F},
\end{equation}
where $m=m_{i}+m_{s}$ is the ice and snow mass per unit area;
$\vek{u}=u\vek{i}+v\vek{j}$ is the ice velocity vector;
$\vek{i}$, $\vek{j}$, and $\vek{k}$ are unit vectors in the $x$, $y$, and $z$
directions, respectively;
$f$ is the Coriolis parameter;
$\vtau_{air}$ and $\vtau_{ocean}$ are the wind-ice and ocean-ice stresses,
respectively;
$g$ is the gravity accelation;
$\nabla\phi(0)$ is the gradient (or tilt) of the sea surface height;
$\phi(0) = g\eta + p_{a}/\rho_{0} + mg/\rho_{0}$ is the sea surface
height potential in response to ocean dynamics ($g\eta$), to
atmospheric pressure loading ($p_{a}/\rho_{0}$, where $\rho_{0}$ is a
reference density) and a term due to snow and ice loading \citep{cam08};
and $\vek{F}=\nabla\cdot\sigma$ is the divergence of the internal ice
stress tensor $\sigma_{ij}$. %
Advection of sea-ice momentum is neglected. The wind and ice-ocean stress
terms are given by
\begin{align*}
  \vtau_{air}   = & \rho_{air}  C_{air}   |\vek{U}_{air}  -\vek{u}|
                   R_{air}  (\vek{U}_{air}  -\vek{u}), \\ 
  \vtau_{ocean} = & \rho_{ocean}C_{ocean} |\vek{U}_{ocean}-\vek{u}| 
                   R_{ocean}(\vek{U}_{ocean}-\vek{u}),
\end{align*}
where $\vek{U}_{air/ocean}$ are the surface winds of the atmosphere
and surface currents of the ocean, respectively; $C_{air/ocean}$ are
air and ocean drag coefficients; $\rho_{air/ocean}$ are reference
densities; and $R_{air/ocean}$ are rotation matrices that act on the
wind/current vectors.

\paragraph{Viscous-Plastic (VP) Rheology\label{sec:pkg:seaice:VPrheology}}~\\
%
For an isotropic system the stress tensor $\sigma_{ij}$ ($i,j=1,2$) can
be related to the ice strain rate and strength by a nonlinear
viscous-plastic (VP) constitutive law \citep{hib79, zhang97}:
\begin{equation}
  \label{eq:vpequation}
  \sigma_{ij}=2\eta(\dot{\epsilon}_{ij},P)\dot{\epsilon}_{ij} 
  + \left[\zeta(\dot{\epsilon}_{ij},P) -
    \eta(\dot{\epsilon}_{ij},P)\right]\dot{\epsilon}_{kk}\delta_{ij}  
  - \frac{P}{2}\delta_{ij}.
\end{equation}
The ice strain rate is given by
\begin{equation*}
  \dot{\epsilon}_{ij} = \frac{1}{2}\left( 
    \frac{\partial{u_{i}}}{\partial{x_{j}}} +
    \frac{\partial{u_{j}}}{\partial{x_{i}}}\right).
\end{equation*}
The maximum ice pressure $P_{\max}$, a measure of ice strength, depends on
both thickness $h$ and compactness (concentration) $c$:
\begin{equation}
  P_{\max} = P^{*}c\,h\,\exp\{-C^{*}\cdot(1-c)\},
\label{eq:icestrength}
\end{equation}
with the constants $P^{*}$ (run-time parameter \code{SEAICE\_strength}) and
$C^{*}=20$. The nonlinear bulk and shear 
viscosities $\eta$ and $\zeta$ are functions of ice strain rate
invariants and ice strength such that the principal components of the
stress lie on an elliptical yield curve with the ratio of major to
minor axis $e$ equal to $2$; they are given by:
\begin{align*}
  \zeta =& \min\left(\frac{P_{\max}}{2\max(\Delta,\Delta_{\min})},
   \zeta_{\max}\right) \\
  \eta =& \frac{\zeta}{e^2} \\
  \intertext{with the abbreviation}
  \Delta = & \left[
    \left(\dot{\epsilon}_{11}^2+\dot{\epsilon}_{22}^2\right)
    (1+e^{-2}) +  4e^{-2}\dot{\epsilon}_{12}^2 + 
    2\dot{\epsilon}_{11}\dot{\epsilon}_{22} (1-e^{-2})
  \right]^{\frac{1}{2}}.
\end{align*}
The bulk viscosities are bounded above by imposing both a minimum
$\Delta_{\min}$ (for numerical reasons, run-time parameter
\code{SEAICE\_EPS} with a default value of
$10^{-10}\text{\,s}^{-1}$) and a maximum $\zeta_{\max} =
P_{\max}/\Delta^*$, where
$\Delta^*=(5\times10^{12}/2\times10^4)\text{\,s}^{-1}$. (There is also
the option of bounding $\zeta$ from below by setting run-time
parameter \code{SEAICE\_zetaMin} $>0$, but this is generally not
recommended). For stress tensor computation the replacement pressure $P
= 2\,\Delta\zeta$ \citep{hibler95} is used so that the stress state
always lies on the elliptic yield curve by definition.

Defining the CPP-flag \code{SEAICE\_ZETA\_SMOOTHREG} in
\code{SEAICE\_OPTIONS.h} before compiling replaces the method for
bounding $\zeta$ by a smooth (differentiable) expression:
\begin{equation}
  \label{eq:zetaregsmooth}
  \begin{split}
  \zeta &= \zeta_{\max}\tanh\left(\frac{P}{2\,\min(\Delta,\Delta_{\min})
      \,\zeta_{\max}}\right)\\
  &= \frac{P}{2\Delta^*}
  \tanh\left(\frac{\Delta^*}{\min(\Delta,\Delta_{\min})}\right) 
  \end{split}
\end{equation}
where $\Delta_{\min}=10^{-20}\text{\,s}^{-1}$ is chosen to avoid divisions
by zero.

\paragraph{LSR and  JFNK solver \label{sec:pkg:seaice:LSRJFNK}}~\\
%
% By default, the VP-model is integrated by a Pickwith the
% semi-implicit line successive over relaxation (LSOR)-solver of
% \citet{zhang97}, which allows for long time steps that, in our case,
% are limited by the explicit treatment of the Coriolis term. The
% explicit treatment of the Coriolis term does not represent a severe
% limitation because it restricts the time step to approximately the
% same length as in the ocean model where the Coriolis term is also
% treated explicitly.

\newcommand{\mat}[1]{\ensuremath{\mathbf{#1}}} 
%
In the matrix notation, the discretized momentum equations can be
written as
\begin{equation}
  \label{eq:matrixmom}
  \mat{A}(\vek{x})\,\vek{x} = \vek{b}(\vek{x}).
\end{equation}
The solution vector $\vek{x}$ consists of the two velocity components
$u$ and $v$ that contain the velocity variables at all grid points and
at one time level. The standard (and default) method for solving
Eq.\,(\ref{eq:matrixmom}) in the sea ice component of the
\mbox{MITgcm}, as in many sea ice models, is an iterative Picard
solver: in the $k$-th iteration a linearized form
$\mat{A}(\vek{x}^{k-1})\,\vek{x}^{k} = \vek{b}(\vek{x}^{k-1})$ is
solved (in the case of the MITgcm it is a Line Successive (over)
Relaxation (LSR) algorithm \citep{zhang97}).  Picard solvers converge
slowly, but generally the iteration is terminated after only a few
non-linear steps \citep{zhang97, lemieux09} and the calculation
continues with the next time level. This method is the default method
in the MITgcm. The number of non-linear iteration steps or pseudo-time
steps can be controlled by the runtime parameter
\code{NPSEUDOTIMESTEPS} (default is 2).

In order to overcome the poor convergence of the Picard-solver,
\citet{lemieux10} introduced a Jacobian-free Newton-Krylov solver for
the sea ice momentum equations. This solver is also implemented in the
MITgcm \citep{losch14:_jfnk}. The Newton method transforms minimizing
the residual $\vek{F}(\vek{x}) = \mat{A}(\vek{x})\,\vek{x} -
\vek{b}(\vek{x})$ to finding the roots of a multivariate Taylor
expansion of the residual \vek{F} around the previous ($k-1$) estimate
$\vek{x}^{k-1}$:
\begin{equation}
  \label{eq:jfnktaylor}
  \vek{F}(\vek{x}^{k-1}+\delta\vek{x}^{k}) = 
  \vek{F}(\vek{x}^{k-1}) + \vek{F}'(\vek{x}^{k-1})\,\delta\vek{x}^{k}
\end{equation}
with the Jacobian $\mat{J}\equiv\vek{F}'$. The root
$\vek{F}(\vek{x}^{k-1}+\delta\vek{x}^{k})=0$ is found by solving
\begin{equation}
  \label{eq:jfnklin}
  \mat{J}(\vek{x}^{k-1})\,\delta\vek{x}^{k} = -\vek{F}(\vek{x}^{k-1})
\end{equation}
for $\delta\vek{x}^{k}$. The next ($k$-th) estimate is given by
$\vek{x}^{k}=\vek{x}^{k-1}+a\,\delta\vek{x}^{k}$. In order to avoid
overshoots the factor $a$ is iteratively reduced in a line search
($a=1, \frac{1}{2}, \frac{1}{4}, \frac{1}{8}, \ldots$) until
$\|\vek{F}(\vek{x}^k)\| < \|\vek{F}(\vek{x}^{k-1})\|$, where
$\|\cdot\|=\int\cdot\,dx^2$ is the $L_2$-norm. In practice, the line
search is stopped at $a=\frac{1}{8}$. The line search starts after
$\code{SEAICE\_JFNK\_lsIter}$ non-linear Newton iterations (off by
default).


Forming the Jacobian $\mat{J}$ explicitly is often avoided as ``too
error prone and time consuming'' \citep{knoll04:_jfnk}. Instead,
Krylov methods only require the action of \mat{J} on an arbitrary
vector \vek{w} and hence allow a matrix free algorithm for solving
Eq.\,(\ref{eq:jfnklin}) \citep{knoll04:_jfnk}. The action of \mat{J}
can be approximated by a first-order Taylor series expansion:
\begin{equation}
  \label{eq:jfnkjacvecfd}
  \mat{J}(\vek{x}^{k-1})\,\vek{w} \approx
  \frac{\vek{F}(\vek{x}^{k-1}+\epsilon\vek{w}) - \vek{F}(\vek{x}^{k-1})}
  {\epsilon} 
\end{equation}
or computed exactly with the help of automatic differentiation (AD)
tools. \code{SEAICE\_JFNKepsilon} sets the step size
$\epsilon$. 

We use the Flexible Generalized Minimum RESidual method
\citep[FGMRES,][]{saad93:_fgmres} with right-hand side preconditioning
to solve Eq.\,(\ref{eq:jfnklin}) iteratively starting from a first
guess of $\delta\vek{x}^{k}_{0} = 0$. For the preconditioning matrix
\mat{P} we choose a simplified form of the system matrix
$\mat{A}(\vek{x}^{k-1})$ \citep{lemieux10} where $\vek{x}^{k-1}$ is
the estimate of the previous Newton step $k-1$. The transformed
equation\,(\ref{eq:jfnklin}) becomes
\begin{equation}
  \label{eq:jfnklinpc}
  \mat{J}(\vek{x}^{k-1})\,\mat{P}^{-1}\delta\vek{z} =
  -\vek{F}(\vek{x}^{k-1}), 
  \quad\text{with}\quad \delta\vek{z}=\mat{P}\delta\vek{x}^{k}.
\end{equation}
The Krylov method iteratively improves the approximate solution
to~(\ref{eq:jfnklinpc}) in subspace ($\vek{r}_0$,
$\mat{J}\mat{P}^{-1}\vek{r}_0$, $(\mat{J}\mat{P}^{-1})^2\vek{r}_0$,
\ldots, $(\mat{J}\mat{P}^{-1})^m\vek{r}_0$) with increasing $m$;
$\vek{r}_0 = -\vek{F}(\vek{x}^{k-1})
-\mat{J}(\vek{x}^{k-1})\,\delta\vek{x}^{k}_{0}$
%-\vek{F}(\vek{x}^{k-1})
%-\mat{J}(\vek{x}^{k-1})\,\mat{P}^{-1}\delta\vek{z}$ 
is the initial residual of
(\ref{eq:jfnklin}); $\vek{r}_0=-\vek{F}(\vek{x}^{k-1})$ with the first
guess $\delta\vek{x}^{k}_{0}=0$. We allow a Krylov-subspace of
dimension~$m=50$ and we do not use restarts. The preconditioning operation
involves applying $\mat{P}^{-1}$ to the basis vectors $\vek{v}_0,
\vek{v}_1, \vek{v}_2, \ldots, \vek{v}_m$ of the Krylov subspace. This
operation is approximated by solving the linear system
$\mat{P}\,\vek{w}=\vek{v}_i$. Because $\mat{P} \approx
\mat{A}(\vek{x}^{k-1})$, we can use the LSR-algorithm \citep{zhang97}
already implemented in the Picard solver. Each preconditioning
operation uses a fixed number of 10~LSR-iterations avoiding any
termination criterion. More details and results can be found in
\citet{lemieux10, losch14:_jfnk}.

To use the JFNK-solver set \code{SEAICEuseJFNK = .TRUE.} in the
namelist file \code{data.seaice}; \code{SEAICE\_ALLOW\_JFNK} needs to
be defined in \code{SEAICE\_OPTIONS.h} and we recommend using a smooth
regularization of $\zeta$ by defining \code{SEAICE\_ZETA\_SMOOTHREG}
(see above) for better convergence.  The non-linear Newton iteration
is terminated when the $L_2$-norm of the residual is reduced by
$\gamma_{\mathrm{nl}}$ (runtime parameter \code{JFNKgamma\_nonlin =
  1.e-4} will already lead to expensive simulations) with respect to
the initial norm: $\|\vek{F}(\vek{x}^k)\| <
\gamma_{\mathrm{nl}}\|\vek{F}(\vek{x}^0)\|$.  Within a non-linear
iteration, the linear FGMRES solver is terminated when the residual is
smaller than $\gamma_k\|\vek{F}(\vek{x}^{k-1})\|$ where $\gamma_k$ is
determined by
\begin{equation}
  \label{eq:jfnkgammalin}
  \gamma_k = 
  \begin{cases} 
    \gamma_0 &\text{for $\|\vek{F}(\vek{x}^{k-1})\| \geq r$},  \\ 
    \max\left(\gamma_{\min},
    \frac{\|\vek{F}(\vek{x}^{k-1})\|}{\|\vek{F}(\vek{x}^{k-2})\|}\right)  
%    \phi\left(\frac{\|\vek{F}(\vek{x}^{k-1})\|}{\|\vek{F}(\vek{x}^{k-2})\|}\right)^\alpha\right)  
    &\text{for $\|\vek{F}(\vek{x}^{k-1})\| < r$,}
  \end{cases}
\end{equation}
so that the linear tolerance parameter $\gamma_k$ decreases with the
non-linear Newton step as the non-linear solution is approached. This
inexact Newton method is generally more robust and computationally
more efficient than exact methods \citep[e.g.,][]{knoll04:_jfnk}.
% \footnote{The general idea behind
%   inexact Newton methods is this: The Krylov solver is ``only''
%   used to find an intermediate solution of the linear
%   equation\,(\ref{eq:jfnklin}) that is used to improve the approximation of
%   the actual equation\,(\ref{eq:matrixmom}). With the choice of a
%   relatively weak lower limit for FGMRES convergence 
%   $\gamma_{\min}$ we make sure that the time spent in the FGMRES
%   solver is reduced at the cost of more Newton iterations. Newton
%   iterations are cheaper than Krylov iterations so that this choice
%   improves the overall efficiency.} 
Typical parameter choices are
$\gamma_0=\code{JFNKgamma\_lin\_max}=0.99$,
$\gamma_{\min}=\code{JFNKgamma\_lin\_min}=0.1$, and $r = 
\code{JFNKres\_tFac}\times\|\vek{F}(\vek{x}^{0})\|$ with
$\code{JFNKres\_tFac} = \frac{1}{2}$. We recommend a maximum number of
non-linear iterations $\code{SEAICEnewtonIterMax} = 100$ and a maximum
number of Krylov iterations $\code{SEAICEkrylovIterMax} = 50$, because
the Krylov subspace has a fixed dimension of 50.

\paragraph{Elastic-Viscous-Plastic (EVP) Dynamics\label{sec:pkg:seaice:EVPdynamics}}~\\
%
\citet{hun97}'s introduced an elastic contribution to the strain
rate in order to regularize Eq.~\ref{eq:vpequation} in such a way that
the resulting elastic-viscous-plastic (EVP) and VP models are
identical at steady state,
\begin{equation}
  \label{eq:evpequation}
  \frac{1}{E}\frac{\partial\sigma_{ij}}{\partial{t}} +
  \frac{1}{2\eta}\sigma_{ij} 
  + \frac{\eta - \zeta}{4\zeta\eta}\sigma_{kk}\delta_{ij}  
  + \frac{P}{4\zeta}\delta_{ij}
  = \dot{\epsilon}_{ij}. 
\end{equation}
%In the EVP model, equations for the components of the stress tensor
%$\sigma_{ij}$ are solved explicitly. Both model formulations will be
%used and compared the present sea-ice model study.
The EVP-model uses an explicit time stepping scheme with a short
timestep. According to the recommendation of \citet{hun97}, the
EVP-model should be stepped forward in time 120 times
($\code{SEAICE\_deltaTevp} = \code{SEAICIE\_deltaTdyn}/120$) within
the physical ocean model time step (although this parameter is under
debate), to allow for elastic waves to disappear.  Because the scheme
does not require a matrix inversion it is fast in spite of the small
internal timestep and simple to implement on parallel computers
\citep{hun97}. For completeness, we repeat the equations for the
components of the stress tensor $\sigma_{1} =
\sigma_{11}+\sigma_{22}$, $\sigma_{2}= \sigma_{11}-\sigma_{22}$, and
$\sigma_{12}$. Introducing the divergence $D_D =
\dot{\epsilon}_{11}+\dot{\epsilon}_{22}$, and the horizontal tension
and shearing strain rates, $D_T =
\dot{\epsilon}_{11}-\dot{\epsilon}_{22}$ and $D_S =
2\dot{\epsilon}_{12}$, respectively, and using the above
abbreviations, the equations~\ref{eq:evpequation} can be written as:
\begin{align}
  \label{eq:evpstresstensor1}
  \frac{\partial\sigma_{1}}{\partial{t}} + \frac{\sigma_{1}}{2T} +
  \frac{P}{2T} &= \frac{P}{2T\Delta} D_D \\
  \label{eq:evpstresstensor2}
  \frac{\partial\sigma_{2}}{\partial{t}} + \frac{\sigma_{2} e^{2}}{2T}
  &= \frac{P}{2T\Delta} D_T \\
  \label{eq:evpstresstensor12}
  \frac{\partial\sigma_{12}}{\partial{t}} + \frac{\sigma_{12} e^{2}}{2T}
  &= \frac{P}{4T\Delta} D_S 
\end{align}
Here, the elastic parameter $E$ is redefined in terms of a damping
timescale $T$ for elastic waves \[E=\frac{\zeta}{T}.\]
$T=E_{0}\Delta{t}$ with the tunable parameter $E_0<1$ and the external
(long) timestep $\Delta{t}$.  $E_{0} = \frac{1}{3}$ is the default
value in the code and close to what \citet{hun97} and
\citet{hun01} recommend.

To use the EVP solver, make sure that both \code{SEAICE\_CGRID} and
\code{SEAICE\_ALLOW\_EVP} are defined in \code{SEAICE\_OPTIONS.h}
(default). The solver is turned on by setting the sub-cycling time
step \code{SEAICE\_deltaTevp} to a value larger than zero. The
choice of this time step is under debate. \citet{hun97} recommend
order(120) time steps for the EVP solver within one model time step
$\Delta{t}$ (\code{deltaTmom}). One can also choose order(120) time
steps within the forcing time scale, but then we recommend adjusting
the damping time scale $T$ accordingly, by setting either
\code{SEAICE\_elasticParm} ($E_{0}$), so that
$E_{0}\Delta{t}=\mbox{forcing time scale}$, or directly
\code{SEAICE\_evpTauRelax} ($T$) to the forcing time scale.

\paragraph{More stable variant of Elastic-Viscous-Plastic Dynamics:  EVP*\label{sec:pkg:seaice:EVPstar}}~\\ 
%
The genuine EVP schemes appears to give noisy solutions \citep{hun01,
  lemieux12, bouillon13}. This has lead to a modified EVP or EVP*
\citep{lemieux12, bouillon13, kimmritz15}; here, refer to these
variants by EVP*. The main idea is to modify the ``natural''
time-discretization of the momentum equations:
\begin{equation}
  \label{eq:evpstar}
  m\frac{D\vec{u}}{Dt} \approx m\frac{u^{p+1}-u^{n}}{\Delta{t}}
  + \beta^{*}\frac{u^{p+1}-u^{p}}{\Delta{t}_{\mathrm{EVP}}}
\end{equation}
where $n$ is the previous time step index, and $p$ is the previous
sub-cycling index. The term allows the definition of a residual
$|u^{p+1}-u^{p}|$ that, as $u^{p+1} \rightarrow u^{n+1}$, converges to
$0$ and a re-interpretation of EVP as a pure iterative solver where
the sub-cycling has lost all time-relation \citep{bouillon13,
  kimmritz15}. Using the terminology of \citet{kimmritz15}, the
evolution equations of stress $\sigma_{ij}$ and momentum $\vec{u}$ can
be written as:
\begin{align}
  \label{eq:evpstarsigma}
  \sigma_{ij}^{p+1}&=\sigma_{ij}^p+\frac{1}{\alpha}
  \Big(\sigma_{ij}(\vec{u}^p)-\sigma_{ij}^p\Big),
  \phantom{\int}\\
  \label{eq:evpstarmom}
  \vec{u}^{p+1}&=\vec{u}^p+\frac{1}{\beta}
  \Big(\frac{\Delta t}{m}\nabla \cdot{\bf \sigma}^{p+1}+
  \frac{\Delta t}{m}\vec{R}^{p+1/2}+\vec{u}_n-\vec{u}^p\Big).
\end{align}
$\vec{R}$ contains all terms in the momentum equations except for the
rheology terms and the time derivative, $\alpha$ and $\beta$ are free
parameters (\code{SEAICE\_evpAlpha}, \code{SEAICE\_evpBeta}) that
replace the time stepping parameters \code{SEAICE\_deltaTevp}
($\Delta{T}_{\mathrm{EVP}}$), \code{SEAICE\_elasticParm} ($E_{0}$), or
\code{SEAICE\_evpTauRelax} ($T$). $\alpha$ and $\beta$ determine the
speed of convergence and the stability. Usually, it makes sense to use
$\alpha = \beta$, and \code{SEAICEnEVPstarSteps} $>> \alpha = \beta$
\citep{kimmritz15}.

In order to use EVP* in the MITgcm, set \code{SEAICEuseEVPstar =
  .TRUE.,} in \code{data.seaice}. \code{SEAICEuseEVPrev =.TRUE.,} uses
the actual form of equations (\ref{eq:evpstarsigma}) and
(\ref{eq:evpstarmom}) with fewer implicit terms and the factor of
$e^{2}$ dropped in the stress equations (\ref{eq:evpstresstensor2})
and (\ref{eq:evpstresstensor12}).  This turns out to improve
convergence \citep{bouillon13}.

Note, that for historical reasons, \code{SEAICE\_deltaTevp} needs to
be set to some value in order to use also EVP*. Also note, that
probably because of the C-grid staggering of velocities and stresses,
EVP* does not converge as successfully as in \citet{kimmritz15}.


\paragraph{Truncated ellipse method (TEM) for yield curve \label{sec:pkg:seaice:TEM}}~\\
%
In the so-called truncated ellipse method the shear viscosity $\eta$
is capped to suppress any tensile stress \citep{hibler97, geiger98}:
\begin{equation}
  \label{eq:etatem}
  \eta = \min\left(\frac{\zeta}{e^2},
  \frac{\frac{P}{2}-\zeta(\dot{\epsilon}_{11}+\dot{\epsilon}_{22})}
  {\sqrt{(\dot{\epsilon}_{11}+\dot{\epsilon}_{22})^2
      +4\dot{\epsilon}_{12}^2}}\right).
\end{equation}
To enable this method, set \code{\#define SEAICE\_ALLOW\_TEM} in
\code{SEAICE\_OPTIONS.h} and turn it on with
\code{SEAICEuseTEM} in \code{data.seaice}. 

\paragraph{Ice-Ocean stress \label{sec:pkg:seaice:iceoceanstress}}~\\
%
Moving sea ice exerts a stress on the ocean which is the opposite of
the stress $\vtau_{ocean}$ in Eq.~\ref{eq:momseaice}. This stess is
applied directly to the surface layer of the ocean model. An
alternative ocean stress formulation is given by \citet{hibler87}.
Rather than applying $\vtau_{ocean}$ directly, the stress is derived
from integrating over the ice thickness to the bottom of the oceanic
surface layer. In the resulting equation for the \emph{combined}
ocean-ice momentum, the interfacial stress cancels and the total
stress appears as the sum of windstress and divergence of internal ice
stresses: $\delta(z) (\vtau_{air} + \vek{F})/\rho_0$, \citep[see also
Eq.\,2 of][]{hibler87}. The disadvantage of this formulation is that
now the velocity in the surface layer of the ocean that is used to
advect tracers, is really an average over the ocean surface
velocity and the ice velocity leading to an inconsistency as the ice
temperature and salinity are different from the oceanic variables.
To turn on the stress formulation of \citet{hibler87}, set
\code{useHB87StressCoupling=.TRUE.} in \code{data.seaice}.


% Our discretization differs from \citet{zhang97, zhang03} in the
% underlying grid, namely the Arakawa C-grid, but is otherwise
% straightforward. The EVP model, in particular, is discretized
% naturally on the C-grid with $\sigma_{1}$ and $\sigma_{2}$ on the
% center points and $\sigma_{12}$ on the corner (or vorticity) points of
% the grid. With this choice all derivatives are discretized as central
% differences and averaging is only involved in computing $\Delta$ and
% $P$ at vorticity points.

\paragraph{Finite-volume discretization of the stress tensor
  divergence\label{sec:pkg:seaice:discretization}}~\\
%
On an Arakawa C~grid, ice thickness and concentration and thus ice
strength $P$ and bulk and shear viscosities $\zeta$ and $\eta$ are
naturally defined a C-points in the center of the grid
cell. Discretization requires only averaging of $\zeta$ and $\eta$ to
vorticity or Z-points (or $\zeta$-points, but here we use Z in order
avoid confusion with the bulk viscosity) at the bottom left corner of
the cell to give $\overline{\zeta}^{Z}$ and $\overline{\eta}^{Z}$. In
the following, the superscripts indicate location at Z or C points,
distance across the cell (F), along the cell edge (G), between
$u$-points (U), $v$-points (V), and C-points (C). The control volumes
of the $u$- and $v$-equations in the grid cell at indices $(i,j)$ are
$A_{i,j}^{w}$ and $A_{i,j}^{s}$, respectively. With these definitions
(which follow the model code documentation except that $\zeta$-points
have been renamed to Z-points), the strain rates are discretized as:
\begin{align}
  \dot{\epsilon}_{11} &= \partial_{1}{u}_{1} + k_{2}u_{2} \\ \notag
  => (\epsilon_{11})_{i,j}^C &= \frac{u_{i+1,j}-u_{i,j}}{\Delta{x}_{i,j}^{F}} 
   + k_{2,i,j}^{C}\frac{v_{i,j+1}+v_{i,j}}{2} \\ 
  \dot{\epsilon}_{22} &= \partial_{2}{u}_{2} + k_{1}u_{1} \\\notag
  => (\epsilon_{22})_{i,j}^C &= \frac{v_{i,j+1}-v_{i,j}}{\Delta{y}_{i,j}^{F}} 
   + k_{1,i,j}^{C}\frac{u_{i+1,j}+u_{i,j}}{2} \\ 
   \dot{\epsilon}_{12} = \dot{\epsilon}_{21} &= \frac{1}{2}\biggl(
   \partial_{1}{u}_{2} + \partial_{2}{u}_{1} - k_{1}u_{2} - k_{2}u_{1}
   \biggr) \\ \notag
  => (\epsilon_{12})_{i,j}^Z &= \frac{1}{2}
  \biggl( \frac{v_{i,j}-v_{i-1,j}}{\Delta{x}_{i,j}^V} 
   + \frac{u_{i,j}-u_{i,j-1}}{\Delta{y}_{i,j}^U} \\\notag
  &\phantom{=\frac{1}{2}\biggl(}
   - k_{1,i,j}^{Z}\frac{v_{i,j}+v_{i-1,j}}{2}
   - k_{2,i,j}^{Z}\frac{u_{i,j}+u_{i,j-1}}{2}
   \biggr),
\end{align}
so that the diagonal terms of the strain rate tensor are naturally
defined at C-points and the symmetric off-diagonal term at
Z-points. No-slip boundary conditions ($u_{i,j-1}+u_{i,j}=0$ and
$v_{i-1,j}+v_{i,j}=0$ across boundaries) are implemented via
``ghost-points''; for free slip boundary conditions
$(\epsilon_{12})^Z=0$ on boundaries.

For a spherical polar grid, the coefficients of the metric terms are
$k_{1}=0$ and $k_{2}=-\tan\phi/a$, with the spherical radius $a$ and
the latitude $\phi$; $\Delta{x}_1 = \Delta{x} = a\cos\phi
\Delta\lambda$, and $\Delta{x}_2 = \Delta{y}=a\Delta\phi$. For a
general orthogonal curvilinear grid, $k_{1}$ and
$k_{2}$ can be approximated by finite differences of the cell widths:
\begin{align}
  k_{1,i,j}^{C} &= \frac{1}{\Delta{y}_{i,j}^{F}}
  \frac{\Delta{y}_{i+1,j}^{G}-\Delta{y}_{i,j}^{G}}{\Delta{x}_{i,j}^{F}} \\
  k_{2,i,j}^{C} &= \frac{1}{\Delta{x}_{i,j}^{F}}
  \frac{\Delta{x}_{i,j+1}^{G}-\Delta{x}_{i,j}^{G}}{\Delta{y}_{i,j}^{F}} \\
  k_{1,i,j}^{Z} &= \frac{1}{\Delta{y}_{i,j}^{U}}
  \frac{\Delta{y}_{i,j}^{C}-\Delta{y}_{i-1,j}^{C}}{\Delta{x}_{i,j}^{V}} \\
  k_{2,i,j}^{Z} &= \frac{1}{\Delta{x}_{i,j}^{V}}
  \frac{\Delta{x}_{i,j}^{C}-\Delta{x}_{i,j-1}^{C}}{\Delta{y}_{i,j}^{U}}
\end{align}

The stress tensor is given by the constitutive viscous-plastic
relation $\sigma_{\alpha\beta} = 2\eta\dot{\epsilon}_{\alpha\beta} +
[(\zeta-\eta)\dot{\epsilon}_{\gamma\gamma} - P/2
]\delta_{\alpha\beta}$ \citep{hib79}. The stress tensor divergence
$(\nabla\sigma)_{\alpha} = \partial_\beta\sigma_{\beta\alpha}$, is
discretized in finite volumes \citep[see
also][]{losch10:_mitsim}. This conveniently avoids dealing with
further metric terms, as these are ``hidden'' in the differential cell
widths. For the $u$-equation ($\alpha=1$) we have:
\begin{align}
  (\nabla\sigma)_{1}: \phantom{=}&
  \frac{1}{A_{i,j}^w}
  \int_{\mathrm{cell}}(\partial_1\sigma_{11}+\partial_2\sigma_{21})\,dx_1\,dx_2
  \\\notag
  =& \frac{1}{A_{i,j}^w} \biggl\{
  \int_{x_2}^{x_2+\Delta{x}_2}\sigma_{11}dx_2\biggl|_{x_{1}}^{x_{1}+\Delta{x}_{1}}
  + \int_{x_1}^{x_1+\Delta{x}_1}\sigma_{21}dx_1\biggl|_{x_{2}}^{x_{2}+\Delta{x}_{2}}
  \biggr\} \\ \notag
  \approx& \frac{1}{A_{i,j}^w} \biggl\{
  \Delta{x}_2\sigma_{11}\biggl|_{x_{1}}^{x_{1}+\Delta{x}_{1}}
  + \Delta{x}_1\sigma_{21}\biggl|_{x_{2}}^{x_{2}+\Delta{x}_{2}}
  \biggr\} \\ \notag
  =& \frac{1}{A_{i,j}^w} \biggl\{
  (\Delta{x}_2\sigma_{11})_{i,j}^C -
  (\Delta{x}_2\sigma_{11})_{i-1,j}^C 
  \\\notag
  \phantom{=}& \phantom{\frac{1}{A_{i,j}^w} \biggl\{}
  + (\Delta{x}_1\sigma_{21})_{i,j+1}^Z - (\Delta{x}_1\sigma_{21})_{i,j}^Z
  \biggr\}
\end{align}
with
\begin{align}
  (\Delta{x}_2\sigma_{11})_{i,j}^C =& \phantom{+}
  \Delta{y}_{i,j}^{F}(\zeta + \eta)^{C}_{i,j}
  \frac{u_{i+1,j}-u_{i,j}}{\Delta{x}_{i,j}^{F}} \\ \notag
  &+ \Delta{y}_{i,j}^{F}(\zeta + \eta)^{C}_{i,j}
  k_{2,i,j}^C \frac{v_{i,j+1}+v_{i,j}}{2} \\ \notag
  \phantom{=}& + \Delta{y}_{i,j}^{F}(\zeta - \eta)^{C}_{i,j}
  \frac{v_{i,j+1}-v_{i,j}}{\Delta{y}_{i,j}^{F}} \\ \notag
  \phantom{=}& + \Delta{y}_{i,j}^{F}(\zeta - \eta)^{C}_{i,j}
  k_{1,i,j}^{C}\frac{u_{i+1,j}+u_{i,j}}{2} \\ \notag
  \phantom{=}& - \Delta{y}_{i,j}^{F} \frac{P}{2} \\
  (\Delta{x}_1\sigma_{21})_{i,j}^Z =& \phantom{+}
  \Delta{x}_{i,j}^{V}\overline{\eta}^{Z}_{i,j}
  \frac{u_{i,j}-u_{i,j-1}}{\Delta{y}_{i,j}^{U}} \\ \notag
  & + \Delta{x}_{i,j}^{V}\overline{\eta}^{Z}_{i,j}
  \frac{v_{i,j}-v_{i-1,j}}{\Delta{x}_{i,j}^{V}} \\ \notag
  & - \Delta{x}_{i,j}^{V}\overline{\eta}^{Z}_{i,j} 
  k_{2,i,j}^{Z}\frac{u_{i,j}+u_{i,j-1}}{2} \\ \notag
  & - \Delta{x}_{i,j}^{V}\overline{\eta}^{Z}_{i,j} 
  k_{1,i,j}^{Z}\frac{v_{i,j}+v_{i-1,j}}{2}
\end{align}

Similarly, we have for the $v$-equation ($\alpha=2$):
\begin{align}
  (\nabla\sigma)_{2}: \phantom{=}&
  \frac{1}{A_{i,j}^s}
  \int_{\mathrm{cell}}(\partial_1\sigma_{12}+\partial_2\sigma_{22})\,dx_1\,dx_2 
  \\\notag
  =& \frac{1}{A_{i,j}^s} \biggl\{
  \int_{x_2}^{x_2+\Delta{x}_2}\sigma_{12}dx_2\biggl|_{x_{1}}^{x_{1}+\Delta{x}_{1}}
  + \int_{x_1}^{x_1+\Delta{x}_1}\sigma_{22}dx_1\biggl|_{x_{2}}^{x_{2}+\Delta{x}_{2}}
  \biggr\} \\ \notag
  \approx& \frac{1}{A_{i,j}^s} \biggl\{
  \Delta{x}_2\sigma_{12}\biggl|_{x_{1}}^{x_{1}+\Delta{x}_{1}}
  + \Delta{x}_1\sigma_{22}\biggl|_{x_{2}}^{x_{2}+\Delta{x}_{2}}
  \biggr\} \\ \notag
  =& \frac{1}{A_{i,j}^s} \biggl\{
  (\Delta{x}_2\sigma_{12})_{i+1,j}^Z - (\Delta{x}_2\sigma_{12})_{i,j}^Z
  \\ \notag
  \phantom{=}& \phantom{\frac{1}{A_{i,j}^s} \biggl\{}
  + (\Delta{x}_1\sigma_{22})_{i,j}^C - (\Delta{x}_1\sigma_{22})_{i,j-1}^C
  \biggr\} 
\end{align}
with
\begin{align}
  (\Delta{x}_1\sigma_{12})_{i,j}^Z =& \phantom{+}
  \Delta{y}_{i,j}^{U}\overline{\eta}^{Z}_{i,j}
  \frac{u_{i,j}-u_{i,j-1}}{\Delta{y}_{i,j}^{U}} 
  \\\notag &
  + \Delta{y}_{i,j}^{U}\overline{\eta}^{Z}_{i,j}
  \frac{v_{i,j}-v_{i-1,j}}{\Delta{x}_{i,j}^{V}} \\\notag
  &- \Delta{y}_{i,j}^{U}\overline{\eta}^{Z}_{i,j}
  k_{2,i,j}^{Z}\frac{u_{i,j}+u_{i,j-1}}{2} 
  \\\notag &
  - \Delta{y}_{i,j}^{U}\overline{\eta}^{Z}_{i,j}
  k_{1,i,j}^{Z}\frac{v_{i,j}+v_{i-1,j}}{2} \\ \notag
  (\Delta{x}_2\sigma_{22})_{i,j}^C =& \phantom{+}
  \Delta{x}_{i,j}^{F}(\zeta - \eta)^{C}_{i,j}
  \frac{u_{i+1,j}-u_{i,j}}{\Delta{x}_{i,j}^{F}} \\ \notag
  &+ \Delta{x}_{i,j}^{F}(\zeta - \eta)^{C}_{i,j}
  k_{2,i,j}^{C} \frac{v_{i,j+1}+v_{i,j}}{2} \\ \notag
  & + \Delta{x}_{i,j}^{F}(\zeta + \eta)^{C}_{i,j}
  \frac{v_{i,j+1}-v_{i,j}}{\Delta{y}_{i,j}^{F}} \\ \notag
  & + \Delta{x}_{i,j}^{F}(\zeta + \eta)^{C}_{i,j}
  k_{1,i,j}^{C}\frac{u_{i+1,j}+u_{i,j}}{2} \\ \notag
  & -\Delta{x}_{i,j}^{F} \frac{P}{2}
\end{align}

Again, no slip boundary conditions are realized via ghost points and
$u_{i,j-1}+u_{i,j}=0$ and $v_{i-1,j}+v_{i,j}=0$ across boundaries. For
free slip boundary conditions the lateral stress is set to zeros. In
analogy to $(\epsilon_{12})^Z=0$ on boundaries, we set
$\sigma_{21}^{Z}=0$, or equivalently $\eta_{i,j}^{Z}=0$, on boundaries.

\paragraph{Thermodynamics\label{sec:pkg:seaice:thermodynamics}}~\\
%
In its original formulation the sea ice model \citep{menemenlis05}
uses simple thermodynamics following the appendix of
\citet{sem76}. This formulation does not allow storage of heat,
that is, the heat capacity of ice is zero. Upward conductive heat flux
is parameterized assuming a linear temperature profile and together
with a constant ice conductivity. It is expressed as
$(K/h)(T_{w}-T_{0})$, where $K$ is the ice conductivity, $h$ the ice
thickness, and $T_{w}-T_{0}$ the difference between water and ice
surface temperatures. This type of model is often refered to as a
``zero-layer'' model. The surface heat flux is computed in a similar
way to that of \citet{parkinson79} and \citet{manabe79}. 

The conductive heat flux depends strongly on the ice thickness $h$.
However, the ice thickness in the model represents a mean over a
potentially very heterogeneous thickness distribution.  In order to
parameterize a sub-grid scale distribution for heat flux
computations, the mean ice thickness $h$ is split into seven thickness
categories $H_{n}$ that are equally distributed between $2h$ and a
minimum imposed ice thickness of $5\text{\,cm}$ by $H_n=
\frac{2n-1}{7}\,h$ for $n\in[1,7]$. The heat fluxes computed for each
thickness category is area-averaged to give the total heat flux
\citep{hibler84}. To use this thickness category parameterization set
\code{\#define SEAICE\_MULTICATEGORY}; note that this requires
different restart files and switching this flag on in the middle of an
integration is not possible.

The atmospheric heat flux is balanced by an oceanic heat flux from
below.  The oceanic flux is proportional to
$\rho\,c_{p}\left(T_{w}-T_{fr}\right)$ where $\rho$ and $c_{p}$ are
the density and heat capacity of sea water and $T_{fr}$ is the local
freezing point temperature that is a function of salinity. This flux
is not assumed to instantaneously melt or create ice, but a time scale
of three days (run-time parameter \code{SEAICE\_gamma\_t}) is used
to relax $T_{w}$ to the freezing point.
%
The parameterization of lateral and vertical growth of sea ice follows
that of \citet{hib79, hib80}; the so-called lead closing parameter
$h_{0}$ (run-time parameter \code{HO}) has a default value of
0.5~meters.

On top of the ice there is a layer of snow that modifies the heat flux
and the albedo \citep{zha98a}. Snow modifies the effective
conductivity according to 
\[\frac{K}{h} \rightarrow \frac{1}{\frac{h_{s}}{K_{s}}+\frac{h}{K}},\]
where $K_s$ is the conductivity of snow and $h_s$ the snow thickness.
If enough snow accumulates so that its weight submerges the ice and
the snow is flooded, a simple mass conserving parameterization of
snowice formation (a flood-freeze algorithm following Archimedes'
principle) turns snow into ice until the ice surface is back at $z=0$
\citep{leppaeranta83}. The flood-freeze algorithm is enabled with the CPP-flag
\code{SEAICE\_ALLOW\_FLOODING} and turned on with run-time parameter
\code{SEAICEuseFlooding=.true.}.

\paragraph{Advection of thermodynamic variables\label{sec:pkg:seaice:advection}}~\\
%
Effective ice thickness (ice volume per unit area,
$c\cdot{h}$), concentration $c$ and effective snow thickness
($c\cdot{h}_{s}$) are advected by ice velocities:
\begin{equation}
  \label{eq:advection}
  \frac{\partial{X}}{\partial{t}} = - \nabla\cdot\left(\vek{u}\,X\right) +
  \Gamma_{X} + D_{X}
\end{equation}
where $\Gamma_X$ are the thermodynamic source terms and $D_{X}$ the
diffusive terms for quantities $X=(c\cdot{h}), c, (c\cdot{h}_{s})$.
%
From the various advection scheme that are available in the MITgcm, we
recommend flux-limited schemes \citep[multidimensional 2nd and
3rd-order advection scheme with flux limiter][]{roe:85, hundsdorfer94}
to preserve sharp gradients and edges that are typical of sea ice
distributions and to rule out unphysical over- and undershoots
(negative thickness or concentration). These schemes conserve volume
and horizontal area and are unconditionally stable, so that we can set
$D_{X}=0$. Run-timeflags: \code{SEAICEadvScheme} (default=2, is the
historic 2nd-order, centered difference scheme), \code{DIFF1} =
$D_{X}/\Delta{x}$
(default=0.004).

The MITgcm sea ice model provides the option to use
the thermodynamics model of \citet{win00}, which in turn is based on
the 3-layer model of \citet{sem76} and which treats brine content by
means of enthalpy conservation; the corresponding package
\code{thsice} is described in section~\ref{sec:pkg:thsice}. This
scheme requires additional state variables, namely the enthalpy of the
two ice layers (instead of effective ice salinity), to be advected by
ice velocities.
%
The internal sea ice temperature is inferred from ice enthalpy.  To
avoid unphysical (negative) values for ice thickness and
concentration, a positive 2nd-order advection scheme with a SuperBee
flux limiter \citep{roe:85} should be used to advect all
sea-ice-related quantities of the \citet{win00} thermodynamic model
(runtime flag \code{thSIceAdvScheme=77} and
\code{thSIce\_diffK}=$D_{X}$=0 in \code{data.ice}, defaults are 0).  Because of the
non-linearity of the advection scheme, care must be taken in advecting
these quantities: when simply using ice velocity to advect enthalpy,
the total energy (i.e., the volume integral of enthalpy) is not
conserved. Alternatively, one can advect the energy content (i.e.,
product of ice-volume and enthalpy) but then false enthalpy extrema
can occur, which then leads to unrealistic ice temperature.  In the
currently implemented solution, the sea-ice mass flux is used to
advect the enthalpy in order to ensure conservation of enthalpy and to
prevent false enthalpy extrema. %

%----------------------------------------------------------------------

\subsubsection{Key subroutines
\label{sec:pkg:seaice:subroutines}}

Top-level routine: \code{seaice\_model.F}

{\footnotesize
\begin{verbatim}

C     !CALLING SEQUENCE:
c ...
c  seaice_model (TOP LEVEL ROUTINE)
c  |
c  |-- #ifdef SEAICE_CGRID
c  |     SEAICE_DYNSOLVER
c  |     |
c  |     |-- < compute proxy for geostrophic velocity >
c  |     |
c  |     |-- < set up mass per unit area and Coriolis terms >
c  |     |
c  |     |-- < dynamic masking of areas with no ice >
c  |     |
c  |     |

c  |   #ELSE
c  |     DYNSOLVER
c  |   #ENDIF
c  |
c  |-- if ( useOBCS ) 
c  |     OBCS_APPLY_UVICE
c  |
c  |-- if ( SEAICEadvHeff .OR. SEAICEadvArea .OR. SEAICEadvSnow .OR. SEAICEadvSalt )
c  |     SEAICE_ADVDIFF
c  |
c  |-- if ( usePW79thermodynamics ) 
c  |     SEAICE_GROWTH
c  |
c  |-- if ( useOBCS ) 
c  |     if ( SEAICEadvHeff ) OBCS_APPLY_HEFF
c  |     if ( SEAICEadvArea ) OBCS_APPLY_AREA
c  |     if ( SEAICEadvSALT ) OBCS_APPLY_HSALT
c  |     if ( SEAICEadvSNOW ) OBCS_APPLY_HSNOW
c  |
c  |-- < do various exchanges >
c  |
c  |-- < do additional diagnostics >
c  |
c  o

\end{verbatim}
}


%----------------------------------------------------------------------

\subsubsection{SEAICE diagnostics
\label{sec:pkg:seaice:diagnostics}}

Diagnostics output is available via the diagnostics package
(see Section \ref{sec:pkg:diagnostics}).
Available output fields are summarized in 
Table \ref{tab:pkg:seaice:diagnostics}.

\begin{table}[!ht]
\centering
\label{tab:pkg:seaice:diagnostics}
{\footnotesize
\begin{verbatim}
---------+----------+----------------+-----------------
 <-Name->|<- grid ->|<--  Units   -->|<- Tile (max=80c)
---------+----------+----------------+-----------------
 sIceLoad|SM      U1|kg/m^2          |sea-ice loading (in Mass of ice+snow / area unit)
---
SEA ICE STATE:
---
 SIarea  |SM      M1|m^2/m^2         |SEAICE fractional ice-covered area [0 to 1]
 SIheff  |SM      M1|m               |SEAICE effective ice thickness
 SIhsnow |SM      M1|m               |SEAICE effective snow thickness
 SIhsalt |SM      M1|g/m^2           |SEAICE effective salinity
 SIuice  |UU      M1|m/s             |SEAICE zonal ice velocity, >0 from West to East
 SIvice  |VV      M1|m/s             |SEAICE merid. ice velocity, >0 from South to North
---
ATMOSPHERIC STATE AS SEEN BY SEA ICE:
---
 SItices |SM  C   M1|K               |Surface Temperature over Sea-Ice (area weighted)
 SIuwind |UM      U1|m/s             |SEAICE zonal 10-m wind speed, >0 increases uVel
 SIvwind |VM      U1|m/s             |SEAICE meridional 10-m wind speed, >0 increases uVel
 SIsnPrcp|SM      U1|kg/m^2/s        |Snow precip. (+=dw) over Sea-Ice (area weighted)
---
FLUXES ACROSS ICE-OCEAN INTERFACE (ATMOS to OCEAN FOR ICE-FREE REGIONS):
---
 SIfu    |UU      U1|N/m^2           |SEAICE zonal surface wind stress, >0 increases uVel
 SIfv    |VV      U1|N/m^2           |SEAICE merid. surface wind stress, >0 increases vVel
 SIqnet  |SM      U1|W/m^2           |Ocean surface heatflux, turb+rad, >0 decreases theta
 SIqsw   |SM      U1|W/m^2           |Ocean surface shortwave radiat., >0 decreases theta
 SIempmr |SM      U1|kg/m^2/s        |Ocean surface freshwater flux, > 0 increases salt
 SIqneto |SM      U1|W/m^2           |Open Ocean Part of SIqnet, turb+rad, >0 decr theta
 SIqneti |SM      U1|W/m^2           |Ice Covered Part of SIqnet, turb+rad, >0 decr theta
---
FLUXES ACROSS ATMOSPHERE-ICE INTERFACE (ATMOS to OCEAN FOR ICE-FREE REGIONS):
---
 SIatmQnt|SM      U1|W/m^2           |Net atmospheric heat flux, >0 decreases theta
 SIatmFW |SM      U1|kg/m^2/s        |Net freshwater flux from atmosphere & land (+=down)
 SIfwSubl|SM      U1|kg/m^2/s        |Freshwater flux of sublimated ice, >0 decreases ice
---
THERMODYNAMIC DIAGNOSTICS:
---
 SIareaPR|SM      M1|m^2/m^2         |SIarea preceeding ridging process
 SIareaPT|SM      M1|m^2/m^2         |SIarea preceeding thermodynamic growth/melt
 SIheffPT|SM      M1|m               |SIheff preceeeding thermodynamic growth/melt
 SIhsnoPT|SM      M1|m               |SIhsnow preceeeding thermodynamic growth/melt
 SIaQbOCN|SM      M1|m/s             |Potential HEFF rate of change by ocean ice flux
 SIaQbATC|SM      M1|m/s             |Potential HEFF rate of change by atm flux over ice
 SIaQbATO|SM      M1|m/s             |Potential HEFF rate of change by open ocn atm flux
 SIdHbOCN|SM      M1|m/s             |HEFF rate of change by ocean ice flux
 SIdSbATC|SM      M1|m/s             |HSNOW rate of change by atm flux over sea ice
 SIdSbOCN|SM      M1|m/s             |HSNOW rate of change by ocean ice flux
 SIdHbATC|SM      M1|m/s             |HEFF rate of change by atm flux over sea ice
 SIdHbATO|SM      M1|m/s             |HEFF rate of change by open ocn atm flux
 SIdHbFLO|SM      M1|m/s             |HEFF rate of change by flooding snow
 SIdAbATO|SM      M1|m^2/m^2/s       |Potential AREA rate of change by open ocn atm flux
 SIdAbATC|SM      M1|m^2/m^2/s       |Potential AREA rate of change by atm flux over ice
 SIdAbOCN|SM      M1|m^2/m^2/s       |Potential AREA rate of change by ocean ice flux
 SIdA    |SM      M1|m^2/m^2/s       |AREA rate of change (net)
---
DYNAMIC/RHEOLOGY DIAGNOSTICS:
---
 SIpress |SM      M1|m^2/s^2         |SEAICE strength (with upper and lower limit)
 SIzeta  |SM      M1|m^2/s           |SEAICE nonlinear bulk viscosity
 SIeta   |SM      M1|m^2/s           |SEAICE nonlinear shear viscosity
 SIsigI  |SM      M1|no units        |SEAICE normalized principle stress, component one
 SIsigII |SM      M1|no units        |SEAICE normalized principle stress, component two
---
ADVECTIVE/DIFFUSIVE FLUXES OF SEA ICE variables:
---
 ADVxHEFF|UU      M1|m.m^2/s         |Zonal      Advective Flux of eff ice thickn
 ADVyHEFF|VV      M1|m.m^2/s         |Meridional Advective Flux of eff ice thickn
 SIuheff |UU      M1|m^2/s           |Zonal      Transport of eff ice thickn (centered)
 SIvheff |VV      M1|m^2/s           |Meridional Transport of eff ice thickn (centered)
 DFxEHEFF|UU      M1|m^2/s           |Zonal      Diffusive Flux of eff ice thickn
 DFyEHEFF|VV      M1|m^2/s           |Meridional Diffusive Flux of eff ice thickn
 ADVxAREA|UU      M1|m^2/m^2.m^2/s   |Zonal      Advective Flux of fract area
 ADVyAREA|VV      M1|m^2/m^2.m^2/s   |Meridional Advective Flux of fract area
 DFxEAREA|UU      M1|m^2/m^2.m^2/s   |Zonal      Diffusive Flux of fract area
 DFyEAREA|VV      M1|m^2/m^2.m^2/s   |Meridional Diffusive Flux of fract area
 ADVxSNOW|UU      M1|m.m^2/s         |Zonal      Advective Flux of eff snow thickn
 ADVySNOW|VV      M1|m.m^2/s         |Meridional Advective Flux of eff snow thickn
 DFxESNOW|UU      M1|m.m^2/s         |Zonal      Diffusive Flux of eff snow thickn
 DFyESNOW|VV      M1|m.m^2/s         |Meridional Diffusive Flux of eff snow thickn
 ADVxSSLT|UU      M1|psu.m^2/s       |Zonal      Advective Flux of seaice salinity
 ADVySSLT|VV      M1|psu.m^2/s       |Meridional Advective Flux of seaice salinity
 DFxESSLT|UU      M1|psu.m^2/s       |Zonal      Diffusive Flux of seaice salinity
 DFyESSLT|VV      M1|psu.m^2/s       |Meridional Diffusive Flux of seaice salinity
\end{verbatim}
}
\caption{Available diagnostics of the seaice-package}
\end{table}


%\subsubsection{Package Reference}

\subsubsection{Experiments and tutorials that use seaice}
\label{sec:pkg:seaice:experiments}

\begin{itemize}
\item{Labrador Sea experiment in \code{lab\_sea} verification directory. }
\item \code{seaice\_obcs}, based on \code{lab\_sea}
\item \code{offline\_exf\_seaice/input.seaicetd}, based on \code{lab\_sea}
\item \code{global\_ocean.cs32x15/input.icedyn} and
  \code{global\_ocean.cs32x15/input.seaice}, global
  cubed-sphere-experiment with combinations of \code{seaice} and
  \code{thsice}
\end{itemize}


%%% Local Variables: 
%%% mode: latex
%%% TeX-master: "../../manual"
%%% End: 
