% \documentclass[12pt]{article}
% \usepackage{amssymb}
% 
% \usepackage{graphics}
% 
% 
% \oddsidemargin -4mm \evensidemargin 0mm
% \textwidth 165mm
% \textheight 230mm
% \topmargin -2mm \headsep -2mm
% \renewcommand{\baselinestretch}{1.5}
% \begin{document}
% 

%%%--------------------------------------%%%
\def\deg{$^o$}
\section{Bulk Formula Package}
\label{sec:pkg:bulk_formula}
\begin{rawhtml}
<!-- CMIREDIR:package_bulk_formula: -->
\end{rawhtml}

author: Stephanie Dutkiewicz\\

\noindent
Instead of forcing the model with heat and fresh water flux data,
this package calculates these fluxes using the changing sea surface
temperature. We need to read in some atmospheric data:
{\bf air temperature, air humidity, down shortwave radiation,
     down longwave radiation, precipitation, wind speed}.
The current setup also reads in {\bf wind stress}, but this
can be changed so that the stresses are calculated from the
wind speed.

The current setup requires that there is the thermodynamic-seaice package
({\it pkg/thsice}, also refered below as seaice)
is also used. It would be useful though to have it also
setup to run with some very simple parametrization of the sea ice.

%%%%%%%%%%%%%%%%%%%%%%%%%%%%%%%%%%%%%%%%%%%%%%%%%%%%%%%%%%%%%%

\vspace{1cm}

\noindent
The heat and fresh water fluxes are calculated in {\it bulkf\_forcing.F}
called from {\it forward\_step.F}. These fluxes are used over open water,
fluxes over seaice are recalculated in the sea-ice package.
Before the call to {\it bulkf\_forcing.F} we call 
{\it bulkf\_fields\_load.F} to find the current atmospheric conditions.
The only other changes to the model code come from the initializing
and writing diagnostics of these fluxes.

%%%%%%%%%%%%%%%%%%%%%%%%%%%%%%%%%%%%%%%%%%%%%%%%%%%%%%%%%%%%%%
\vspace{1cm}
\noindent
{\bf \underline{subroutine BULKF\_FIELDS\_LOAD}}

\noindent
Here we find the atmospheric data needed for the bulk formula
calculations. These are read in at periodic intervals and
values are interpolated to the current time. The data file names
come from {\bf data.blk}. The values that can be read in are:
air temperature, air humidity, precipitation, 
down solar radiation, down long
wave radiation, zonal and meridional wind speeds, total wind
speed, net heat flux, net freshwater forcing, cloud cover,
snow fall, zonal and meridional wind stresses, and SST and SSS
used for relaxation terms.
Not all these files are necessary or used. For instance cloud
cover and snow fall are not used in the current bulk formula
calculation. If total wind speed is not supplied, wind speed
is calculate from the zonal and meridional components. If
wind stresses are not read in, then the stresses are calculated
from the wind speed. Net heat flux and net freshwater can be
read in and used over open ocean instead of the bulk formula
calculations (but over seaice the bulkf formula is always
used). This is "hardwired" into {\it bulkf\_forcing} and
the "ch" in the variable names suggests that this is "cheating".
SST and SSS need to be read in if there is any relaxation used.


%%%%%%%%%%%%%%%%%%%%%%%%%%%%%%%%%%%%%%%%%%%%%%%%%%%%%%%%%%%%%%


\vspace{1cm}
\noindent
{\bf \underline{subroutine BULKF\_FORCING}}

\noindent
In {\it bulkf\_forcing.F}, we calculate  heat and fresh water
fluxes (and wind stress, if necessary) for each grid cell.
First we determine if the grid cell is open water or seaice
and this information is carried by {\bf iceornot}. There is
a provision here for a different designation if there is
snow cover (but currently this does not make any difference).
We then call {\it bulkf\_formula\_lanl.F} which provides
values for: up long wave radiation, latent and sensible heat
fluxes, the derivative of these three with respect to surface
temperature, wind stress, evaporation. 
Net long wave radiation is calculated from the combination
of the down long wave read in and the up long wave calculated.

We then find the albedo of the surface - with a call to
{\it sfc\_albedo} if there is sea-ice (see the seaice package
for information on the subroutine). If the grid cell is open
ocean the albedo is set as 0.1. Note that this is a parameter
that can be used to tune the results. The net short wave
radiation is then the down shortwave radiation minus the 
amount reflected.

If the wind stress needed to be calculated in {\it bulkf\_formula\_lanl.F},
it was calculated to grid cell center points, so in {\it bulkf\_forcing.F}
we regrid to {\bf u} and {\bf v} points. We let the model know
if it has read in stresses or calculated stresses by the switch
{\bf readwindstress} which is can be set in data.blk, and defaults
to {\bf .TRUE.}.

We then calculate {\bf Qnet} and {\bf EmPmR} that will be used
as the fluxes over the open ocean. There is a provision for
using runoff. If we are "cheating" and using observed fluxes 
over the open ocean, then there is a provision here to
use read in {\bf Qnet} and {\bf EmPmR}.

The final call is to calculate averages of the terms found
in this subroutine.

%%%%%%%%%%%%%%%%%%%%%%%%%%%%%%%%%%%%%%%%%%%%%%%%%%%%%%%%%%%%%%
\vspace{1cm}
\noindent
{\bf {\underline{ subroutine BULKF\_FORMULA\_LANL}}}

\noindent
This is the main program of the package where the
heat fluxes and freshwater fluxes over ice and
open water are calculated. Note that this subroutine
is also called from the seaice package during the
iterations to find the ice surface temperature.

Latent heat ($L$) used in this subroutine 
depends on the state of the surface: vaporization for
open water, fusion and vaporization for ice surfaces.
Air temperature is converted from Celsius to Kelvin.
If there is no wind speed ($u_s$) given, then the wind speed
is calculated from the zonal and meridional components.

We calculate the virtual temperature:
\[
T_o = T_{air} (1+\gamma q_{air})
\]
where $T_{air}$ is the air temperature at $h_T$, $q_{air}$ is
humidity at $h_q$ and $\gamma$ is a constant.

The saturated vapor pressure is calculate (QQ ref):
\[
q_{sat} = \frac{a}{p_o} e^{L (b-\frac{c}{T_{srf}})}
\]
where $a,b,c$ are constants, $T_{srf}$ is surface temperature
and $p_o$ is the surface pressure.

The two values crucial for the bulk formula calculations are
the difference between air at sea surface and sea surface temperature:
\[
\Delta T = T_{air} - T_{srf} +\alpha h_T
\]
where $\alpha$ is adiabatic lapse rate and $h_T$ is the  height
where the air temperature was taken; and the difference
between the air humidity and the saturated humidity
\[
\Delta q = q_{air} - q_{sat}.
\]

We then calculate the turbulent exchange coefficients
following Bryan et al (1996) and the numerical scheme
of Hunke and Lipscombe (1998). 
We estimate initial values for the exchange coefficients, $c_u$,
$c_T$ and $c_q$ as
\[
\frac{\kappa}{ln(z_{ref}/z_{rou})}
\]
where $\kappa$ is the Von Karman constant, $z_{ref}$ is a
reference height and $z_{rou}$ is a roughness length scale
which could be a function of type of surface, but is here set
as a constant. Turbulent scales are:
\begin{eqnarray}
u^* & = & c_u u_s \nonumber\\
T^* & = & c_T \Delta T \nonumber\\
q^* & = & c_q \Delta q \nonumber
\end{eqnarray}

We find the "integrated flux profile" for momentum and stability
if there are stable QQ conditions ($\Upsilon>0$) :
\[
\psi_m = \psi_s = -5 \Upsilon
\]
and for unstable QQ conditions ($\Upsilon<0$):
\begin{eqnarray}
\psi_m & = & 2 ln(0.5(1+\chi)) + ln(0.5(1+\chi^2)) - 2 \tan^{-1} \chi + \pi/2
\nonumber \\
\psi_s & = & 2 ln(0.5(1+\chi^2)) \nonumber
\end{eqnarray}
where
\[
\Upsilon = \frac{\kappa g z_{ref}}{u^{*2}} (\frac{T^*}{T_o} + 
\frac{q^*}{1/\gamma + q_a})
\]
and $\chi=(1-16\Upsilon)^{1/2}$.

The coefficients are updated through 5 iterations as:
\begin{eqnarray}
c_u & = & \frac {\hat{c_u}}{1+\hat{c_u}(\lambda - \psi_m)/\kappa} \nonumber \\
c_T & = & \frac {\hat{c_T}}{1+\hat{c_T}(\lambda - \psi_s)/\kappa} \nonumber \\
c_q & = & c'_T
\end{eqnarray}
where $\lambda =ln(h_T/z_{ref})$.

We can then find the bulk formula heat fluxes:

\vspace{.2cm}
\noindent
Sensible heat flux:
\[
Q_s=\rho_{air} c_{p_{air}} u_s c_u c_T \Delta T
\]

\vspace{.2cm}
\noindent
Latent heat flux:
\[
Q_l=\rho_{air} L u_s c_u c_q \Delta q
\]

\vspace{.2cm}
\noindent
Up long wave radiation
\[
Q_{lw}^{up}=\epsilon \sigma T_{srf}^4
\]
where $\epsilon$ is emissivity (which can be different for
open ocean, ice and snow), $\sigma$ is Stefan-Boltzman constant.

We calculate the derivatives of the three above functions
with respect to surface temperature
\begin{eqnarray}
\frac{dQ_s}{d_T} & = & \rho_{air} c_{p_{air}} u_s c_u c_T \nonumber \\
\frac{dQ_l}{d_T} & = & \frac{\rho_{air} L^2 u_s c_u c_q c}{T_{srf}^2} \nonumber \\
\frac{dQ_{]lw}^{up}}{d_T} & = &  4 \epsilon \sigma t_{srf}^3 \nonumber
\end{eqnarray}

And total derivative $\frac{dQ_o}{dT}= \frac{dQ_s}{dT} +
\frac{dQ_l}{dT} + \frac{dQ_{lw}^{up}}{dT}$.




If we do not read in the wind stress, it is calculated here.

%%%%%%%%%%%%%%%%%%%%%%%%%%%%%%%%%%%%%%%%%%%%%%%%%%%%%%%%%%%%%%%

\vspace{1cm}

\noindent
{\bf {\underline{Initializing subroutines}}}

\noindent
{\it bulkf\_init.F}:
Set bulkf variables to zero.

\noindent
{\it bulkf\_readparms.F}:
Reads {\bf data.blk}

%%%%%%%%%%%%%%%%%%%%%%%%%%%%%%%%%%%%%%%%%%%%%%%%%%%%%%%%%%%%%%%

\vspace{1cm}

\noindent
{\bf {\underline{Diagnostic subroutines}}}

\noindent
{\it bulkf\_ave.F}:
Keeps track of means of the bulkf variables

\noindent
{\it bulkf\_diags.F}:
Finds averages and writes out diagnostics

%%%%%%%%%%%%%%%%%%%%%%%%%%%%%%%%%%%%%%%%%%%%%%%%%%%%%%%%%%%%%%%%
\vspace{1cm}

\noindent
{\bf {\underline{Common Blocks}}}

\noindent
{\it BULKF.h}: BULKF Variables,  data file names, and logicals
{\bf readwindstress} and {\bf readsurface}

\noindent
{\it BULKF\_DIAGS.h}: matrices for diagnostics: averages of fields
from {\it bulkf\_diags.F}

\noindent
{\it BULKF\_ICE\_CONSTANTS.h}: 
all the parameters need by the ice model and in the bulkf formula
calculations.

%%%%%%%%%%%%%%%%%%%%%%%%%%%%%%%%%%%%%%%%%%%%%%%%%%%%%%%%%%%%%%%%%%
\vspace{1cm}

\noindent
{\bf {\underline{Input file DATA.ICE}}}

\noindent
We read in the file names of atmospheric data used in
the  bulk formula calculations. Here we can also set
the logicals: {\bf readwindstress} if we read in the
wind stress rather than calculate it from the wind
speed; and {\bf readsurface} to read in the surface
temperature and salinity if these will be used as
part of a relaxing term.

%%%%%%%%%%%%%%%%%%%%%%%%%%%%%%%%%%%%%%%%%%%%%%%%%%%%%%%%%%%%
\vspace{1cm}

\noindent
{\bf {\underline{Important Notes}}}

\noindent
{\bf 1)} heat fluxes have different signs in the ocean and ice
models.

\noindent
{\bf 2)} {\bf StartIceModel} must be changed in {\bf data.ice}:
1 (if starting from no ice), 0 (if using pickup.ic file).

%%%%%%%%%%%%%%%%%%%%%%%%%%%%%%%%%%%%%%%%%%%%%%%%%%%%%%%%%%%%%%

\vspace{1cm}

\noindent 
{\bf {\underline{References}}}


\vspace{.2cm}

\noindent
Bryan F.O., B.G Kauffman, W.G. Large, P.R. Gent, 1996:
The NCAR CSM flux coupler. Technical note TN-425+STR,
NCAR.

\vspace{.2cm}

\noindent
Hunke, E.C and W.H. Lipscomb, circa 2001: CICE: the Los Alamos
Sea Ice Model Documentation and Software User's Manual.
LACC-98-16v.2.\\
(note: this documentation is no longer available as CICE has progressed
to a very different version 3)





%%%%%%%%%%%%%%%%%%%%%%%%%%%%%%%%%%%%%%%%%%%
% \end{document}
