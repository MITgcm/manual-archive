% $Header: /u/gcmpack/manual/s_getstarted/text/getting_started.tex,v 1.24 2004/04/09 15:06:18 edhill Exp $
% $Name:  $

%\section{Getting started}

In this section, we describe how to use the model. In the first
section, we provide enough information to help you get started with
the model. We believe the best way to familiarize yourself with the
model is to run the case study examples provided with the base
version. Information on how to obtain, compile, and run the code is
found there as well as a brief description of the model structure
directory and the case study examples.  The latter and the code
structure are described more fully in chapters
\ref{chap:discretization} and \ref{chap:sarch}, respectively. Here, in
this section, we provide information on how to customize the code when
you are ready to try implementing the configuration you have in mind.

\section{Where to find information}
\label{sect:whereToFindInfo}

A web site is maintained for release 2 (``Pelican'') of MITgcm:
\begin{rawhtml} <A href=http://mitgcm.org/pelican/ target="idontexist"> \end{rawhtml}
\begin{verbatim}
http://mitgcm.org/pelican
\end{verbatim}
\begin{rawhtml} </A> \end{rawhtml}
Here you will find an on-line version of this document, a
``browsable'' copy of the code and a searchable database of the model
and site, as well as links for downloading the model and
documentation, to data-sources, and other related sites.

There is also a web-archived support mailing list for the model that
you can email at \texttt{MITgcm-support@mitgcm.org} or browse at:
\begin{rawhtml} <A href=http://mitgcm.org/mailman/listinfo/mitgcm-support/ target="idontexist"> \end{rawhtml}
\begin{verbatim}
http://mitgcm.org/mailman/listinfo/mitgcm-support/
http://mitgcm.org/pipermail/mitgcm-support/
\end{verbatim}
\begin{rawhtml} </A> \end{rawhtml}
Essentially all of the MITgcm web pages can be searched using a
popular web crawler such as Google or through our own search facility:
\begin{rawhtml} <A href=http://mitgcm.org/mailman/htdig/ target="idontexist"> \end{rawhtml}
\begin{verbatim}
http://mitgcm.org/htdig/
\end{verbatim}
\begin{rawhtml} </A> \end{rawhtml}
%%% http://www.google.com/search?q=hydrostatic+site%3Amitgcm.org



\section{Obtaining the code}
\label{sect:obtainingCode}

MITgcm can be downloaded from our system by following
the instructions below. As a courtesy we ask that you send e-mail to us at
\begin{rawhtml} <A href=mailto:MITgcm-support@mitgcm.org> \end{rawhtml}
MITgcm-support@mitgcm.org
\begin{rawhtml} </A> \end{rawhtml}
to enable us to keep track of who's using the model and in what application.
You can download the model two ways:

\begin{enumerate}
\item Using CVS software. CVS is a freely available source code management
tool. To use CVS you need to have the software installed. Many systems
come with CVS pre-installed, otherwise good places to look for
the software for a particular platform are
\begin{rawhtml} <A href=http://www.cvshome.org/ target="idontexist"> \end{rawhtml}
cvshome.org
\begin{rawhtml} </A> \end{rawhtml}
and
\begin{rawhtml} <A href=http://www.wincvs.org/ target="idontexist"> \end{rawhtml}
wincvs.org
\begin{rawhtml} </A> \end{rawhtml}
.

\item Using a tar file. This method is simple and does not
require any special software. However, this method does not
provide easy support for maintenance updates.

\end{enumerate}

\subsubsection{Checkout from CVS}
\label{sect:cvs_checkout}

If CVS is available on your system, we strongly encourage you to use it. CVS
provides an efficient and elegant way of organizing your code and keeping
track of your changes. If CVS is not available on your machine, you can also
download a tar file.

Before you can use CVS, the following environment variable(s) should
be set within your shell.  For a csh or tcsh shell, put the following 
\begin{verbatim}
% setenv CVSROOT :pserver:cvsanon@mitgcm.org:/u/gcmpack
\end{verbatim}
in your .cshrc or .tcshrc file.  For bash or sh shells, put:
\begin{verbatim}
% export CVSROOT=':pserver:cvsanon@mitgcm.org:/u/gcmpack'
\end{verbatim}
in your \texttt{.profile} or \texttt{.bashrc} file.


To get MITgcm through CVS, first register with the MITgcm CVS server
using command:
\begin{verbatim}
% cvs login ( CVS password: cvsanon )
\end{verbatim}
You only need to do a ``cvs login'' once.

To obtain the latest sources type:
\begin{verbatim}
% cvs co MITgcm
\end{verbatim}
or to get a specific release type:
\begin{verbatim}
% cvs co -P -r checkpoint52i_post  MITgcm
\end{verbatim}
The MITgcm web site contains further directions concerning the source
code and CVS.  It also contains a web interface to our CVS archive so
that one may easily view the state of files, revisions, and other
development milestones:
\begin{rawhtml} <A href=''http://mitgcm.org/download'' target="idontexist"> \end{rawhtml}
\begin{verbatim}
http://mitgcm.org/source_code.html
\end{verbatim}
\begin{rawhtml} </A> \end{rawhtml}

As a convenience, the MITgcm CVS server contains aliases which are
named subsets of the codebase.  These aliases can be especially
helpful when used over slow internet connections or on machines with
restricted storage space.  Table \ref{tab:cvsModules} contains a list
of CVS aliases
\begin{table}[htb]
  \centering
  \begin{tabular}[htb]{|lp{3.25in}|}\hline
    \textbf{Alias Name}    &  \textbf{Information (directories) Contained}  \\\hline
    \texttt{MITgcm\_code}  &  Only the source code -- none of the verification examples.  \\
    \texttt{MITgcm\_verif\_basic}
    &  Source code plus a small set of the verification examples 
    (\texttt{global\_ocean.90x40x15}, \texttt{aim.5l\_cs}, \texttt{hs94.128x64x5}, 
    \texttt{front\_relax}, and \texttt{plume\_on\_slope}).  \\
    \texttt{MITgcm\_verif\_atmos}  &  Source code plus all of the atmospheric examples.  \\
    \texttt{MITgcm\_verif\_ocean}  &  Source code plus all of the oceanic examples.  \\
    \texttt{MITgcm\_verif\_all}    &  Source code plus all of the
    verification examples. \\\hline
  \end{tabular}
  \caption{MITgcm CVS Modules}
  \label{tab:cvsModules}
\end{table}

The checkout process creates a directory called \textit{MITgcm}. If
the directory \textit{MITgcm} exists this command updates your code
based on the repository. Each directory in the source tree contains a
directory \textit{CVS}. This information is required by CVS to keep
track of your file versions with respect to the repository. Don't edit
the files in \textit{CVS}!  You can also use CVS to download code
updates.  More extensive information on using CVS for maintaining
MITgcm code can be found
\begin{rawhtml} <A href=''http://mitgcm.org/usingcvstoget.html'' target="idontexist"> \end{rawhtml}
here
\begin{rawhtml} </A> \end{rawhtml} 
.
It is important to note that the CVS aliases in Table
\ref{tab:cvsModules} cannot be used in conjunction with the CVS
\texttt{-d DIRNAME} option.  However, the \texttt{MITgcm} directories
they create can be changed to a different name following the check-out:
\begin{verbatim}
   %  cvs co MITgcm_verif_basic
   %  mv MITgcm MITgcm_verif_basic
\end{verbatim}


\subsubsection{Conventional download method}
\label{sect:conventionalDownload}

If you do not have CVS on your system, you can download the model as a
tar file from the web site at:
\begin{rawhtml} <A href=http://mitgcm.org/download target="idontexist"> \end{rawhtml}
\begin{verbatim}
http://mitgcm.org/download/
\end{verbatim}
\begin{rawhtml} </A> \end{rawhtml}
The tar file still contains CVS information which we urge you not to
delete; even if you do not use CVS yourself the information can help
us if you should need to send us your copy of the code.  If a recent
tar file does not exist, then please contact the developers through
the 
\begin{rawhtml} <A href=''mailto:MITgcm-support@mitgcm.org"> \end{rawhtml}
MITgcm-support@mitgcm.org
\begin{rawhtml} </A> \end{rawhtml}
mailing list.

\subsubsection{Upgrading from an earlier version}

If you already have an earlier version of the code you can ``upgrade''
your copy instead of downloading the entire repository again. First,
``cd'' (change directory) to the top of your working copy:
\begin{verbatim}
% cd MITgcm
\end{verbatim}
and then issue the cvs update command such as:
\begin{verbatim}
% cvs -q update -r checkpoint52i_post -d -P
\end{verbatim}
This will update the ``tag'' to ``checkpoint52i\_post'', add any new
directories (-d) and remove any empty directories (-P). The -q option
means be quiet which will reduce the number of messages you'll see in
the terminal. If you have modified the code prior to upgrading, CVS
will try to merge your changes with the upgrades. If there is a
conflict between your modifications and the upgrade, it will report
that file with a ``C'' in front, e.g.:
\begin{verbatim}
C model/src/ini_parms.F
\end{verbatim}
If the list of conflicts scrolled off the screen, you can re-issue the
cvs update command and it will report the conflicts. Conflicts are
indicated in the code by the delimites ``$<<<<<<<$'', ``======='' and
``$>>>>>>>$''. For example,
{\small
\begin{verbatim}
<<<<<<< ini_parms.F
     & bottomDragLinear,myOwnBottomDragCoefficient,
=======
     & bottomDragLinear,bottomDragQuadratic,
>>>>>>> 1.18
\end{verbatim}
}
means that you added ``myOwnBottomDragCoefficient'' to a namelist at
the same time and place that we added ``bottomDragQuadratic''. You
need to resolve this conflict and in this case the line should be
changed to:
{\small
\begin{verbatim}
     & bottomDragLinear,bottomDragQuadratic,myOwnBottomDragCoefficient,
\end{verbatim}
}
and the lines with the delimiters ($<<<<<<$,======,$>>>>>>$) be deleted.
Unless you are making modifications which exactly parallel
developments we make, these types of conflicts should be rare.

\paragraph*{Upgrading to the current pre-release version}

We don't make a ``release'' for every little patch and bug fix in
order to keep the frequency of upgrades to a minimum. However, if you
have run into a problem for which ``we have already fixed in the
latest code'' and we haven't made a ``tag'' or ``release'' since that
patch then you'll need to get the latest code:
\begin{verbatim}
% cvs -q update -A -d -P
\end{verbatim}
Unlike, the ``check-out'' and ``update'' procedures above, there is no
``tag'' or release name. The -A tells CVS to upgrade to the
very latest version. As a rule, we don't recommend this since you
might upgrade while we are in the processes of checking in the code so
that you may only have part of a patch. Using this method of updating
also means we can't tell what version of the code you are working
with. So please be sure you understand what you're doing.

\section{Model and directory structure}

The ``numerical'' model is contained within a execution environment
support wrapper. This wrapper is designed to provide a general
framework for grid-point models. MITgcmUV is a specific numerical
model that uses the framework. Under this structure the model is split
into execution environment support code and conventional numerical
model code. The execution environment support code is held under the
\textit{eesupp} directory. The grid point model code is held under the
\textit{model} directory. Code execution actually starts in the
\textit{eesupp} routines and not in the \textit{model} routines. For
this reason the top-level \textit{MAIN.F} is in the
\textit{eesupp/src} directory. In general, end-users should not need
to worry about this level. The top-level routine for the numerical
part of the code is in \textit{model/src/THE\_MODEL\_MAIN.F}. Here is
a brief description of the directory structure of the model under the
root tree (a detailed description is given in section 3: Code
structure).

\begin{itemize}

\item \textit{bin}: this directory is initially empty. It is the
  default directory in which to compile the code.
  
\item \textit{diags}: contains the code relative to time-averaged
  diagnostics. It is subdivided into two subdirectories \textit{inc}
  and \textit{src} that contain include files (*.\textit{h} files) and
  Fortran subroutines (*.\textit{F} files), respectively.

\item \textit{doc}: contains brief documentation notes.
  
\item \textit{eesupp}: contains the execution environment source code.
  Also subdivided into two subdirectories \textit{inc} and
  \textit{src}.
  
\item \textit{exe}: this directory is initially empty. It is the
  default directory in which to execute the code.
  
\item \textit{model}: this directory contains the main source code.
  Also subdivided into two subdirectories \textit{inc} and
  \textit{src}.
  
\item \textit{pkg}: contains the source code for the packages. Each
  package corresponds to a subdirectory. For example, \textit{gmredi}
  contains the code related to the Gent-McWilliams/Redi scheme,
  \textit{aim} the code relative to the atmospheric intermediate
  physics. The packages are described in detail in section 3.
  
\item \textit{tools}: this directory contains various useful tools.
  For example, \textit{genmake2} is a script written in csh (C-shell)
  that should be used to generate your makefile. The directory
  \textit{adjoint} contains the makefile specific to the Tangent
  linear and Adjoint Compiler (TAMC) that generates the adjoint code.
  The latter is described in details in part V.
  
\item \textit{utils}: this directory contains various utilities. The
  subdirectory \textit{knudsen2} contains code and a makefile that
  compute coefficients of the polynomial approximation to the knudsen
  formula for an ocean nonlinear equation of state. The
  \textit{matlab} subdirectory contains matlab scripts for reading
  model output directly into matlab. \textit{scripts} contains C-shell
  post-processing scripts for joining processor-based and tiled-based
  model output.
  
\item \textit{verification}: this directory contains the model
  examples. See section \ref{sect:modelExamples}.

\end{itemize}

\section{Example experiments}
\label{sect:modelExamples}

%% a set of twenty-four pre-configured numerical experiments

The MITgcm distribution comes with more than a dozen pre-configured
numerical experiments. Some of these example experiments are tests of
individual parts of the model code, but many are fully fledged
numerical simulations. A few of the examples are used for tutorial
documentation in sections \ref{sect:eg-baro} - \ref{sect:eg-global}.
The other examples follow the same general structure as the tutorial
examples. However, they only include brief instructions in a text file
called {\it README}.  The examples are located in subdirectories under
the directory \textit{verification}. Each example is briefly described
below.

\subsection{Full list of model examples}

\begin{enumerate}
  
\item \textit{exp0} - single layer, ocean double gyre (barotropic with
  free-surface). This experiment is described in detail in section
  \ref{sect:eg-baro}.

\item \textit{exp1} - Four layer, ocean double gyre. This experiment
  is described in detail in section \ref{sect:eg-baroc}.
  
\item \textit{exp2} - 4x4 degree global ocean simulation with steady
  climatological forcing. This experiment is described in detail in
  section \ref{sect:eg-global}.
  
\item \textit{exp4} - Flow over a Gaussian bump in open-water or
  channel with open boundaries.
  
\item \textit{exp5} - Inhomogenously forced ocean convection in a
  doubly periodic box.

\item \textit{front\_relax} - Relaxation of an ocean thermal front (test for
Gent/McWilliams scheme). 2D (Y-Z).

\item \textit{internal wave} - Ocean internal wave forced by open
  boundary conditions.
  
\item \textit{natl\_box} - Eastern subtropical North Atlantic with KPP
  scheme; 1 month integration
  
\item \textit{hs94.1x64x5} - Zonal averaged atmosphere using Held and
  Suarez '94 forcing.
  
\item \textit{hs94.128x64x5} - 3D atmosphere dynamics using Held and
  Suarez '94 forcing.
  
\item \textit{hs94.cs-32x32x5} - 3D atmosphere dynamics using Held and
  Suarez '94 forcing on the cubed sphere.
  
\item \textit{aim.5l\_zon-ave} - Intermediate Atmospheric physics.
  Global Zonal Mean configuration, 1x64x5 resolution.
  
\item \textit{aim.5l\_XZ\_Equatorial\_Slice} - Intermediate
  Atmospheric physics, equatorial Slice configuration.  2D (X-Z).
  
\item \textit{aim.5l\_Equatorial\_Channel} - Intermediate Atmospheric
  physics. 3D Equatorial Channel configuration.
  
\item \textit{aim.5l\_LatLon} - Intermediate Atmospheric physics.
  Global configuration, on latitude longitude grid with 128x64x5 grid
  points ($2.8^\circ{\rm degree}$ resolution).
  
\item \textit{adjustment.128x64x1} Barotropic adjustment problem on
  latitude longitude grid with 128x64 grid points ($2.8^\circ{\rm
    degree}$ resolution).
  
\item \textit{adjustment.cs-32x32x1} Barotropic adjustment problem on
  cube sphere grid with 32x32 points per face ( roughly $2.8^\circ{\rm
    degree}$ resolution).
  
\item \textit{advect\_cs} Two-dimensional passive advection test on
  cube sphere grid.
  
\item \textit{advect\_xy} Two-dimensional (horizontal plane) passive
  advection test on Cartesian grid.
  
\item \textit{advect\_yz} Two-dimensional (vertical plane) passive
  advection test on Cartesian grid.
  
\item \textit{carbon} Simple passive tracer experiment. Includes
  derivative calculation. Described in detail in section
  \ref{sect:eg-carbon-ad}.

\item \textit{flt\_example} Example of using float package.
  
\item \textit{global\_ocean.90x40x15} Global circulation with GM, flux
  boundary conditions and poles.

\item \textit{global\_ocean\_pressure} Global circulation in pressure
  coordinate (non-Boussinesq ocean model). Described in detail in
  section \ref{sect:eg-globalpressure}.
  
\item \textit{solid-body.cs-32x32x1} Solid body rotation test for cube
  sphere grid.

\end{enumerate}

\subsection{Directory structure of model examples}

Each example directory has the following subdirectories:

\begin{itemize}
\item \textit{code}: contains the code particular to the example. At a
  minimum, this directory includes the following files:

  \begin{itemize}
  \item \textit{code/CPP\_EEOPTIONS.h}: declares CPP keys relative to
    the ``execution environment'' part of the code. The default
    version is located in \textit{eesupp/inc}.
  
  \item \textit{code/CPP\_OPTIONS.h}: declares CPP keys relative to
    the ``numerical model'' part of the code. The default version is
    located in \textit{model/inc}.
  
  \item \textit{code/SIZE.h}: declares size of underlying
    computational grid.  The default version is located in
    \textit{model/inc}.
  \end{itemize}
  
  In addition, other include files and subroutines might be present in
  \textit{code} depending on the particular experiment. See Section 2
  for more details.
  
\item \textit{input}: contains the input data files required to run
  the example. At a minimum, the \textit{input} directory contains the
  following files:

  \begin{itemize}
  \item \textit{input/data}: this file, written as a namelist,
    specifies the main parameters for the experiment.
  
  \item \textit{input/data.pkg}: contains parameters relative to the
    packages used in the experiment.
  
  \item \textit{input/eedata}: this file contains ``execution
    environment'' data. At present, this consists of a specification
    of the number of threads to use in $X$ and $Y$ under multithreaded
    execution.
  \end{itemize}
  
  In addition, you will also find in this directory the forcing and
  topography files as well as the files describing the initial state
  of the experiment.  This varies from experiment to experiment. See
  section 2 for more details.

\item \textit{results}: this directory contains the output file
  \textit{output.txt} produced by the simulation example. This file is
  useful for comparison with your own output when you run the
  experiment.
\end{itemize}

Once you have chosen the example you want to run, you are ready to
compile the code.

\section{Building the code}
\label{sect:buildingCode}

To compile the code, we use the {\em make} program. This uses a file
({\em Makefile}) that allows us to pre-process source files, specify
compiler and optimization options and also figures out any file
dependencies. We supply a script ({\em genmake2}), described in
section \ref{sect:genmake}, that automatically creates the {\em
  Makefile} for you. You then need to build the dependencies and
compile the code.

As an example, let's assume that you want to build and run experiment
\textit{verification/exp2}. The are multiple ways and places to
actually do this but here let's build the code in
\textit{verification/exp2/input}:
\begin{verbatim}
% cd verification/exp2/input
\end{verbatim}
First, build the {\em Makefile}:
\begin{verbatim}
% ../../../tools/genmake2 -mods=../code
\end{verbatim}
The command line option tells {\em genmake} to override model source
code with any files in the directory {\em ./code/}.

On many systems, the {\em genmake2} program will be able to
automatically recognize the hardware, find compilers and other tools
within the user's path (``echo \$PATH''), and then choose an
appropriate set of options from the files contained in the {\em
  tools/build\_options} directory.  Under some circumstances, a user
may have to create a new ``optfile'' in order to specify the exact
combination of compiler, compiler flags, libraries, and other options
necessary to build a particular configuration of MITgcm.  In such
cases, it is generally helpful to read the existing ``optfiles'' and
mimic their syntax.

Through the MITgcm-support list, the MITgcm developers are willing to
provide help writing or modifing ``optfiles''.  And we encourage users
to post new ``optfiles'' (particularly ones for new machines or
architectures) to the 
\begin{rawhtml} <A href=''mailto:MITgcm-support@mitgcm.org"> \end{rawhtml}
MITgcm-support@mitgcm.org
\begin{rawhtml} </A> \end{rawhtml}
list.

To specify an optfile to {\em genmake2}, the syntax is:
\begin{verbatim}
% ../../../tools/genmake2 -mods=../code -of /path/to/optfile
\end{verbatim}

Once a {\em Makefile} has been generated, we create the dependencies:
\begin{verbatim}
% make depend
\end{verbatim}
This modifies the {\em Makefile} by attaching a [long] list of files
upon which other files depend. The purpose of this is to reduce
re-compilation if and when you start to modify the code. The {\tt make
  depend} command also creates links from the model source to this
directory.

Next compile the code:
\begin{verbatim}
% make
\end{verbatim}
The {\tt make} command creates an executable called \textit{mitgcmuv}.
Additional make ``targets'' are defined within the makefile to aid in
the production of adjoint and other versions of MITgcm.

Now you are ready to run the model. General instructions for doing so are
given in section \ref{sect:runModel}. Here, we can run the model with:
\begin{verbatim}
./mitgcmuv > output.txt
\end{verbatim}
where we are re-directing the stream of text output to the file {\em
output.txt}.


\subsection{Building/compiling the code elsewhere}

In the example above (section \ref{sect:buildingCode}) we built the
executable in the {\em input} directory of the experiment for
convenience. You can also configure and compile the code in other
locations, for example on a scratch disk with out having to copy the
entire source tree. The only requirement to do so is you have {\tt
  genmake2} in your path or you know the absolute path to {\tt
  genmake2}.

The following sections outline some possible methods of organizing
your source and data.

\subsubsection{Building from the {\em ../code directory}}

This is just as simple as building in the {\em input/} directory:
\begin{verbatim}
% cd verification/exp2/code
% ../../../tools/genmake2
% make depend
% make
\end{verbatim}
However, to run the model the executable ({\em mitgcmuv}) and input
files must be in the same place. If you only have one calculation to make:
\begin{verbatim}
% cd ../input
% cp ../code/mitgcmuv ./
% ./mitgcmuv > output.txt
\end{verbatim}
or if you will be making multiple runs with the same executable:
\begin{verbatim}
% cd ../
% cp -r input run1
% cp code/mitgcmuv run1
% cd run1
% ./mitgcmuv > output.txt
\end{verbatim}

\subsubsection{Building from a new directory}

Since the {\em input} directory contains input files it is often more
useful to keep {\em input} pristine and build in a new directory
within {\em verification/exp2/}:
\begin{verbatim}
% cd verification/exp2
% mkdir build
% cd build
% ../../../tools/genmake2 -mods=../code
% make depend
% make
\end{verbatim}
This builds the code exactly as before but this time you need to copy
either the executable or the input files or both in order to run the
model. For example,
\begin{verbatim}
% cp ../input/* ./
% ./mitgcmuv > output.txt
\end{verbatim}
or if you tend to make multiple runs with the same executable then
running in a new directory each time might be more appropriate:
\begin{verbatim}
% cd ../
% mkdir run1
% cp build/mitgcmuv run1/
% cp input/* run1/
% cd run1
% ./mitgcmuv > output.txt
\end{verbatim}

\subsubsection{Building on a scratch disk}

Model object files and output data can use up large amounts of disk
space so it is often the case that you will be operating on a large
scratch disk. Assuming the model source is in {\em ~/MITgcm} then the
following commands will build the model in {\em /scratch/exp2-run1}:
\begin{verbatim}
% cd /scratch/exp2-run1
% ~/MITgcm/tools/genmake2 -rootdir=~/MITgcm \
  -mods=~/MITgcm/verification/exp2/code
% make depend
% make
\end{verbatim}
To run the model here, you'll need the input files:
\begin{verbatim}
% cp ~/MITgcm/verification/exp2/input/* ./
% ./mitgcmuv > output.txt
\end{verbatim}

As before, you could build in one directory and make multiple runs of
the one experiment:
\begin{verbatim}
% cd /scratch/exp2
% mkdir build
% cd build
% ~/MITgcm/tools/genmake2 -rootdir=~/MITgcm \
  -mods=~/MITgcm/verification/exp2/code
% make depend
% make
% cd ../
% cp -r ~/MITgcm/verification/exp2/input run2
% cd run2
% ./mitgcmuv > output.txt
\end{verbatim}


\subsection{Using \texttt{genmake2}}
\label{sect:genmake}

To compile the code, first use the program \texttt{genmake2} (located
in the \texttt{tools} directory) to generate a Makefile.
\texttt{genmake2} is a shell script written to work with all
``sh''--compatible shells including bash v1, bash v2, and Bourne.
Internally, \texttt{genmake2} determines the locations of needed
files, the compiler, compiler options, libraries, and Unix tools.  It
relies upon a number of ``optfiles'' located in the
\texttt{tools/build\_options} directory.

The purpose of the optfiles is to provide all the compilation options
for particular ``platforms'' (where ``platform'' roughly means the
combination of the hardware and the compiler) and code configurations.
Given the combinations of possible compilers and library dependencies
({\it eg.}  MPI and NetCDF) there may be numerous optfiles available
for a single machine.  The naming scheme for the majority of the
optfiles shipped with the code is
\begin{center}
  {\bf OS\_HARDWARE\_COMPILER }
\end{center}
where
\begin{description}
\item[OS] is the name of the operating system (generally the
  lower-case output of the {\tt 'uname'} command)
\item[HARDWARE] is a string that describes the CPU type and
  corresponds to output from the  {\tt 'uname -m'} command:
  \begin{description}
  \item[ia32] is for ``x86'' machines such as i386, i486, i586, i686,
    and athlon
  \item[ia64] is for Intel IA64 systems (eg. Itanium, Itanium2)
  \item[amd64] is AMD x86\_64 systems
  \item[ppc] is for Mac PowerPC systems
  \end{description}
\item[COMPILER] is the compiler name (generally, the name of the
  FORTRAN executable)
\end{description}

In many cases, the default optfiles are sufficient and will result in
usable Makefiles.  However, for some machines or code configurations,
new ``optfiles'' must be written. To create a new optfile, it is
generally best to start with one of the defaults and modify it to suit
your needs.  Like \texttt{genmake2}, the optfiles are all written
using a simple ``sh''--compatible syntax.  While nearly all variables
used within \texttt{genmake2} may be specified in the optfiles, the
critical ones that should be defined are:

\begin{description}
\item[FC] the FORTRAN compiler (executable) to use
\item[DEFINES] the command-line DEFINE options passed to the compiler
\item[CPP] the C pre-processor to use
\item[NOOPTFLAGS] options flags for special files that should not be
  optimized
\end{description}

For example, the optfile for a typical Red Hat Linux machine (``ia32''
architecture) using the GCC (g77) compiler is
\begin{verbatim}
FC=g77
DEFINES='-D_BYTESWAPIO -DWORDLENGTH=4'
CPP='cpp  -traditional -P'
NOOPTFLAGS='-O0'
#  For IEEE, use the "-ffloat-store" option
if test "x$IEEE" = x ; then
    FFLAGS='-Wimplicit -Wunused -Wuninitialized'
    FOPTIM='-O3 -malign-double -funroll-loops'
else
    FFLAGS='-Wimplicit -Wunused -ffloat-store'
    FOPTIM='-O0 -malign-double'
fi
\end{verbatim}

If you write an optfile for an unrepresented machine or compiler, you
are strongly encouraged to submit the optfile to the MITgcm project
for inclusion.  Please send the file to the
\begin{rawhtml} <A href="mail-to:MITgcm-support@mitgcm.org"> \end{rawhtml}
\begin{center}
  MITgcm-support@mitgcm.org
\end{center}
\begin{rawhtml} </A> \end{rawhtml}
mailing list.

In addition to the optfiles, \texttt{genmake2} supports a number of
helpful command-line options.  A complete list of these options can be
obtained from:
\begin{verbatim}
% genmake2 -h
\end{verbatim}

The most important command-line options are:
\begin{description}
  
\item[\texttt{--optfile=/PATH/FILENAME}] specifies the optfile that
  should be used for a particular build.
  
  If no "optfile" is specified (either through the command line or the
  MITGCM\_OPTFILE environment variable), genmake2 will try to make a
  reasonable guess from the list provided in {\em
    tools/build\_options}.  The method used for making this guess is
  to first determine the combination of operating system and hardware
  (eg. "linux\_ia32") and then find a working FORTRAN compiler within
  the user's path.  When these three items have been identified,
  genmake2 will try to find an optfile that has a matching name.
  
\item[\texttt{--pdefault='PKG1 PKG2 PKG3 ...'}] specifies the default
  set of packages to be used.  The normal order of precedence for
  packages is as follows:
  \begin{enumerate}
  \item If available, the command line (\texttt{--pdefault}) settings
    over-rule any others.

  \item Next, \texttt{genmake2} will look for a file named
    ``\texttt{packages.conf}'' in the local directory or in any of the
    directories specified with the \texttt{--mods} option.
    
  \item Finally, if neither of the above are available,
    \texttt{genmake2} will use the \texttt{/pkg/pkg\_default} file.
  \end{enumerate}
  
\item[\texttt{--pdepend=/PATH/FILENAME}] specifies the dependency file
  used for packages.
  
  If not specified, the default dependency file {\em pkg/pkg\_depend}
  is used.  The syntax for this file is parsed on a line-by-line basis
  where each line containes either a comment ("\#") or a simple
  "PKGNAME1 (+|-)PKGNAME2" pairwise rule where the "+" or "-" symbol
  specifies a "must be used with" or a "must not be used with"
  relationship, respectively.  If no rule is specified, then it is
  assumed that the two packages are compatible and will function
  either with or without each other.
  
\item[\texttt{--adof=/path/to/file}] specifies the "adjoint" or
  automatic differentiation options file to be used.  The file is
  analogous to the ``optfile'' defined above but it specifies
  information for the AD build process.
  
  The default file is located in {\em
    tools/adjoint\_options/adjoint\_default} and it defines the "TAF"
  and "TAMC" compilers.  An alternate version is also available at
  {\em tools/adjoint\_options/adjoint\_staf} that selects the newer
  "STAF" compiler.  As with any compilers, it is helpful to have their
  directories listed in your {\tt \$PATH} environment variable.
  
\item[\texttt{--mods='DIR1 DIR2 DIR3 ...'}] specifies a list of
  directories containing ``modifications''.  These directories contain
  files with names that may (or may not) exist in the main MITgcm
  source tree but will be overridden by any identically-named sources
  within the ``MODS'' directories.
  
  The order of precedence for this "name-hiding" is as follows:
  \begin{itemize}
  \item ``MODS'' directories (in the order given)
  \item Packages either explicitly specified or provided by default
    (in the order given)
  \item Packages included due to package dependencies (in the order
    that that package dependencies are parsed)
  \item The "standard dirs" (which may have been specified by the
    ``-standarddirs'' option)
  \end{itemize}
  
\item[\texttt{--mpi}] This option enables certain MPI features (using
  CPP \texttt{\#define}s) within the code and is necessary for MPI
  builds (see Section \ref{sect:mpi-build}).
  
\item[\texttt{--make=/path/to/gmake}] Due to the poor handling of
  soft-links and other bugs common with the \texttt{make} versions
  provided by commercial Unix vendors, GNU \texttt{make} (sometimes
  called \texttt{gmake}) should be preferred.  This option provides a
  means for specifying the make executable to be used.
  
\item[\texttt{--bash=/path/to/sh}] On some (usually older UNIX)
  machines, the ``bash'' shell is unavailable.  To run on these
  systems, \texttt{genmake2} can be invoked using an ``sh'' (that is,
  a Bourne, POSIX, or compatible) shell.  The syntax in these
  circumstances is:
  \begin{center}
    \texttt{\%  /bin/sh genmake2 -bash=/bin/sh [...options...]}
  \end{center}
  where \texttt{/bin/sh} can be replaced with the full path and name
  of the desired shell.

\end{description}


\subsection{Building with MPI}
\label{sect:mpi-build}

Building MITgcm to use MPI libraries can be complicated due to the
variety of different MPI implementations available, their dependencies
or interactions with different compilers, and their often ad-hoc
locations within file systems.  For these reasons, its generally a
good idea to start by finding and reading the documentation for your
machine(s) and, if necessary, seeking help from your local systems
administrator.

The steps for building MITgcm with MPI support are:
\begin{enumerate}
  
\item Determine the locations of your MPI-enabled compiler and/or MPI
  libraries and put them into an options file as described in Section
  \ref{sect:genmake}.  One can start with one of the examples in:
  \begin{rawhtml} <A
    href="http://mitgcm.org/cgi-bin/viewcvs.cgi/MITgcm/tools/build_options/">
  \end{rawhtml}
  \begin{center}
    \texttt{MITgcm/tools/build\_options/}
  \end{center}
  \begin{rawhtml} </A> \end{rawhtml}
  such as \texttt{linux\_ia32\_g77+mpi\_cg01} or
  \texttt{linux\_ia64\_efc+mpi} and then edit it to suit the machine at
  hand.  You may need help from your user guide or local systems
  administrator to determine the exact location of the MPI libraries.
  If libraries are not installed, MPI implementations and related
  tools are available including:
  \begin{itemize}
  \item \begin{rawhtml} <A
      href="http://www-unix.mcs.anl.gov/mpi/mpich/">
    \end{rawhtml}
    MPICH
    \begin{rawhtml} </A> \end{rawhtml}

  \item \begin{rawhtml} <A
      href="http://www.lam-mpi.org/">
    \end{rawhtml}
    LAM/MPI
    \begin{rawhtml} </A> \end{rawhtml}

  \item \begin{rawhtml} <A
      href="http://www.osc.edu/~pw/mpiexec/">
    \end{rawhtml}
    MPIexec
    \begin{rawhtml} </A> \end{rawhtml}
  \end{itemize}
  
\item Build the code with the \texttt{genmake2} \texttt{-mpi} option
  (see Section \ref{sect:genmake}) using commands such as:
{\footnotesize \begin{verbatim}
  %  ../../../tools/genmake2 -mods=../code -mpi -of=YOUR_OPTFILE
  %  make depend
  %  make
\end{verbatim} }
  
\item Run the code with the appropriate MPI ``run'' or ``exec''
  program provided with your particular implementation of MPI.
  Typical MPI packages such as MPICH will use something like:
\begin{verbatim}
  %  mpirun -np 4 -machinefile mf ./mitgcmuv
\end{verbatim}
  Sightly more complicated scripts may be needed for many machines
  since execution of the code may be controlled by both the MPI
  library and a job scheduling and queueing system such as PBS,
  LoadLeveller, Condor, or any of a number of similar tools.  A few
  example scripts (those used for our \begin{rawhtml} <A
    href="http://mitgcm.org/testing.html"> \end{rawhtml}regular
  verification runs\begin{rawhtml} </A> \end{rawhtml}) are available
  at:
  \begin{rawhtml} <A
    href="http://mitgcm.org/cgi-bin/viewcvs.cgi/MITgcm_contrib/test_scripts/">
  \end{rawhtml}
  {\footnotesize \tt
    http://mitgcm.org/cgi-bin/viewcvs.cgi/MITgcm\_contrib/test\_scripts/ }
  \begin{rawhtml} </A> \end{rawhtml}

\end{enumerate}



\section{Running the model}
\label{sect:runModel}

If compilation finished succesfuully (section \ref{sect:buildingCode})
then an executable called \texttt{mitgcmuv} will now exist in the
local directory.

To run the model as a single process (ie. not in parallel) simply
type:
\begin{verbatim}
% ./mitgcmuv
\end{verbatim}
The ``./'' is a safe-guard to make sure you use the local executable
in case you have others that exist in your path (surely odd if you
do!). The above command will spew out many lines of text output to
your screen.  This output contains details such as parameter values as
well as diagnostics such as mean Kinetic energy, largest CFL number,
etc. It is worth keeping this text output with the binary output so we
normally re-direct the {\em stdout} stream as follows:
\begin{verbatim}
% ./mitgcmuv > output.txt
\end{verbatim}

For the example experiments in {\em verification}, an example of the
output is kept in {\em results/output.txt} for comparison. You can compare
your {\em output.txt} with this one to check that the set-up works.



\subsection{Output files}

The model produces various output files. At a minimum, the instantaneous
``state'' of the model is written out, which is made of the following files:

\begin{itemize}
\item \textit{U.00000nIter} - zonal component of velocity field (m/s and $>
0 $ eastward).

\item \textit{V.00000nIter} - meridional component of velocity field (m/s
and $> 0$ northward).

\item \textit{W.00000nIter} - vertical component of velocity field (ocean:
m/s and $> 0$ upward, atmosphere: Pa/s and $> 0$ towards increasing pressure
i.e. downward).

\item \textit{T.00000nIter} - potential temperature (ocean: $^{0}$C,
atmosphere: $^{0}$K).

\item \textit{S.00000nIter} - ocean: salinity (psu), atmosphere: water vapor
(g/kg).

\item \textit{Eta.00000nIter} - ocean: surface elevation (m), atmosphere:
surface pressure anomaly (Pa).
\end{itemize}

The chain \textit{00000nIter} consists of ten figures that specify the
iteration number at which the output is written out. For example, \textit{%
U.0000000300} is the zonal velocity at iteration 300.

In addition, a ``pickup'' or ``checkpoint'' file called:

\begin{itemize}
\item \textit{pickup.00000nIter}
\end{itemize}

is written out. This file represents the state of the model in a condensed
form and is used for restarting the integration. If the C-D scheme is used,
there is an additional ``pickup'' file:

\begin{itemize}
\item \textit{pickup\_cd.00000nIter}
\end{itemize}

containing the D-grid velocity data and that has to be written out as well
in order to restart the integration. Rolling checkpoint files are the same
as the pickup files but are named differently. Their name contain the chain 
\textit{ckptA} or \textit{ckptB} instead of \textit{00000nIter}. They can be
used to restart the model but are overwritten every other time they are
output to save disk space during long integrations.

\subsection{Looking at the output}

All the model data are written according to a ``meta/data'' file format.
Each variable is associated with two files with suffix names \textit{.data}
and \textit{.meta}. The \textit{.data} file contains the data written in
binary form (big\_endian by default). The \textit{.meta} file is a
``header'' file that contains information about the size and the structure
of the \textit{.data} file. This way of organizing the output is
particularly useful when running multi-processors calculations. The base
version of the model includes a few matlab utilities to read output files
written in this format. The matlab scripts are located in the directory 
\textit{utils/matlab} under the root tree. The script \textit{rdmds.m} reads
the data. Look at the comments inside the script to see how to use it.

Some examples of reading and visualizing some output in {\em Matlab}:
\begin{verbatim}
% matlab
>> H=rdmds('Depth');
>> contourf(H');colorbar;
>> title('Depth of fluid as used by model');

>> eta=rdmds('Eta',10);
>> imagesc(eta');axis ij;colorbar;
>> title('Surface height at iter=10');

>> eta=rdmds('Eta',[0:10:100]);
>> for n=1:11; imagesc(eta(:,:,n)');axis ij;colorbar;pause(.5);end
\end{verbatim}

\section{Doing it yourself: customizing the code}

When you are ready to run the model in the configuration you want, the
easiest thing is to use and adapt the setup of the case studies
experiment (described previously) that is the closest to your
configuration. Then, the amount of setup will be minimized. In this
section, we focus on the setup relative to the ``numerical model''
part of the code (the setup relative to the ``execution environment''
part is covered in the parallel implementation section) and on the
variables and parameters that you are likely to change.

\subsection{Configuration and setup}

The CPP keys relative to the ``numerical model'' part of the code are
all defined and set in the file \textit{CPP\_OPTIONS.h }in the
directory \textit{ model/inc }or in one of the \textit{code
}directories of the case study experiments under
\textit{verification.} The model parameters are defined and declared
in the file \textit{model/inc/PARAMS.h }and their default values are
set in the routine \textit{model/src/set\_defaults.F. }The default
values can be modified in the namelist file \textit{data }which needs
to be located in the directory where you will run the model. The
parameters are initialized in the routine
\textit{model/src/ini\_parms.F}.  Look at this routine to see in what
part of the namelist the parameters are located.

In what follows the parameters are grouped into categories related to
the computational domain, the equations solved in the model, and the
simulation controls.

\subsection{Computational domain, geometry and time-discretization}

\begin{description}
\item[dimensions] \ 
  
  The number of points in the x, y, and r directions are represented
  by the variables \textbf{sNx}, \textbf{sNy} and \textbf{Nr}
  respectively which are declared and set in the file
  \textit{model/inc/SIZE.h}.  (Again, this assumes a mono-processor
  calculation. For multiprocessor calculations see the section on
  parallel implementation.)

\item[grid] \ 
  
  Three different grids are available: cartesian, spherical polar, and
  curvilinear (which includes the cubed sphere). The grid is set
  through the logical variables \textbf{usingCartesianGrid},
  \textbf{usingSphericalPolarGrid}, and \textbf{usingCurvilinearGrid}.
  In the case of spherical and curvilinear grids, the southern
  boundary is defined through the variable \textbf{phiMin} which
  corresponds to the latitude of the southern most cell face (in
  degrees). The resolution along the x and y directions is controlled
  by the 1D arrays \textbf{delx} and \textbf{dely} (in meters in the
  case of a cartesian grid, in degrees otherwise).  The vertical grid
  spacing is set through the 1D array \textbf{delz} for the ocean (in
  meters) or \textbf{delp} for the atmosphere (in Pa).  The variable
  \textbf{Ro\_SeaLevel} represents the standard position of Sea-Level
  in ``R'' coordinate. This is typically set to 0m for the ocean
  (default value) and 10$^{5}$Pa for the atmosphere. For the
  atmosphere, also set the logical variable \textbf{groundAtK1} to
  \texttt{'.TRUE.'} which puts the first level (k=1) at the lower
  boundary (ground).
  
  For the cartesian grid case, the Coriolis parameter $f$ is set
  through the variables \textbf{f0} and \textbf{beta} which correspond
  to the reference Coriolis parameter (in s$^{-1}$) and
  $\frac{\partial f}{ \partial y}$(in m$^{-1}$s$^{-1}$) respectively.
  If \textbf{beta } is set to a nonzero value, \textbf{f0} is the
  value of $f$ at the southern edge of the domain.

\item[topography - full and partial cells] \ 
  
  The domain bathymetry is read from a file that contains a 2D (x,y)
  map of depths (in m) for the ocean or pressures (in Pa) for the
  atmosphere. The file name is represented by the variable
  \textbf{bathyFile}. The file is assumed to contain binary numbers
  giving the depth (pressure) of the model at each grid cell, ordered
  with the x coordinate varying fastest. The points are ordered from
  low coordinate to high coordinate for both axes. The model code
  applies without modification to enclosed, periodic, and double
  periodic domains. Periodicity is assumed by default and is
  suppressed by setting the depths to 0m for the cells at the limits
  of the computational domain (note: not sure this is the case for the
  atmosphere). The precision with which to read the binary data is
  controlled by the integer variable \textbf{readBinaryPrec} which can
  take the value \texttt{32} (single precision) or \texttt{64} (double
  precision). See the matlab program \textit{gendata.m} in the
  \textit{input} directories under \textit{verification} to see how
  the bathymetry files are generated for the case study experiments.
  
  To use the partial cell capability, the variable \textbf{hFacMin}
  needs to be set to a value between 0 and 1 (it is set to 1 by
  default) corresponding to the minimum fractional size of the cell.
  For example if the bottom cell is 500m thick and \textbf{hFacMin} is
  set to 0.1, the actual thickness of the cell (i.e. used in the code)
  can cover a range of discrete values 50m apart from 50m to 500m
  depending on the value of the bottom depth (in \textbf{bathyFile})
  at this point.
  
  Note that the bottom depths (or pressures) need not coincide with
  the models levels as deduced from \textbf{delz} or \textbf{delp}.
  The model will interpolate the numbers in \textbf{bathyFile} so that
  they match the levels obtained from \textbf{delz} or \textbf{delp}
  and \textbf{hFacMin}.
  
  (Note: the atmospheric case is a bit more complicated than what is
  written here I think. To come soon...)

\item[time-discretization] \ 
  
  The time steps are set through the real variables \textbf{deltaTMom}
  and \textbf{deltaTtracer} (in s) which represent the time step for
  the momentum and tracer equations, respectively. For synchronous
  integrations, simply set the two variables to the same value (or you
  can prescribe one time step only through the variable
  \textbf{deltaT}). The Adams-Bashforth stabilizing parameter is set
  through the variable \textbf{abEps} (dimensionless). The stagger
  baroclinic time stepping can be activated by setting the logical
  variable \textbf{staggerTimeStep} to \texttt{'.TRUE.'}.

\end{description}


\subsection{Equation of state}

First, because the model equations are written in terms of
perturbations, a reference thermodynamic state needs to be specified.
This is done through the 1D arrays \textbf{tRef} and \textbf{sRef}.
\textbf{tRef} specifies the reference potential temperature profile
(in $^{o}$C for the ocean and $^{o}$K for the atmosphere) starting
from the level k=1. Similarly, \textbf{sRef} specifies the reference
salinity profile (in ppt) for the ocean or the reference specific
humidity profile (in g/kg) for the atmosphere.

The form of the equation of state is controlled by the character
variables \textbf{buoyancyRelation} and \textbf{eosType}.
\textbf{buoyancyRelation} is set to \texttt{'OCEANIC'} by default and
needs to be set to \texttt{'ATMOSPHERIC'} for atmosphere simulations.
In this case, \textbf{eosType} must be set to \texttt{'IDEALGAS'}.
For the ocean, two forms of the equation of state are available:
linear (set \textbf{eosType} to \texttt{'LINEAR'}) and a polynomial
approximation to the full nonlinear equation ( set \textbf{eosType} to
\texttt{'POLYNOMIAL'}). In the linear case, you need to specify the
thermal and haline expansion coefficients represented by the variables
\textbf{tAlpha} (in K$^{-1}$) and \textbf{sBeta} (in ppt$^{-1}$). For
the nonlinear case, you need to generate a file of polynomial
coefficients called \textit{POLY3.COEFFS}. To do this, use the program
\textit{utils/knudsen2/knudsen2.f} under the model tree (a Makefile is
available in the same directory and you will need to edit the number
and the values of the vertical levels in \textit{knudsen2.f} so that
they match those of your configuration).

There there are also higher polynomials for the equation of state:
\begin{description}
\item[\texttt{'UNESCO'}:] The UNESCO equation of state formula of
  Fofonoff and Millard \cite{fofonoff83}. This equation of state
  assumes in-situ temperature, which is not a model variable; {\em its
    use is therefore discouraged, and it is only listed for
    completeness}.
\item[\texttt{'JMD95Z'}:] A modified UNESCO formula by Jackett and
  McDougall \cite{jackett95}, which uses the model variable potential
  temperature as input. The \texttt{'Z'} indicates that this equation
  of state uses a horizontally and temporally constant pressure
  $p_{0}=-g\rho_{0}z$. 
\item[\texttt{'JMD95P'}:] A modified UNESCO formula by Jackett and
  McDougall \cite{jackett95}, which uses the model variable potential
  temperature as input. The \texttt{'P'} indicates that this equation
  of state uses the actual hydrostatic pressure of the last time
  step. Lagging the pressure in this way requires an additional pickup
  file for restarts.
\item[\texttt{'MDJWF'}:] The new, more accurate and less expensive
  equation of state by McDougall et~al. \cite{mcdougall03}. It also
  requires lagging the pressure and therefore an additional pickup
  file for restarts.
\end{description}
For none of these options an reference profile of temperature or
salinity is required.

\subsection{Momentum equations}

In this section, we only focus for now on the parameters that you are
likely to change, i.e. the ones relative to forcing and dissipation
for example.  The details relevant to the vector-invariant form of the
equations and the various advection schemes are not covered for the
moment. We assume that you use the standard form of the momentum
equations (i.e. the flux-form) with the default advection scheme.
Also, there are a few logical variables that allow you to turn on/off
various terms in the momentum equation. These variables are called
\textbf{momViscosity, momAdvection, momForcing, useCoriolis,
  momPressureForcing, momStepping} and \textbf{metricTerms }and are
assumed to be set to \texttt{'.TRUE.'} here.  Look at the file
\textit{model/inc/PARAMS.h }for a precise definition of these
variables.

\begin{description}
\item[initialization] \ 
  
  The velocity components are initialized to 0 unless the simulation
  is starting from a pickup file (see section on simulation control
  parameters).

\item[forcing] \ 
  
  This section only applies to the ocean. You need to generate
  wind-stress data into two files \textbf{zonalWindFile} and
  \textbf{meridWindFile} corresponding to the zonal and meridional
  components of the wind stress, respectively (if you want the stress
  to be along the direction of only one of the model horizontal axes,
  you only need to generate one file). The format of the files is
  similar to the bathymetry file. The zonal (meridional) stress data
  are assumed to be in Pa and located at U-points (V-points). As for
  the bathymetry, the precision with which to read the binary data is
  controlled by the variable \textbf{readBinaryPrec}.  See the matlab
  program \textit{gendata.m} in the \textit{input} directories under
  \textit{verification} to see how simple analytical wind forcing data
  are generated for the case study experiments.
  
  There is also the possibility of prescribing time-dependent periodic
  forcing. To do this, concatenate the successive time records into a
  single file (for each stress component) ordered in a (x,y,t) fashion
  and set the following variables: \textbf{periodicExternalForcing }to
  \texttt{'.TRUE.'}, \textbf{externForcingPeriod }to the period (in s)
  of which the forcing varies (typically 1 month), and
  \textbf{externForcingCycle} to the repeat time (in s) of the forcing
  (typically 1 year -- note: \textbf{ externForcingCycle} must be a
  multiple of \textbf{externForcingPeriod}).  With these variables set
  up, the model will interpolate the forcing linearly at each
  iteration.

\item[dissipation] \ 
  
  The lateral eddy viscosity coefficient is specified through the
  variable \textbf{viscAh} (in m$^{2}$s$^{-1}$). The vertical eddy
  viscosity coefficient is specified through the variable
  \textbf{viscAz} (in m$^{2}$s$^{-1}$) for the ocean and
  \textbf{viscAp} (in Pa$^{2}$s$^{-1}$) for the atmosphere.  The
  vertical diffusive fluxes can be computed implicitly by setting the
  logical variable \textbf{implicitViscosity }to \texttt{'.TRUE.'}.
  In addition, biharmonic mixing can be added as well through the
  variable \textbf{viscA4} (in m$^{4}$s$^{-1}$). On a spherical polar
  grid, you might also need to set the variable \textbf{cosPower}
  which is set to 0 by default and which represents the power of
  cosine of latitude to multiply viscosity. Slip or no-slip conditions
  at lateral and bottom boundaries are specified through the logical
  variables \textbf{no\_slip\_sides} and \textbf{no\_slip\_bottom}. If
  set to \texttt{'.FALSE.'}, free-slip boundary conditions are
  applied. If no-slip boundary conditions are applied at the bottom, a
  bottom drag can be applied as well. Two forms are available: linear
  (set the variable \textbf{bottomDragLinear} in s$ ^{-1}$) and
  quadratic (set the variable \textbf{bottomDragQuadratic} in
  m$^{-1}$).

  The Fourier and Shapiro filters are described elsewhere.

\item[C-D scheme] \ 
  
  If you run at a sufficiently coarse resolution, you will need the
  C-D scheme for the computation of the Coriolis terms. The
  variable\textbf{\ tauCD}, which represents the C-D scheme coupling
  timescale (in s) needs to be set.
  
\item[calculation of pressure/geopotential] \ 
  
  First, to run a non-hydrostatic ocean simulation, set the logical
  variable \textbf{nonHydrostatic} to \texttt{'.TRUE.'}. The pressure
  field is then inverted through a 3D elliptic equation. (Note: this
  capability is not available for the atmosphere yet.) By default, a
  hydrostatic simulation is assumed and a 2D elliptic equation is used
  to invert the pressure field. The parameters controlling the
  behaviour of the elliptic solvers are the variables
  \textbf{cg2dMaxIters} and \textbf{cg2dTargetResidual } for
  the 2D case and \textbf{cg3dMaxIters} and
  \textbf{cg3dTargetResidual} for the 3D case. You probably won't need to
  alter the default values (are we sure of this?).
  
  For the calculation of the surface pressure (for the ocean) or
  surface geopotential (for the atmosphere) you need to set the
  logical variables \textbf{rigidLid} and \textbf{implicitFreeSurface}
  (set one to \texttt{'.TRUE.'} and the other to \texttt{'.FALSE.'}
  depending on how you want to deal with the ocean upper or atmosphere
  lower boundary).

\end{description} 

\subsection{Tracer equations}

This section covers the tracer equations i.e. the potential
temperature equation and the salinity (for the ocean) or specific
humidity (for the atmosphere) equation. As for the momentum equations,
we only describe for now the parameters that you are likely to change.
The logical variables \textbf{tempDiffusion} \textbf{tempAdvection}
\textbf{tempForcing}, and \textbf{tempStepping} allow you to turn
on/off terms in the temperature equation (same thing for salinity or
specific humidity with variables \textbf{saltDiffusion},
\textbf{saltAdvection} etc.). These variables are all assumed here to
be set to \texttt{'.TRUE.'}. Look at file \textit{model/inc/PARAMS.h}
for a precise definition.

\begin{description}
\item[initialization] \ 
  
  The initial tracer data can be contained in the binary files
  \textbf{hydrogThetaFile} and \textbf{hydrogSaltFile}. These files
  should contain 3D data ordered in an (x,y,r) fashion with k=1 as the
  first vertical level.  If no file names are provided, the tracers
  are then initialized with the values of \textbf{tRef} and
  \textbf{sRef} mentioned above (in the equation of state section). In
  this case, the initial tracer data are uniform in x and y for each
  depth level.

\item[forcing] \ 
  
  This part is more relevant for the ocean, the procedure for the
  atmosphere not being completely stabilized at the moment.
  
  A combination of fluxes data and relaxation terms can be used for
  driving the tracer equations.  For potential temperature, heat flux
  data (in W/m$ ^{2}$) can be stored in the 2D binary file
  \textbf{surfQfile}.  Alternatively or in addition, the forcing can
  be specified through a relaxation term. The SST data to which the
  model surface temperatures are restored to are supposed to be stored
  in the 2D binary file \textbf{thetaClimFile}. The corresponding
  relaxation time scale coefficient is set through the variable
  \textbf{tauThetaClimRelax} (in s). The same procedure applies for
  salinity with the variable names \textbf{EmPmRfile},
  \textbf{saltClimFile}, and \textbf{tauSaltClimRelax} for freshwater
  flux (in m/s) and surface salinity (in ppt) data files and
  relaxation time scale coefficient (in s), respectively. Also for
  salinity, if the CPP key \textbf{USE\_NATURAL\_BCS} is turned on,
  natural boundary conditions are applied i.e. when computing the
  surface salinity tendency, the freshwater flux is multiplied by the
  model surface salinity instead of a constant salinity value.
  
  As for the other input files, the precision with which to read the
  data is controlled by the variable \textbf{readBinaryPrec}.
  Time-dependent, periodic forcing can be applied as well following
  the same procedure used for the wind forcing data (see above).

\item[dissipation] \ 
  
  Lateral eddy diffusivities for temperature and salinity/specific
  humidity are specified through the variables \textbf{diffKhT} and
  \textbf{diffKhS} (in m$^{2}$/s). Vertical eddy diffusivities are
  specified through the variables \textbf{diffKzT} and
  \textbf{diffKzS} (in m$^{2}$/s) for the ocean and \textbf{diffKpT
  }and \textbf{diffKpS} (in Pa$^{2}$/s) for the atmosphere. The
  vertical diffusive fluxes can be computed implicitly by setting the
  logical variable \textbf{implicitDiffusion} to \texttt{'.TRUE.'}.
  In addition, biharmonic diffusivities can be specified as well
  through the coefficients \textbf{diffK4T} and \textbf{diffK4S} (in
  m$^{4}$/s). Note that the cosine power scaling (specified through
  \textbf{cosPower}---see the momentum equations section) is applied to
  the tracer diffusivities (Laplacian and biharmonic) as well. The
  Gent and McWilliams parameterization for oceanic tracers is
  described in the package section. Finally, note that tracers can be
  also subject to Fourier and Shapiro filtering (see the corresponding
  section on these filters).

\item[ocean convection] \ 
  
  Two options are available to parameterize ocean convection: one is
  to use the convective adjustment scheme. In this case, you need to
  set the variable \textbf{cadjFreq}, which represents the frequency
  (in s) with which the adjustment algorithm is called, to a non-zero
  value (if set to a negative value by the user, the model will set it
  to the tracer time step). The other option is to parameterize
  convection with implicit vertical diffusion. To do this, set the
  logical variable \textbf{implicitDiffusion} to \texttt{'.TRUE.'}
  and the real variable \textbf{ivdc\_kappa} to a value (in m$^{2}$/s)
  you wish the tracer vertical diffusivities to have when mixing
  tracers vertically due to static instabilities. Note that
  \textbf{cadjFreq} and \textbf{ivdc\_kappa}can not both have non-zero
  value.

\end{description}

\subsection{Simulation controls}

The model ''clock'' is defined by the variable \textbf{deltaTClock}
(in s) which determines the IO frequencies and is used in tagging
output.  Typically, you will set it to the tracer time step for
accelerated runs (otherwise it is simply set to the default time step
\textbf{deltaT}).  Frequency of checkpointing and dumping of the model
state are referenced to this clock (see below).

\begin{description}
\item[run duration] \ 
  
  The beginning of a simulation is set by specifying a start time (in
  s) through the real variable \textbf{startTime} or by specifying an
  initial iteration number through the integer variable
  \textbf{nIter0}. If these variables are set to nonzero values, the
  model will look for a ''pickup'' file \textit{pickup.0000nIter0} to
  restart the integration. The end of a simulation is set through the
  real variable \textbf{endTime} (in s).  Alternatively, you can
  specify instead the number of time steps to execute through the
  integer variable \textbf{nTimeSteps}.

\item[frequency of output] \
  
  Real variables defining frequencies (in s) with which output files
  are written on disk need to be set up. \textbf{dumpFreq} controls
  the frequency with which the instantaneous state of the model is
  saved. \textbf{chkPtFreq} and \textbf{pchkPtFreq} control the output
  frequency of rolling and permanent checkpoint files, respectively.
  See section 1.5.1 Output files for the definition of model state and
  checkpoint files. In addition, time-averaged fields can be written
  out by setting the variable \textbf{taveFreq} (in s).  The precision
  with which to write the binary data is controlled by the integer
  variable w\textbf{riteBinaryPrec} (set it to \texttt{32} or
  \texttt{64}).

\end{description}


%%% Local Variables: 
%%% mode: latex
%%% TeX-master: t
%%% End: 
