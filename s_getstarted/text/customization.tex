\section[Customizing MITgcm]{Doing it yourself: customizing the model configuration}
\label{sect:customize}
\begin{rawhtml}
<!-- CMIREDIR:customizing_mitgcm: -->
\end{rawhtml}

When you are ready to run the model in the configuration you want, the
easiest thing is to use and adapt the setup of the case studies
experiment (described previously) that is the closest to your
configuration. Then, the amount of setup will be minimized. In this
section, we focus on the setup relative to the ``numerical model''
part of the code (the setup relative to the ``execution environment''
part is covered in the parallel implementation section) and on the
variables and parameters that you are likely to change.

The CPP keys relative to the ``numerical model'' part of the code are
all defined and set in the file \textit{CPP\_OPTIONS.h }in the
directory \textit{ model/inc }or in one of the \textit{code
}directories of the case study experiments under
\textit{verification.} The model parameters are defined and declared
in the file \textit{model/inc/PARAMS.h }and their default values are
set in the routine \textit{model/src/set\_defaults.F. }The default
values can be modified in the namelist file \textit{data }which needs
to be located in the directory where you will run the model. The
parameters are initialized in the routine
\textit{model/src/ini\_parms.F}.  Look at this routine to see in what
part of the namelist the parameters are located. Here is a complete list 
of the model parameters related to the main model (namelist parameters
for the packages are located in the package descriptions), their meaning, 
and their default values:

\newpage

\begin{table}
%\hspace*{-1.5in}
\begin{tabular}{lllc}

  \textbf{Name}  &  \textbf{value}  
    &  \textbf{Description}   &  \textbf{Reference}  \\
  & & & \\

   buoyancyRelation   &   
    &   buoyancyRelation = OCEANIC
    &  %---ref---
    \\
   fluidIsAir    &                     F
    &   fluid major constituent is Air 
    &  %---ref---
    \\
   fluidIsWater  &                     T
    &   fuild major constituent is Water 
    &  %---ref---
    \\
   usingPCoords   &                     F
    &   use p (or p
    &  %---ref---
    \\
   usingZCoords   &                     T
    &   use z (or z
    &  %---ref---
    \\
   tRef   &      2.0E+01 at K= top
    &   Reference temperature profile ( oC or K ) 
    &  %---ref---
    \\
   sRef   &      3.0E+01  at K= top
    &   Reference salinity profile ( psu ) 
    &  %---ref---
    \\
   viscAh   &                   0.0E+00
    &   Lateral eddy viscosity ( $m^2/s$ ) 
    &  %---ref---
    \\
   viscAhMax   &                   1.0E+21
    &   Maximum lateral eddy viscosity ( $m^2/s$ ) 
    &  %---ref---
    \\
   viscAhGrid   &                   0.0E+00
    &   Grid dependent lateral eddy viscosity ( non-dim. ) 
    &  %---ref---
    \\
   useFullLeith   &                     F
    &   Use Full Form of Leith Viscosity on/off flag
    &  %---ref---
    \\
   useStrainTensionVisc   &                     F
    &   Use StrainTension Form of Viscous Operator on/off flag
    &  %---ref---
    \\
   useAreaViscLength   &                     F
    &   Use area for visc length instead of geom. mean
    &  %---ref---
    \\
   viscC2leith   &                   0.0E+00
    &   Leith harmonic visc. factor (on grad(vort),non-dim.) 
    &  %---ref---
    \\
   viscC2leithD   &                   0.0E+00
    &   Leith harmonic viscosity factor (on grad(div),non-dim.) 
    &  %---ref---
    \\
   viscC2smag   &                   0.0E+00
    &   Smagorinsky harmonic viscosity factor (non-dim.) 
    &  %---ref---
    \\
   viscA4   &                   0.0E+00
    &   Lateral biharmonic viscosity ( $m^4/s$ ) 
    &  %---ref---
    \\
   viscA4Max   &                   1.0E+21
    &   Maximum biharmonic viscosity ( $m^2/s$ ) 
    &  %---ref---
    \\
   viscA4Grid   &                   0.0E+00
    &   Grid dependent biharmonic viscosity ( non-dim. ) 
    &  %---ref---
    \\
   viscC4leith   &                   0.0E+00
    &   Leith biharm viscosity factor (on grad(vort), non-dim.) 
    &  %---ref---
    \\
   viscC4leithD   &                   0.0E+00
    &   Leith biharm viscosity factor (on grad(div), non-dim.) 
    &  %---ref---
    \\
   viscC4Smag   &                   0.0E+00
    &   Smagorinsky biharm viscosity factor (non-dim) 
    &  %---ref---
    \\
   no\_slip\_sides   &                     T
    &   Viscous BCs: No-slip sides 
    &  %---ref---
    \\
   sideDragFactor   &                   2.0E+00
    &   side-drag scaling factor (non-dim) 
    &  %---ref---
    \\
   viscAr   &                   0.0E+00
    &   Vertical eddy viscosity ( units of $r^2/s$ ) 
    &  %---ref---
    \\
   no\_slip\_bottom   &                     T
    &   Viscous BCs: No-slip bottom 
    &  %---ref---
    \\
   bottomDragLinear   &                   0.0E+00
    &   linear bottom-drag coefficient ( 1/s ) 
    &  %---ref---
    \\
   bottomDragQuadratic   &                   0.0E+00
    &   quadratic bottom-drag coeff. ( 1/m ) 
    &  %---ref---
    \\
   diffKhT   &                   0.0E+00
    &   Laplacian diffusion of heat laterally ( $m^2/s$ ) 
    &  %---ref---
    \\
   diffK4T   &                   0.0E+00
    &   Bihaarmonic diffusion of heat laterally ( $m^4/s$ ) 
    &  %---ref---
    \\
   diffKhS   &                   0.0E+00
    &   Laplacian diffusion of salt laterally ( $m^2/s$ ) 
    &  %---ref---
    \\
   diffK4S   &                   0.0E+00
    &   Bihaarmonic diffusion of salt laterally ( $m^4/s$ ) 
    &  %---ref---
    \\
   diffKrNrT   &      0.0E+00 at K= top
    &   vertical profile of vertical diffusion of Temp ( $m^2/s$ )
    &  %---ref---
    \\
   diffKrNrS   &      0.0E+00 at K= top
    &   vertical profile of vertical diffusion of Salt ( $m^2/s$ )
    &  %---ref---
    \\
   diffKrBL79surf   &                   0.0E+00
    &   Surface diffusion for Bryan and Lewis 1979 ( $m^2/s$ ) 
    &  %---ref---
    \\
   diffKrBL79deep   &                   0.0E+00
    &   Deep diffusion for Bryan and Lewis 1979 ( $m^2/s$ ) 
    &  %---ref---
    \\
   diffKrBL79scl   &                   2.0E+02
    &   Depth scale for Bryan and Lewis 1979 ( m ) 
    &  %---ref---
    \\
   diffKrBL79Ho   &                  -2.0E+03
    &   Turning depth for Bryan and Lewis 1979 ( m ) 
    &  %---ref---
    \\
   eosType   &  LINEAR 
    &   Equation of State
    &  %---ref---
    \\
   tAlpha   &                   2.0E-04
    &   Linear EOS thermal expansion coefficient ( 1/oC ) 
    &  %---ref---
    \\
\end{tabular}
\end{table}

\newpage
\begin{table}
%\hspace*{-1.5in}
\begin{tabular}{lllc}

  \textbf{Name}  &  \textbf{value}  
    &  \textbf{Description}   &  \textbf{Reference}  \\
  & & & \\
   sBeta   &                   7.4E-04
    &   Linear EOS haline contraction coef ( 1/psu ) 
    &  %---ref---
    \\
   rhonil   &                   9.998E+02
    &   Reference density ( $kg/m^3$ ) 
    &  %---ref---
    \\
   rhoConst   &                   9.998E+02
    &   Reference density ( $kg/m^3$ ) 
    &  %---ref---
    \\
   rhoConstFresh   &                   9.998E+02
    &   Reference density ( $kg/m^3$ ) 
    &  %---ref---
    \\
   gravity   &                   9.81E+00
    &   Gravitational acceleration ( $m/s^2$ ) 
    &  %---ref---
    \\
   gBaro   &                   9.81E+00
    &   Barotropic gravity ( $m/s^2$ ) 
    &  %---ref---
    \\
   rotationPeriod   &                   8.6164E+04
    &   Rotation Period ( s ) 
    &  %---ref---
    \\
   omega   &                   7.292123516990375E-05
    &   Angular velocity ( rad/s ) 
    &  %---ref---
    \\
   f0   &                   1.0E-04
    &   Reference coriolis parameter ( 1/s ) 
    &  %---ref---
    \\
   beta   &                   9.999999999999999E-12
    &   Beta ( 1/(m.s) ) 
    &  %---ref---
    \\
   freeSurfFac   &                   1.0E+00
    &   Implicit free surface factor 
    &  %---ref---
    \\
   implicitFreeSurface   &                     T
    &   Implicit free surface on/off flag 
    &  %---ref---
    \\
   rigidLid   &                     F
    &   Rigid lid on/off flag 
    &  %---ref---
    \\
   implicSurfPress   &                   1.0E+00
    &   Surface Pressure implicit factor (0-1)
    &  %---ref---
    \\
   implicDiv2Dflow   &                   1.0E+00
    &   Barot. Flow Div. implicit factor (0-1)
    &  %---ref---
    \\
   exactConserv   &                     F
    &   Exact Volume Conservation on/off flag
    &  %---ref---
    \\
   uniformLin\_PhiSurf   &                     T
    &   use uniform Bo\_surf on/off flag
    &  %---ref---
    \\
   nonlinFreeSurf   &                         0
    &   Non-linear Free Surf. options (-1,0,1,2,3)
    &  %---ref---
    \\
   hFacInf   &                   2.0E-01
    &   lower threshold for hFac (nonlinFreeSurf only)
    &  %---ref---
    \\
   hFacSup   &                   2.0E+00
    &   upper threshold for hFac (nonlinFreeSurf only)
    &  %---ref---
    \\
   select\_rStar   &                         0
    &   r
    &  %---ref---
    \\
   useRealFreshWaterFlux   &                     F
    &   Real Fresh Water Flux on/off flag
    &  %---ref---
    \\
   convertFW2Salt   &                   3.5E+01
    &   convert F.W. Flux to Salt Flux (-1=use local S)
    &  %---ref---
    \\
   use3Dsolver   &                     F
    &   use 3-D pressure solver on/off flag 
    &  %---ref---
    \\
   nonHydrostatic   &                     F
    &   Non-Hydrostatic on/off flag 
    &  %---ref---
    \\
   nh\_Am2   &                   1.0E+00
    &   Non-Hydrostatic terms scaling factor 
    &  %---ref---
    \\
   quasiHydrostatic   &                     F
    &   Quasi-Hydrostatic on/off flag 
    &  %---ref---
    \\
   momStepping   &                     T
    &   Momentum equation on/off flag 
    &  %---ref---
    \\
   vectorInvariantMomentum  &                     F
    &   Vector-Invariant Momentum on/off 
    &  %---ref---
    \\
   momAdvection   &                     T
    &   Momentum advection on/off flag 
    &  %---ref---
    \\
   momViscosity   &                     T
    &   Momentum viscosity on/off flag 
    &  %---ref---
    \\
   momImplVertAdv   &                     F
    &   Momentum implicit vert. advection on/off
    &  %---ref---
    \\
   implicitViscosity   &                     F
    &   Implicit viscosity on/off flag 
    &  %---ref---
    \\
   metricTerms   &                     F
    &   metric-Terms on/off flag 
    &  %---ref---
    \\
   useNHMTerms   &                     F
    &   Non-Hydrostatic Metric-Terms on/off 
    &  %---ref---
    \\
   useCoriolis   &                     T
    &   Coriolis on/off flag 
    &  %---ref---
    \\
   useCDscheme   &                     F
    &   CD scheme on/off flag 
    &  %---ref---
    \\
   useJamartWetPoints  &                     F
    &   Coriolis WetPoints method flag 
    &  %---ref---
    \\
   useJamartMomAdv  &                     F
    &   V.I. Non-linear terms Jamart flag 
    &  %---ref---
    \\
\end{tabular}
\end{table}

\newpage
\begin{table}
%\hspace*{-1.5in}
\begin{tabular}{lllc}

  \textbf{Name}  &  \textbf{value}  
    &  \textbf{Description}   &  \textbf{Reference}  \\
  & & & \\
   SadournyCoriolis  &                     F
    &   Sadourny Coriolis discr. flag 
    &  %---ref---
    \\
   upwindVorticity  &                     F
    &   Upwind bias vorticity flag 
    &  %---ref---
    \\
   useAbsVorticity  &                     F
    &   Work with f
    &  %---ref---
    \\
   highOrderVorticity  &                     F
    &   High order interp. of vort. flag 
    &  %---ref---
    \\
   upwindShear  &                     F
    &   Upwind vertical Shear advection flag 
    &  %---ref---
    \\
   selectKEscheme  &                         0
    &   Kinetic Energy scheme selector 
    &  %---ref---
    \\
   momForcing   &                     T
    &   Momentum forcing on/off flag 
    &  %---ref---
    \\
   momPressureForcing   &                     T
    &   Momentum pressure term on/off flag 
    &  %---ref---
    \\
   implicitIntGravWave  &                     F
    &   Implicit Internal Gravity Wave flag 
    &  %---ref---
    \\
   staggerTimeStep   &                     F
    &   Stagger time stepping on/off flag 
    &  %---ref---
    \\
   multiDimAdvection   &                     T
    &   enable/disable Multi-Dim Advection 
    &  %---ref---
    \\
   useMultiDimAdvec   &                     F
    &   Multi-Dim Advection is/is-not used 
    &  %---ref---
    \\
   implicitDiffusion   &                     F
    &   Implicit Diffusion on/off flag 
    &  %---ref---
    \\
   tempStepping   &                     T
    &   Temperature equation on/off flag 
    &  %---ref---
    \\
   tempAdvection  &                     T
    &   Temperature advection on/off flag 
    &  %---ref---
    \\
   tempImplVertAdv   &                     F
    &   Temp. implicit vert. advection on/off 
    &  %---ref---
    \\
   tempForcing    &                     T
    &   Temperature forcing on/off flag 
    &  %---ref---
    \\
   saltStepping   &                     T
    &   Salinity equation on/off flag 
    &  %---ref---
    \\
   saltAdvection  &                     T
    &   Salinity advection on/off flag 
    &  %---ref---
    \\
   saltImplVertAdv   &                     F
    &   Sali. implicit vert. advection on/off 
    &  %---ref---
    \\
   saltForcing    &                     T
    &   Salinity forcing on/off flag 
    &  %---ref---
    \\
    readBinaryPrec   &                        32
    &   Precision used for reading binary files 
    &  %---ref---
    \\
   writeBinaryPrec   &                        32
    &   Precision used for writing binary files 
    &  %---ref---
    \\
    globalFiles   &                     F
    &   write "global" (=not per tile) files 
    &  %---ref---
    \\
    useSingleCpuIO   &                     F
    &   only master MPI process does I/O 
    &  %---ref---
    \\
    debugMode    &                     F
    &   Debug Mode on/off flag 
    &  %---ref---
    \\
      debLevA    &                         1
    &   1rst level of debugging 
    &  %---ref---
    \\
      debLevB    &                         2
    &   2nd  level of debugging 
    &  %---ref---
    \\
    debugLevel   &                         1
    &   select debugging level 
    &  %---ref---
    \\
   cg2dMaxIters   &                       150
    &   Upper limit on 2d con. grad iterations  
    &  %---ref---
    \\
   cg2dChkResFreq   &                         1
    &   2d con. grad convergence test frequency 
    &  %---ref---
    \\
   cg2dTargetResidual   &                   1.0E-07
    &   2d con. grad target residual  
    &  %---ref---
    \\
   cg2dTargetResWunit   &                  -1.0E+00
    &   CG2d target residual [W units] 
    &  %---ref---
    \\
   cg2dPreCondFreq   &                         1
    &   Freq. for updating cg2d preconditioner 
    &  %---ref---
    \\
   nIter0   &                         0
    &   Run starting timestep number  
    &  %---ref---
    \\
   nTimeSteps   &                         0
    &   Number of timesteps 
    &  %---ref---
    \\
   deltatTmom   &                   6.0E+01
    &   Momentum equation timestep ( s ) 
    &  %---ref---
    \\
   deltaTfreesurf   &                   6.0E+01
    &   FreeSurface equation timestep ( s ) 
    &  %---ref---
    \\
   dTtracerLev   &      6.0E+01 at K= top
    &   Tracer equation timestep ( s ) 
    &  %---ref---
    \\
   deltatTClock    &                   6.0E+01
    &   Model clock timestep ( s ) 
    &  %---ref---
    \\
\end{tabular}
\end{table}

\newpage
\begin{table}
%\hspace*{-1.5in}
\begin{tabular}{lllc}

  \textbf{Name}  &  \textbf{value}  
    &  \textbf{Description}   &  \textbf{Reference}  \\
  & & & \\
   cAdjFreq   &                   0.0E+00
    &   Convective adjustment interval ( s ) 
    &  %---ref---
    \\
   momForcingOutAB   &                         0
    &   =1: take Momentum Forcing out of Adams-Bash.
    &  %---ref---
    \\
   tracForcingOutAB   &                         0
    &   =1: take T,S,pTr Forcing out of Adams-Bash.
    &  %---ref---
    \\
   momDissip\_In\_AB   &                     T
    &   put Dissipation Tendency in Adams-Bash. 
    &  %---ref---
    \\
   doAB\_onGtGs   &                     T
    &   apply AB on Tendencies (rather than on T,S)
    &  %---ref---
    \\
   abEps   &                   1.0E-02
    &   Adams-Bashforth-2 stabilizing weight 
    &  %---ref---
    \\
   baseTime   &                   0.0E+00
    &   Model base time ( s ). 
    &  %---ref---
    \\
   startTime   &                   0.0E+00
    &   Run start time ( s ). 
    &  %---ref---
    \\
   endTime   &                   0.0E+00
    &   Integration ending time ( s ). 
    &  %---ref---
    \\
   pChkPtFreq   &                   0.0E+00
    &   Permanent restart/checkpoint file interval ( s ). 
    &  %---ref---
    \\
   chkPtFreq   &                   0.0E+00
    &   Rolling restart/checkpoint file interval ( s ). 
    &  %---ref---
    \\
   pickup\_write\_mdsio   &                     T
    &   Model IO flag. 
    &  %---ref---
    \\
   pickup\_read\_mdsio   &                     T
    &   Model IO flag. 
    &  %---ref---
    \\
   pickup\_write\_immed   &                     F
    &   Model IO flag. 
    &  %---ref---
    \\
   dumpFreq   &                   0.0E+00
    &   Model state write out interval ( s ). 
    &  %---ref---
    \\
   dumpInitAndLast  &                     T
    &   write out Initial and Last iter. model state 
    &  %---ref---
    \\
   snapshot\_mdsio   &                     T
    &   Model IO flag. 
    &  %---ref---
    \\
   monitorFreq   &                   6.0E+01
    &   Monitor output interval ( s ). 
    &  %---ref---
    \\
   monitor\_stdio   &                     T
    &   Model IO flag. 
    &  %---ref---
    \\
   externForcingPeriod   &                   0.0E+00
    &   forcing period (s) 
    &  %---ref---
    \\
   externForcingCycle   &                   0.0E+00
    &   period of the cyle (s). 
    &  %---ref---
    \\
   tauThetaClimRelax   &                   0.0E+00
    &   relaxation time scale (s) 
    &  %---ref---
    \\
   tauSaltClimRelax   &                   0.0E+00
    &   relaxation time scale (s) 
    &  %---ref---
    \\
   latBandClimRelax   &                   3.703701E+05
    &   max. Lat. where relaxation 
    &  %---ref---
    \\
   usingCartesianGrid   &                     T
    &   Cartesian coordinates flag ( True / False ) 
    &  %---ref---
    \\
   usingSphericalPolarGrid   &                     F
    &   Spherical coordinates flag ( True / False ) 
    &  %---ref---
    \\
   usingCylindricalGrid   &                     F
    &   Spherical coordinates flag ( True / False ) 
    &  %---ref---
    \\
   Ro\_SeaLevel   &                   0.0E+00
    &   r(1) ( units of r ) 
    &  %---ref---
    \\
   rkSign   &                  -1.0E+00
    &   index orientation relative to vertical coordinate 
    &  %---ref---
    \\
   horiVertRatio   &                   1.0E+00
    &   Ratio on units : Horiz - Vertical 
    &  %---ref---
    \\
   drC   &                   5.0E+03 at K=1
    &   C spacing ( units of r ) 
    &  %---ref---
    \\
   drF   &      1.0E+04 at K= top
    &   W spacing ( units of r ) 
    &  %---ref---
    \\
   delX   &      1.234567E+05  at I= east
    &   U spacing ( m - cartesian, degrees - spherical ) 
    &  %---ref---
    \\
   delY   &                   1.234567E+05  at J=1 
    &   V spacing ( m - cartesian, degrees - spherical ) 
    &  %---ref---
    \\
   phiMin   &                   0.0E+00
    &   South edge (ignored - cartesian, degrees - spherical ) 
    &  %---ref---
    \\
   thetaMin   &                   0.0E+00
    &   West edge ( ignored - cartesian, degrees - spherical ) 
    &  %---ref---
    \\
   rSphere   &                   6.37E+06
    &   Radius ( ignored - cartesian, m - spherical ) 
    &  %---ref---
    \\
   xcoord   &                   6.172835E+04 at I=1
    &   P-point X coord ( m - cartesian, degrees - spherical ) 
    &  %---ref---
    \\
   ycoord   &                   6.172835E+04 at J=1
    &   P-point Y coord ( m - cartesian, degrees - spherical ) 
    &  %---ref---
    \\
   rcoord   &                  -5.0E+03 at K=1
    &   P-point R coordinate (  units of r ) 
    &  %---ref---
    \\
   rF   &                   0.0E+00 at K=1
    &   W-Interf. R coordinate (  units of r ) 
    &  %---ref---
    \\
   dBdrRef   &      0.0E+00  at K= top
    &   Vertical gradient of reference boyancy [$(m/s/r)^2)$] 
    &  %---ref---
    \\
\end{tabular}
\end{table}

\newpage
\begin{table}
%\hspace*{-1.5in}
\begin{tabular}{lllc}

  \textbf{Name}  &  \textbf{value}  
    &  \textbf{Description}   &  \textbf{Reference}  \\
  & & & \\
   dxF   &       1.234567E+05  at K= top
    &   dxF(:,1,:,1) ( m - cartesian, degrees - spherical ) 
    &  %---ref---
    \\
   dyF   &      1.234567E+05  at I= east
    &   dyF(:,1,:,1) ( m - cartesian, degrees - spherical ) 
    &  %---ref---
    \\
   dxG   &      1.234567E+05   at I= east
    &   dxG(:,1,:,1) ( m - cartesian, degrees - spherical ) 
    &  %---ref---
    \\
   dyG   &      1.234567E+05   at I= east
    &   dyG(:,1,:,1) ( m - cartesian, degrees - spherical ) 
    &  %---ref---
    \\
   dxC   &      1.234567E+05  at I= east
    &   dxC(:,1,:,1) ( m - cartesian, degrees - spherical ) 
    &  %---ref---
    \\
   dyC   &      1.234567E+05  at I= east
    &   dyC(:,1,:,1) ( m - cartesian, degrees - spherical ) 
    &  %---ref---
    \\
   dxV   &      1.234567E+05   at I= east
    &   dxV(:,1,:,1) ( m - cartesian, degrees - spherical ) 
    &  %---ref---
    \\
   dyU   &      1.234567E+05   at I= east
    &   dyU(:,1,:,1) ( m - cartesian, degrees - spherical ) 
    &  %---ref---
    \\
   rA   &      1.524155677489E+10  at I= east
    &   rA(:,1,:,1) ( m - cartesian, degrees - spherical ) 
    &  %---ref---
    \\
   rAw   &      1.524155677489E+10   at K= top
    &   rAw(:,1,:,1) ( m - cartesian, degrees - spherical ) 
    &  %---ref---
    \\
   rAs   &      1.524155677489E+10   at K= top
    &   rAs(:,1,:,1) ( m - cartesian, degrees - spherical ) 
    &  %---ref---
    \\

\end{tabular}
\end{table}

\begin{table}
%\hspace*{-1.5in}
\begin{tabular}{lllc}

  \textbf{Name}  &  \textbf{Default value}  
    &  \textbf{Description}   &  \textbf{Reference}  \\
  & & & \\

   tempAdvScheme   &                         2
    &   Temp. Horiz.Advection scheme selector 
    &  %---ref---
    \\
   tempVertAdvScheme   &                         2
    &   Temp. Vert. Advection scheme selector 
    &  %---ref---
    \\
   tempMultiDimAdvec   &                     F
    &   use Muti-Dim Advec method for Temp 
    &  %---ref---
    \\
   tempAdamsBashforth   &                     T
    &   use Adams-Bashforth time-stepping for Temp 
    &  %---ref---
    \\
   saltAdvScheme   &                         2
    &   Salt. Horiz.advection scheme selector 
    &  %---ref---
    \\
   saltVertAdvScheme   &                         2
    &   Salt. Vert. Advection scheme selector 
    &  %---ref---
    \\
   saltMultiDimAdvec   &                     F
    &   use Muti-Dim Advec method for Salt 
    &  %---ref---
    \\
   saltAdamsBashforth   &                     T
    &   use Adams-Bashforth time-stepping for Salt 
    &  %---ref---
    \\

\end{tabular}
\end{table}



In what follows the parameters are grouped into categories related to
the computational domain, the equations solved in the model, and the
simulation controls.

\subsection{Parameters: Computational domain, geometry and time-discretization}

\begin{description}
\item[dimensions] \ 
  
  The number of points in the x, y, and r directions are represented
  by the variables \textbf{sNx}, \textbf{sNy} and \textbf{Nr}
  respectively which are declared and set in the file
  \textit{model/inc/SIZE.h}.  (Again, this assumes a mono-processor
  calculation. For multiprocessor calculations see the section on
  parallel implementation.)

\item[grid] \ 
  
  Three different grids are available: cartesian, spherical polar, and
  curvilinear (which includes the cubed sphere). The grid is set
  through the logical variables \textbf{usingCartesianGrid},
  \textbf{usingSphericalPolarGrid}, and \textbf{usingCurvilinearGrid}.
  In the case of spherical and curvilinear grids, the southern
  boundary is defined through the variable \textbf{phiMin} which
  corresponds to the latitude of the southern most cell face (in
  degrees). The resolution along the x and y directions is controlled
  by the 1D arrays \textbf{delx} and \textbf{dely} (in meters in the
  case of a cartesian grid, in degrees otherwise).  The vertical grid
  spacing is set through the 1D array \textbf{delz} for the ocean (in
  meters) or \textbf{delp} for the atmosphere (in Pa).  The variable
  \textbf{Ro\_SeaLevel} represents the standard position of Sea-Level
  in ``R'' coordinate. This is typically set to 0m for the ocean
  (default value) and 10$^{5}$Pa for the atmosphere. For the
  atmosphere, also set the logical variable \textbf{groundAtK1} to
  \texttt{'.TRUE.'} which puts the first level (k=1) at the lower
  boundary (ground).
  
  For the cartesian grid case, the Coriolis parameter $f$ is set
  through the variables \textbf{f0} and \textbf{beta} which correspond
  to the reference Coriolis parameter (in s$^{-1}$) and
  $\frac{\partial f}{ \partial y}$(in m$^{-1}$s$^{-1}$) respectively.
  If \textbf{beta } is set to a nonzero value, \textbf{f0} is the
  value of $f$ at the southern edge of the domain.

\item[topography - full and partial cells] \ 
  
  The domain bathymetry is read from a file that contains a 2D (x,y)
  map of depths (in m) for the ocean or pressures (in Pa) for the
  atmosphere. The file name is represented by the variable
  \textbf{bathyFile}. The file is assumed to contain binary numbers
  giving the depth (pressure) of the model at each grid cell, ordered
  with the x coordinate varying fastest. The points are ordered from
  low coordinate to high coordinate for both axes. The model code
  applies without modification to enclosed, periodic, and double
  periodic domains. Periodicity is assumed by default and is
  suppressed by setting the depths to 0m for the cells at the limits
  of the computational domain (note: not sure this is the case for the
  atmosphere). The precision with which to read the binary data is
  controlled by the integer variable \textbf{readBinaryPrec} which can
  take the value \texttt{32} (single precision) or \texttt{64} (double
  precision). See the matlab program \textit{gendata.m} in the
  \textit{input} directories under \textit{verification} to see how
  the bathymetry files are generated for the case study experiments.
  
  To use the partial cell capability, the variable \textbf{hFacMin}
  needs to be set to a value between 0 and 1 (it is set to 1 by
  default) corresponding to the minimum fractional size of the cell.
  For example if the bottom cell is 500m thick and \textbf{hFacMin} is
  set to 0.1, the actual thickness of the cell (i.e. used in the code)
  can cover a range of discrete values 50m apart from 50m to 500m
  depending on the value of the bottom depth (in \textbf{bathyFile})
  at this point.
  
  Note that the bottom depths (or pressures) need not coincide with
  the models levels as deduced from \textbf{delz} or \textbf{delp}.
  The model will interpolate the numbers in \textbf{bathyFile} so that
  they match the levels obtained from \textbf{delz} or \textbf{delp}
  and \textbf{hFacMin}.
  
  (Note: the atmospheric case is a bit more complicated than what is
  written here I think. To come soon...)

\item[time-discretization] \ 
  
  The time steps are set through the real variables \textbf{deltaTMom}
  and \textbf{deltaTtracer} (in s) which represent the time step for
  the momentum and tracer equations, respectively. For synchronous
  integrations, simply set the two variables to the same value (or you
  can prescribe one time step only through the variable
  \textbf{deltaT}). The Adams-Bashforth stabilizing parameter is set
  through the variable \textbf{abEps} (dimensionless). The stagger
  baroclinic time stepping can be activated by setting the logical
  variable \textbf{staggerTimeStep} to \texttt{'.TRUE.'}.

\end{description}


\subsection{Parameters: Equation of state}

First, because the model equations are written in terms of
perturbations, a reference thermodynamic state needs to be specified.
This is done through the 1D arrays \textbf{tRef} and \textbf{sRef}.
\textbf{tRef} specifies the reference potential temperature profile
(in $^{o}$C for the ocean and $^{o}$K for the atmosphere) starting
from the level k=1. Similarly, \textbf{sRef} specifies the reference
salinity profile (in ppt) for the ocean or the reference specific
humidity profile (in g/kg) for the atmosphere.

The form of the equation of state is controlled by the character
variables \textbf{buoyancyRelation} and \textbf{eosType}.
\textbf{buoyancyRelation} is set to \texttt{'OCEANIC'} by default and
needs to be set to \texttt{'ATMOSPHERIC'} for atmosphere simulations.
In this case, \textbf{eosType} must be set to \texttt{'IDEALGAS'}.
For the ocean, two forms of the equation of state are available:
linear (set \textbf{eosType} to \texttt{'LINEAR'}) and a polynomial
approximation to the full nonlinear equation ( set \textbf{eosType} to
\texttt{'POLYNOMIAL'}). In the linear case, you need to specify the
thermal and haline expansion coefficients represented by the variables
\textbf{tAlpha} (in K$^{-1}$) and \textbf{sBeta} (in ppt$^{-1}$). For
the nonlinear case, you need to generate a file of polynomial
coefficients called \textit{POLY3.COEFFS}. To do this, use the program
\textit{utils/knudsen2/knudsen2.f} under the model tree (a Makefile is
available in the same directory and you will need to edit the number
and the values of the vertical levels in \textit{knudsen2.f} so that
they match those of your configuration).

There there are also higher polynomials for the equation of state:
\begin{description}
\item[\texttt{'UNESCO'}:] The UNESCO equation of state formula of
  Fofonoff and Millard \cite{fofonoff83}. This equation of state
  assumes in-situ temperature, which is not a model variable; {\em its
    use is therefore discouraged, and it is only listed for
    completeness}.
\item[\texttt{'JMD95Z'}:] A modified UNESCO formula by Jackett and
  McDougall \cite{jackett95}, which uses the model variable potential
  temperature as input. The \texttt{'Z'} indicates that this equation
  of state uses a horizontally and temporally constant pressure
  $p_{0}=-g\rho_{0}z$. 
\item[\texttt{'JMD95P'}:] A modified UNESCO formula by Jackett and
  McDougall \cite{jackett95}, which uses the model variable potential
  temperature as input. The \texttt{'P'} indicates that this equation
  of state uses the actual hydrostatic pressure of the last time
  step. Lagging the pressure in this way requires an additional pickup
  file for restarts.
\item[\texttt{'MDJWF'}:] The new, more accurate and less expensive
  equation of state by McDougall et~al. \cite{mcdougall03}. It also
  requires lagging the pressure and therefore an additional pickup
  file for restarts.
\end{description}
For none of these options an reference profile of temperature or
salinity is required.

\subsection{Parameters: Momentum equations}

In this section, we only focus for now on the parameters that you are
likely to change, i.e. the ones relative to forcing and dissipation
for example.  The details relevant to the vector-invariant form of the
equations and the various advection schemes are not covered for the
moment. We assume that you use the standard form of the momentum
equations (i.e. the flux-form) with the default advection scheme.
Also, there are a few logical variables that allow you to turn on/off
various terms in the momentum equation. These variables are called
\textbf{momViscosity, momAdvection, momForcing, useCoriolis,
  momPressureForcing, momStepping} and \textbf{metricTerms }and are
assumed to be set to \texttt{'.TRUE.'} here.  Look at the file
\textit{model/inc/PARAMS.h }for a precise definition of these
variables.

\begin{description}
\item[initialization] \ 
  
  The velocity components are initialized to 0 unless the simulation
  is starting from a pickup file (see section on simulation control
  parameters).

\item[forcing] \ 
  
  This section only applies to the ocean. You need to generate
  wind-stress data into two files \textbf{zonalWindFile} and
  \textbf{meridWindFile} corresponding to the zonal and meridional
  components of the wind stress, respectively (if you want the stress
  to be along the direction of only one of the model horizontal axes,
  you only need to generate one file). The format of the files is
  similar to the bathymetry file. The zonal (meridional) stress data
  are assumed to be in Pa and located at U-points (V-points). As for
  the bathymetry, the precision with which to read the binary data is
  controlled by the variable \textbf{readBinaryPrec}.  See the matlab
  program \textit{gendata.m} in the \textit{input} directories under
  \textit{verification} to see how simple analytical wind forcing data
  are generated for the case study experiments.
  
  There is also the possibility of prescribing time-dependent periodic
  forcing. To do this, concatenate the successive time records into a
  single file (for each stress component) ordered in a (x,y,t) fashion
  and set the following variables: \textbf{periodicExternalForcing }to
  \texttt{'.TRUE.'}, \textbf{externForcingPeriod }to the period (in s)
  of which the forcing varies (typically 1 month), and
  \textbf{externForcingCycle} to the repeat time (in s) of the forcing
  (typically 1 year -- note: \textbf{ externForcingCycle} must be a
  multiple of \textbf{externForcingPeriod}).  With these variables set
  up, the model will interpolate the forcing linearly at each
  iteration.

\item[dissipation] \ 
  
  The lateral eddy viscosity coefficient is specified through the
  variable \textbf{viscAh} (in m$^{2}$s$^{-1}$). The vertical eddy
  viscosity coefficient is specified through the variable
  \textbf{viscAz} (in m$^{2}$s$^{-1}$) for the ocean and
  \textbf{viscAp} (in Pa$^{2}$s$^{-1}$) for the atmosphere.  The
  vertical diffusive fluxes can be computed implicitly by setting the
  logical variable \textbf{implicitViscosity }to \texttt{'.TRUE.'}.
  In addition, biharmonic mixing can be added as well through the
  variable \textbf{viscA4} (in m$^{4}$s$^{-1}$). On a spherical polar
  grid, you might also need to set the variable \textbf{cosPower}
  which is set to 0 by default and which represents the power of
  cosine of latitude to multiply viscosity. Slip or no-slip conditions
  at lateral and bottom boundaries are specified through the logical
  variables \textbf{no\_slip\_sides} and \textbf{no\_slip\_bottom}. If
  set to \texttt{'.FALSE.'}, free-slip boundary conditions are
  applied. If no-slip boundary conditions are applied at the bottom, a
  bottom drag can be applied as well. Two forms are available: linear
  (set the variable \textbf{bottomDragLinear} in m/s) and
  quadratic (set the variable \textbf{bottomDragQuadratic}, dimensionless).

  The Fourier and Shapiro filters are described elsewhere.

\item[C-D scheme] \ 
  
  If you run at a sufficiently coarse resolution, you will need the
  C-D scheme for the computation of the Coriolis terms. The
  variable\textbf{\ tauCD}, which represents the C-D scheme coupling
  timescale (in s) needs to be set.
  
\item[calculation of pressure/geopotential] \ 
  
  First, to run a non-hydrostatic ocean simulation, set the logical
  variable \textbf{nonHydrostatic} to \texttt{'.TRUE.'}. The pressure
  field is then inverted through a 3D elliptic equation. (Note: this
  capability is not available for the atmosphere yet.) By default, a
  hydrostatic simulation is assumed and a 2D elliptic equation is used
  to invert the pressure field. The parameters controlling the
  behaviour of the elliptic solvers are the variables
  \textbf{cg2dMaxIters} and \textbf{cg2dTargetResidual } for
  the 2D case and \textbf{cg3dMaxIters} and
  \textbf{cg3dTargetResidual} for the 3D case. You probably won't need to
  alter the default values (are we sure of this?).
  
  For the calculation of the surface pressure (for the ocean) or
  surface geopotential (for the atmosphere) you need to set the
  logical variables \textbf{rigidLid} and \textbf{implicitFreeSurface}
  (set one to \texttt{'.TRUE.'} and the other to \texttt{'.FALSE.'}
  depending on how you want to deal with the ocean upper or atmosphere
  lower boundary).

\end{description} 

\subsection{Parameters: Tracer equations}

This section covers the tracer equations i.e. the potential
temperature equation and the salinity (for the ocean) or specific
humidity (for the atmosphere) equation. As for the momentum equations,
we only describe for now the parameters that you are likely to change.
The logical variables \textbf{tempDiffusion} \textbf{tempAdvection}
\textbf{tempForcing}, and \textbf{tempStepping} allow you to turn
on/off terms in the temperature equation (same thing for salinity or
specific humidity with variables \textbf{saltDiffusion},
\textbf{saltAdvection} etc.). These variables are all assumed here to
be set to \texttt{'.TRUE.'}. Look at file \textit{model/inc/PARAMS.h}
for a precise definition.

\begin{description}
\item[initialization] \ 
  
  The initial tracer data can be contained in the binary files
  \textbf{hydrogThetaFile} and \textbf{hydrogSaltFile}. These files
  should contain 3D data ordered in an (x,y,r) fashion with k=1 as the
  first vertical level.  If no file names are provided, the tracers
  are then initialized with the values of \textbf{tRef} and
  \textbf{sRef} mentioned above (in the equation of state section). In
  this case, the initial tracer data are uniform in x and y for each
  depth level.

\item[forcing] \ 
  
  This part is more relevant for the ocean, the procedure for the
  atmosphere not being completely stabilized at the moment.
  
  A combination of fluxes data and relaxation terms can be used for
  driving the tracer equations.  For potential temperature, heat flux
  data (in W/m$ ^{2}$) can be stored in the 2D binary file
  \textbf{surfQfile}.  Alternatively or in addition, the forcing can
  be specified through a relaxation term. The SST data to which the
  model surface temperatures are restored to are supposed to be stored
  in the 2D binary file \textbf{thetaClimFile}. The corresponding
  relaxation time scale coefficient is set through the variable
  \textbf{tauThetaClimRelax} (in s). The same procedure applies for
  salinity with the variable names \textbf{EmPmRfile},
  \textbf{saltClimFile}, and \textbf{tauSaltClimRelax} for freshwater
  flux (in m/s) and surface salinity (in ppt) data files and
  relaxation time scale coefficient (in s), respectively. Also for
  salinity, if the CPP key \textbf{USE\_NATURAL\_BCS} is turned on,
  natural boundary conditions are applied i.e. when computing the
  surface salinity tendency, the freshwater flux is multiplied by the
  model surface salinity instead of a constant salinity value.
  
  As for the other input files, the precision with which to read the
  data is controlled by the variable \textbf{readBinaryPrec}.
  Time-dependent, periodic forcing can be applied as well following
  the same procedure used for the wind forcing data (see above).

\item[dissipation] \ 
  
  Lateral eddy diffusivities for temperature and salinity/specific
  humidity are specified through the variables \textbf{diffKhT} and
  \textbf{diffKhS} (in m$^{2}$/s). Vertical eddy diffusivities are
  specified through the variables \textbf{diffKzT} and
  \textbf{diffKzS} (in m$^{2}$/s) for the ocean and \textbf{diffKpT
  }and \textbf{diffKpS} (in Pa$^{2}$/s) for the atmosphere. The
  vertical diffusive fluxes can be computed implicitly by setting the
  logical variable \textbf{implicitDiffusion} to \texttt{'.TRUE.'}.
  In addition, biharmonic diffusivities can be specified as well
  through the coefficients \textbf{diffK4T} and \textbf{diffK4S} (in
  m$^{4}$/s). Note that the cosine power scaling (specified through
  \textbf{cosPower}---see the momentum equations section) is applied to
  the tracer diffusivities (Laplacian and biharmonic) as well. The
  Gent and McWilliams parameterization for oceanic tracers is
  described in the package section. Finally, note that tracers can be
  also subject to Fourier and Shapiro filtering (see the corresponding
  section on these filters).

\item[ocean convection] \ 
  
  Two options are available to parameterize ocean convection: one is
  to use the convective adjustment scheme. In this case, you need to
  set the variable \textbf{cadjFreq}, which represents the frequency
  (in s) with which the adjustment algorithm is called, to a non-zero
  value (if set to a negative value by the user, the model will set it
  to the tracer time step). The other option is to parameterize
  convection with implicit vertical diffusion. To do this, set the
  logical variable \textbf{implicitDiffusion} to \texttt{'.TRUE.'}
  and the real variable \textbf{ivdc\_kappa} to a value (in m$^{2}$/s)
  you wish the tracer vertical diffusivities to have when mixing
  tracers vertically due to static instabilities. Note that
  \textbf{cadjFreq} and \textbf{ivdc\_kappa}can not both have non-zero
  value.

\end{description}

\subsection{Parameters: Simulation controls}

The model ''clock'' is defined by the variable \textbf{deltaTClock}
(in s) which determines the IO frequencies and is used in tagging
output.  Typically, you will set it to the tracer time step for
accelerated runs (otherwise it is simply set to the default time step
\textbf{deltaT}).  Frequency of checkpointing and dumping of the model
state are referenced to this clock (see below).

\begin{description}
\item[run duration] \ 
  
  The beginning of a simulation is set by specifying a start time (in
  s) through the real variable \textbf{startTime} or by specifying an
  initial iteration number through the integer variable
  \textbf{nIter0}. If these variables are set to nonzero values, the
  model will look for a ''pickup'' file \textit{pickup.0000nIter0} to
  restart the integration. The end of a simulation is set through the
  real variable \textbf{endTime} (in s).  Alternatively, you can
  specify instead the number of time steps to execute through the
  integer variable \textbf{nTimeSteps}.

\item[frequency of output] \
  
  Real variables defining frequencies (in s) with which output files
  are written on disk need to be set up. \textbf{dumpFreq} controls
  the frequency with which the instantaneous state of the model is
  saved. \textbf{chkPtFreq} and \textbf{pchkPtFreq} control the output
  frequency of rolling and permanent checkpoint files, respectively.
  See section 1.5.1 Output files for the definition of model state and
  checkpoint files. In addition, time-averaged fields can be written
  out by setting the variable \textbf{taveFreq} (in s).  The precision
  with which to write the binary data is controlled by the integer
  variable w\textbf{riteBinaryPrec} (set it to \texttt{32} or
  \texttt{64}).

\end{description}


%%% Local Variables: 
%%% mode: latex
%%% TeX-master: t
%%% End: 
