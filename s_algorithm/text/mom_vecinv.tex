% $Header: /u/gcmpack/manual/s_algorithm/text/mom_vecinv.tex,v 1.2 2001/10/25 18:36:53 cnh Exp $
% $Name:  $

\section{Vector invariant momentum equations}

The finite volume method lends itself to describing the continuity and
tracer equations in curvilinear coordinate systems but the appearance
of new metric terms in the flux-form momentum equations makes
generalizing them far from elegant. The vector invariant form of the
momentum equations are exactly that; invariant under coordinate
transformations.

The non-hydrostatic vector invariant equations read:
\begin{equation}
\partial_t \vec{v} + ( 2\vec{\Omega} + \vec{\zeta}) \wedge \vec{v}
- b \hat{r}
+ \vec{\nabla} B = \vec{\nabla} \cdot \vec{\bf \tau}
\end{equation}
which describe motions in any orthogonal curvilinear coordinate
system. Here, $B$ is the Bernoulli function and $\vec{\zeta}=\nabla
\wedge \vec{v}$ is the vorticity vector. We can take advantage of the
elegance of these equations when discretizing them and use the
discrete definitions of the grad, curl and divergence operators to
satisfy constraints. We can also consider the analogy to forming
derived equations, such as the vorticity equation, and examine how the
discretization can be adjusted to give suitable vorticity advection
among other things.

The underlying algorithm is the same as for the flux form
equations. All that has changed is the contents of the ``G's''. For
the time-being, only the hydrostatic terms have been coded but we will
indicate the points where non-hydrostatic contributions will enter:
\begin{eqnarray}
G_u & = & G_u^{fv} + G_u^{\zeta_3 v} + G_u^{\zeta_2 w} + G_u^{\partial_x B}
+ G_u^{\partial_z \tau^x} + G_u^{h-dissip} + G_u^{v-dissip} \\
G_v & = & G_v^{fu} + G_v^{\zeta_3 u} + G_v^{\zeta_1 w} + G_v^{\partial_y B}
+ G_v^{\partial_z \tau^y} + G_v^{h-dissip} + G_v^{v-dissip} \\
G_w & = & G_w^{fu} + G_w^{\zeta_1 v} + G_w^{\zeta_2 u} + G_w^{\partial_z B}
+ G_w^{h-dissip} + G_w^{v-dissip}
\end{eqnarray}

\fbox{ \begin{minipage}{4.75in}
{\em S/R CALC\_MOM\_RHS} ({\em pkg/mom\_vecinv/calc\_mom\_rhs.F})

$G_u$: {\bf Gu} ({\em DYNVARS.h})

$G_v$: {\bf Gv} ({\em DYNVARS.h})

$G_w$: {\bf Gw} ({\em DYNVARS.h})
\end{minipage} }

\subsection{Relative vorticity}

The vertical component of relative vorticity is explicitly calculated
and use in the discretization. The particular form is crucial for
numerical stability; alternative definitions break the conservation
properties of the discrete equations.

Relative vorticity is defined:
\begin{equation}
\zeta_3 = \frac{\Gamma}{A_\zeta}
= \frac{1}{{\cal A}_\zeta} ( \delta_i \Delta y_c v - \delta_j \Delta x_c u )
\end{equation}
where ${\cal A}_\zeta$ is the area of the vorticity cell presented in
the vertical and $\Gamma$ is the circulation about that cell.

\fbox{ \begin{minipage}{4.75in}
{\em S/R MOM\_VI\_CALC\_RELVORT3} ({\em mom\_vi\_calc\_relvort3.F})

$\zeta_3$: {\bf vort3} (local to {\em calc\_mom\_rhs.F})
\end{minipage} }


\subsection{Kinetic energy}

The kinetic energy, denoted $KE$, is defined:
\begin{equation}
KE = \frac{1}{2} ( \overline{ u^2 }^i + \overline{ v^2 }^j 
+ \epsilon_{nh} \overline{ w^2 }^k )
\end{equation}

\fbox{ \begin{minipage}{4.75in}
{\em S/R MOM\_VI\_CALC\_KE} ({\em mom\_vi\_calc\_ke.F})

$KE$: {\bf KE} (local to {\em calc\_mom\_rhs.F})
\end{minipage} }


\subsection{Coriolis terms}

The potential enstrophy conserving form of the linear Coriolis terms
are written:
\begin{eqnarray}
G_u^{fv} & = &
\frac{1}{\Delta x_c}
\overline{ \frac{f}{h_\zeta} }^j \overline{ \overline{ \Delta x_g h_s v }^j }^i \\
G_v^{fu} & = & -
\frac{1}{\Delta y_c}
\overline{ \frac{f}{h_\zeta} }^i \overline{ \overline{ \Delta y_g h_w u }^i }^j
\end{eqnarray}
Here, the Coriolis parameter $f$ is defined at vorticity (corner)
points.
\marginpar{$f$: {\bf fCoriG}}
\marginpar{$h_\zeta$: {\bf hFacZ}}

The potential enstrophy conserving form of the non-linear Coriolis
terms are written:
\begin{eqnarray}
G_u^{\zeta_3 v} & = &
\frac{1}{\Delta x_c}
\overline{ \frac{\zeta_3}{h_\zeta} }^j \overline{ \overline{ \Delta x_g h_s v }^j }^i \\
G_v^{\zeta_3 u} & = & -
\frac{1}{\Delta y_c}
\overline{ \frac{\zeta_3}{h_\zeta} }^i \overline{ \overline{ \Delta y_g h_w u }^i }^j
\end{eqnarray}
\marginpar{$\zeta_3$: {\bf vort3}}

The Coriolis terms can also be evaluated together and expressed in
terms of absolute vorticity $f+\zeta_3$. The potential enstrophy
conserving form using the absolute vorticity is written:
\begin{eqnarray}
G_u^{fv} + G_u^{\zeta_3 v} & = &
\frac{1}{\Delta x_c}
\overline{ \frac{f + \zeta_3}{h_\zeta} }^j \overline{ \overline{ \Delta x_g h_s v }^j }^i \\
G_v^{fu} + G_v^{\zeta_3 u} & = & -
\frac{1}{\Delta y_c}
\overline{ \frac{f + \zeta_3}{h_\zeta} }^i \overline{ \overline{ \Delta y_g h_w u }^i }^j
\end{eqnarray}

\marginpar{Run-time control needs to be added for these options} The
distinction between using absolute vorticity or relative vorticity is
useful when constructing higher order advection schemes; monotone
advection of relative vorticity behaves differently to monotone
advection of absolute vorticity. Currently the choice of
relative/absolute vorticity, centered/upwind/high order advection is
available only through commented subroutine calls.

\fbox{ \begin{minipage}{4.75in}
{\em S/R MOM\_VI\_CORIOLIS} ({\em mom\_vi\_coriolis.F})

{\em S/R MOM\_VI\_U\_CORIOLIS} ({\em mom\_vi\_u\_coriolis.F})

{\em S/R MOM\_VI\_V\_CORIOLIS} ({\em mom\_vi\_v\_coriolis.F})

$G_u^{fv}$, $G_u^{\zeta_3 v}$: {\bf uCf} (local to {\em calc\_mom\_rhs.F})

$G_v^{fu}$, $G_v^{\zeta_3 u}$: {\bf vCf} (local to {\em calc\_mom\_rhs.F})
\end{minipage} }


\subsection{Shear terms}

The shear terms ($\zeta_2w$ and $\zeta_1w$) are are discretized to
guarantee that no spurious generation of kinetic energy is possible;
the horizontal gradient of Bernoulli function has to be consistent
with the vertical advection of shear:
\marginpar{N-H terms have not been tried!}
\begin{eqnarray}
G_u^{\zeta_2 w} & = &
\frac{1}{ {\cal A}_w \Delta r_f h_w } \overline{
\overline{ {\cal A}_c w }^i ( \delta_k u - \epsilon_{nh} \delta_j w )
}^k \\
G_v^{\zeta_1 w} & = &
\frac{1}{ {\cal A}_s \Delta r_f h_s } \overline{
\overline{ {\cal A}_c w }^i ( \delta_k u - \epsilon_{nh} \delta_j w )
}^k
\end{eqnarray}

\fbox{ \begin{minipage}{4.75in}
{\em S/R MOM\_VI\_U\_VERTSHEAR} ({\em mom\_vi\_u\_vertshear.F})

{\em S/R MOM\_VI\_V\_VERTSHEAR} ({\em mom\_vi\_v\_vertshear.F})

$G_u^{\zeta_2 w}$: {\bf uCf} (local to {\em calc\_mom\_rhs.F})

$G_v^{\zeta_1 w}$: {\bf vCf} (local to {\em calc\_mom\_rhs.F})
\end{minipage} }



\subsection{Gradient of Bernoulli function}

\begin{eqnarray}
G_u^{\partial_x B} & = &
\frac{1}{\Delta x_c} \delta_i ( \phi' + KE ) \\
G_v^{\partial_y B} & = &
\frac{1}{\Delta x_y} \delta_j ( \phi' + KE )
%G_w^{\partial_z B} & = &
%\frac{1}{\Delta r_c} h_c \delta_k ( \phi' + KE )
\end{eqnarray}

\fbox{ \begin{minipage}{4.75in}
{\em S/R MOM\_VI\_U\_GRAD\_KE} ({\em mom\_vi\_u\_grad\_ke.F})

{\em S/R MOM\_VI\_V\_GRAD\_KE} ({\em mom\_vi\_v\_grad\_ke.F})

$G_u^{\partial_x KE}$: {\bf uCf} (local to {\em calc\_mom\_rhs.F})

$G_v^{\partial_y KE}$: {\bf vCf} (local to {\em calc\_mom\_rhs.F})
\end{minipage} }



\subsection{Horizontal dissipation}

The horizontal divergence, a complimentary quantity to relative
vorticity, is used in parameterizing the Reynolds stresses and is
discretized:
\begin{equation}
D = \frac{1}{{\cal A}_c h_c} (
  \delta_i \Delta y_g h_w u
+ \delta_j \Delta x_g h_s v )
\end{equation}

\fbox{ \begin{minipage}{4.75in}
{\em S/R MOM\_VI\_CALC\_HDIV} ({\em mom\_vi\_calc\_hdiv.F})

$D$: {\bf hDiv} (local to {\em calc\_mom\_rhs.F})
\end{minipage} }


\subsection{Horizontal dissipation}

The following discretization of horizontal dissipation conserves
potential vorticity (thickness weighted relative vorticity) and
divergence and dissipates energy, enstrophy and divergence squared:
\begin{eqnarray}
G_u^{h-dissip} & = &
  \frac{1}{\Delta x_c} \delta_i ( A_D D - A_{D4} D^*)
- \frac{1}{\Delta y_u h_w} \delta_j h_\zeta ( A_\zeta \zeta - A_{\zeta4} \zeta^* )
\\
G_v^{h-dissip} & = &
  \frac{1}{\Delta x_v h_s} \delta_i h_\zeta ( A_\zeta \zeta - A_\zeta \zeta^* )
+ \frac{1}{\Delta y_c} \delta_j ( A_D D - A_{D4} D^* )
\end{eqnarray}
where
\begin{eqnarray}
D^* & = & \frac{1}{{\cal A}_c h_c} (
  \delta_i \Delta y_g h_w \nabla^2 u
+ \delta_j \Delta x_g h_s \nabla^2 v ) \\
\zeta^* & = & \frac{1}{{\cal A}_\zeta} (
  \delta_i \Delta y_c \nabla^2 v
- \delta_j \Delta x_c \nabla^2 u )
\end{eqnarray}

\fbox{ \begin{minipage}{4.75in}
{\em S/R MOM\_VI\_HDISSIP} ({\em mom\_vi\_hdissip.F})

$G_u^{h-dissip}$: {\bf uDiss} (local to {\em calc\_mom\_rhs.F})

$G_v^{h-dissip}$: {\bf vDiss} (local to {\em calc\_mom\_rhs.F})
\end{minipage} }


\subsection{Vertical dissipation}

Currently, this is exactly the same code as the flux form equations.
\begin{eqnarray}
G_u^{v-diss} & = &
\frac{1}{\Delta r_f h_w} \delta_k \tau_{13} \\
G_v^{v-diss} & = &
\frac{1}{\Delta r_f h_s} \delta_k \tau_{23}
\end{eqnarray}
represents the general discrete form of the vertical dissipation terms.

In the interior the vertical stresses are discretized:
\begin{eqnarray}
\tau_{13} & = & A_v \frac{1}{\Delta r_c} \delta_k u \\
\tau_{23} & = & A_v \frac{1}{\Delta r_c} \delta_k v
\end{eqnarray}

\fbox{ \begin{minipage}{4.75in}
{\em S/R MOM\_U\_RVISCLFUX} ({\em mom\_u\_rviscflux.F})

{\em S/R MOM\_V\_RVISCLFUX} ({\em mom\_v\_rviscflux.F})

$\tau_{13}$: {\bf urf} (local to {\em calc\_mom\_rhs.F})

$\tau_{23}$: {\bf vrf} (local to {\em calc\_mom\_rhs.F})
\end{minipage} }
