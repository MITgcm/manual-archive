% $Header: /u/gcmpack/manual/s_algorithm/text/nonlin_frsurf.tex,v 1.11 2005/08/09 18:21:53 jmc Exp $
% $Name:  $



\subsection{Non-linear free-surface}
\label{sect:nonlinear-freesurface}

Recently, new options have been added to the model
that concern the free surface formulation.


\subsubsection{pressure/geo-potential and free surface}
\label{sect:phi-freesurface}

For the atmosphere, since $\phi = \phi_{topo} - \int^p_{p_s} \alpha dp$,
subtracting the reference state defined in section
\ref{sec:hpe-p-geo-potential-split}~:\\
$$
\phi_o = \phi_{topo} - \int^p_{p_o} \alpha_o dp  
\hspace{5mm}\mathrm{with}\hspace{3mm} \phi_o(p_o)=\phi_{topo}
$$
we get:
$$
\phi' = \phi - \phi_o = \int^{p_s}_p \alpha dp - \int^{p_o}_p \alpha_o dp
$$
For the ocean, the reference state is simpler since $\rho_c$ does not dependent
on $z$ ($b_o=g$) and the surface reference position is uniformly $z=0$ ($R_o=0$),
and the same subtraction leads to a similar relation.
For both fluid, using the isomorphic notations, we can write:
$$ 
\phi' = \int^{r_{surf}}_r b~ dr - \int^{R_o}_r b_o dr
$$
\begin{eqnarray}
\mathrm{and~re~write:}\hspace{10mm}
\phi' = \int^{r_{surf}}_{R_o} b~ dr & + & \int^{R_o}_r (b - b_o) dr
\label{eq:split-phi-Ro} \\ 
\mathrm{or:}\hspace{10mm}
\phi' = \int^{r_{surf}}_{R_o} b_o dr & + & \int^{r_{surf}}_r (b - b_o) dr
\label{eq:split-phi-bo}
\end{eqnarray}

In section \ref{sec:finding_the_pressure_field}, following eq.\ref{eq:split-phi-Ro},
the pressure/geo-potential $\phi'$ has been separated into surface ($\phi_s$),
and hydrostatic anomaly ($\phi'_{hyd}$).
In this section, the split between $\phi_s$ and $\phi'_{hyd}$ is 
made according to equation \ref{eq:split-phi-bo}. This slightly
different definition reflects the actual implementation in the code 
and is valid for both linear and non-linear
free-surface formulation, in both r-coordinate and r*-coordinate.

Because the linear free-surface approximation ignore the tracer content
of the fluid parcel between $R_o$ and $r_{surf}=R_o+\eta$, 
for consistency reasons, this part is also neglected in $\phi'_{hyd}$~:
$$
\phi'_{hyd} = \int^{r_{surf}}_r (b - b_o) dr \simeq \int^{R_o}_r (b - b_o) dr
$$
Note that in this case, the two definitions of $\phi_s$ and $\phi'_{hyd}$ 
from equation \ref{eq:split-phi-Ro} and \ref{eq:split-phi-bo} converge toward 
the same (approximated) expressions: $\phi_s = \int^{r_{surf}}_{R_o} b_o dr$ 
and $\phi'_{hyd}=\int^{R_o}_r b' dr$.\\
On the contrary, the unapproximated formulation ("non-linear free-surface",
see the next section) retains the full expression:
$\phi'_{hyd} = \int^{r_{surf}}_r (b - b_o) dr $~.
This is obtained by selecting {\bf nonlinFreeSurf}=4 in parameter
file {\em data}.\\

Regarding the surface potential: 
$$\phi_s = \int_{R_o}^{R_o+\eta} b_o dr = b_s \eta
\hspace{5mm}\mathrm{with}\hspace{5mm} 
b_s = \frac{1}{\eta} \int_{R_o}^{R_o+\eta} b_o dr $$
$b_s \simeq b_o(R_o)$ is an excellent approximation (better than 
the usual numerical truncation, since generally $|\eta|$ is smaller 
than the vertical grid increment).

For the ocean, $\phi_s = g \eta$ and $b_s = g$ is uniform.
For the atmosphere, however, because of topographic effects, the
reference surface pressure $R_o=p_o$ has large spatial variations that
are responsible for significant $b_s$ variations (from 0.8 to 1.2
$[m^3/kg]$). For this reason, when {\bf uniformLin\_PhiSurf} {\em=.FALSE.} 
(parameter file {\em data}, namelist {\em PARAM01})
a non-uniform linear coefficient $b_s$ is used and computed
({\it S/R INI\_LINEAR\_PHISURF}) according to the reference surface 
pressure $p_o$:
$b_s = b_o(R_o) = c_p \kappa (p_o / P^o_{SL})^{(\kappa - 1)} \theta_{ref}(p_o)$.
with $P^o_{SL}$ the mean sea-level pressure.


\subsubsection{Free surface effect on column total thickness
(Non-linear free-surface)}

The total thickness of the fluid column is $r_{surf} - R_{fixed} =
\eta + R_o - R_{fixed}$. In most applications, the free surface 
displacements are small compared to the total thickness
$\eta \ll H_o = R_o - R_{fixed}$. 
In the previous sections and in older version of the model,
the linearized free-surface approximation was made, assuming
$r_{surf} - R_{fixed} \simeq H_o$ when computing horizontal transports,
either in the continuity equation or in tracer and momentum 
advection terms.
This approximation is dropped when using the non-linear free-surface
formulation and the total thickness, including the time varying part
$\eta$, is considered when computing horizontal transports.
Implications for the barotropic part are presented hereafter.
In section \ref{sect:freesurf-tracer-advection} consequences for 
tracer conservation is briefly discussed (more details can be 
found in \cite{campin:02})~; the general time-stepping is presented 
in section \ref{sect:nonlin-freesurf-timestepping} with some 
limitations regarding the vertical resolution in section 
\ref{sect:nonlin-freesurf-dz_surf}.

In the non-linear formulation, the continuous form of the model 
equations remains unchanged, except for the 2D continuity equation 
(\ref{eq:discrete-time-backward-free-surface}) which is now
integrated from $R_{fixed}(x,y)$ up to $r_{surf}=R_o+\eta$ :

\begin{displaymath}
\epsilon_{fs} \partial_t \eta =
\left. \dot{r} \right|_{r=r_{surf}} + \epsilon_{fw} (P-E) =
- {\bf \nabla}_h \cdot \int_{R_{fixed}}^{R_o+\eta} \vec{\bf v} dr
+ \epsilon_{fw} (P-E)
\end{displaymath}

Since $\eta$ has a direct effect on the horizontal velocity (through
$\nabla_h \Phi_{surf}$), this adds a non-linear term to the free
surface equation. Several options for the time discretization of this
non-linear part can be considered, as detailed below.

If the column thickness is evaluated at time step $n$, and with
implicit treatment of the surface potential gradient, equations
(\ref{eq-solve2D} and \ref{eq-solve2D_rhs}) becomes:
\begin{eqnarray*}
\epsilon_{fs} {\eta}^{n+1} -
{\bf \nabla}_h \cdot \Delta t^2 (\eta^{n}+R_o-R_{fixed})
{\bf \nabla}_h b_s {\eta}^{n+1}
= {\eta}^*
\end{eqnarray*}
where
\begin{eqnarray*}
{\eta}^* = \epsilon_{fs} \: {\eta}^{n} -
\Delta t {\bf \nabla}_h \cdot \int_{R_{fixed}}^{R_o+\eta^n} \vec{\bf v}^* dr
\: + \: \epsilon_{fw} \Delta_t (P-E)^{n}
\end{eqnarray*} 
This method requires us to update the solver matrix at each time step.

Alternatively, the non-linear contribution can be evaluated fully
explicitly:
\begin{eqnarray*}
\epsilon_{fs} {\eta}^{n+1} -
{\bf \nabla}_h \cdot \Delta t^2 (R_o-R_{fixed})
{\bf \nabla}_h b_s {\eta}^{n+1}
= {\eta}^*
+{\bf \nabla}_h \cdot \Delta t^2 (\eta^{n})
{\bf \nabla}_h b_s {\eta}^{n}
\end{eqnarray*} 
This formulation allows one to keep the initial solver matrix
unchanged though throughout the integration, since the non-linear free
surface only affects the RHS.

Finally, another option is a "linearized" formulation where the total
column thickness appears only in the integral term of the RHS
(\ref{eq-solve2D_rhs}) but not directly in the equation
(\ref{eq-solve2D}).

Those different options (see Table \ref{tab:nonLinFreeSurf_flags})
have been tested and show little differences. However, we recommend
the use of the most precise method (the 1rst one) since the 
computation cost involved in the solver matrix update is negligible.

\begin{table}[htb]
%\begin{center}
\centering
 \begin{tabular}[htb]{|l|c|l|}
   \hline
   parameter & value & description \\ 
   \hline
                   & -1 & linear free-surface, restart from a pickup file \\
                   &    & produced with \#undef EXACT\_CONSERV code\\
   \cline{2-3}
                   & 0 & Linear free-surface \\ 
   \cline{2-3}
    nonlinFreeSurf & 4 & Non-linear free-surface \\
   \cline{2-3}
                   & 3 & same as 4 but neglecting
                           $\int_{R_o}^{R_o+\eta} b' dr $ in $\Phi'_{hyd}$ \\
   \cline{2-3}
                   & 2 & same as 3 but do not update cg2d solver matrix \\
   \cline{2-3}
                  & 1 & same as 2 but treat momentum as in Linear FS \\
   \hline
                  & 0 & do not use $r*$ vertical coordinate (= default)\\
   \cline{2-3}
    select\_rStar & 2 & use $r^*$ vertical coordinate \\
   \cline{2-3}
                  & 1 & same as 2 but without the contribution of the\\
                  &   & slope of the coordinate in $\nabla \Phi$ \\
   \hline
  \end{tabular}
  \caption{Non-linear free-surface flags}
  \label{tab:nonLinFreeSurf_flags}
%\end{center}
\end{table}


\subsubsection{Tracer conservation with non-linear free-surface}
\label{sect:freesurf-tracer-advection}

To ensure global tracer conservation (i.e., the total amount) as well
as local conservation, the change in the surface level thickness must
be consistent with the way the continuity equation is integrated, both
in the barotropic part (to find $\eta$) and baroclinic part (to find
$w = \dot{r}$).

To illustrate this, consider the shallow water model, with a source
of fresh water (P):
$$
\partial_t h + \nabla \cdot h \vec{\bf v} = P
$$
where $h$ is the total thickness of the water column.
To conserve the tracer $\theta$ we have to discretize:
$$
\partial_t (h \theta) + \nabla \cdot ( h \theta \vec{\bf v})
  = P \theta_{\mathrm{rain}}
$$
Using the implicit (non-linear) free surface described above (section
\ref{sect:pressure-method-linear-backward}) we have:
\begin{eqnarray*}
h^{n+1} = h^{n} - \Delta t \nabla \cdot (h^n \, \vec{\bf v}^{n+1} ) + \Delta t P \\
\end{eqnarray*}
The discretized form of the tracer equation must adopt the same
``form'' in the computation of tracer fluxes, that is, the same value
of $h$, as used in the continuity equation:
\begin{eqnarray*}
h^{n+1} \, \theta^{n+1} = h^n \, \theta^n 
        - \Delta t \nabla \cdot (h^n \, \theta^n \, \vec{\bf v}^{n+1})
        + \Delta t P \theta_{rain}
\end{eqnarray*}

The use of a 3 time-levels time-stepping scheme such as the Adams-Bashforth
make the conservation sightly tricky.
The current implementation with the Adams-Bashforth time-stepping
provides an exact local conservation and prevents any drift in
the global tracer content (\cite{campin:02}).
Compared to the linear free-surface method, an additional step is required:
the variation of the water column thickness (from $h^n$ to $h^{n+1}$) is
not incorporated directly into the tracer equation.  Instead, the
model uses the $G_\theta$ terms (first step) as in the linear free
surface formulation (with the "{\it surface correction}" turned "on",
see tracer section):
$$
G_\theta^n = \left(- \nabla \cdot (h^n \, \theta^n \, \vec{\bf v}^{n+1}) 
         - \dot{r}_{surf}^{n+1} \theta^n \right) / h^n
$$
Then, in a second step, the thickness variation (expansion/reduction)
is taken into account:
$$
\theta^{n+1} = \theta^n + \Delta t \frac{h^n}{h^{n+1}} 
   \left( G_\theta^{(n+1/2)} + P (\theta_{\mathrm{rain}} - \theta^n )/h^n \right)
%\theta^{n+1} = \theta^n + \frac{\Delta t}{h^{n+1}} 
%   \left( h^n G_\theta^{(n+1/2)} + P (\theta_{\mathrm{rain}} - \theta^n ) \right)
$$
Note that with a simple forward time step (no Adams-Bashforth), 
these two formulations are equivalent,  
since
$
(h^{n+1} - h^{n})/ \Delta t = 
P - \nabla \cdot (h^n \, \vec{\bf v}^{n+1} ) = P + \dot{r}_{surf}^{n+1}
$

\subsubsection{Time stepping implementation of the
non-linear free-surface}     
\label{sect:nonlin-freesurf-timestepping}

The grid cell thickness was hold constant with the linear
free-surface~; with the non-linear free-surface, it is now varying 
in time, at least at the surface level.
This implies some modifications of the general algorithm described 
earlier in sections \ref{sect:adams-bashforth-sync} and 
\ref{sect:adams-bashforth-staggered}.

A simplified version of the staggered in time, non-linear 
free-surface algorithm is detailed hereafter, and can be compared 
to the equivalent linear free-surface case (eq. \ref{eq:Gv-n-staggered}
to \ref{eq:t-n+1-staggered}) and can also be easily transposed
to the synchronous time-stepping case.
Among the simplifications, salinity equation, implicit operator
and detailed elliptic equation are omitted. Surface forcing is
explicitly written as fluxes of temperature, fresh water and
momentum, $Q^{n+1/2}, P^{n+1/2}, F_{\bf v}^n$ respectively.
$h^n$ and $dh^n$ are the column and grid box thickness in r-coordinate.
%-------------------------------------------------------------
\begin{eqnarray}
\phi^{n}_{hyd} & = & \int b(\theta^{n},S^{n},r) dr
\label{eq:phi-hyd-nlfs} \\
\vec{\bf G}_{\vec{\bf v}}^{n-1/2}\hspace{-2mm} & = & 
\vec{\bf G}_{\vec{\bf v}} (dh^{n-1},\vec{\bf v}^{n-1/2})
\hspace{+2mm};\hspace{+2mm}
\vec{\bf G}_{\vec{\bf v}}^{(n)} =  
   \frac{3}{2} \vec{\bf G}_{\vec{\bf v}}^{n-1/2} 
-  \frac{1}{2} \vec{\bf G}_{\vec{\bf v}}^{n-3/2}
\label{eq:Gv-n-nlfs} \\
%\vec{\bf G}_{\vec{\bf v}}^{(n)} & = & 
%   \frac{3}{2} \vec{\bf G}_{\vec{\bf v}}^{n-1/2} 
%-  \frac{1}{2} \vec{\bf G}_{\vec{\bf v}}^{n-3/2}
%\label{eq:Gv-n+5-nlfs} \\
%\vec{\bf v}^{*} & = & \vec{\bf v}^{n-1/2} + \frac{\Delta t}{dh^{n}} \left(
%dh^{n-1}\vec{\bf G}_{\vec{\bf v}}^{(n)} + F_{\vec{\bf v}}^{n} \right)
\vec{\bf v}^{*} & = & \vec{\bf v}^{n-1/2} + \Delta t \frac{dh^{n-1}}{dh^{n}} \left(
\vec{\bf G}_{\vec{\bf v}}^{(n)} + F_{\vec{\bf v}}^{n}/dh^{n-1} \right)
- \Delta t \nabla \phi_{hyd}^{n}
\label{eq:vstar-nlfs} \\
  \mathrm{update}\hspace{-4mm} & & \hspace{-4mm}\mathrm{
  model~geometry~:~{\bf hFac}}(dh^n)\nonumber \\
\eta^{n+1/2} \hspace{-2mm} & = & 
\eta^{n-1/2} + \Delta t P^{n+1/2} - \Delta t
  \nabla \cdot \int \vec{\bf v}^{n+1/2} dh^{n} \nonumber \\
             & = & \eta^{n-1/2} + \Delta t P^{n+1/2} - \Delta t
  \nabla \cdot \int \!\!\! \left( \vec{\bf v}^* - g \Delta t \nabla \eta^{n+1/2} \right) dh^{n}
\label{eq:nstar-nlfs} \\
\vec{\bf v}^{n+1/2}\hspace{-2mm} & = & 
\vec{\bf v}^{*} - g \Delta t \nabla \eta^{n+1/2}
\label{eq:v-n+1-nlfs} \\
h^{n+1} & = & h^{n} + \Delta t P^{n+1/2} - \Delta t
  \nabla \cdot \int \vec{\bf v}^{n+1/2} dh^{n}
\label{eq:h-n+1-nlfs} \\
G_{\theta}^{n} & = & G_{\theta} ( dh^{n}, u^{n+1/2}, \theta^{n} )
\hspace{+2mm};\hspace{+2mm}
G_{\theta}^{(n+1/2)} = \frac{3}{2} G_{\theta}^{n} - \frac{1}{2} G_{\theta}^{n-1}
\label{eq:Gt-n-nlfs} \\
%\theta^{n+1} & = &\theta^{n} + \frac{\Delta t}{dh^{n+1}} \left( dh^n 
%G_{\theta}^{(n+1/2)} + Q^{n+1/2} + P^{n+1/2} (\theta_{\mathrm{rain}}-\theta^n) \right)
\theta^{n+1} & = &\theta^{n} + \Delta t \frac{dh^n}{dh^{n+1}} \left( 
G_{\theta}^{(n+1/2)}
+( P^{n+1/2} (\theta_{\mathrm{rain}}-\theta^n) + Q^{n+1/2})/dh^n \right)
\nonumber \\
& & \label{eq:t-n+1-nlfs}
\end{eqnarray}
%-------------------------------------------------------------
Two steps have been added to linear free-surface algorithm 
(eq. \ref{eq:Gv-n-staggered} to \ref{eq:t-n+1-staggered}):
Firstly, the model ``geometry''
(here the {\bf hFacC,W,S}) is updated just before entering {\it
SOLVE\_FOR\_PRESSURE}, using the current $dh^{n}$ field.
Secondly, the vertically integrated continuity equation 
(eq.\ref{eq:h-n+1-nlfs}) has been added ({\bf exactConserv}{\em =TRUE},
in parameter file {\em data}, namelist {\em PARM01})
just before computing the vertical velocity, in subroutine
{\em INTEGR\_CONTINUITY}. This ensures that tracer and continuity equation
discretization  a Although this equation might appear 
redundant with eq.\ref{eq:nstar-nlfs}, the integrated column
thickness $h^{n+1}$ can be different from $\eta^{n+1/2} + H$~:
\begin{itemize}
\item when Crank-Nickelson time-stepping is used (see section
\ref{sect:freesurf-CrankNick}).
\item when filters are applied to the flow field, after
(\ref{eq:v-n+1-nlfs}) and alter the divergence of the flow.
\item when the solver does not iterate until convergence~;
 for example, because a too large residual target was set
 ({\bf cg2dTargetResidual}, parameter file {\em data}, namelist
 {\em PARM02}).
\end{itemize}\noindent
In this staggered time-stepping algorithm, the momentum tendencies
are computed using $dh^{n-1}$ geometry factors.
(eq.\ref{eq:Gv-n-nlfs}) and then rescaled in subroutine {\it TIMESTEP},
(eq.\ref{eq:vstar-nlfs}), similarly to tracer tendencies (see section
\ref{sect:freesurf-tracer-advection}).
The tracers are stepped forward later, using the recently updated
flow field ${\bf v}^{n+1/2}$ and the corresponding model geometry
$dh^{n}$ to compute the tendencies (eq.\ref{eq:Gt-n-nlfs});
Then the tendencies are rescaled by $dh^n/dh^{n+1}$ to derive
the new tracers values $(\theta,S)^{n+1}$ (eq.\ref{eq:t-n+1-nlfs},
in subroutine {\em CALC\_GT, CALC\_GS}).

Note that the fresh-water input is added in a consistent way in the 
continuity equation and in the tracer equation, taking into account
the fresh-water temperature $\theta_{\mathrm{rain}}$.

Regarding the restart procedure,
two 2.D fields $h^{n-1}$ and $(h^n-h^{n-1})/\Delta t$ 
in addition to the standard
state variables and tendencies ($\eta^{n-1/2}$, ${\bf v}^{n-1/2}$,
$\theta^n$, $S^n$, ${\bf G}_{\bf v}^{n-3/2}$, $G_{\theta,S}^{n-1}$)
are stored in a "{\em pickup}" file.
The model restarts reading this "{\em pickup}" file,
then update the model geometry according to $h^{n-1}$,
and compute $h^n$ and the vertical velocity
%$h^n=h^{n-1} + \Delta t [(h^n-h^{n-1})/\Delta t]$,
before starting the main calling sequence (eq.\ref{eq:phi-hyd-nlfs} 
to \ref{eq:t-n+1-nlfs}, {\em S/R FORWARD\_STEP}).
\\

\fbox{ \begin{minipage}{4.75in}
{\em INTEGR\_CONTINUITY} ({\em integr\_continuity.F})

$h^{n+1} -H_o$: {\bf etaH} ({\em DYNVARS.h})

$h^{n} -H_o$: {\bf etaHnm1} ({\em SURFACE.h})

$h^{n+1}-h^{n}/\Delta t$: {\bf dEtaHdt} ({\em SURFACE.h})

\end{minipage} }

\subsubsection{Non-linear free-surface and vertical resolution}
\label{sect:nonlin-freesurf-dz_surf}

When the amplitude of the free-surface variations becomes
as large as the vertical resolution near the surface,
the surface layer thickness can decrease to nearly zero or
can even vanish completely. 
This later possibility has not been implemented, and a 
minimum relative thickness is imposed ({\bf hFacInf}, 
parameter file {\em data}, namelist {\em PARM01}) to prevent 
numerical instabilities caused by very thin surface level.

A better alternative to the vanishing level problem has been 
found and implemented recently, and rely on a different 
vertical coordinate $r^*$~:
The time variation ot the total column thickness becomes
part of the r* coordinate motion, as in a $\sigma_{z},\sigma_{p}$
model, but the fixed part related to topography is treated
as in a height or pressure coordinate model.
A complete description is given in \cite{adcroft:04a}. 

The time-stepping implementation of the $r^*$ coordinate is
identical to the non-linear free-surface in $r$ coordinate,
and differences appear only in the spacial discretization.
\marginpar{needs a subsection ref. here}

