% $Header: /u/gcmpack/manual/s_algorithm/text/tracer.tex,v 1.3 2001/09/25 20:13:42 adcroft Exp $
% $Name:  $

\section{Tracer equations}

The basic discretization used for the tracer equations is the second
order piece-wise constant finite volume form of the forced
advection-diussion equations. There are many alternatives to second
order method for advection and alternative parameterizations for the
sub-grid scale processes. The Gent-McWilliams eddy parameterization,
KPP mixing scheme and PV flux parameterization are all dealt with in
separate sections. The basic discretization of the advection-diffusion
part of the tracer equations and the various advection schemes will be
described here.

\subsection{Time-stepping of tracers: ABII}

The default advection scheme is the centered second order method which
requires a second order or quasi-second order time-stepping scheme to
be stable. Historically this has been the quasi-second order
Adams-Bashforth method (ABII) and applied to all terms. For an
arbitrary tracer, $\tau$, the forced advection-diffusion equation
reads:
\begin{equation}
\partial_t \tau + G_{adv}^\tau = G_{diff}^\tau + G_{forc}^\tau
\end{equation}
where $G_{adv}^\tau$, $G_{diff}^\tau$ and $G_{forc}^\tau$ are the
tendencies due to advection, diffusion and forcing, respectively,
namely:
\begin{eqnarray}
G_{adv}^\tau & = & \partial_x u \tau + \partial_y v \tau + \partial_r w \tau
- \tau \nabla \cdot {\bf v} \\
G_{diff}^\tau & = & \nabla \cdot {\bf K} \nabla \tau
\end{eqnarray}
and the forcing can be some arbitrary function of state, time and
space.

The term, $\tau \nabla \cdot {\bf v}$, is required to retain local
conservation in conjunction with the linear implicit free-surface. It
only affects the surface layer since the flow is non-divergent
everywhere else. This term is therefore referred to as the surface
correction term. Global conservation is not possible using the
flux-form (as here) and a linearized free-surface
(\cite{Griffies00,Campin02}).

The continuity equation can be recovered by setting
$G_{diff}=G_{forc}=0$ and $\tau=1$.

The driver routine that calls the routines to calculate tendancies are
{\em S/R CALC\_GT} and {\em S/R CALC\_GS} for temperature and salt
(moisture), respectively. These in turn call a generic advection
diffusion routine {\em S/R GAD\_CALC\_RHS} that is called with the
flow field and relevent tracer as arguments and returns the collective
tendancy due to advection and diffusion. Forcing is add subsequently
in {\em S/R CALC\_GT} or {\em S/R CALC\_GS} to the same tendancy
array.

\fbox{ \begin{minipage}{4.75in}
{\em S/R GAD\_CALC\_RHS} ({\em pkg/generic\_advdiff/gad\_calc\_rhs.F})

$\tau$: {\bf tracer} (argument)

$G^{(n)}$: {\bf gTracer} (argument)

$F_r$: {\bf fVerT} (argument)

\end{minipage} }


The space and time discretizations are treated seperately (method of
lines). The Adams-Bashforth time discretization reads:
\marginpar{$\epsilon$: {\bf AB\_eps}}
\marginpar{$\Delta t$: {\bf deltaTtracer}}
\begin{equation}
\tau^{(n+1)} = \tau^{(n)} + \Delta t \left(
(\frac{3}{2} + \epsilon) G^{(n)} - (\frac{1}{2} + \epsilon) G^{(n-1)}
\right)
\end{equation}
where $G^{(n)} = G_{adv}^\tau + G_{diff}^\tau + G_{src}^\tau$ at time
step $n$.

Strictly speaking the ABII scheme should be applied only to the
advection terms. However, this scheme is only used in conjuction with
the standard second, third and fourth order advection
schemes. Selection of any other advection scheme disables
Adams-Bashforth for tracers so that explicit diffusion and forcing use
the forward method.

\fbox{ \begin{minipage}{4.75in}
{\em S/R TIMESTEP\_TRACER} ({\em model/src/timestep\_tracer.F})

$\tau$: {\bf tracer} (argument)

$G^{(n)}$: {\bf gTracer} (argument)

$G^{(n-1)}$: {\bf gTrNm1} (argument)

$\Delta t$: {\bf deltaTtracer} (PARAMS.h)

\end{minipage} }

\begin{figure}
\resizebox{5.5in}{!}{\includegraphics{part2/advect-1d-lo.eps}}
\caption{
Comparison of 1-D advection schemes. Courant number is 0.05 with 60
points and solutions are shown for T=1 (one complete period).
a) Shows the upwind biased schemes; first order upwind, DST3,
third order upwind and second order upwind.
b) Shows the centered schemes; Lax-Wendroff, DST4, centered second order,
centered fourth order and finite volume fourth order.
c) Shows the second order flux limiters: minmod, Superbee,
MC limiter and the van Leer limiter.
d) Shows the DST3 method with flux limiters due to Sweby with
$\mu=1$, $\mu=c/(1-c)$ and a fourth order DST method with Sweby limiter,
$\mu=c/(1-c)$.
\label{fig:advect-1d-lo}
}
\end{figure}

\begin{figure}
\resizebox{5.5in}{!}{\includegraphics{part2/advect-1d-hi.eps}}
\caption{
Comparison of 1-D advection schemes. Courant number is 0.89 with 60
points and solutions are shown for T=1 (one complete period).
a) Shows the upwind biased schemes; first order upwind and DST3.
Third order upwind and second order upwind are unstable at this Courant number.
b) Shows the centered schemes; Lax-Wendroff, DST4. Centered second order,
centered fourth order and finite volume fourth order and unstable at this
Courant number.
c) Shows the second order flux limiters: minmod, Superbee,
MC limiter and the van Leer limiter.
d) Shows the DST3 method with flux limiters due to Sweby with
$\mu=1$, $\mu=c/(1-c)$ and a fourth order DST method with Sweby limiter,
$\mu=c/(1-c)$.
\label{fig:advect-1d-hi}
}
\end{figure}

\section{Linear advection schemes}

The advection schemes known as centered second order, centered fourth
order, first order upwind and upwind biased third order are known as
linear advection schemes because the coefficient for interpolation of
the advected tracer are linear and a function only of the flow, not
the tracer field it self. We discuss these first since they are most
commonly used in the field and most familiar.

\subsection{Centered second order advection-diffusion}

The basic discretization, centered second order, is the default. It is
designed to be consistant with the continuity equation to facilitate
conservation properties analogous to the continuum. However, centered
second order advection is notoriously noisey and must be used in
conjuction with some finite amount of diffusion to produce a sensible
solution.

The advection operator is discretized:
\begin{equation}
{\cal A}_c \Delta r_f h_c G_{adv}^\tau = 
\delta_i F_x + \delta_j F_y + \delta_k F_r
\end{equation}
where the area integrated fluxes are given by:
\begin{eqnarray}
F_x & = & U \overline{ \tau }^i \\
F_y & = & V \overline{ \tau }^j \\
F_r & = & W \overline{ \tau }^k
\end{eqnarray}
The quantities $U$, $V$ and $W$ are volume fluxes defined:
\marginpar{$U$: {\bf uTrans} }
\marginpar{$V$: {\bf vTrans} }
\marginpar{$W$: {\bf rTrans} }
\begin{eqnarray}
U & = & \Delta y_g \Delta r_f h_w u \\
V & = & \Delta x_g \Delta r_f h_s v \\
W & = & {\cal A}_c w
\end{eqnarray}

For non-divergent flow, this discretization can be shown to conserve
the tracer both locally and globally and to globally conserve tracer
variance, $\tau^2$. The proof is given in \cite{Adcroft95,Adcroft97}.

\fbox{ \begin{minipage}{4.75in}
{\em S/R GAD\_C2\_ADV\_X} ({\em gad\_c2\_adv\_x.F})

$F_x$: {\bf uT} (argument)

$U$: {\bf uTrans} (argument)

$\tau$: {\bf tracer} (argument)

{\em S/R GAD\_C2\_ADV\_Y} ({\em gad\_c2\_adv\_y.F})

$F_y$: {\bf vT} (argument)

$V$: {\bf vTrans} (argument)

$\tau$: {\bf tracer} (argument)

{\em S/R GAD\_C2\_ADV\_R} ({\em gad\_c2\_adv\_r.F})

$F_r$: {\bf wT} (argument)

$W$: {\bf rTrans} (argument)

$\tau$: {\bf tracer} (argument)

\end{minipage} }


\subsection{Third order upwind bias advection}

Upwind biased third order advection offers a relatively good
compromise between accuracy and smoothness. It is not a ``positive''
scheme meaning false extrema are permitted but the amplitude of such
are significantly reduced over the centered second order method.

The third order upwind fluxes are discretized:
\begin{eqnarray}
F_x & = & U \overline{\tau - \frac{1}{6} \delta_{ii} \tau}^i
         + \frac{1}{2} |U| \delta_i \frac{1}{6} \delta_{ii} \tau \\
F_y & = & V \overline{\tau - \frac{1}{6} \delta_{ii} \tau}^j
         + \frac{1}{2} |V| \delta_j \frac{1}{6} \delta_{jj} \tau \\
F_r & = & W \overline{\tau - \frac{1}{6} \delta_{ii} \tau}^k
         + \frac{1}{2} |W| \delta_k \frac{1}{6} \delta_{kk} \tau 
\end{eqnarray}

At boundaries, $\delta_{\hat{n}} \tau$ is set to zero allowing
$\delta_{nn}$ to be evaluated. We are currently examing the accuracy
of this boundary condition and the effect on the solution.


\subsection{Centered fourth order advection}

Centered fourth order advection is formally the most accurate scheme
we have implemented and can be used to great effect in high resolution
simultation where dynamical scales are well resolved. However, the
scheme is noisey like the centered second order method and so must be
used with some finite amount of diffusion. Bi-harmonic is recommended
since it is more scale selective and less likely to diffuse away the
well resolved gradient the fourth order scheme worked so hard to
create.

The centered fourth order fluxes are discretized:
\begin{eqnarray}
F_x & = & U \overline{\tau - \frac{1}{6} \delta_{ii} \tau}^i \\
F_y & = & V \overline{\tau - \frac{1}{6} \delta_{ii} \tau}^j \\
F_r & = & W \overline{\tau - \frac{1}{6} \delta_{ii} \tau}^k
\end{eqnarray}

As for the third order scheme, the best discretization near boundaries
is under investigation but currenlty $\delta_i \tau=0$ on a boundary.

\subsection{First order upwind advection}

Although the upwind scheme is the underlying scheme for the robust or
non-linear methods given later, we haven't actually supplied this
method for general use. It would be very diffusive and it is unlikely
that it could ever produce more useful results than the positive
higher order schemes.

Upwind bias is introduced into many schemes using the {\em abs}
function and is allows the first order upwind flux to be written:
\begin{eqnarray}
F_x & = & U \overline{ \tau }^i - \frac{1}{2} |U| \delta_i \tau \\
F_y & = & V \overline{ \tau }^j - \frac{1}{2} |V| \delta_j \tau \\
F_r & = & W \overline{ \tau }^k - \frac{1}{2} |W| \delta_k \tau
\end{eqnarray}

If for some reason, the above method is required, then the second
order flux limiter scheme described later reduces to the above scheme
if the limiter is set to zero.


\section{Non-linear advection schemes}

Non-linear advection schemes invoke non-linear interpolation and are
widely used in computational fluid dynamics (non-linear does not refer
to the non-linearity of the advection operator). The flux limited
advection schemes belong to the class of finite volume methods which
neatly ties into the spatial discretization of the model.

When employing the flux limited schemes, first order upwind or
direct-space-time method the time-stepping is switched to forward in
time.

\subsection{Second order flux limiters}

The second order flux limiter method can be cast in several ways but
is generally expressed in terms of other flux approximations. For
example, in terms of a first order upwind flux and second order
Lax-Wendroff flux, the limited flux is given as:
\begin{equation}
F = F_1 + \psi(r) F_{LW}
\end{equation}
where $\psi(r)$ is the limiter function,
\begin{equation}
F_1 = u \overline{\tau}^i - \frac{1}{2} |u| \delta_i \tau
\end{equation}
is the upwind flux,
\begin{equation}
F_{LW} = F_1 + \frac{|u|}{2} (1-c) \delta_i \tau
\end{equation}
is the Lax-Wendroff flux and $c = \frac{u \Delta t}{\Delta x}$ is the
Courant (CFL) number.

The limiter function, $\psi(r)$, takes the slope ratio
\begin{eqnarray}
r = \frac{ \tau_{i-1} - \tau_{i-2} }{ \tau_{i} - \tau_{i-1} } & \forall & u > 0
\\
r = \frac{ \tau_{i+1} - \tau_{i} }{ \tau_{i} - \tau_{i-1} } & \forall & u < 0
\end{eqnarray}
as it's argument. There are many choices of limiter function but we
only provide the Superbee limiter \cite{Roe85}:
\begin{equation}
\psi(r) = \max[0,\min[1,2r],\min[2,r]]
\end{equation}


\subsection{Third order direct space time}

The direct-space-time method deals with space and time discretization
together (other methods that treat space and time seperately are known
collectively as the ``Method of Lines''). The Lax-Wendroff scheme
falls into this category; it adds sufficient diffusion to a second
order flux that the forward-in-time method is stable. The upwind
biased third order DST scheme is:
\begin{eqnarray}
F = u \left( \tau_{i-1}
        + d_0 (\tau_{i}-\tau_{i-1}) + d_1 (\tau_{i-1}-\tau_{i-2}) \right)
& \forall & u > 0 \\
F = u \left( \tau_{i}
        - d_0 (\tau_{i}-\tau_{i-1}) - d_1 (\tau_{i+1}-\tau_{i}) \right)
& \forall & u < 0
\end{eqnarray}
where
\begin{eqnarray}
d_1 & = & \frac{1}{6} ( 2 - |c| ) ( 1 - |c| ) \\
d_2 & = & \frac{1}{6} ( 1 - |c| ) ( 1 + |c| )
\end{eqnarray}
The coefficients $d_0$ and $d_1$ approach $1/3$ and $1/6$ repectively
as the Courant number, $c$, vanishes. In this limit, the conventional
third order upwind method is recovered. For finite Courant number, the
deviations from the linear method are analogous to the diffusion added
to centered second order advection in the Lax-Wendroff scheme.

The DST3 method described above must be used in a forward-in-time
manner and is stable for $0 \le |c| \le 1$. Although the scheme
appears to be forward-in-time, it is in fact second order in time and
the accuracy increases with the Courant number! For low Courant
number, DST3 produces very similar results (indistinguishable in
Fig.~\ref{fig:advect-1d-lo}) to the linear third order method but for
large Courant number, where the linear upwind third order method is
unstable, the scheme is extremely accurate
(Fig.~\ref{fig:advect-1d-hi}) with only minor overshoots.

\subsection{Third order direct space time with flux limiting}

The overshoots in the DST3 method can be controlled with a flux limiter.
The limited flux is written:
\begin{equation}
F =
\frac{1}{2}(u+|u|)\left( \tau_{i-1} + \psi(r^+)(\tau_{i} - \tau_{i-1} )\right)
+
\frac{1}{2}(u-|u|)\left( \tau_{i-1} + \psi(r^-)(\tau_{i} - \tau_{i-1} )\right)
\end{equation}
where
\begin{eqnarray}
r^+ & = & \frac{\tau_{i-1} - \tau_{i-2}}{\tau_{i} - \tau_{i-1}} \\
r^- & = & \frac{\tau_{i+1} - \tau_{i}}{\tau_{i} - \tau_{i-1}}
\end{eqnarray}
and the limiter is the Sweby limiter:
\begin{equation}
\psi(r) = \max[0, \min[\min(1,d_0+d_1r],\frac{1-c}{c}r ]]
\end{equation}

\subsection{Multi-dimensional advection}

In many of the aforementioned advection schemes the behaviour in
multiple dimensions is not necessarily as good as the one dimensional
behaviour. For instance, a shape preserving monotonic scheme in one
dimension can have severe shape distortion in two dimensions if the
two components of horizontal fluxes are treated independently. There
is a large body of literature on the subject dealing with this problem
and among the fixes are operator and flux splitting methods, corner
flux methods and more. We have adopted a variant on the standard
splitting methods that allows the flux calculations to be implemented
as if in one dimension:
\begin{eqnarray}
\tau^{n+1/3} & = & \tau^{n}
- \Delta t \left( \frac{1}{\Delta x} \delta_i F^x(\tau^{n})
           + \tau^{n} \frac{1}{\Delta x} \delta_i u \right) \\
\tau^{n+2/3} & = & \tau^{n}
- \Delta t \left( \frac{1}{\Delta y} \delta_j F^y(\tau^{n+1/3})
           + \tau^{n} \frac{1}{\Delta y} \delta_i v \right) \\
\tau^{n+3/3} & = & \tau^{n}
- \Delta t \left( \frac{1}{\Delta r} \delta_k F^x(\tau^{n+2/3})
           + \tau^{n} \frac{1}{\Delta r} \delta_i w \right)
\end{eqnarray}

In order to incorporate this method into the general model algorithm,
we compute the effective tendancy rather than update the tracer so
that other terms such as diffusion are using the $n$ time-level and
not the updated $n+3/3$ quantities:
\begin{equation}
G^{n+1/2}_{adv} = \frac{1}{\Delta t} ( \tau^{n+3/3} - \tau^{n} )
\end{equation}
So that the over all time-stepping looks likes:
\begin{equation}
\tau^{n+1} = \tau^{n} + \Delta t \left( G^{n+1/2}_{adv} + G_{diff}(\tau^{n}) + G^{n}_{forcing} \right)
\end{equation}
