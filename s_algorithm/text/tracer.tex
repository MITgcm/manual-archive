% $Header: /u/gcmpack/manual/s_algorithm/text/tracer.tex,v 1.2 2001/08/09 20:45:27 adcroft Exp $
% $Name:  $

\section{Tracer equations}

The basic discretization used for the tracer equations is the second
order piece-wise constant finite volume form of the forced
advection-diussion equations. There are many alternatives to second
order method for advection and alternative parameterizations for the
sub-grid scale processes. The Gent-McWilliams eddy parameterization,
KPP mixing scheme and PV flux parameterization are all dealt with in
separate sections. The basic discretization of the advection-diffusion
part of the tracer equations and the various advection schemes will be
described here.

\subsection{Centered second order advection-diffusion}

The basic discretization, centered second order, is the default. It is
designed to be consistant with the continuity equation to facilitate
conservation properties analogous to the continuum:
\begin{equation}
{\cal A}_c \Delta r_f h_c \partial_\theta
+ \delta_i F_x
+ \delta_j F_y
+ \delta_k F_r
= {\cal A}_c \Delta r_f h_c {\cal S}_\theta
+ \theta {\cal A}_c \delta_k (P-E)_{r=0}
\end{equation}
where the area integrated fluxes are given by:
\begin{eqnarray}
F_x & = & U \overline{ \theta }^i
- \kappa_h \frac{\Delta y_g \Delta r_f h_w}{\Delta x_c} \delta_i \theta \\
F_y & = & V \overline{ \theta }^j
- \kappa_h \frac{\Delta x_g \Delta r_f h_s}{\Delta y_c} \delta_j \theta \\
F_r & = & W \overline{ \theta }^k
- \kappa_v \frac{\Delta x_g \Delta y_g}{\Delta r_c} \delta_k \theta
\end{eqnarray}
The quantities $U$, $V$ and $W$ are volume fluxes defined:
\marginpar{$U$: {\bf uTrans} }
\marginpar{$V$: {\bf vTrans} }
\marginpar{$W$: {\bf rTrans} }
\begin{eqnarray}
U & = & \Delta y_g \Delta r_f h_w u \\
V & = & \Delta x_g \Delta r_f h_s v \\
W & = & {\cal A}_c w
\end{eqnarray}
${\cal S}$ represents the ``parameterized'' SGS processes and physics
and sources associated with the tracer. For instance, potential
temperature equation in the ocean has is forced by surface and
partially penetrating heat fluxes:
\begin{equation}
{\cal A}_c \Delta r_f h_c {\cal S}_\theta = \frac{1}{c_p \rho_o} \delta_k {\cal A}_c {\cal Q}
\end{equation}
while the salt equation has no real sources, ${\cal S}=0$, which
leaves just the $P-E$ term.

The continuity equation can be recovered by setting ${\cal Q}=0$, $\kappa_h = \kappa_v = 0$ and
$\theta=1$. The term $\theta (P-E)_{r=0}$ is required to retain local
conservation of $\theta$. Global conservation is not possible using
the flux-form (as here) and a linearized free-surface
(\cite{Griffies00,Campin02}).




