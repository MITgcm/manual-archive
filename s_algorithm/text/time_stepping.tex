% $Header: /u/gcmpack/manual/s_algorithm/text/time_stepping.tex,v 1.6 2001/09/26 20:19:52 adcroft Exp $
% $Name:  $

The convention used in this section is as follows:
Time is ``discretized'' using a time step $\Delta t$   
and $\Phi^n$ refers to the variable $\Phi$ 
at time $t = n \Delta t$ . We use the notation $\Phi^{(n)}$
when time interpolation is required to estimate the value of $\phi$
at the time $n \Delta t$.

\section{Time integration}

The discretization in time of the model equations (cf section I )
does not depend of the discretization in space of each
term and so  this section can be read independently.
 
The continuous form of the model equations is:

\begin{eqnarray}
\partial_t \theta & = & G_\theta
\label{eq-tCsC-theta}
\\
\partial_t S & = & G_s
\label{eq-tCsC-salt}
\\
b' & = & b'(\theta,S,r)
\\
\partial_r \phi'_{hyd} & = & -b'
\label{eq-tCsC-hyd}
\\
\partial_t \vec{\bf v}
+ {\bf \nabla}_h b_s \eta
+ \epsilon_{nh} {\bf \nabla}_h \phi'_{nh}
& = & \vec{\bf G}_{\vec{\bf v}} 
- {\bf \nabla}_h \phi'_{hyd}
\label{eq-tCsC-Hmom}
\\
\epsilon_{nh} \frac {\partial{\dot{r}}}{\partial{t}}
+ \epsilon_{nh} \partial_r \phi'_{nh}
& = & \epsilon_{nh} G_{\dot{r}} 
\label{eq-tCsC-Vmom}
\\
{\bf \nabla}_h \cdot \vec{\bf v} + \partial_r \dot{r}
& = & 0
\label{eq-tCsC-cont}
\end{eqnarray}
where
\begin{eqnarray*}
G_\theta & = &
- \vec{\bf v} \cdot {\bf \nabla} \theta + {\cal Q}_\theta
\\
G_S & = &
- \vec{\bf v} \cdot {\bf \nabla} S + {\cal Q}_S
\\
\vec{\bf G}_{\vec{\bf v}}
& = &
- \vec{\bf v} \cdot {\bf \nabla} \vec{\bf v}
- f \hat{\bf k} \wedge \vec{\bf v}
+ \vec{\cal F}_{\vec{\bf v}}
\\
G_{\dot{r}}
& = &
- \vec{\bf v} \cdot {\bf \nabla} \dot{r}
+ {\cal F}_{\dot{r}}
\end{eqnarray*}
The exact form of all the ``{\it G}''s terms is described in the next
section \ref{sect:discrete}. Here its sufficient to mention that they contains
all the RHS terms except the pressure/geo-potential terms.

The switch $\epsilon_{nh}$ allows one to activate the non-hydrostatic
mode ($\epsilon_{nh}=1$) for the ocean model. Otherwise, in the
hydrostatic limit $\epsilon_{nh} = 0$ and equation \ref{eq-tCsC-Vmom}
is not used.

As discussed in section \ref{sect:1.3.6.2}, the equation for $\eta$ is
obtained by integrating the continuity equation over the entire depth
of the fluid, from $R_{fixed}(x,y)$ up to $R_o(x,y)$. The linear free
surface evolves according to:
\begin{eqnarray}
\epsilon_{fs} \partial_t \eta =
\left. \dot{r} \right|_{r=r_{surf}} + \epsilon_{fw} (P-E) =
- {\bf \nabla} \cdot \int_{R_{fixed}}^{R_o} \vec{\bf v} dr
+ \epsilon_{fw} (P-E)
\label{eq-tCsC-eta}
\end{eqnarray}

Here, $\epsilon_{fs}$ is a flag to switch between the free-surface,
$\epsilon_{fs}=1$, and a rigid-lid, $\epsilon_{fs}=0$. The flag
$\epsilon_{fw}$ determines whether an exchange of fresh water is
included at the ocean surface (natural BC) ($\epsilon_{fw} = 1$) or
not ($\epsilon_{fw} = 0$).

The hydrostatic potential is found by integrating (equation
\ref{eq-tCsC-hyd}) with the boundary condition that
$\phi'_{hyd}(r=R_o) = 0$:
\begin{eqnarray*}
& &
\int_{r'}^{R_o} \partial_r \phi'_{hyd} dr =
\left[ \phi'_{hyd} \right]_{r'}^{R_o} =
\int_{r'}^{R_o} - b' dr
\\
\Rightarrow & &
\phi'_{hyd}(x,y,r') = \int_{r'}^{R_o} b' dr
\end{eqnarray*}

\subsection{General method}
 
An overview of the general method is now presented with explicit
references to the Fortran code. We often refer to the discretized
equations of the model that are detailed in the following sections.

The general algorithm consist of a ``predictor step'' that computes
the forward tendencies ("G" terms") comprising of explicit-in-time
terms and the ``first guess'' values (star notation): $\theta^*, S^*,
\vec{\bf v}^*$ (and $\dot{r}^*$ in non-hydrostatic mode). This is done
in the two routines {\it THERMODYNAMICS} and {\it DYNAMICS}.

Terms that are integrated implicitly in time are handled at the end of
the {\it THERMODYNAMICS} and {\it DYNAMICS} routines. Then the
surface pressure and non hydrostatic pressure are solved for in ({\it
SOLVE\_FOR\_PRESSURE}).

Finally, in the ``corrector step'', (routine {\it
THE\_CORRECTION\_STEP}) the new values of $u,v,w,\theta,S$ are
determined (see details in \ref{sect:II.1.3}).

At this point, the regular time stepping process is complete. However,
after the correction step there are optional adjustments such as
convective adjustment or filters (zonal FFT filter, shapiro filter)
that can be applied on both momentum and tracer fields, just prior to
incrementing the time level to $n+1$.

Since the pressure solver precision is of the order of the ``target
residual'' and can be lower than the the computer truncation error,
and also because some filters might alter the divergence part of the
flow field, a final evaluation of the surface r anomaly $\eta^{n+1}$
is performed in {\it CALC\_EXACT\_ETA}. This ensures exact volume
conservation. Note that there is no need for an equivalent
non-hydrostatic ``exact conservation'' step, since by default $w$ is
already computed after the filters are applied.

Optional forcing terms (usually part of a physics ``package''), that
account for a specific source or sink process (e.g. condensation as a
sink of water vapor Q) are generally incorporated in the main
algorithm as follows: at the the beginning of the time step, the
additional tendencies are computed as a function of the present state
(time step $n$) and external forcing; then within the main part of
model, only those new tendencies are added to the model variables.

[more details needed]\\

The atmospheric physics follows this general scheme.

[more about C\_grid, A\_grid conversion \& drag term]\\



\subsection{Standard synchronous time stepping}

In the standard formulation, the surface pressure is evaluated at time
step n+1 (an implicit method).  Equations \ref{eq-tCsC-theta} to
\ref{eq-tCsC-cont} are then discretized in time as follows:
\begin{eqnarray}
\left[ 1 - \partial_r \kappa_v^\theta \partial_r \right]
\theta^{n+1} & = & \theta^*
\label{eq-tDsC-theta}
\\
\left[ 1 - \partial_r \kappa_v^S \partial_r \right]
S^{n+1} & = & S^*
\label{eq-tDsC-salt}
\\
%{b'}^{n} & = & b'(\theta^{n},S^{n},r)
%\partial_r {\phi'_{hyd}}^{n} & = & {-b'}^{n}
%\\
{\phi'_{hyd}}^{n} & = & \int_{r'}^{R_o} b'(\theta^{n},S^{n},r) dr
\label{eq-tDsC-hyd}
\\
\vec{\bf v} ^{n+1}
+ \Delta t {\bf \nabla}_h b_s {\eta}^{n+1}
+ \epsilon_{nh} \Delta t {\bf \nabla} {\phi'_{nh}}^{n+1}
- \partial_r A_v \partial_r \vec{\bf v}^{n+1}
& = &
\vec{\bf v}^*
\label{eq-tDsC-Hmom}
\\
\epsilon_{fs} {\eta}^{n+1} + \Delta t
{\bf \nabla}_h \cdot \int_{R_{fixed}}^{R_o} \vec{\bf v}^{n+1} dr
& = & 
    \epsilon_{fs} {\eta}^{n} + \epsilon_{fw} \Delta_t (P-E)^{n} 
\nonumber
\\
% = \epsilon_{fs} {\eta}^{n} & + & \epsilon_{fw} \Delta_t (P-E)^{n} 
\label{eq-tDsC-eta}
\\
\epsilon_{nh} \left( \dot{r} ^{n+1}
+ \Delta t \partial_r {\phi'_{nh}} ^{n+1}
\right)
& = & \epsilon_{nh} \dot{r}^*
\label{eq-tDsC-Vmom}
\\
{\bf \nabla}_h \cdot \vec{\bf v}^{n+1} + \partial_r \dot{r}^{n+1}
& = & 0
\label{eq-tDsC-cont}
\end{eqnarray}
where
\begin{eqnarray}
\theta^* & = &
\theta ^{n} + \Delta t G_{\theta} ^{(n+1/2)}
\\
S^* & = &
S ^{n} + \Delta t G_{S} ^{(n+1/2)}
\\
\vec{\bf v}^* & = &
\vec{\bf v}^{n} + \Delta t \vec{\bf G}_{\vec{\bf v}} ^{(n+1/2)}
+ \Delta t  {\bf \nabla}_h {\phi'_{hyd}}^{(n+1/2)}
\\
\dot{r}^* & = &
\dot{r} ^{n} + \Delta t G_{\dot{r}} ^{(n+1/2)}
\end{eqnarray}

Note that implicit vertical viscosity and diffusivity terms are not
considered as part of the ``{\it G}'' terms, but are written
separately here.

The default time-stepping method is the Adams-Bashforth quasi-second
order scheme in which the ``G'' terms are extrapolated forward in time
from time-levels $n-1$ and $n$ to time-level $n+1/2$ and provides a
simple but stable algorithm:
\begin{equation}
G^{(n+1/2)} = G^n + (1/2+\epsilon_{AB}) (G^n - G^{n-1})
\end{equation}
where $\epsilon_{AB}$ is a small number used to stabilize the time
stepping.

In the standard non-staggered formulation, the Adams-Bashforth time
stepping is also applied to the hydrostatic pressure/geo-potential
gradient, $\nabla_h \Phi'_{hyd}$.  Note that presently, this term is in
fact incorporated to the $\vec{\bf G}_{\vec{\bf v}}$ arrays ({\bf
gU,gV}).
\marginpar{JMC: Clarify this term?}


\subsection{Stagger baroclinic time stepping}

An alternative to synchronous time-stepping is to stagger the momentum
and tracer fields in time. This allows the evaluation and gradient of
$\phi'_{hyd}$ to be centered in time with out needing to use the
Adams-Bashforth extrapoltion. This option is known as staggered
baroclinic time stepping because tracer and momentum are stepped
forward-in-time one after the other.  It can be activated by turning
on a run-time parameter {\bf staggerTimeStep} in namelist ``{\it
PARM01}''.

The main advantage of staggered time-stepping compared to synchronous,
is improved stability to internal gravity wave motions and a very
natural implementation of a 2nd order in time hydrostatic
pressure/geo-potential gradient term. However, synchronous
time-stepping might be better for convection problems, time dependent
forcing and diagnosing the model are also easier and it allows a more
efficient parallel decomposition.

The staggered baroclinic time-stepping scheme is equations
\ref{eq-tDsC-theta} to \ref{eq-tDsC-cont} except that \ref{eq-tDsC-hyd} is replaced with:
\begin{equation}
{\phi'_{hyd}}^{n+1/2} = \int_{r'}^{R_o} b'(\theta^{n+1/2},S^{n+1/2},r)
dr
\end{equation}
and the time-level for tracers has been shifted back by half:
\begin{eqnarray*}
\theta^* & = &
\theta ^{(n-1/2)} + \Delta t G_{\theta} ^{(n)}
\\
S^* & = &
S ^{(n-1/2)} + \Delta t G_{S} ^{(n)}
\\
\left[ 1 - \partial_r \kappa_v^\theta \partial_r \right]
\theta^{n+1/2} & = & \theta^*
\\
\left[ 1 - \partial_r \kappa_v^S \partial_r \right]
S^{n+1/2} & = & S^*
\end{eqnarray*}


\subsection{Surface pressure}

Substituting \ref{eq-tDsC-Hmom} into \ref{eq-tDsC-cont}, assuming
$\epsilon_{nh} = 0$ yields a Helmholtz equation for ${\eta}^{n+1}$:
\begin{eqnarray}
\epsilon_{fs} {\eta}^{n+1} -
{\bf \nabla}_h \cdot \Delta t^2 (R_o-R_{fixed})
{\bf \nabla}_h b_s {\eta}^{n+1}
= {\eta}^*
\label{eq-solve2D}
\end{eqnarray}
where
\begin{eqnarray}
{\eta}^* = \epsilon_{fs} \: {\eta}^{n} -
\Delta t {\bf \nabla}_h \cdot \int_{R_{fixed}}^{R_o} \vec{\bf v}^* dr
\: + \: \epsilon_{fw} \Delta_t (P-E)^{n} 
\label{eq-solve2D_rhs}
\end{eqnarray}

Once ${\eta}^{n+1}$ has been found, substituting into
\ref{eq-tDsC-Hmom} yields $\vec{\bf v}^{n+1}$ if the model is
hydrostatic ($\epsilon_{nh}=0$):
$$
\vec{\bf v}^{n+1} = \vec{\bf v}^{*}
- \Delta t {\bf \nabla}_h b_s {\eta}^{n+1}
$$

This is known as the correction step. However, when the model is
non-hydrostatic ($\epsilon_{nh}=1$) we need an additional step and an
additional equation for $\phi'_{nh}$. This is obtained by substituting
\ref{eq-tDsC-Hmom} and \ref{eq-tDsC-Vmom} into
\ref{eq-tDsC-cont}:
\begin{equation}
\left[ {\bf \nabla}_h^2 + \partial_{rr} \right] {\phi'_{nh}}^{n+1}
= \frac{1}{\Delta t} \left(
{\bf \nabla}_h \cdot \vec{\bf v}^{**} + \partial_r \dot{r}^* \right)
\end{equation}
where
\begin{displaymath}
\vec{\bf v}^{**} = \vec{\bf v}^* - \Delta t {\bf \nabla}_h b_s {\eta}^{n+1}
\end{displaymath}
Note that $\eta^{n+1}$ is also used to update the second RHS term
$\partial_r \dot{r}^* $ since
the vertical velocity at the surface ($\dot{r}_{surf}$) 
is evaluated as $(\eta^{n+1} - \eta^n) / \Delta t$.

Finally, the horizontal velocities at the new time level are found by:
\begin{equation}
\vec{\bf v}^{n+1} = \vec{\bf v}^{**}
- \epsilon_{nh} \Delta t {\bf \nabla}_h {\phi'_{nh}}^{n+1}
\end{equation}
and the vertical velocity is found by integrating the continuity
equation vertically.  Note that, for the convenience of the restart
procedure, the vertical integration of the continuity equation has
been moved to the beginning of the time step (instead of at the end),
without any consequence on the solution.

Regarding the implementation of the surface pressure solver, all
computation are done within the routine {\it SOLVE\_FOR\_PRESSURE} and
its dependent calls.  The standard method to solve the 2D elliptic
problem (\ref{eq-solve2D}) uses the conjugate gradient method (routine
{\it CG2D}); the solver matrix and conjugate gradient operator are
only function of the discretized domain and are therefore evaluated
separately, before the time iteration loop, within {\it INI\_CG2D}.
The computation of the RHS $\eta^*$ is partly done in {\it
CALC\_DIV\_GHAT} and in {\it SOLVE\_FOR\_PRESSURE}.

The same method is applied for the non hydrostatic part, using a
conjugate gradient 3D solver ({\it CG3D}) that is initialized in {\it
INI\_CG3D}. The RHS terms of 2D and 3D problems are computed together
at the same point in the code.


\subsection{Crank-Nickelson barotropic time stepping}

The full implicit time stepping described previously is
unconditionally stable but damps the fast gravity waves, resulting in
a loss of potential energy.  The modification presented now allows one
to combine an implicit part ($\beta,\gamma$) and an explicit part
($1-\beta,1-\gamma$) for the surface pressure gradient ($\beta$) and
for the barotropic flow divergence ($\gamma$).
\\
For instance, $\beta=\gamma=1$ is the previous fully implicit scheme;
$\beta=\gamma=1/2$ is the non damping (energy conserving), unconditionally
stable, Crank-Nickelson scheme; $(\beta,\gamma)=(1,0)$ or $=(0,1)$
corresponds to the forward - backward scheme that conserves energy but is
only stable for small time steps.\\
In the code, $\beta,\gamma$ are defined as parameters, respectively 
{\it implicSurfPress}, {\it implicDiv2DFlow}. They are read from
the main data file "{\it data}" and are set by default to 1,1.

Equations \ref{eq-tDsC-Hmom} and \ref{eq-tDsC-eta} are modified as follows:
$$
\frac{ \vec{\bf v}^{n+1} }{ \Delta t }
+ {\bf \nabla}_h b_s [ \beta {\eta}^{n+1} + (1-\beta) {\eta}^{n} ] 
+ \epsilon_{nh} {\bf \nabla}_h {\phi'_{nh}}^{n+1}
 = \frac{ \vec{\bf v}^* }{ \Delta t }
$$
$$
\epsilon_{fs} \frac{ {\eta}^{n+1} - {\eta}^{n} }{ \Delta t}
+ {\bf \nabla}_h \cdot \int_{R_{fixed}}^{R_o} 
[ \gamma \vec{\bf v}^{n+1} + (1-\gamma) \vec{\bf v}^{n}] dr
= \epsilon_{fw} (P-E)
$$
where:
\begin{eqnarray*}
\vec{\bf v}^* & = &
\vec{\bf v} ^{n} + \Delta t \vec{\bf G}_{\vec{\bf v}} ^{(n+1/2)}
+ (\beta-1) \Delta t {\bf \nabla}_h b_s {\eta}^{n}
+ \Delta t {\bf \nabla}_h {\phi'_{hyd}}^{(n+1/2)}
\\
{\eta}^* & = &
\epsilon_{fs} {\eta}^{n} + \epsilon_{fw} \Delta t (P-E) 
- \Delta t {\bf \nabla}_h \cdot \int_{R_{fixed}}^{R_o} 
[ \gamma \vec{\bf v}^* + (1-\gamma) \vec{\bf v}^{n}] dr
\end{eqnarray*}
\\
In the hydrostatic case ($\epsilon_{nh}=0$), allowing us to find
${\eta}^{n+1}$, thus:
$$
\epsilon_{fs} {\eta}^{n+1} -
{\bf \nabla}_h \cdot \beta\gamma \Delta t^2 b_s (R_o - R_{fixed})
{\bf \nabla}_h {\eta}^{n+1}
= {\eta}^*
$$ 
and then to compute (correction step):
$$
\vec{\bf v}^{n+1} = \vec{\bf v}^{*}
- \beta \Delta t {\bf \nabla}_h b_s {\eta}^{n+1}
$$

The non-hydrostatic part is solved as described previously. 

Note that:
\begin{enumerate}
\item The non-hydrostatic part of the code has not yet been 
updated, so that this option cannot be used with $(\beta,\gamma) \neq (1,1)$.
\item The stability criteria with Crank-Nickelson time stepping
for the pure linear gravity wave problem in cartesian coordinates is:
\begin{itemize}
\item $\beta + \gamma < 1$ : unstable
\item $\beta \geq 1/2$ and $ \gamma \geq 1/2$ : stable
\item $\beta + \gamma \geq 1$ : stable if
$$ 
c_{max}^2 (\beta - 1/2)(\gamma - 1/2) + 1 \geq 0
$$
$$
\mbox{with }~
%c^2 = 2 g H {\Delta t}^2 
%(\frac{1-cos 2 \pi / k}{\Delta x^2} 
%+\frac{1-cos 2 \pi / l}{\Delta y^2})
%$$
%Practically, the most stringent condition is obtained with $k=l=2$ :
%$$
c_{max} =  2 \Delta t \: \sqrt{g H} \: 
\sqrt{ \frac{1}{\Delta x^2} + \frac{1}{\Delta y^2} }
$$
\end{itemize}
\end{enumerate}

