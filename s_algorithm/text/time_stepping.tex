% $Header: /u/gcmpack/manual/s_algorithm/text/time_stepping.tex,v 1.4 2001/08/17 18:38:10 jmc Exp $
% $Name:  $

The convention used in this section is as follows:
Time is "discretize" using a time step $\Delta t$   
and $\Phi^n$ refers to the variable $\Phi$ 
at time $t = n \Delta t$ . We used the notation $\Phi^{(n)}$
when time interpolation is required to estimate the value of $\phi$
at the time $n \Delta t$.

\section{Time integration}

The discretization in time of the model equations (cf section I )
does not depend of the discretization in space of each
term, so that this section can be read independently.
Also for this purpose, we will refers to the continuous 
space-derivative form of model equations, and focus on 
the time discretization.
 
The continuous form of the model equations is:

\begin{eqnarray}
\partial_t \theta & = & G_\theta
\label{eq-tCsC-theta}
\\
\partial_t S & = & G_s
\label{eq-tCsC-salt}
\\
b' & = & b'(\theta,S,r)
\\
\partial_r \phi'_{hyd} & = & -b'
\label{eq-tCsC-hyd}
\\
\partial_t \vec{\bf v}
+ {\bf \nabla}_h b_s \eta
+ \epsilon_{nh} {\bf \nabla}_h \phi'_{nh}
& = & \vec{\bf G}_{\vec{\bf v}} 
- {\bf \nabla}_h \phi'_{hyd}
\label{eq-tCsC-Hmom}
\\
\epsilon_{nh} \frac {\partial{\dot{r}}}{\partial{t}}
+ \epsilon_{nh} \partial_r \phi'_{nh}
& = & \epsilon_{nh} G_{\dot{r}} 
\label{eq-tCsC-Vmom}
\\
{\bf \nabla}_h \cdot \vec{\bf v} + \partial_r \dot{r}
& = & 0
\label{eq-tCsC-cont}
\end{eqnarray}
where
\begin{eqnarray*}
G_\theta & = &
- \vec{\bf v} \cdot {\bf \nabla} \theta + {\cal Q}_\theta
\\
G_S & = &
- \vec{\bf v} \cdot {\bf \nabla} S + {\cal Q}_S
\\
\vec{\bf G}_{\vec{\bf v}}
& = &
- \vec{\bf v} \cdot {\bf \nabla} \vec{\bf v}
- f \hat{\bf k} \wedge \vec{\bf v}
+ \vec{\cal F}_{\vec{\bf v}}
\\
G_{\dot{r}}
& = &
- \vec{\bf v} \cdot {\bf \nabla} \dot{r}
+ {\cal F}_{\dot{r}}
\end{eqnarray*}
The exact form of all the "{\it G}"s terms is described in the next
section (). Here its sufficient to mention that they contains
all the RHS terms except the pressure / geo- potential terms.

The switch $\epsilon_{nh}$ allows to activate the non hydrostatic
mode ($\epsilon_{nh}=1$) for the ocean model. Otherwise, 
in the hydrostatic limit $\epsilon_{nh} = 0$ 
and equation \ref{eq-tCsC-Vmom} vanishes.

The equation for $\eta$ is obtained by integrating the 
continuity equation over the entire depth of the fluid, 
from $R_{min}(x,y)$ up to $R_o(x,y)$ 
(Linear free surface):
\begin{eqnarray}
\epsilon_{fs} \partial_t \eta =
\left. \dot{r} \right|_{r=r_{surf}} + \epsilon_{fw} (P-E) =
- {\bf \nabla} \cdot \int_{R_{min}}^{R_o} \vec{\bf v} dr
+ \epsilon_{fw} (P-E)
\label{eq-tCsC-eta}
\end{eqnarray}

Where $\epsilon_{fs}$,$\epsilon_{fw}$ are two flags to 
distinguish between a free-surface equation ($\epsilon_{fs}=1$) 
or the rigid-lid approximation ($\epsilon_{fs}=0$);  
and to distinguish when exchange of Fresh-Water is included 
at the ocean surface (natural BC) ($\epsilon_{fw} = 1$) 
or not ($\epsilon_{fw} = 0$).

The hydrostatic potential is found by
integrating \ref{eq-tCsC-hyd} with the boundary condition that
$\phi'_{hyd}(r=R_o) = 0$:
\begin{eqnarray*}
& &
\int_{r'}^{R_o} \partial_r \phi'_{hyd} dr =
\left[ \phi'_{hyd} \right]_{r'}^{R_o} =
\int_{r'}^{R_o} - b' dr
\\
\Rightarrow & &
\phi'_{hyd}(x,y,r') = \int_{r'}^{R_o} b' dr
\end{eqnarray*}

\subsection{General method}
 
An overview of the general method is presented hereafter, 
with explicit references to the Fortran code. This part
often refers to the discretized equations of the model
that are detailed in the following sections.

The general algorithm consist in  a "predictor step" that computes
the forward tendencies ("G" terms") and all 
the "first guess" values (star notation): 
$\theta^*, S^*, \vec{\bf v}^*$ (and $\dot{r}^*$
in non-hydrostatic mode). This is done in the two routines 
{\it THERMODYNAMICS} and {\it DYNAMICS}.

Then the implicit terms that appear on the left hand side (LHS)
of equations \ref{eq-tDsC-theta} - \ref{eq-tDsC-cont},
are solved as follows:
Since implicit vertical diffusion and viscosity terms 
are independent from the barotropic flow adjustment,
they are computed first, solving a 3 diagonal Nr x Nr linear system, 
and incorporated at the end of the {\it THERMODYNAMICS} and 
{\it DYNAMICS} routines.
Then the surface pressure and non hydrostatic pressure
are evaluated ({\it SOLVE\_FOR\_PRESSURE}); 

Finally, within a "corrector step', 
(routine {\it THE\_CORRECTION\_STEP})
the new values of $u,v,w,\theta,S$ 
are derived according to the above equations
(see details in II.1.3). 

At this point, the regular time step is over, but  
the correction step contains also other optional
adjustments such as convective adjustment algorithm, or filters 
(zonal FFT filter, shapiro filter)
that applied on both momentum and tracer fields.
just before setting the $n+1$ new time step value.

Since the pressure solver precision is of the order of
the "target residual" that could be lower than the 
the computer truncation error, and also because some filters 
might alter the divergence part of the flow field,
a final evaluation of the surface r anomaly $\eta^{n+1}$
is performed, according to \ref{eq-tDsC-eta} ({\it CALC\_EXACT\_ETA}).
This ensures a perfect volume conservation.
Note that there is no need for an equivalent Non-hydrostatic
"exact conservation" step, since W is already computed after 
the filters applied.

Regarding optional forcing terms (usually part of a "package"), 
that account for a specific source or sink term (e.g.: condensation
as a sink of water vapor Q), they are generally incorporated 
in the main algorithm as follows;
At the the beginning of the time step,
the additional tendencies are computed
as function of the present state (time step $n$) and external forcing ;
Then within the main part of model,
only those new tendencies are added to the model variables.

[more details needed]\\

The atmospheric physics follows this general scheme.

[more about C\_grid, A\_grid conversion \& drag term]\\

\subsection{Standard synchronous time stepping}

In the standard formulation, the surface pressure is 
evaluated at time step n+1 (implicit method).
The above set of equations is then discretized in time 
as follows:
\begin{eqnarray}
\left[ 1 - \partial_r \kappa_v^\theta \partial_r \right]
\theta^{n+1} & = & \theta^*
\label{eq-tDsC-theta}
\\
\left[ 1 - \partial_r \kappa_v^S \partial_r \right]
S^{n+1} & = & S^*
\label{eq-tDsC-salt}
\\
%{b'}^{n} & = & b'(\theta^{n},S^{n},r)
%\partial_r {\phi'_{hyd}}^{n} & = & {-b'}^{n}
%\\
{\phi'_{hyd}}^{n} & = & \int_{r'}^{R_o} b'(\theta^{n},S^{n},r) dr
\label{eq-tDsC-hyd}
\\
\vec{\bf v} ^{n+1}
+ \Delta t {\bf \nabla}_h b_s {\eta}^{n+1}
+ \epsilon_{nh} \Delta t {\bf \nabla} {\phi'_{nh}}^{n+1}
- \partial_r A_v \partial_r \vec{\bf v}^{n+1}
& = &
\vec{\bf v}^*
\label{eq-tDsC-Hmom}
\\
\epsilon_{fs} {\eta}^{n+1} + \Delta t
{\bf \nabla}_h \cdot \int_{R_{min}}^{R_o} \vec{\bf v}^{n+1} dr
& = & 
    \epsilon_{fs} {\eta}^{n} + \epsilon_{fw} \Delta_t (P-E)^{n} 
\nonumber
\\
% = \epsilon_{fs} {\eta}^{n} & + & \epsilon_{fw} \Delta_t (P-E)^{n} 
\label{eq-tDsC-eta}
\\
\epsilon_{nh} \left( \dot{r} ^{n+1}
+ \Delta t \partial_r {\phi'_{nh}} ^{n+1}
\right)
& = & \epsilon_{nh} \dot{r}^*
\label{eq-tDsC-Vmom}
\\
{\bf \nabla}_h \cdot \vec{\bf v}^{n+1} + \partial_r \dot{r}^{n+1}
& = & 0
\label{eq-tDsC-cont}
\end{eqnarray}
where
\begin{eqnarray}
\theta^* & = &
\theta ^{n} + \Delta t G_{\theta} ^{(n+1/2)}
\\
S^* & = &
S ^{n} + \Delta t G_{S} ^{(n+1/2)}
\\
\vec{\bf v}^* & = &
\vec{\bf v}^{n} + \Delta t \vec{\bf G}_{\vec{\bf v}} ^{(n+1/2)}
+ \Delta t  {\bf \nabla}_h {\phi'_{hyd}}^{(n+1/2)}
\\
\dot{r}^* & = &
\dot{r} ^{n} + \Delta t G_{\dot{r}} ^{(n+1/2)}
\end{eqnarray}

Note that implicit vertical terms (viscosity and diffusivity) are 
not considered as part of the "{\it G}" terms, but are 
written separately here.

To ensure a second order time discretization for both 
momentum and tracer,
The "{\it G}" terms are "extrapolated" forward in time
(Adams Bashforth time stepping)
from the values computed at time step $n$ and $n-1$
to the time $n+1/2$:
$$G^{(n+1/2)} = G^n + (1/2+\epsilon_{AB}) (G^n - G^{n-1})$$
A small number for the parameter $\epsilon_{AB}$ is generally used 
to stabilize this time stepping.

In the standard non-stagger formulation, 
the Adams-Bashforth time stepping is also applied to the 
hydrostatic (pressure / geo-) potential term $\nabla_h \Phi'_{hyd}$. 
Note that presently, this term is in fact incorporated to the 
$\vec{\bf G}_{\vec{\bf v}}$ arrays ({\bf gU,gV}).

\subsection{Stagger baroclinic time stepping}

An alternative is to evaluate $\phi'_{hyd}$ with the 
new tracer fields, and step forward the momentum after.
This option is known as stagger baroclinic time stepping, 
since tracer and momentum are step forward in time one after the other.
It can be activated turning on a running flag parameter 
{\bf staggerTimeStep} in file "{\it data}"). 

The main advantage of this time stepping compared to a synchronous one,
is a better stability, specially regarding internal gravity waves,
and a very natural implementation of a 2nd order in time 
hydrostatic pressure / geo- potential term.
In the other hand, a synchronous time step might be  better
for convection problems; Its also make simpler time dependent forcing
and diagnostic implementation ; and allows a more efficient threading.

Although the stagger time step does not affect deeply the 
structure of the code --- a switch allows to evaluate the 
hydrostatic pressure / geo- potential from new $\theta,S$ 
instead of the Adams-Bashforth estimation ---
this affect the way the time discretization is presented :

\begin{eqnarray*}
\left[ 1 - \partial_r \kappa_v^\theta \partial_r \right]
\theta^{n+1/2} & = & \theta^*
\\
\left[ 1 - \partial_r \kappa_v^S \partial_r \right]
S^{n+1/2} & = & S^*
\end{eqnarray*}
with
\begin{eqnarray*}
\theta^* & = &
\theta ^{(n-1/2)} + \Delta t G_{\theta} ^{(n)}
\\
S^* & = &
S ^{(n-1/2)} + \Delta t G_{S} ^{(n)}
\end{eqnarray*}
And 
\begin{eqnarray*}
%{b'}^{n+1/2} & = & b'(\theta^{n+1/2},S^{n+1/2},r)
%\\
%\partial_r {\phi'_{hyd}}^{n+1/2} & = & {-b'}^{n+1/2}
{\phi'_{hyd}}^{n+1/2} & = & \int_{r'}^{R_o} b'(\theta^{n+1/2},S^{n+1/2},r) dr
%\label{eq-tDsC-hyd}
\\
\vec{\bf v} ^{n+1}
+ \Delta t {\bf \nabla}_h b_s {\eta}^{n+1}
+ \epsilon_{nh} \Delta t {\bf \nabla}_h {\phi'_{nh}}^{n+1}
- \partial_r A_v \partial_r \vec{\bf v}^{n+1}
& = &
\vec{\bf v}^*
%\label{eq-tDsC-Hmom}
\\
\epsilon_{fs} {\eta}^{n+1} + \Delta t
{\bf \nabla}_h \cdot \int_{R_{min}}^{R_o} \vec{\bf v}^{n+1} dr
& = & 
\epsilon_{fs} {\eta}^{n} + \epsilon_{fw} \Delta_t (P-E)^{n} 
\\
\epsilon_{nh} \left( \dot{r} ^{n+1}
+ \Delta t \partial_r {\phi'_{nh}} ^{n+1}
\right)
& = & \epsilon_{nh} \dot{r}^*
%\label{eq-tDsC-Vmom}
\\
{\bf \nabla}_h \cdot \vec{\bf v}^{n+1} + \partial_r \dot{r}^{n+1}
& = & 0
%\label{eq-tDsC-cont}
\end{eqnarray*}
with
\begin{eqnarray*}
\vec{\bf v}^* & = &
\vec{\bf v} ^{n} + \Delta t \vec{\bf G}_{\vec{\bf v}} ^{(n+1/2)}
+ \Delta t  {\bf \nabla}_h {\phi'_{hyd}}^{n+1/2}
\\
\dot{r}^* & = &
\dot{r} ^{n} + \Delta t G_{\dot{r}} ^{(n+1/2)}
\end{eqnarray*}

%---------------------------------------------------------------------

\subsection{Surface pressure}

Substituting \ref{eq-tDsC-Hmom} into \ref{eq-tDsC-cont}, assuming
$\epsilon_{nh} = 0$ yields a Helmholtz equation for ${\eta}^{n+1}$:
\begin{eqnarray}
\epsilon_{fs} {\eta}^{n+1} -
{\bf \nabla}_h \cdot \Delta t^2 (R_o-R_{min})
{\bf \nabla}_h b_s {\eta}^{n+1}
= {\eta}^*
\label{eq-solve2D}
\end{eqnarray}
where
\begin{eqnarray}
{\eta}^* = \epsilon_{fs} \: {\eta}^{n} -
\Delta t {\bf \nabla}_h \cdot \int_{R_{min}}^{R_o} \vec{\bf v}^* dr
\: + \: \epsilon_{fw} \Delta_t (P-E)^{n} 
\label{eq-solve2D_rhs}
\end{eqnarray}

Once ${\eta}^{n+1}$ has been found substituting into \ref{eq-tDsC-Hmom}
would yield $\vec{\bf v}^{n+1}$ if the model is hydrostatic
($\epsilon_{nh}=0$):
$$
\vec{\bf v}^{n+1} = \vec{\bf v}^{*}
- \Delta t {\bf \nabla}_h b_s {\eta}^{n+1}
$$

This is known as the correction step. However, when the model is
non-hydrostatic ($\epsilon_{nh}=1$) we need an additional step and an
additional equation for $\phi'_{nh}$. This is obtained by
substituting \ref{eq-tDsC-Hmom} and \ref{eq-tDsC-Vmom} into
\ref{eq-tDsC-cont}:
\begin{equation}
\left[ {\bf \nabla}_h^2 + \partial_{rr} \right] {\phi'_{nh}}^{n+1}
= \frac{1}{\Delta t} \left(
{\bf \nabla}_h \cdot \vec{\bf v}^{**} + \partial_r \dot{r}^* \right)
\end{equation}
where
\begin{displaymath}
\vec{\bf v}^{**} = \vec{\bf v}^* - \Delta t {\bf \nabla}_h b_s {\eta}^{n+1}
\end{displaymath}
Note that $\eta^{n+1}$ is also used to update the second RHS term
$\partial_r \dot{r}^* $ since
the vertical velocity at the surface ($\dot{r}_{surf}$) 
is evaluated as $(\eta^{n+1} - \eta^n) / \Delta t$.

Finally, the horizontal velocities at the new time level are found by:
\begin{equation}
\vec{\bf v}^{n+1} = \vec{\bf v}^{**}
- \epsilon_{nh} \Delta t {\bf \nabla}_h {\phi'_{nh}}^{n+1}
\end{equation}
and the vertical velocity is found by integrating the continuity
equation vertically.
Note that, for convenience regarding the restart procedure,
the integration of the continuity equation has been 
moved at the beginning of the time step (instead of at the end),
without any consequence on the solution.

Regarding the implementation, all those computation are done
within the routine {\it SOLVE\_FOR\_PRESSURE} and its dependent 
{\it CALL}s.
The standard method to solve the 2D elliptic problem (\ref{eq-solve2D})
uses the conjugate gradient method (routine {\it CG2D}); The 
the solver matrix and conjugate gradient operator are only function
of the discretized domain and are therefore evaluated separately,
before the time iteration loop, within {\it INI\_CG2D}. 
The computation of the RHS $\eta^*$ is partly 
done in {\it CALC\_DIV\_GHAT} and in {\it SOLVE\_FOR\_PRESSURE}.

The same method is applied for the non hydrostatic part, using
a conjugate gradient 3D solver ({\it CG3D}) that is initialized 
in {\it INI\_CG3D}. The RHS terms of 2D and 3D problems 
are computed together, within the same part of the code.

\newpage
%-----------------------------------------------------------------------------------
\subsection{Crank-Nickelson barotropic time stepping}

The full implicit time stepping described previously is unconditionally stable
but damps the fast gravity waves, resulting in a loss of 
gravity potential energy.
The modification presented hereafter allows to combine an implicit part
($\beta,\gamma$) and an explicit part ($1-\beta,1-\gamma$) for the surface
pressure gradient ($\beta$) and for the barotropic flow divergence ($\gamma$).
\\
For instance, $\beta=\gamma=1$ is the previous fully implicit scheme;
$\beta=\gamma=1/2$ is the non damping (energy conserving), unconditionally
stable, Crank-Nickelson scheme; $(\beta,\gamma)=(1,0)$ or $=(0,1)$
corresponds to the forward - backward scheme that conserves energy but is
only stable for small time steps.\\
In the code, $\beta,\gamma$ are defined as parameters, respectively 
{\it implicSurfPress}, {\it implicDiv2DFlow}. They are read from
the main data file "{\it data}" and are set by default to 1,1.

Equations \ref{eq-tDsC-Hmom} and \ref{eq-tDsC-eta} are modified as follows:
$$
\frac{ \vec{\bf v}^{n+1} }{ \Delta t }
+ {\bf \nabla}_h b_s [ \beta {\eta}^{n+1} + (1-\beta) {\eta}^{n} ] 
+ \epsilon_{nh} {\bf \nabla}_h {\phi'_{nh}}^{n+1}
 = \frac{ \vec{\bf v}^* }{ \Delta t }
$$
$$
\epsilon_{fs} \frac{ {\eta}^{n+1} - {\eta}^{n} }{ \Delta t}
+ {\bf \nabla}_h \cdot \int_{R_{min}}^{R_o} 
[ \gamma \vec{\bf v}^{n+1} + (1-\gamma) \vec{\bf v}^{n}] dr
= \epsilon_{fw} (P-E)
$$
where:
\begin{eqnarray*}
\vec{\bf v}^* & = &
\vec{\bf v} ^{n} + \Delta t \vec{\bf G}_{\vec{\bf v}} ^{(n+1/2)}
+ (\beta-1) \Delta t {\bf \nabla}_h b_s {\eta}^{n}
+ \Delta t {\bf \nabla}_h {\phi'_{hyd}}^{(n+1/2)}
\\
{\eta}^* & = &
\epsilon_{fs} {\eta}^{n} + \epsilon_{fw} \Delta t (P-E) 
- \Delta t {\bf \nabla}_h \cdot \int_{R_{min}}^{R_o} 
[ \gamma \vec{\bf v}^* + (1-\gamma) \vec{\bf v}^{n}] dr
\end{eqnarray*}
\\
In the hydrostatic case ($\epsilon_{nh}=0$),
this allow to find ${\eta}^{n+1}$, according to:
$$
\epsilon_{fs} {\eta}^{n+1} -
{\bf \nabla}_h \cdot \beta\gamma \Delta t^2 b_s (R_o - R_{min})
{\bf \nabla}_h {\eta}^{n+1}
= {\eta}^*
$$ 
and then to compute (correction step):
$$
\vec{\bf v}^{n+1} = \vec{\bf v}^{*}
- \beta \Delta t {\bf \nabla}_h b_s {\eta}^{n+1}
$$

The non-hydrostatic part is solved as described previously. 
\\ \\
N.B:
\\
 a) The non-hydrostatic part of the code has not yet been 
updated, %since it falls out of the purpose of this test,
so that this option cannot be used with $(\beta,\gamma) \neq (1,1)$.
\\
b) To remind, the stability criteria with the Crank-Nickelson time stepping
for the pure linear gravity wave problem in cartesian coordinate is:
\\
$\star$~ $\beta + \gamma < 1$ : unstable
\\
$\star$~ $\beta \geq 1/2$ and $ \gamma \geq 1/2$ : stable
\\
$\star$~ $\beta + \gamma \geq 1$ : stable if
%, for all wave length $(k\Delta x,l\Delta y)$
$$ 
c_{max}^2 (\beta - 1/2)(\gamma - 1/2) + 1 \geq 0
$$
$$
\mbox{with }~
%c^2 = 2 g H {\Delta t}^2 
%(\frac{1-cos 2 \pi / k}{\Delta x^2} 
%+\frac{1-cos 2 \pi / l}{\Delta y^2})
%$$
%Practically, the most stringent condition is obtained with $k=l=2$ :
%$$
c_{max} =  2 \Delta t \: \sqrt{g H} \: 
\sqrt{ \frac{1}{\Delta x^2} + \frac{1}{\Delta y^2} }
$$
