% $Header: /u/gcmpack/manual/s_algorithm/text/notation.tex,v 1.8 2011/05/04 22:42:48 jmc Exp $
% $Name:  $

\section{Notation} 

Because of the particularity of the vertical direction in stratified fluid
context, in this chapter, the vector notations are mostly used
for the horizontal component:
the horizontal part of a vector is simply written 
$\vec{\bf v}$ (instead of ${\bf v_h}$ or $\vec{\mathbf{v}}_{h}$ in chaper 1)
and a 3.D vector is simply written $\vec{v}$ (instead of $\vec{\mathbf{v}}$ 
in chapter 1).

The notations we use to describe the discrete formulation 
of the model are summarized hereafter:\\
general notation:
\\ $\Delta x, \Delta y, \Delta r$ grid spacing in X,Y,R directions.
\\ $A_c,A_w,A_s,A_{\zeta}$ : 
horizontal area of a grid cell surrounding $\theta,u,v,\zeta$ point.
\\ ${\cal V}_u , {\cal V}_v , {\cal V}_w , {\cal V}_\theta$ :
Volume of the grid box surrounding $u,v,w,\theta$ point;
\\ $i,j,k$ : current index relative to X,Y,R directions;
\\basic operator:
\\ $\delta_i $ : $\delta_i \Phi = \Phi_{i+1/2} - \Phi_{i-1/2} $
\label{eq:delta_i}
\\ $~^{-i}$ : $\overline{\Phi}^i = ( \Phi_{i+1/2} + \Phi_{i-1/2} ) / 2 $ 
\label{eq:bar_i}
\\ $\delta_x $ : $\delta_x \Phi = \frac{1}{\Delta x} \delta_i \Phi $
\label{eq:delta_x}
\\
\\ $\overline{\nabla}$ = horizontal gradient operator :  
$\overline{\nabla} \Phi = \{ \delta_x \Phi , \delta_y \Phi \}$
\label{eq:d_grad}
\\ $\overline{\nabla} \cdot$ = horizontal divergence operator :  
$\overline{\nabla}\cdot \vec{\mathrm{f}}  = 
\frac{1}{\cal A} \{ \delta_i \Delta y \, \mathrm{f}_x 
                  + \delta_j \Delta x \, \mathrm{f}_y \} $
\label{eq:d_div}
\\ $\overline{\nabla}^2 $ = horizontal Laplacian operator :
$ \overline{\nabla}^2 \Phi = 
   \overline{\nabla}\cdot \overline{\nabla}\Phi $
\label{eq:d_lap}
