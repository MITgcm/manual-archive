% $Header: /u/gcmpack/manual/s_algorithm/text/notation.tex,v 1.3 2001/08/09 19:24:04 jmc Exp $
% $Name:  $

\section{Notation} 

The notations we use to discribe the discrete formulation 
of the model are summarised hereafter:\\
general notation:
\\ $\Delta x, \Delta y, \Delta r$ grid spacing in X,Y,R directions.
\\ $A_o$ : Area of the face orthogonal to "o" direction (o=u,v,w ...).
\\ ${\cal V}_u , {\cal V}_v , {\cal V}_v , {\cal V}_\theta$ :
Volume of the grid box surrounding $u,v,w,\theta$ point;
\\ $i,j,k$ : current index relative to X,Y,R directions;
\\basic operator:
\\ $\delta_i $ : $\delta_i \Phi = \Phi_{i+1/2} - \Phi_{i-1/2} $
\\ $\overline{~}i$ : $\overline{\Phi}^i = ( \Phi_{i+1/2} + \Phi_{i-1/2} ) / 2 $ 
\\ $\delta_x $ : $\delta_x \Phi = \frac{1}{\Delta x} \delta_i \Phi $
\\
\\ $\overline{\nabla}$ = gradient operator :  
$\overline{\nabla} \Phi = \{ \delta_x \Phi , \delta_y \Phi \}$
\\ $\overline{\nabla} \cdot$ = divergence operator :  
$\overline{\nabla}\cdot \vec{\mathrm{f}}  = 
\frac{1}{\cal A} \{ \delta_i \Delta y \mathrm{f}_x 
                  + \delta_j \Delta x \mathrm{f}_y \} $
\\ $\overline{\nabla}^2 $ = Laplacien operator :
$ \overline{\nabla}^2 \Phi = 
   \overline{\nabla}\cdot \overline{\nabla}\Phi $
