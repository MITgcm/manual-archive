% $Header: /u/gcmpack/manual/s_algorithm/text/mom_fluxform.tex,v 1.3 2001/09/28 03:28:32 adcroft Exp $
% $Name:  $

\section{Flux-form momentum equations}

The original finite volume model was based on the Eulerian flux form
momentum equations. This is the default though the vector invariant
form is optionally available (and recommended in some cases).

The ``G's'' (our colloquial name for all terms on rhs!) are broken
into the various advective, Coriolis, horizontal dissipation, vertical
dissipation and metric forces:
\marginpar{$G_u$: {\bf Gu} }
\marginpar{$G_v$: {\bf Gv} }
\marginpar{$G_w$: {\bf Gw} }
\begin{eqnarray}
G_u & = & G_u^{adv} + G_u^{cor} + G_u^{h-diss} + G_u^{v-diss} +
G_u^{metric} + G_u^{nh-metric} \label{eq:gsplit_momu} \\
G_v & = & G_v^{adv} + G_v^{cor} + G_v^{h-diss} + G_v^{v-diss} +
G_v^{metric} + G_v^{nh-metric} \label{eq:gsplit_momv} \\
G_w & = & G_w^{adv} + G_w^{cor} + G_w^{h-diss} + G_w^{v-diss} +
G_w^{metric} + G_w^{nh-metric} \label{eq:gsplit_momw}
\end{eqnarray}
In the hydrostatic limit, $G_w=0$ and $\epsilon_{nh}=0$, reducing the
vertical momentum to hydrostatic balance.

These terms are calculated in routines called from subroutine {\em
CALC\_MOM\_RHS} a collected into the global arrays {\bf Gu}, {\bf Gv},
and {\bf Gw}.

\fbox{ \begin{minipage}{4.75in}
{\em S/R CALC\_MOM\_RHS} ({\em pkg/mom\_fluxform/calc\_mom\_rhs.F})

$G_u$: {\bf Gu} ({\em DYNVARS.h})

$G_v$: {\bf Gv} ({\em DYNVARS.h})

$G_w$: {\bf Gw} ({\em DYNVARS.h})
\end{minipage} }


\subsection{Advection of momentum}

The advective operator is second order accurate in space:
\begin{eqnarray}
{\cal A}_w \Delta r_f h_w G_u^{adv} & = &
  \delta_i \overline{ U }^i \overline{ u }^i
+ \delta_j \overline{ V }^i \overline{ u }^j
+ \delta_k \overline{ W }^i \overline{ u }^k \label{eq:discrete-momadvu} \\
{\cal A}_s \Delta r_f h_s G_v^{adv} & = &
  \delta_i \overline{ U }^j \overline{ v }^i
+ \delta_j \overline{ V }^j \overline{ v }^j
+ \delta_k \overline{ W }^j \overline{ v }^k \label{eq:discrete-momadvv} \\
{\cal A}_c \Delta r_c G_w^{adv} & = &
  \delta_i \overline{ U }^k \overline{ w }^i
+ \delta_j \overline{ V }^k \overline{ w }^j
+ \delta_k \overline{ W }^k \overline{ w }^k \label{eq:discrete-momadvw}
\end{eqnarray}
and because of the flux form does not contribute to the global budget
of linear momentum. The quantities $U$, $V$ and $W$ are volume fluxes
defined:
\marginpar{$U$: {\bf uTrans} }
\marginpar{$V$: {\bf vTrans} }
\marginpar{$W$: {\bf rTrans} }
\begin{eqnarray}
U & = & \Delta y_g \Delta r_f h_w u \label{eq:utrans} \\
V & = & \Delta x_g \Delta r_f h_s v \label{eq:vtrans} \\
W & = & {\cal A}_c w \label{eq:rtrans}
\end{eqnarray}
The advection of momentum takes the same form as the advection of
tracers but by a translated advective flow. Consequently, the
conservation of second moments, derived for tracers later, applies to
$u^2$ and $v^2$ and $w^2$ so that advection of momentum correctly
conserves kinetic energy.

\fbox{ \begin{minipage}{4.75in}
{\em S/R MOM\_U\_ADV\_UU} ({\em mom\_u\_adv\_uu.F})

{\em S/R MOM\_U\_ADV\_VU} ({\em mom\_u\_adv\_vu.F})

{\em S/R MOM\_U\_ADV\_WU} ({\em mom\_u\_adv\_wu.F})

{\em S/R MOM\_U\_ADV\_UV} ({\em mom\_u\_adv\_uv.F})

{\em S/R MOM\_U\_ADV\_VV} ({\em mom\_u\_adv\_vv.F})

{\em S/R MOM\_U\_ADV\_WV} ({\em mom\_u\_adv\_wv.F})

$uu$, $uv$, $vu$, $vv$: {\bf aF} (local to {\em calc\_mom\_rhs.F})
\end{minipage} }



\subsection{Coriolis terms}

The ``pure C grid'' Coriolis terms (i.e. in absence of C-D scheme) are
discretized:
\begin{eqnarray}
{\cal A}_w \Delta r_f h_w G_u^{Cor} & = &
  \overline{ f {\cal A}_c \Delta r_f h_c \overline{ v }^j }^i
- \epsilon_{nh} \overline{ f' {\cal A}_c \Delta r_f h_c \overline{ w }^k }^i \\
{\cal A}_s \Delta r_f h_s G_v^{Cor} & = &
- \overline{ f {\cal A}_c \Delta r_f h_c \overline{ u }^i }^j \\
{\cal A}_c \Delta r_c G_w^{Cor} & = &
 \epsilon_{nh} \overline{ f' {\cal A}_c \Delta r_f h_c \overline{ u }^i }^k
\end{eqnarray}
where the Coriolis parameters $f$ and $f'$ are defined:
\begin{eqnarray}
f & = & 2 \Omega \sin{\phi} \\
f' & = & 2 \Omega \cos{\phi}
\end{eqnarray}
where $\phi$ is geographic latitude when using spherical geometry,
otherwise the $\beta$-plane definition is used:
\begin{eqnarray}
f & = & f_o + \beta y \\
f' & = & 0
\end{eqnarray}

This discretization globally conserves kinetic energy. It should be
noted that despite the use of this discretization in former
publications, all calculations to date have used the following
different discretization:
\begin{eqnarray}
G_u^{Cor} & = &
  f_u \overline{ v }^{ji}
- \epsilon_{nh} f_u' \overline{ w }^{ik} \\
G_v^{Cor} & = &
- f_v \overline{ u }^{ij} \\
G_w^{Cor} & = &
 \epsilon_{nh} f_w' \overline{ u }^{ik}
\end{eqnarray}
\marginpar{Need to change the default in code to match this}
where the subscripts on $f$ and $f'$ indicate evaluation of the
Coriolis parameters at the appropriate points in space. The above
discretization does {\em not} conserve anything, especially energy and
for historical reasons is the default for the code. A
flag controls this discretization: set run-time logical {\bf
useEnergyConservingCoriolis} to {\em true} which otherwise defaults to
{\em false}.

\fbox{ \begin{minipage}{4.75in}
{\em S/R MOM\_CDSCHEME} ({\em mom\_cdscheme.F})

{\em S/R MOM\_U\_CORIOLIS} ({\em mom\_u\_coriolis.F})

{\em S/R MOM\_V\_CORIOLIS} ({\em mom\_v\_coriolis.F})

$G_u^{Cor}$, $G_v^{Cor}$: {\bf cF} (local to {\em calc\_mom\_rhs.F})
\end{minipage} }


\subsection{Curvature metric terms}

The most commonly used coordinate system on the sphere is the
geographic system $(\lambda,\phi)$. The curvilinear nature of these
coordinates on the sphere lead to some ``metric'' terms in the
component momentum equations. Under the thin-atmosphere and
hydrostatic approximations these terms are discretized:
\begin{eqnarray}
{\cal A}_w \Delta r_f h_w G_u^{metric} & = &
  \overline{ \frac{ \overline{u}^i }{a} \tan{\phi} {\cal A}_c \Delta r_f h_c \overline{ v }^j }^i \\
{\cal A}_s \Delta r_f h_s G_v^{metric} & = &
- \overline{ \frac{ \overline{u}^i }{a} \tan{\phi} {\cal A}_c \Delta r_f h_c \overline{ u }^i }^j \\
G_w^{metric} & = & 0
\end{eqnarray}
where $a$ is the radius of the planet (sphericity is assumed) or the
radial distance of the particle (i.e. a function of height).  It is
easy to see that this discretization satisfies all the properties of
the discrete Coriolis terms since the metric factor $\frac{u}{a}
\tan{\phi}$ can be viewed as a modification of the vertical Coriolis
parameter: $f \rightarrow f+\frac{u}{a} \tan{\phi}$.

However, as for the Coriolis terms, a non-energy conserving form has
exclusively been used to date:
\begin{eqnarray}
G_u^{metric} & = & \frac{u \overline{v}^{ij} }{a} \tan{\phi} \\
G_v^{metric} & = & \frac{ \overline{u}^{ij} \overline{u}^{ij}}{a} \tan{\phi}
\end{eqnarray}
where $\tan{\phi}$ is evaluated at the $u$ and $v$ points
respectively.

\fbox{ \begin{minipage}{4.75in}
{\em S/R MOM\_U\_METRIC\_SPHERE} ({\em mom\_u\_metric\_sphere.F})

{\em S/R MOM\_V\_METRIC\_SPHERE} ({\em mom\_v\_metric\_sphere.F})

$G_u^{metric}$, $G_v^{metric}$: {\bf mT} (local to {\em calc\_mom\_rhs.F})
\end{minipage} }



\subsection{Non-hydrostatic metric terms}

For the non-hydrostatic equations, dropping the thin-atmosphere
approximation re-introduces metric terms involving $w$ and are
required to conserve anglular momentum:
\begin{eqnarray}
{\cal A}_w \Delta r_f h_w G_u^{metric} & = &
- \overline{ \frac{ \overline{u}^i \overline{w}^k }{a} {\cal A}_c \Delta r_f h_c }^i \\
{\cal A}_s \Delta r_f h_s G_v^{metric} & = &
- \overline{ \frac{ \overline{v}^j \overline{w}^k }{a} {\cal A}_c \Delta r_f h_c}^j \\
{\cal A}_c \Delta r_c G_w^{metric} & = &
  \overline{ \frac{ {\overline{u}^i}^2 + {\overline{v}^j}^2}{a} {\cal A}_c \Delta r_f h_c }^k
\end{eqnarray}

Because we are always consistent, even if consistently wrong, we have,
in the past, used a different discretization in the model which is:
\begin{eqnarray}
G_u^{metric} & = &
- \frac{u}{a} \overline{w}^{ik} \\
G_v^{metric} & = &
- \frac{v}{a} \overline{w}^{jk} \\
G_w^{metric} & = &
  \frac{1}{a} ( {\overline{u}^{ik}}^2 + {\overline{v}^{jk}}^2 )
\end{eqnarray}

\fbox{ \begin{minipage}{4.75in}
{\em S/R MOM\_U\_METRIC\_NH} ({\em mom\_u\_metric\_nh.F})

{\em S/R MOM\_V\_METRIC\_NH} ({\em mom\_v\_metric\_nh.F})

$G_u^{metric}$, $G_v^{metric}$: {\bf mT} (local to {\em calc\_mom\_rhs.F})
\end{minipage} }


\subsection{Lateral dissipation}

Historically, we have represented the SGS Reynolds stresses as simply
down gradient momentum fluxes, ignoring constraints on the stress
tensor such as symmetry.
\begin{eqnarray}
{\cal A}_w \Delta r_f h_w G_u^{h-diss} & = &
  \delta_i  \Delta y_f \Delta r_f h_c \tau_{11}
+ \delta_j  \Delta x_v \Delta r_f h_\zeta \tau_{12} \\
{\cal A}_s \Delta r_f h_s G_v^{h-diss} & = &
  \delta_i  \Delta y_u \Delta r_f h_\zeta \tau_{21}
+ \delta_j  \Delta x_f \Delta r_f h_c \tau_{22}
\end{eqnarray}
\marginpar{Check signs of stress definitions}

The lateral viscous stresses are discretized:
\begin{eqnarray}
\tau_{11} & = & A_h c_{11\Delta}(\phi) \frac{1}{\Delta x_f} \delta_i u
               -A_4 c_{11\Delta^2}(\phi) \frac{1}{\Delta x_f} \delta_i \nabla^2 u \\
\tau_{12} & = & A_h c_{12\Delta}(\phi) \frac{1}{\Delta y_u} \delta_j u
               -A_4 c_{12\Delta^2}(\phi)\frac{1}{\Delta y_u} \delta_j \nabla^2 u \\
\tau_{21} & = & A_h c_{21\Delta}(\phi) \frac{1}{\Delta x_v} \delta_i v
               -A_4 c_{21\Delta^2}(\phi) \frac{1}{\Delta x_v} \delta_i \nabla^2 v \\
\tau_{22} & = & A_h c_{22\Delta}(\phi) \frac{1}{\Delta y_f} \delta_j v
               -A_4 c_{22\Delta^2}(\phi) \frac{1}{\Delta y_f} \delta_j \nabla^2 v
\end{eqnarray}
where the non-dimensional factors $c_{lm\Delta^n}(\phi), \{l,m,n\} \in
\{1,2\}$ define the ``cosine'' scaling with latitude which can be
applied in various ad-hoc ways. For instance, $c_{11\Delta} =
c_{21\Delta} = (\cos{\phi})^{3/2}$, $c_{12\Delta}=c_{22\Delta}=0$ would
represent the an-isotropic cosine scaling typically used on the
``lat-lon'' grid for Laplacian viscosity.
\marginpar{Need to tidy up method for controlling this in code}

It should be noted that dispite the ad-hoc nature of the scaling, some
scaling must be done since on a lat-lon grid the converging meridians
make it very unlikely that a stable viscosity parameter exists across
the entire model domain.

The Laplacian viscosity coefficient, $A_h$ ({\bf viscAh}), has units
of $m^2 s^{-1}$. The bi-harmonic viscosity coefficient, $A_4$ ({\bf
viscA4}), has units of $m^4 s^{-1}$.

\fbox{ \begin{minipage}{4.75in}
{\em S/R MOM\_U\_XVISCFLUX} ({\em mom\_u\_xviscflux.F})

{\em S/R MOM\_U\_YVISCFLUX} ({\em mom\_u\_yviscflux.F})

{\em S/R MOM\_V\_XVISCFLUX} ({\em mom\_v\_xviscflux.F})

{\em S/R MOM\_V\_YVISCFLUX} ({\em mom\_v\_yviscflux.F})

$\tau_{11}$, $\tau_{12}$, $\tau_{22}$, $\tau_{22}$: {\bf vF}, {\bf
v4F} (local to {\em calc\_mom\_rhs.F})
\end{minipage} }

Two types of lateral boundary condition exist for the lateral viscous
terms, no-slip and free-slip.

The free-slip condition is most convenient to code since it is
equivalent to zero-stress on boundaries. Simple masking of the stress
components sets them to zero. The fractional open stress is properly
handled using the lopped cells.

The no-slip condition defines the normal gradient of a tangential flow
such that the flow is zero on the boundary. Rather than modify the
stresses by using complicated functions of the masks and ``ghost''
points (see \cite{Adcroft+Marshall98}) we add the boundary stresses as
an additional source term in cells next to solid boundaries. This has
the advantage of being able to cope with ``thin walls'' and also makes
the interior stress calculation (code) independent of the boundary
conditions. The ``body'' force takes the form:
\begin{eqnarray}
G_u^{side-drag} & = &
\frac{4}{\Delta z_f} \overline{ (1-h_\zeta) \frac{\Delta x_v}{\Delta y_u} }^j
\left( A_h c_{12\Delta}(\phi) u - A_4 c_{12\Delta^2}(\phi) \nabla^2 u \right)
\\
G_v^{side-drag} & = &
\frac{4}{\Delta z_f} \overline{ (1-h_\zeta) \frac{\Delta y_u}{\Delta x_v} }^i
\left( A_h c_{21\Delta}(\phi) v - A_4 c_{21\Delta^2}(\phi) \nabla^2 v \right)
\end{eqnarray}

In fact, the above discretization is not quite complete because it
assumes that the bathymetry at velocity points is deeper than at
neighbouring vorticity points, e.g. $1-h_w < 1-h_\zeta$

\fbox{ \begin{minipage}{4.75in}
{\em S/R MOM\_U\_SIDEDRAG} ({\em mom\_u\_sidedrag.F})

{\em S/R MOM\_V\_SIDEDRAG} ({\em mom\_v\_sidedrag.F})

$G_u^{side-drag}$, $G_v^{side-drag}$: {\bf vF} (local to {\em calc\_mom\_rhs.F})
\end{minipage} }


\subsection{Vertical dissipation}

Vertical viscosity terms are discretized with only partial adherence
to the variable grid lengths introduced by the finite volume
formulation. This reduces the formal accuracy of these terms to just
first order but only next to boundaries; exactly where other terms
appear such as linar and quadratic bottom drag.
\begin{eqnarray}
G_u^{v-diss} & = &
\frac{1}{\Delta r_f h_w} \delta_k \tau_{13} \\
G_v^{v-diss} & = &
\frac{1}{\Delta r_f h_s} \delta_k \tau_{23} \\
G_w^{v-diss} & = & \epsilon_{nh}
\frac{1}{\Delta r_f h_d} \delta_k \tau_{33}
\end{eqnarray}
represents the general discrete form of the vertical dissipation terms.

In the interior the vertical stresses are discretized:
\begin{eqnarray}
\tau_{13} & = & A_v \frac{1}{\Delta r_c} \delta_k u \\
\tau_{23} & = & A_v \frac{1}{\Delta r_c} \delta_k v \\
\tau_{33} & = & A_v \frac{1}{\Delta r_f} \delta_k w
\end{eqnarray}
It should be noted that in the non-hydrostatic form, the stress tensor
is even less consistent than for the hydrostatic (see Wazjowicz
\cite{Waojz}). It is well known how to do this properly (see Griffies
\cite{Griffies}) and is on the list of to-do's.

\fbox{ \begin{minipage}{4.75in}
{\em S/R MOM\_U\_RVISCLFUX} ({\em mom\_u\_rviscflux.F})

{\em S/R MOM\_V\_RVISCLFUX} ({\em mom\_v\_rviscflux.F})

$\tau_{13}$: {\bf urf} (local to {\em calc\_mom\_rhs.F})

$\tau_{23}$: {\bf vrf} (local to {\em calc\_mom\_rhs.F})
\end{minipage} }


As for the lateral viscous terms, the free-slip condition is
equivalent to simply setting the stress to zero on boundaries.  The
no-slip condition is implemented as an additional term acting on top
of the interior and free-slip stresses. Bottom drag represents
additional friction, in addition to that imposed by the no-slip
condition at the bottom. The drag is cast as a stress expressed as a
linear or quadratic function of the mean flow in the layer above the
topography:
\begin{eqnarray}
\tau_{13}^{bottom-drag} & = &
\left(
2 A_v \frac{1}{\Delta r_c}
+ r_b
+ C_d \sqrt{ \overline{2 KE}^i }
\right) u \\
\tau_{23}^{bottom-drag} & = &
\left(
2 A_v \frac{1}{\Delta r_c}
+ r_b
+ C_d \sqrt{ \overline{2 KE}^j }
\right) v
\end{eqnarray}
where these terms are only evaluated immediately above topography.
$r_b$ ({\bf bottomDragLinear}) has units of $m s^{-1}$ and a typical value
of the order 0.0002 $m s^{-1}$. $C_d$ ({\bf bottomDragQuadratic}) is
dimensionless with typical values in the range 0.001--0.003.

\fbox{ \begin{minipage}{4.75in}
{\em S/R MOM\_U\_BOTTOMDRAG} ({\em mom\_u\_bottomdrag.F})

{\em S/R MOM\_V\_BOTTOMDRAG} ({\em mom\_v\_bottomdrag.F})

$\tau_{13}^{bottom-drag}$, $\tau_{23}^{bottom-drag}$: {\bf vf} (local to {\em calc\_mom\_rhs.F})
\end{minipage} }

\subsection{Derivation of discrete energy conservation}

These discrete equations conserve kinetic plus potential energy using the
following definitions:
\begin{equation}
KE = \frac{1}{2} \left( \overline{ u^2 }^i + \overline{ v^2 }^j +
\epsilon_{nh} \overline{ w^2 }^k \right)
\end{equation}

