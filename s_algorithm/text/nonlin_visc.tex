%  $Header: /u/gcmpack/manual/s_algorithm/text/nonlin_visc.tex,v 1.1 2005/10/05 19:52:47 edhill Exp $
%  $Name:  $

\def\del{{\mathbf \nabla}}
\def\av#1{\overline{#1}}
\def\pd#1#2{{\frac{\partial{#2}}{\partial#1}}}
\def\pds#1#2{{\frac{\partial^2{#2}}{{\partial#1}^2}}}
\def\Dt#1{\frac{D{#1}}{Dt}}
\def\aDt#1{\frac{\av D{#1}}{\av{Dt}}}
\def\d#1{{\,\rm d#1}}
\def\Ro{{\rm Ro}}
\def\Re{{\rm Re}}
\def\Fr{{\rm Fr}}
\def\mr{{m_{Ro}}}
\def\Mr{{M_{Ro}}}
\def\eg{{\emph{e.g.,}\ }}
\def\ie{{\emph{i.e.,}\ }}
\def\tu{{\tilde u}}
\def\tv{{\tilde v}}
\def\atu{{\tilde {\av u}}}
\def\atv{{\tilde {\av v}}}
\def\lesssim{{<\atop\sim}}


\section{Nonlinear Viscosities for Large Eddy Simulation}
\label{sect:nonlin-visc}

In Large Eddy Simulations (LES), a turbulent closure needs to be
provided that accounts for the effects of subgridscale motions on the
large scale.  With sufficiently powerful computers, we could resolve
the entire flow down to the molecular viscosity scales
{($L_{\nu}\approx 1 \rm cm$)}.  Current computation allows perhaps
four decades to be resolved, so the largest problem computationally
feasible would be about 10m. Most oceanographic problems are much
larger in scale, so some form of LES is required, where only the
largest scales of motion are resolved, and the subgridscale's effects
on the large-scale are parameterized.

To formalize this process, we can introduce a filter over the
subgridscale L: $u_\alpha\rightarrow \av u_\alpha$ and $L:
b\rightarrow \av b$.  This filter has some intrinsic length and time
scales, and we assume that the flow at that scale can be characterized
with a single velocity scale ($V$) and vertical buoyancy gradient
($N^2$). The filtered equations of motion in a local Mercator
projection about the gridpoint in question (see Appendix for notation
and details of approximation) are: \newpage
\begin{eqnarray}
\aDt \atu- \frac{\atv \sin\theta}{\Ro\sin\theta_0}+\frac{\Mr}{\Ro}\pd{x}{\av\pi}=-\left({\av{\Dt \tu}}-{\aDt \atu}\right)+\frac{\nabla^2{\atu}}{\Re}\label{eq:mercat}\\
\aDt\atv+ \frac{\atu\sin\theta}{\Ro\sin\theta_0}+\frac{\Mr}{\Ro}\pd{y}{\av\pi}=-\left({\av{\Dt \tv}}-{\aDt \atv}\right)+\frac{\nabla^2{\atv}}{\Re}\nonumber\\
\aDt {\av w} +\frac{\pd{z}{\av\pi}-\av b}{\Fr^2\lambda^2}=-\left(\av{\Dt w}-\aDt {\av{w}}\right)+\frac{\nabla^2\av w}{\Re}\nonumber\\
\aDt{\ \av b}+\av w=-\left(\av{\Dt{b}}-\aDt{\ \av b} \right)+\frac{\nabla^2 \av b}{\Pr\Re}\nonumber \\
\mu^2\left(\pd x\atu  + \pd y\atv \right)+\pd z {\av w} =0\label{eq:cont}
\end{eqnarray}
Tildes denote multiplication by $\cos\theta/\cos\theta_0$ to account
for converging meridians.

The ocean is usually turbulent, and an operational definition of
turbulence is that the terms in parentheses (the 'eddy' terms) on the
right of (\ref{eq:mercat}) are of comparable magnitude to the terms on
the left-hand side.  The terms proportional to the inverse of \Re,
instead, are many orders of magnitude smaller than all of the other
terms in virtually every oceanic application.

\subsection{Eddy Viscosity} 
A turbulent closure provides an approximation to the 'eddy' terms on
the right of the preceding equations.  The simplest form of LES is
just to increase the viscosity and diffusivity until the viscous and
diffusive scales are resolved.  That is, we approximate:
\begin{eqnarray}
\left({\av{\Dt \tu}}-{\aDt \atu}\right)\approx\frac{\nabla^2_h{\atu}}{\Re_h}+\frac{\pds{z}{\atu}}{\Re_v}\label{eq:eddyvisc},\qquad
\left({\av{\Dt \tv}}-{\aDt \atv}\right)\approx\frac{\nabla^2_h{\atv}}{\Re_h}+\frac{\pds{z}{\atv}}{\Re_v}\nonumber\\
\left(\av{\Dt w}-\aDt {\av{w}}\right)\approx\frac{\nabla^2_h\av w}{\Re_h}+\frac{\pds{z}{\av w}}{\Re_v}\nonumber,\qquad
\left(\av{\Dt{b}}-\aDt{\ \av b} \right)\approx\frac{\nabla^2_h \av b}{\Pr\Re_h}+\frac{\pds{z} {\av b}}{\Pr\Re_v}\nonumber
\end{eqnarray}
   
\subsubsection{Reynolds-Number Limited Eddy Viscosity}   
One way of ensuring that the gridscale is sufficiently viscous (\ie
resolved) is to choose the eddy viscosity $A_h$ so that the gridscale
horizontal Reynolds number based on this eddy viscosity, $\Re_h$, to
is O(1).  That is, if the gridscale is to be viscous, then the
viscosity should be chosen to make the viscous terms as large as the
advective ones.  \citet{Bryanetal75} note that a computational mode is
squelched by using $\Re_h<$2.

The MITgcm user can select an horizontal eddy viscosity based on
$\Re_h$ by two methods.  1) The user may estimate the velocity scale
expected from the calculation and grid spacing and set the {\sf
  viscAh} to satisfy $\Re_h<2$.  2) The user may use {\sf
  viscAhReMax}, which ensures that the viscosity is always chosen so
that $\Re_h<{\sf viscAhReMax}$.  This last option should be used with
caution, however, since it effectively implies that viscous terms are
fixed in magnitude relative to advective terms.  While it may be a
useful method for specifying a minimum viscosity with little effort,
tests have shown that setting {\sf viscAhReMax}=2
\citep[per][]{Bryanetal75} often tends to increase the viscosity
substantially over other more 'physical' parameterizations below,
especially in regions where gradients of velocity are small (and thus
turbulence may be weak), so perhaps a more liberal value should be
used, \eg {\sf viscAhReMax}=10.
  
While it is certainly necessary that viscosity be active at the
gridscale, the wavelength where dissipation of energy or enstrophy
occurs is not necessarily $L=A_h/U$.  In fact, it is by ensuring that
the either the dissipation of energy in a 3-d turbulent cascade
(Smagorinsky) or dissipation of enstrophy in a 2-d turbulent cascade
(Leith) is resolved that these parameterizations derive their physical
meaning.
  
\subsubsection{Vertical Eddy Viscosities}
Vertical eddy viscosities are often chosen in a more subjective way,
as model stability is not usually as sensitive to vertical viscosity.
Usually the 'observed' value from finescale measurements, etc., is
used (\eg {\sf viscAr}$\approx1\times10^{-4} m^2/s$).  However,
\citet{Smagorinsky93} notes that the Smagorinsky parameterization of
isotropic turbulence implies a value of the vertical viscosity as well
as the horizontal viscosity (see below).
 
\subsubsection{Smagorinsky Viscosity}
\citet{sm63} and \citet{Smagorinsky93} suggest choosing a viscosity
that \emph{depends on the resolved motions}.  Thus, the overall
viscous operator has a nonlinear dependence on velocity.  Smagorinsky
chose his form of viscosity by considering Kolmogorov's ideas about
the energy spectrum of 3-d isotropic turbulence.

Kolmogorov suppposed that is that energy is injected into the flow at
large scales (small $k$) and is 'cascaded' or transferred
conservatively by nonlinear processes to smaller and smaller scales
until it is dissipated near the viscous scale.  By setting the energy
flux through a particular wavenumber $k$, $\epsilon$, to be a constant
in $k$, there is only one combination of viscosity and energy flux
that has the units of length, the Kolmogorov wavelength.  It is
$L_\epsilon(\nu)\propto\pi\epsilon^{-1/4}\nu^{3/4}$ (the $\pi$ stems
from conversion from wavenumber to wavelength).  To ensure that this
viscous scale is resolved in a numerical model, the gridscale should
be decreased until $L_\epsilon(\nu)>L$ (so-called Direct Numerical
Simulation, or DNS).  Alternatively, an eddy viscosity can be used and
the corresponding Kolmogorov length can be made larger than the
gridscale, $L_\epsilon(A_h)\propto\pi\epsilon^{-1/4}A_h^{3/4}$ (for
Large Eddy Simulation or LES).

There are two methods of ensuring that the Kolmogorov length is
resolved in the MITgcm.  1) The user can estimate the flux of energy
through spectral space for a given simulation and adjust grid spacing
or {\sf viscAh} to ensure that $L_\epsilon(A_h)>L$.  2) The user may
use the approach of Smagorinsky with {\sf viscC2Smag}, which estimates
the energy flux at every grid point, and adjusts the viscosity
accordingly.

Smagorinsky formed the energy equation from the momentum equations by
dotting them with velocity.  \citep[There are some complications when
using the hydrostatic approximation, see][]{Smagorinsky93}.  The
positive definite energy dissipation by horizontal viscosity in a
hydrostatic flow is $\nu D^2$, where D is the deformation rate at the
viscous scale.  According to Kolmogorov's theory, this should be a
good approximation to the energy flux at any wavenumber
$\epsilon\approx\nu D^2$.  Kolmogorov and Smagorinsky noted that using
an eddy viscosity that exceeds the molecular value $\nu$ should ensure
that the energy flux through viscous scale set by the eddy viscosity
is the same as it would have been had we resolved all the way to the
true viscous scale.  That is, $\epsilon\approx A_{hSmag} \av D^2$.  If
we use this approximation to estimate the Kolmogorov viscous length,
then
\begin{eqnarray}
L_\epsilon(A_{hSmag})\propto\pi\epsilon^{-1/4}A_{hSmag}^{3/4}\approx\pi(A_{hSmag} \av D^2)^{-1/4}A_{hSmag}^{3/4}=\pi A_{hSmag}^{1/2}\av D^{-1/2}
\end{eqnarray}
To make $L_\epsilon(A_{hSmag})$ scale with the gridscale, then
\begin{eqnarray}
A_{hSmag}=\left(\frac{{\sf viscC2Smag}}{\pi}\right)^2L^2|\av D|
\end{eqnarray}
Where the deformation rate appropriate for hydrostatic flows with
shallow-water scaling is
\begin{eqnarray}
|\av D|=\sqrt{\left(\pd{x}{\av \tu}-\pd{y}{\av \tv}\right)^2+\left(\pd{y}{\av \tu}+\pd{x}{\av \tv}\right)^2}
\end{eqnarray}
The coefficient {\sf viscC2Smag} is what the MITgcm user sets, and it
replaces the proportionality in the Kolmogorov length with an
equality.  \citet{grha00} suggest values of {\sf viscC2Smag} from 2.2
to 4 for oceanic problems.  \citet{Smagorinsky93} shows that values
from 0.2 to 0.9 have been used in atmospheric modeling.

\citet{Smagorinsky93} shows that a corresponding vertical viscosity
should be used:
\begin{eqnarray}
A_{vSmag}=\left(\frac{{\sf viscC2Smag}}{\pi}\right)^2H^2\sqrt{\left(\pd{z}{\av \tu}\right)^2+\left(\pd{z}{\av \tv}\right)^2}\nonumber
\end{eqnarray}
This vertical viscosity is currently not implemented in the MITgcm
(although it may be soon).

\subsubsection{Leith Viscosity}
\citet{Leith68,Leith96} notes that 2-d turbulence is quite different
from 3-d.  In two-dimensional turbulence, energy cascades to larger
scales, so there is no concern about resolving the scales of energy
dissipation.  Instead, another quantity, enstrophy, (which is the
vertical component of vorticity squared) is conserved in 2-d
turbulence, and it cascades to smaller scales where it is dissipated.

Following a similar argument to that above about energy flux, the
enstrophy flux is estimated to be equal to the positive-definite
gridscale dissipation rate of enstrophy $\eta\approx A_{hLeith}
|\nabla\av \omega_3|^2$.  By dimensional analysis, the
enstrophy-dissipation scale is $L_\eta(A_{hLeith})\propto\pi
A_{hLeith}^{1/2}\eta^{-1/6}$.  Thus, the Leith-estimated length scale
of enstrophy-dissipation and the resulting eddy viscosity are
\begin{eqnarray}
L_\eta(A_{hLeith})\propto\pi A_{hLeith}^{1/2}\eta^{-1/6}=\pi A_{hLeith}^{1/3}|\nabla \av \omega_3|^{-1/3}\\
A_{hLeith}={\sf viscC2Leith}|\nabla \av \omega_3|L^3\\
|\nabla\omega_3|\equiv\sqrt{\left[\pd{x}{\ }\left(\pd{x}{\av \tv}-\pd{y}{\av \tu}\right)\right]^2+\left[\pd{y}{\ }\left(\pd{x}{\av \tv}-\pd{y}{\av \tu}\right)\right]^2}
\end{eqnarray}
NOTE:: may be useful to redefine viscC2Leith for consistency with Smag...
\begin{eqnarray}
A_{hLeith}=\left(\frac{{\sf viscC2Leith}}{\pi}\right)^3L^3|\nabla \av \omega_3|
\end{eqnarray}

\subsubsection{Modified Leith Viscosity}
The argument above for the Leith viscosity parameterization uses
concepts from purely 2-dimensional turbulence, where the horizontal
flow field is assumed to be divergenceless.  However, oceanic flows
are only quasi-two dimensional.  While the barotropic flow, or the
flow within isopycnal layers may behave nearly as two-dimensional
turbulence, there is a possibility that these flows will be divergent.
In a high-resolution numerical model, these flows may be substantially
divergent near the grid scale, and in fact, numerical instabilities
exist which are only horizontally divergent and have little vertical
vorticity.  This causes a difficulty with the Leith viscosity, which
can only responds to buildup of vorticity at the grid scale.

The MITgcm offers two options for dealing with this problem.  1) The
Smagorinsky viscosity can be used instead of Leith, or in conjunction
with Leith--a purely divergent flow does cause an increase in
Smagorinsky viscosity.  2) The {\sf viscC2LeithD} parameter can be
set.  This is a damping specifically targeting purely divergent
instabilities near the gridscale.  The combined viscosity has the
form:
\begin{eqnarray}
A_{hLeith}=L^3\sqrt{\left(\frac{{\sf viscC2Leith}}{\pi}\right)^6|\nabla \av \omega_3|^2+\left(\frac{{\sf viscC2LeithD}}{\pi}\right)^6|\nabla \nabla\cdot \av {\tilde u}_h|^2}\nonumber\\
|\nabla \nabla\cdot \av {\tilde u}_h|\equiv\sqrt{\left[\pd{x}{\ }\left(\pd{x}{\av \tu}+\pd{y}{\av \tv}\right)\right]^2+\left[\pd{y}{\ }\left(\pd{x}{\av \tu}+\pd{y}{\av \tv}\right)\right]^2}
\end{eqnarray}
Whether there is any physical rationale for this correction is unclear
at the moment, but the numerical consequences are good.  The
divergence in flows with the grid scale larger or comparable to the
Rossby radius is typically much smaller than the vorticity, so this
adjustment only rarely adjusts the viscosity if ${\sf
  viscC2LeithD}={\sf viscC2Leith}$.  However, the rare regions where
this viscosity acts are often the locations for the largest vales of
vertical velocity in the domain.  Since the CFL condition on vertical
velocity is often what sets the maximum timestep, this viscosity may
substantially increase the allowable timestep without severely
compromising the verity of the simulation.  Tests have shown that in
some calculations, a timestep three times larger was allowed when
${\sf viscC2LeithD}={\sf viscC2Leith}$.

\subsubsection{Courant--Freidrichs--Lewy Constraint on Viscosity}
Whatever viscosities are used in the model, the choice is constrained
by gridscale and timestep by the Courant--Freidrichs--Lewy (CFL)
constraint on stability:
\begin{eqnarray}
A_h<\frac{L^2}{4\Delta t}\nonumber\\
A_4 \le \frac{L^4}{32\Delta t}\nonumber
%%  A_4\lesssim\frac{L^4}{32\Delta t}\nonumber
\end{eqnarray}
The viscosities may be automatically limited to be no greater than
these values in the MITgcm by specifying {\sf viscAhGridMax}$<1$ and
{\sf viscA4GridMax}$<1$.  Similarly-scaled minimum values of
viscosities are provided by {\sf viscAhGridMin} and {\sf
  viscA4GridMin}, which if used, should be set to values $\ll 1$. $L$
is roughly the gridscale (see below).

Following \citet{grha00}, we note that there is a factor of $\Delta
x^2/8$ difference between the harmonic and biharmonic viscosities.
Thus, whenever a non-dimensional harmonic coefficient is used in the
MITgcm (\eg {\sf viscAhGridMax}$<1$), the biharmonic equivalent is
scaled so that the same non-dimensional value can be used (\eg {\sf
  viscA4GridMax}$<1$).

\subsubsection{Biharmonic Viscosity}
\citet{ho78} suggested that eddy viscosities ought to be focuses on
the dynamics at the grid scale, as larger motions would be 'resolved'.
To enhance the scale selectivity of the viscous operator, he suggested
a biharmonic eddy viscosity instead of a harmonic (or Laplacian)
viscosity:
\begin{eqnarray}
\left({\av{\Dt \tu}}-{\aDt \atu}\right)\approx\frac{-\nabla^4_h{\atu}}{\Re_4}+\frac{\pds{z}{\atu}}{\Re_v}\label{eq:bieddyvisc},\qquad
\left({\av{\Dt \tv}}-{\aDt \atv}\right)\approx\frac{-\nabla^4_h{\atv}}{\Re_4}+\frac{\pds{z}{\atv}}{\Re_v}\nonumber\\
\left(\av{\Dt w}-\aDt {\av{w}}\right)\approx\frac{-\nabla^4_h\av w}{\Re_4}+\frac{\pds{z}{\av w}}{\Re_v}\nonumber,\qquad
\left(\av{\Dt{b}}-\aDt{\ \av b} \right)\approx\frac{-\nabla^4_h \av b}{\Pr\Re_4}+\frac{\pds{z} {\av b}}{\Pr\Re_v}\nonumber
\end{eqnarray}
\citet{grha00} propose that if one scales the biharmonic viscosity by
stability considerations, then the biharmonic viscous terms will be
similarly active to harmonic viscous terms at the gridscale of the
model, but much less active on larger scale motions.  Similarly, a
biharmonic diffusivity can be used for less diffusive flows.

In practice, biharmonic viscosity and diffusivity allow a less
viscous, yet numerically stable, simulation than harmonic viscosity
and diffusivity.  However, there is no physical rationale for such
operators being of leading order, and more boundary conditions must be
specified than for the harmonic operators. If one considers the
approximations of \ref{eq:eddyvisc} and \ref{eq:bieddyvisc} to be
terms in the Taylor series expansions of the eddy terms as functions
of the large-scale gradient, then one can argue that both harmonic and
biharmonic terms would occur in the series, and the only question is
the choice of coefficients.  Using biharmonic viscosity alone implies
that one zeros the first non-vanishing term in the Taylor series,
which is unsupported by any fluid theory or observation.

Nonetheless, the MITgcm supports a plethora of biharmonic viscosities
and diffusivities, which are controlled with parameters named
similarly to the harmonic viscosities and diffusivities with the
substitution $h\rightarrow 4$.  The MITgcm also supports a biharmonic
Leith and Smagorinsky viscosities:
\begin{eqnarray}
A_{4Smag}=\left(\frac{{\sf viscC4Smag}}{\pi}\right)^2\frac{L^4}{8}|D|\nonumber\\
A_{4Leith}=\frac{L^5}{8}\sqrt{\left(\frac{{\sf viscC4Leith}}{\pi}\right)^6|\nabla \av \omega_3|^2+\left(\frac{{\sf viscC4LeithD}}{\pi}\right)^6|\nabla \nabla\cdot \av {\bf \tu}_h|^2}\nonumber
\end{eqnarray}
However, it should be noted that unlike the harmonic forms, the
biharmonic scaling does not easily relate to whether
energy-dissipation or enstrophy-dissipation scales are resolved.  If
similar arguments are used to estimate these scales and scale them to
the gridscale, the resulting biharmonic viscosities should be:
\begin{eqnarray}
A_{4Smag}=\left(\frac{{\sf viscC4Smag}}{\pi}\right)^5L^5|\nabla^2\av {\bf \tu}_h|\nonumber\\
A_{4Leith}=L^6\sqrt{\left(\frac{{\sf viscC4Leith}}{\pi}\right)^{12}|\nabla^2 \av \omega_3|^2+\left(\frac{{\sf viscC4LeithD}}{\pi}\right)^{12}|\nabla^2 \nabla\cdot \av {\bf \tu}_h|^2}\nonumber
\end{eqnarray}
Thus, the biharmonic scaling suggested by \citet{grha00} implies:
\begin{eqnarray}
|D|\propto L|\nabla^2\av {\bf \tu}_h|\\
|\nabla \av \omega_3|\propto L|\nabla^2 \av \omega_3|
\end{eqnarray}
It is not at all clear that these assumptions ought to hold.  Only the \citet{grha00} forms are currently implemented in the MITgcm.

\subsubsection{Selection of Length Scale}
Above, the length scale of the grid has been denoted $L$.  However, in
strongly anisotropic grids, $L_x$ and $L_y$ will be quite different in
some locations.  In that case, the CFL condition suggests that the
minimum of $L_x$ and $L_y$ be used.  On the other hand, other
viscosities which involve whether a particular wavelength is
'resolved' might be better suited to use the maximum of $L_x$ and
$L_y$.  Currently the MITgcm uses {\sf useAreaViscLength} to select
between two options.  If false, the geometric mean of $L^2_x$ and
$L^2_y$ is used for all viscosities, which is closer to the minimum
and occurs naturally in the CFL constraint.  If {\sf
  useAreaViscLength} is true, then the square root of the area of the
grid cell is used.

% The Appendices part is started with the command \appendix;
% appendix sections are then done as normal sections
% \appendix

\subsection{Mercator, Nondimensional Equations}
The rotating, incompressible, Boussinesq equations of motion
\citep{Gill1982} on a sphere can be written in Mercator projection
about a latitude $\theta_0$ and geopotential height $z=r-r_0$.  The
nondimensional form of these equations is:
\begin{eqnarray}
\Ro\Dt\tu- \frac{\tv \sin\theta}{\sin\theta_0}+\Mr\pd{x}{\pi}+\frac{\lambda\Fr^2\Mr\cos \theta}{\mu\sin\theta_0} w=-\frac{\Fr^2\Mr \tu w}{r/H}+\frac{\Ro{\bf \hat x}\cdot\nabla^2{\bf u}}{\Re}\nonumber\\
\Ro\Dt\tv+ \frac{\tu\sin\theta}{\sin\theta_0}+\Mr\pd{y}{\pi}=-\frac{\mu\Ro\tan\theta(\tu^2+\tv^2)}{r/L} -\frac{\Fr^2\Mr \tv w}{r/H}+\frac{\Ro{\bf \hat y}\cdot\nabla^2{\bf u}}{\Re}\nonumber\\
\Fr^2\lambda^2\Dt w -b+\pd{z}{\pi}-\frac{\lambda\cot \theta_0 \tu}{\Mr}=\frac{\lambda\mu^2(\tu^2+\tv^2)}{\Mr(r/L)}+\frac{\Fr^2\lambda^2{\bf \hat z}\cdot\nabla^2{\bf u}}{\Re}\nonumber\\
\Dt b+w=\frac{\nabla^2 b}{\Pr\Re}\nonumber, \qquad
\mu^2\left(\pd x\tu  + \pd y\tv \right)+\pd z w =0\nonumber
\end{eqnarray}
Where 
\begin{eqnarray}
\mu\equiv\frac{\cos\theta_0}{\cos\theta},\qquad\tu=\frac{u^*}{V\mu},\qquad\tv=\frac{v^*}{V\mu} , \qquad \Dt\ \equiv \mu^2\left(\tu\pd x\  +\tv \pd y\ \right)+\frac{\Fr^2\Mr}{\Ro} w\pd z \nonumber \\
f_0\equiv2\Omega\sin\theta_0,\qquad x\equiv \frac{r}{L} \phi \cos \theta_0, \qquad y\equiv \frac{r}{L} \int_{\theta_0}^\theta\frac{\cos \theta_0 \d \theta'}{\cos\theta'}, \qquad z\equiv \lambda\frac{r-r_0}{L}\nonumber\\
t^*=t \frac{L}{V},\qquad b^*= b\frac{V f_0\Mr}{\lambda},\qquad \pi^*=\pi V f_0 L\Mr,\qquad w^*=w V \frac{\Fr^2\lambda\Mr}{\Ro}\nonumber\\
\Ro\equiv\frac{V}{f_0 L},\qquad \Mr\equiv \max[1,\Ro], \qquad \Fr\equiv\frac{V}{N \lambda L}, \qquad \Re\equiv\frac{VL}{\nu}, \qquad \Pr\equiv\frac{\nu}{\kappa}\nonumber
\end{eqnarray}
Dimensional variables are denoted by an asterisk where necessary.  If
we filter over a grid scale typical for ocean models ($1m<L<100km$,
$0.0001<\lambda<1$, $0.001m/s <V<1 m/s$, $f_0<0.0001 s^{-1}$, $0.01
s^{-1}<N<0.0001 s^{-1}$), these equations are very well approximated
by
\begin{eqnarray}
\Ro{\Dt\tu}- \frac{\tv \sin\theta}{\sin\theta_0}+\Mr\pd{x}{\pi}=-\frac{\lambda\Fr^2\Mr\cos \theta}{\mu\sin\theta_0} w+\frac{\Ro\nabla^2{\tu}}{\Re}\nonumber\\
\Ro\Dt\tv+ \frac{\tu\sin\theta}{\sin\theta_0}+\Mr\pd{y}{\pi}=\frac{\Ro\nabla^2{\tv}}{\Re}\nonumber\\
\Fr^2\lambda^2\Dt w -b+\pd{z}{\pi}=\frac{\lambda\cot \theta_0 \tu}{\Mr}\nonumber+\frac{\Fr^2\lambda^2\nabla^2w}{\Re}\\
\Dt b+w=\frac{\nabla^2 b}{\Pr\Re}\nonumber, \qquad
\mu^2\left(\pd x\tu  + \pd y\tv \right)+\pd z w =0\nonumber\\
\nabla^2\approx\left(\frac{\partial^2}{\partial x^2}+\frac{\partial^2}{\partial y^2}+\frac{\partial^2}{\lambda^2\partial z^2}\right)\nonumber
\end{eqnarray}
Neglecting the non-frictional terms on the right-hand side is usually
called the 'traditional' approximation.  It is appropriate, with
either large aspect ratio or far from the tropics.  This approximation
is used here, as it does not affect the form of the eddy stresses
which is the main topic.  The frictional terms are preserved in this
approximate form for later comparison with eddy stresses.
% \label{}

% Bibliographic references with the natbib package:
% Parenthetical: \citep{Bai92} produces (Bailyn 1992).
% Textual: \citet{Bai95} produces Bailyn et al. (1995).
% An affix and part of a reference:
%   \citep[e.g.][Ch. 2]{Bar76}
%   produces (e.g. Barnes et al. 1976, Ch. 2).
%\bibliography{biblio}
%\begin{thebibliography}{}

% \bibitem[Names(Year)]{label} or \bibitem[Names(Year)Long names]{label}.
% (\harvarditem{Name}{Year}{label} is also supported.)
% Text of bibliographic item

%\bibitem[]{}

%\end{thebibliography}


