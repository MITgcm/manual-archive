% \documentstyle[12pt]{report}
% \renewcommand{\baselinestretch}{1}
% \textwidth 190mm
% \textheight 250mm
% \topmargin -20mm
% \oddsidemargin -15mm
% \begin{document}

\section[Offline Example]{Offline Experiments}
%\label{www:tutorials}
\label{sec:eg-offline}
\begin{center}
(in directory: {\it verification/tutorial\_offline/})\\
\end{center}

%\begin{center} 
%{\Large \bf Using MITgcm to Perform Offline Tracer Exeriments}
%
%\vspace*{4mm}
%
%\vspace*{3mm}
%{\large September 2005}
%\end{center}

This document describes a simple experiment using the offline form of
the MITgcm.

\subsection{Overview}
%\label{www:tutorials}

This experiment demonstrates use of the offline form of the MITgcm to
study advection of a passive tracer. Time-averaged flow-fields and
mixing coefficients, deriving from a prior online run, are re-used
leaving only the tracer equation to be integrated.

Figure {\it --- missing figure ---} %\ref{fig:eg-ofline_config} 
shows a movie of tracer being
advected using the offline package of the MITgcm. In the top panel the
frames of the movie show the monthly surface evolution of an initially
local source of passive tracer. In the lower panel, the frames of the
movie show the changing monthly surface evolution where the initial
tracer field had a global distribution.

\subsection{Time-stepping of tracers}
%\label{www:tutorials}

\noindent see section 2.15 through 2.18 for details of available
tracer time-stepping schemes and their characteristics.

\subsection{Code Configuration}
%\label{www:tutorials}
\label{sec:eg_offl_code_config}

\noindent The model configuration for this experiment resides under the 
directory {\it verification/tutorial\_offline}.  The experiment files 
\begin {itemize}
\item {\it input/data}
\item {\it input/data.off}
\item {\it input/data.pkg}
\item {\it input/data.ptracers}
\item {\it input/eedata}
\item {\it input/packages.conf}
\item {\it code/PTRACERS\_SIZE.h}
\item {\it code/SIZE.h}. 
\end {itemize}

\noindent contain the code customisations and parameter settings
required to run the example. In addition the following binary data
files are required:

\begin {itemize}
\item {\it input/depth\_g77.bin}
\item {\it input/tracer1\_.bin}
\item {\it input/tracer2\_.bin}
\item {\it input/input\_off/uVeltave.0000000001-12.data}
\item {\it input/input\_off/vVeltave.0000000001-12.data}
\item {\it input/input\_off/wVeltave.0000000001-12.data}
\item {\it input/input\_off/Convtave.0000000001-12.data}
\end {itemize}


\subsubsection{File {\it input/data}}
%\label{www:tutorials}

\noindent This file, reproduced completely below, specifies the main
parameters for the experiment.

\begin{itemize}
\item Line 18, 19
\begin {verbatim}
 nIter0  = 0,
 nTimeSteps = 720,
\end{verbatim}
\end{itemize}

\noindent nIter0 and nTimesteps control the start time and the length
of the run (in timesteps). If nIter0 is non-zero the model will
require appropriate pickup files to be present in the run directory.
Where nIter0 is zero, as here, the model makes a fresh start. In this
case the model has been prescribed to run for 720 timesteps or 1 year.

\begin{itemize}
\item Line 20
\begin {verbatim}
 deltaTtracer= 43200.0,
\end{verbatim}
\end{itemize}

\noindent deltaTtracer is the tracer timestep in seconds, in this
case, 12 hours (43200s = 12 hours). Note that deltatTracer must be
specified in {\it input/data} as well as specifying deltaToffline in
{\it input/data.off}.

\begin{itemize}
\item Line 21
\begin {verbatim}
 deltaTClock= 43200.0,
\end{verbatim}
\end{itemize}

\noindent When using the MITgcm in offline mode deltaTClock (an
internal model counter) should be made equal to the value assigned to
deltatTtracer.

\begin{itemize}
\item Line 27
\begin {verbatim}
 periodicExternalForcing=.TRUE.,
\end{verbatim}
\end{itemize}

\noindent periodicExternalForcing is a flag telling the model whether
to cyclically re-use forcing data where there is external forcing (see
`A More Complcated Example', below). Where there is no external
forcing, as here, but where there is to be cyclic re-use of the
offline flow and mixing fields, periodicExternalForcing must be
assigned the value .TRUE.

\begin{itemize}
\item Line 28
\begin {verbatim}
 externForcingPeriod=2592000.,
\end{verbatim}
\end{itemize}

\noindent externForcingPeriod specifies the period of the external
forcing data in seconds. In the absence of external forcing, as in
this example, it must be made equal to the value of
externForcingPeriod in {\it input/data.off}, in this case, monthly
(2592000s = 1 month).

\begin{itemize}
\item Line 29
\begin {verbatim}
 externForcingCycle=31104000.,
\end{verbatim}
\end{itemize}

\noindent externForcingCycle specifies the duration of the external
forcing data cycle in seconds. In the absence of external forcing, as
in this example, it must be made equal to the value of
externForcingCycle in {\it input/data.off}, in this case, the cycle is
one year(31104000s = 1 year).

\begin{itemize}
\item Line 35
\begin {verbatim}
 usingSphericalPolarGrid=.TRUE.,
\end{verbatim}
\end{itemize}

\noindent This line requests that the simulation be performed in a
spherical polar coordinate system. It affects the interpretation of
grid input parameters and causes the grid generation routines to
initialize an internal grid based on spherical polar geometry.

\begin{itemize}
\item Line 36
\begin {verbatim}
 delR=  50.,  70., 100., 140., 190., 
       240., 290., 340., 390., 440., 
       490., 540., 590., 640., 690.,
\end{verbatim}
\end{itemize}

\noindent This line sets the vertical grid spacing between each
z-coordinate line in the discrete grid. Here the total model depth is
$5200\,{\rm m}$.

\begin{itemize}
\item Line 39
\begin {verbatim}
 ygOrigin=-90.,
\end{verbatim}
\end{itemize}

\noindent This line sets the southern boundary of the modeled domain
to $-90^{\circ}$ latitude N ($90^{\circ}$ S). This value affects both 
the generation of the
locally orthogonal grid that the model uses internally and affects the
initialization of the coriolis force.  Note - it is not required to
set a longitude boundary, since the absolute longitude does not alter
the kernel equation discretisation.

\begin{itemize}
\item Line 40
\begin {verbatim}
 dxSpacing=2.8125,
\end{verbatim}
\end{itemize}

\noindent This line sets the horizontal grid spacing between each
y-coordinate line in the discrete grid to $2.8125^{\circ}$ in
longitude.

\begin{itemize}
\item Line 41
\begin {verbatim}
 dySpacing=2.8125,
\end{verbatim}
\end{itemize}

\noindent This line sets the vertical grid spacing between each
x-coordinate line in the discrete grid to $2.8125^{\circ}$ in
latitude.

\begin{itemize}
\item Line 46
\begin {verbatim}
bathyFile='depth_g77.bin',
\end{verbatim}
\end{itemize}

\noindent This line specifies the name of the file, in this case {\it
  depth\_g77.bin}, from which the domain bathymetry is read. This file
contains a two-dimensional ($x,y$) map of (assumed 64-bit) binary
numbers giving the depth of the model at each grid cell, ordered with
the x coordinate varying fastest. The points are ordered from low
coordinate to high coordinate for both axes. The units and orientation
of the depths in this file are the same as used in the MITgcm code. In
this experiment, a depth of $0m$ indicates land.
  

\begin {verbatim}
1    # ====================
2    # | Model parameters |
3    # ====================
4    #
5    # Continuous equation parameters
6    &PARM01
7    &
8    #
9    # Elliptic solver parameters
10   &PARM02
11   cg2dMaxIters=1000,
12   cg2dTargetResidual=1.E-13,
13   &
14   #
15
16   # Time stepping parameters
17   &PARM03
18   nIter0  = 0,
19   nTimeSteps = 720,
20   deltaTtracer= 43200.0,
21   deltaTClock = 43200.0,
22   pChkptFreq=31104000.,
23   chkptFreq= 31104000.,
24   dumpFreq=  2592000.,
25   taveFreq=  311040000.,
26   monitorFreq= 1.,
27   periodicExternalForcing=.TRUE.,
28   externForcingPeriod=2592000.,
29   externForcingCycle=31104000.,
30   &
31   #
32   # Gridding parameters
33   &PARM04
34   usingCartesianGrid=.FALSE.,
35   usingSphericalPolarGrid=.TRUE.,
36   delR=  50.,  70., 100., 140., 190., 
37         240., 290., 340., 390., 440., 
38         490., 540., 590., 640., 690.,
39   ygOrigin=-90.,
40   dxSpacing=2.8125,
41   dySpacing=2.8125,
42   &
43   #
44   # Input datasets
45   &PARM05
46   bathyFile=      'depth_g77.bin',
47   &
\end{verbatim}

\subsubsection{File {\it input/data.off}}
%\label{www:tutorials}

\noindent {\it input/data.off} provides the MITgcm offline package
with package specific parameters. {\it input/data.off} specifies the
location (relative to the run directory) and prefix of files
describing the flow field (UvelFile, VvelFile, WvelFile) and the
corresponding convective mixing coefficients (ConvFile) which together
prescribe the three dimensional, time varying dynamic system within
which the offline model will advect the tracer.

\begin{itemize}
\item Lines 2 to 5
\begin {verbatim}
  UvelFile= '../input/input_off/uVeltave',
  VvelFile= '../input/input_off/vVeltave',
  WvelFile= '../input/input_off/wVeltave',
  ConvFile= '../input/input_off/Convtave',
\end{verbatim}
\end{itemize}

\noindent In the example the offline data is located in the
sub-directory {\it input/input\_off}. In this directory are fields
describing the velocity and convective mixing histories of a prior
forward integration of the MITgcm required for the offline package and
identified in {\it input/data\_off}. Based on the values of {\it
  deltatToffline}, {\it offlineForcingPeriod} and {\it
  offlineForcingCycle} specified in {\it input/data.off}, since {\it
  offlineForcingCycle} corresponds to 12 forcing periods {\it
  offlineForcingPeriod} and since offlineIter0 is zero, there needs to
be 12 uVeltave, 12 vVeltave, 12 wVeltave and 12 Convtave files each
having a 10 digit sequence identifier between 0000000001 to
0000000012, that is, a total of 48 files.


\begin{itemize}
\item Line 9
\begin {verbatim}
  offlineIter0=0,
\end{verbatim}
\end{itemize}

\noindent offlineIter0, here specified to be 0 timesteps, corresponds
to the timestep at which the tracer model is initialised. Note that
offlineIter0 and nIter0 (set in {\it input/data}) need not be the
same.

\begin{itemize}
\item Line 10
\begin {verbatim}
  deltaToffline=43200.,
\end{verbatim}
\end{itemize}


\noindent deltatToffline sets the timestep associated with the offline
model data in seconds, here 12 hours (43200s = 12 hours).

\begin{itemize}
\item Line 11
\begin {verbatim}
  offlineForcingPeriod=43200.,
\end{verbatim}
\end{itemize}

\noindent offlineForcingPeriod sets the forcing period associated with
the offline model data in seconds.

\begin{itemize}
\item Line 12
\begin {verbatim}
  offlineForcingCycle=518400.,
\end{verbatim}
\end{itemize}

\noindent offlineForcingCycle sets the forcing cycle length associated
with the offline model data in seconds. In this example the offline
forcing cycle is 6 days, or 12 offline forcing periods.  Together
deltatToffline, offlineForcingPeriod and offlineForcingCycle determine
the value of the 10 digit sequencing tag the model expects files in
{\it input/input\_off} to have.


\begin {verbatim}
1   &OFFLINE_PARM01
2    UvelFile= 'input_off/uVeltave',
3    VvelFile= 'input_off/vVeltave',
4    WvelFile= 'input_off/wVeltave',
5    ConvFile= 'input_off/Convtave',
6   &end

7   &OFFLINE_PARM02
8    offlineIter0=0,
9    deltaToffline=43200.,
10   offlineForcingPeriod=43200.,
11   offlineForcingCycle=518400.,
12  &end
\end{verbatim}


\subsubsection{File {\it input/data.pkg}}
%\label{www:tutorials}

\noindent File {\it input/data.pkg}, reproduced completely below,
specifies which MITgcm packages ({\it ~/MITgcm/pkg}) are to be used.


\begin {itemize}
\item Line 3
\begin {verbatim}
 usePTRACERS=.TRUE.,
\end{verbatim}
\end{itemize}

\noindent usePTRACERS is a flag invoking the ptracers package which is
responsible for the advection of the tracer within the model.

\begin {verbatim}
1  # Packages
2   &PACKAGES
3   usePTRACERS=.TRUE.,
4   &
\end{verbatim}

\subsubsection{File {\it input/data.ptracers}}
%\label{www:tutorials}

\noindent File {\it input/data.ptracers}, reproduced completely below,
provides the MITgcm ptracers package with package specific parameters,
prescribing the nature of the the tracer/tracers as well as the
variables associated with their advection.

\begin{itemize}
\item Line 2
\begin {verbatim}
 PTRACERS_numInUse=2,
\end{verbatim}
\end{itemize}

\noindent PTRACERS\_numInUse tells the model how many separate tracers
are to be advected, in this case 2. Note: The value of
PTRACERS\_numInUse must agree with the value specified in {\it
  code/PTRACERS\_SIZE.h} - see {\it code/PTRACERS\_SIZE.h} below.

\begin{itemize}
\item Line 3
\begin {verbatim}
 PTRACERS_Iter0= 0,
\end{verbatim}
\end{itemize}

\noindent PTRACERS\_Iter0 specifies the iteration at which the tracer
is to be introduced. In this case the tracer is initialised at the
start of the simulation. i.e. PTRACERS\_Iter0 = PTRACERS\_nIter0.

\begin{itemize}
\item Lines 5 and 10
\begin {verbatim}
 PTRACERS_advScheme(1)=77,
\end{verbatim}
\end{itemize}

\noindent PTRACERS\_advScheme(n) identifies which advection scheme
will be used for tracer n, where n is the number of the tracer up to
PTRACERS\_numInUse. See section 2.18, 'Comparison of advection
schemes', to identify the numerical codes used to specify different
advection schemes (e.g. centered 2nd order, 3rd order upwind) as well
as details of each.
 
\begin{itemize}
\item Lines 6 and 11
\begin{verbatim}
 PTRACERS_diffKh(1)=1.E3,
\end{verbatim}
\end{itemize}

\noindent PTRACERS\_diffKh(n) is the horizontal diffusion coefficient
for tracer n, where n is the number of the tracer up to
PTRACERS\_numInUse.

\begin {itemize}
\item Lines 7 and 12
\begin{verbatim}
 PTRACERS_diffKr(1)=5.E-5,
\end{verbatim}
\end{itemize}

\noindent PTRACERS\_diffKr(n) is the vertical diffusion coefficient
for tracer n, where n is the number of the tracer up to
PTRACERS\_numInUse.

\begin {itemize}
\item Lines 8 and 13
\begin{verbatim}
 PTRACERS_initialFile(1)='tracer1.bin',
\end{verbatim}
\end{itemize}

\noindent PTRACERS\_initialFile(n) identifies the initial tracer field
to be associated with tracer n, where n is the number of the tracer up
to PTRACERS\_numInUse. In this example file {\it input/tracer1.bin}
contains localised tracer, {\it input/tracer2.bin} contains an
arbitrary global distribution. Included Matlab script {\it
  input/makeinitialtracer.m} provides a template for generating or
manipulating initial tracer fields.

\begin {verbatim}
1   &PTRACERS_PARM01
2   PTRACERS_numInUse=2,
3   PTRACERS_Iter0= 0,
4  # tracer 1 
5   PTRACERS_advScheme(1)=77,
6   PTRACERS_diffKh(1)=1.E3,
7   PTRACERS_diffKr(1)=5.E-5,
8   PTRACERS_initialFile(1)='tracer1.bin',
9  # tracer 2 
10  PTRACERS_advScheme(2)=77,
11  PTRACERS_diffKh(2)=1.E3,
12  PTRACERS_diffKr(2)=5.E-5,
13  PTRACERS_initialFile(2)='tracer2.bin',
14  &
\end{verbatim}

\noindent Note {\it input/data.ptracers} requires a set of entries for
each tracer.

\subsubsection{File {\it input/eedata}}
%\label{www:tutorials}
\noindent This file uses standard default values and does not contain
customisations for this experiment.

\noindent The following code changes are required to run this exaperiment. 

\subsubsection{File {\it code/packages.conf}}
%\label{www:tutorials}

\noindent This file is used to invoke the model components required
for a particuylar implementation of the MITgcm. In this case the {\it
  code/packages.conf} contains the component names:

\begin {verbatim}
ptracers
generic_advdiff
mdsio
mom_fluxform
mom_vecinv
timeave
rw
monitor
offline
\end{verbatim}

\subsubsection{File {\it code/PTRACERS\_SIZE.h}}
%\label{www:tutorials}
\begin{itemize}
\item Line
\begin{verbatim}
     PARAMETER(PTRACERS_num = 2 )
\end{verbatim}
\end{itemize}

\noindent This line sets the parameters PTRACERS\_num (the number of
tracers to be integrated) to 2 (in agreement with {\it
  input/data.ptracers}).

\subsubsection{File {\it code/SIZE.h}}
%\label{www:tutorials}

\noindent Two lines are customized in this file for the current
experiment

\begin{itemize}

\item Line 39, 
\begin{verbatim} sNx=128, \end{verbatim} \noindent this line sets
the lateral domain extent in grid points for the
axis aligned with the x-coordinate.

\item Line 40, 
\begin{verbatim} sNy=64, \end{verbatim} 
\noindent this line sets
the lateral domain extent in grid points for the
axis aligned with the y-coordinate.

\item Line 49, 
\begin{verbatim} Nr=15,   \end{verbatim} 
\noindent this line sets
the vertical domain extent in grid points.

\end{itemize}


\subsection{Running The Example}
%\label{www:tutorials}
%\label{sec:running_the_example}

\subsubsection{Code Download}
%\label{www:tutorials}

In order to run the examples you must first download the code
distribution.  Instructions for downloading the code can be found in
section \ref{sec:obtainingCode}.

\subsubsection{Experiment Location}
%\label{www:tutorials}

This example experiment is located under the release sub-directory

\vspace{5mm}
{\it verification/offline/ }

\subsubsection{Running the Experiment}
%\label{www:tutorials}

To run the experiment

\begin{enumerate}
\item Set the current directory to {\it input/ }

\begin{verbatim}
% cd input
\end{verbatim}

\item Verify that current directory is now correct

\begin{verbatim}
% pwd
\end{verbatim}

  You should see a response on the screen ending in {\it
    verification/offline/input }

\item Copy the contents of {\it input/} including subdirectory {\it
    input/input\_off/} to a new directory called {\it run/}

\item Listing directory {\it run/} you should see:

\begin {verbatim}
data      data.pkg       depth_g77.bin  input_off    tracer2.bin
data.off  data.ptracers  eedata         tracer1.bin
\end{verbatim}


\item Set the current directory to {\it run/ }


\item Run the genmake script to create the experiment {\it Makefile}

\begin {verbatim}
% ../../../tools/genmake2 -mods=../code
\end{verbatim}

\item Create a list of header file dependencies in {\it Makefile}

\begin {verbatim}
% make depend
\end{verbatim}

\item Build the executable file.

\begin {verbatim}
% make
\end{verbatim}

\item Run the {\it mitgcmuv} executable

\begin {verbatim}
% ./mitgcmuv
\end{verbatim}

\end{enumerate}

Besides the input files and the files the model generates describing
the grid (prefixed Depth, DXC, DXG, hFacC, hFacS and hFacW, you should
now have 26 single precision binary files
PTRACER01.0000000000-0000000720.001.001.data and
PTRACER02.0000000000-0000000720.001.001.data and their 26
corresponding meta files as well as a single pickup file,
pickup\_ptracers.0000000720.001.001.data and its corresponding meta
file pickup\_ptracers.0000000720.001.001.meta. To run on simply change
nIter0 in file {\it run/data} to 720...

\subsection{A more complicated example}
\label{sec:eg-offline-cfc}
\begin{center}
(in directory: {\it verification/tutorial\_cfc\_offline/})\\
\end{center}

\noindent The last example demonstrated simple advection of a passive
tracer using the offline form of the MITgcm. Now we present a more
complicated example in which the model is used to explore
contamination of the global ocean through surface exposure to CFC's
during the last century. In invoking packages gchem, gmredi and cfc it
provides a starting point and template for more complicated offline
modeling, involving as it does surface forcing through wind and ice
fields, more sophisticated mixing and a time-varying forcing funtion.

\vspace{0.5cm}

\noindent The model configuration for this experiment resides under
the directory {\it verification/cfc\_offline}.  The experiment files

\begin{itemize}
\item {\it input/data}
\item {\it input/data.gchem}
\item {\it input/data.gmredi}
\item {\it input/data.off}
\item {\it input/data.pkg}
\item {\it input/data.ptracers}
\item {\it input/eedata}
\item {\it code/GCHEM\_OPTIONS.h}
\item {\it code/GMREDI\_OPTIONS.h}
\item {\it input/packages.conf}
\item {\it code/PTRACERS\_SIZE.h}
\item {\it code/SIZE.h}. 
\end{itemize}

\noindent contain all the code customisations and parameter settings
required.

\noindent The full list of other files required becomes:

\begin{verbatim}
cfc1112.atm  data.ptracers   
data         depth_g77.bin   pickup.0004269600.data
data.gchem   eedata          
data.gmredi  ice.bin         pickup_ptracers.0004269600.data
data.off     data.pkg        tren_speed.bin
\end{verbatim}

\noindent and

\begin{verbatim}
input_off/:
Convtave.0004248060.data  GM_Kwz-T.0004248060.data  uVeltave.0004248060.data
Convtave.0004248060.meta  GM_Kwz-T.0004248060.meta  uVeltave.0004248060.meta
Convtave.0004248720.data  GM_Kwz-T.0004248720.data  uVeltave.0004248720.data
Convtave.0004248720.meta  GM_Kwz-T.0004248720.meta  uVeltave.0004248720.meta
GM_Kwx-T.0004248060.data  Stave.0004248060.data     vVeltave.0004248060.data
GM_Kwx-T.0004248060.meta  Stave.0004248060.meta     vVeltave.0004248060.meta
GM_Kwx-T.0004248720.data  Stave.0004248720.data     vVeltave.0004248720.data
GM_Kwx-T.0004248720.meta  Stave.0004248720.meta     vVeltave.0004248720.meta
GM_Kwy-T.0004248060.data  Ttave.0004248060.data     wVeltave.0004248060.data
GM_Kwy-T.0004248060.meta  Ttave.0004248060.meta     wVeltave.0004248060.meta
GM_Kwy-T.0004248720.data  Ttave.0004248720.data     wVeltave.0004248720.data
GM_Kwy-T.0004248720.meta  Ttave.0004248720.meta     wVeltave.0004248720.meta
\end{verbatim}


\subsubsection{File {\it input/data}}
%\label{www:tutorials}

\noindent A single line must be added (under PARM01, between lines 6
and 7) in file {\it input/data} from the previous example

\begin{verbatim}
 &PARM01
 implicitDiffusion=.TRUE.,
 &
\end{verbatim}

\noindent When package GMREDI is used, the flag implicitDiffusion must
be assigned the value .TRUE. For information about MITgcm package
GMREDI see...

\vspace{0.5cm}
\noindent In this example the starting timestep nIter0 is set to
4269600 requiring model access to pickup files with the timetag
0004269600. The model will run for 4 timesteps (nTimeSteps = 4). In
this case the frequencies with which permanent and rolling checkpoints
(pChkptFreq and chkptFreq) have been set is sufficiently long to
ensure that only one from the last timestep will be written. This is
also true of the values that have been assigned to the frequency with
which dumps are written (dumpFreq) and time averaging (taveFreq) is
performed however since the model always dumps the state of the model
when it stops without error a dump will be written with timetag
0004269604 upon completion.


\subsubsection{File {\it input/data.off}}
%\label{www:tutorials}

\noindent File {\it input/data.off}, reproduced in full below,
specifies the prefixes and locations of additional input files
required to run the offline model. Note that {\it input/input\_off}
contains only as many offline files as are required to successfully
run for 4 timesteps. Where the GMREDI scheme was used in the forward
run, as here, package GMREDI must again be invoked when running
offline. In this example tracer is specified as having beeen
introduced with a non-zero starttime, at timestep 4248000.

\begin {verbatim}
1   &OFFLINE_PARM01
2    UvelFile= '../input/input_off/uVeltave',
3    VvelFile= '../input/input_off/vVeltave',
4    WvelFile= '../input/input_off/wVeltave',
5    GMwxFile= '../input/input_off/GM_Kwx-T',
6    GMwyFile= '../input/input_off/GM_Kwy-T',
7    GMwzFile= '../input/input_off/GM_Kwz-T',
8    ConvFile= '../input/input_off/Convtave',
9   &end

10  &OFFLINE_PARM02
11   offlineIter0=4248000,
12   deltaToffline=43200.,
13   offlineForcingPeriod=2592000.,
14   offlineForcingCycle=31104000.,
15  &end
\end{verbatim}

\subsubsection{File {\it input/data.pkg}}
%\label{www:tutorials}

\noindent File {\it input/data.pkg}, reproduced completely below,
specifies which MITgcm packages ({\it ~/MITgcm/pkg}) are to be used.
It now invokes additional packages {\it pkg/gmredi} and {pkg/gchem}.

\begin {verbatim}
1  # Packages
2   &PACKAGES
3   useGMRedi=.TRUE.,
4   usePTRACERS=.TRUE.,
5   useGCHEM=.TRUE.,
6   &

\end{verbatim}


\subsubsection{File {\it input/data.ptracers}}
%\label{www:tutorials}

\noindent File {\it input/data.ptracers}, reproduced completely below,
specifies parameters associated with the CFC11 and CFC12 tracer fields
advected in this example.


\begin{itemize}
\item Line 3
\begin {verbatim}
  PTRACERS_Iter0= 4248000,
\end{verbatim}
\end{itemize}

\noindent In this example the tracers were introduced at iteration
4248000.

\begin{itemize}
\item Lines 7 and 14
\begin {verbatim}
  PTRACERS_diffKh(n)=0.E3,
\end{verbatim}
\end{itemize}

\noindent Since package GMREDI is being used, regular horizontal
diffusion is set to zero.

\begin{itemize}
\item Lines 9,10 and 16,17
\begin {verbatim}
  PTRACERS_useGMRedi(n)=.TRUE. ,
  PTRACERS_useKPP(n)=.FALSE. ,
 \end{verbatim}
\end{itemize}

\noindent Setting flag PTRACERS\_useGMRedi(n) to .TRUE. identifies
that package GMREDI is to be used. Setting flag PTRACERS\_useKPP(n) to
.FALSE. explicitly turns off KPP mixing.

\begin{itemize}
\item Lines 11 and 18
\begin {verbatim}
 PTRACERS_initialFile(n)=' ',
 \end{verbatim}
\end{itemize}

\noindent Since this is a `pickup' run the initial tracer files
PTRACERS\_initialFile(1) and PTRACERS\_initialFile(2) aree not needed.
The model will obtain the tracer state from
pickup\_ptracers.0004269600.data

\begin {verbatim}
1   &PTRACERS_PARM01
2   PTRACERS_numInUse=2,
3   PTRACERS_Iter0= 4248000,
4  #
5  # tracer 1 - CFC11
6   PTRACERS_advScheme(1)=77,
7   PTRACERS_diffKh(1)=0.E3,
8   PTRACERS_diffKr(1)=5.E-5,
9   PTRACERS_useGMRedi(1)=.TRUE. ,
10  PTRACERS_useKPP(1)=.FALSE. ,
11  PTRACERS_initialFile(1)=' ',
12 # tracer 2 - CFC12
13  PTRACERS_advScheme(2)=77,
14  PTRACERS_diffKh(2)=0.E3,
15  PTRACERS_diffKr(2)=5.E-5,
16  PTRACERS_useGMRedi(2)=.TRUE. ,
17  PTRACERS_useKPP(2)=.FALSE. ,
18  PTRACERS_initialFile(2)=' ',
19  &
\end{verbatim}


\subsubsection{File {\it input/data.gchem}}
%\label{www:tutorials}

\noindent File {\it input/data.gchem}, reproduced completely below,
names the forcing files needed in package GCHEM.

\begin{itemize}
\item Line 3
\begin {verbatim}
   iceFile='ice.bin',
\end{verbatim}
\end{itemize}

\noindent File {\it input/ice.bin} contains 12, monthly surface ice
fields.

\begin{itemize}
\item Line 3
\begin {verbatim}
   iceFile='tren_speed.bin',
\end{verbatim}
\end{itemize}

\noindent File {\it input/tren\_speed.bin} contains 12, monthly
surface wind fields.

\begin {verbatim}
1   &GCHEM_PARM01
2    iceFile='fice.bin',
3    windFile='tren_speed.bin',
4   &
\end{verbatim}

\noindent Package GCHEM is described in detail in section ??

\subsubsection{File {\it input/data.gmredi}}
%\label{www:tutorials}

\noindent File {\it input/data.gmredi}, reproduced completely below,
provides the parameters required for package GMREDI.

\begin {verbatim}
1   &GM_PARM01
2    GM_background_K    = 1.e+3,
3    GM_taper_scheme    = 'gkw91',
4    GM_maxSlope        = 1.e-2,
5    GM_Kmin_horiz      = 100.,
6    GM_Scrit           = 4.e-3,
7    GM_Sd              = 1.e-3,
8   &
\end{verbatim}

\noindent Package GMREDI is described in detail in section ??

\subsubsection{File {\it input/cfc1112.atm}}
%\label{www:tutorials}

\noindent File {\it input/cfc1112.atm} is a text file containing the
CFC source functions over the northern and southern hemispheres
annually from 1931 through 1998.

\subsubsection{File {\it code/packages.conf}}
%\label{www:tutorials}

\noindent In this example {\it code/packages.conf} additionally
invokes components gchem, cfc and rmredi:

\begin {verbatim}
ptracers
generic_advdiff
mdsio
mom_fluxform
mom_vecinv
timeave
rw
monitor
offline
gchem
cfc
gmredi
\end{verbatim}

\subsubsection{File {\it code/GCHEM\_OPTIONS.h}}
%\label{www:tutorials}

\noindent File {\it code/GCHEM\_OPTIONS.h}, specifies options for
package GCHEM. In this case defining the flag ALLOW\_CFC to activate
the cfc code.

\subsubsection{File {\it code/GMREDI\_OPTIONS.h}}
%\label{www:tutorials}

\noindent File {\it code/GCHEM\_OPTIONS.h}, specifies options for
package GMREDI.

\subsubsection{File {\it code/PTRACERS\_SIZE.h}}
%\label{www:tutorials}

\noindent File {\it code/PTRACERS\_SIZE.h} is unchanged from the
simpler example.

\subsubsection{File {\it code/SIZE.h}}
%\label{www:tutorials}

\noindent File {\it code/SIZE.h} is unchanged from the simpler
example.


\subsubsection{Running the Experiment}
%\label{www:tutorials}

The model is run as before and produces the files the model generates
describing the grid (prefixed Depth, DXC, DXG, hFacC, hFacS and hFacW)
as well as 2, single precision, binary files
PTRACER01.0004269600-0004269604.001.001.data and
PTRACER02.0004269600-0004269604.001.001.data and their 2 corresponding
meta files as well as a single pickup file,
pickup\_ptracers.ckptA.001.001.data and its corresponding meta file
pickup\_ptracers.ckptA.001.001.meta from which you could run the model
on.

%% \end{document}



