% $Header: /u/gcmpack/manual/s_examples/tracer_adjsens/Attic/co2sens.tex,v 1.5 2001/11/13 19:01:42 adcroft Exp $
% $Name:  $

\section{Example: Centennial Time Scale Sensitivities}

\bodytext{bgcolor="#FFFFFFFF"}

%\begin{center} 
%{\Large \bf Using MITgcm to Look at Centennial Time Scale
%Sensitivities}
%
%\vspace*{4mm}
%
%\vspace*{3mm}
%{\large May 2001}
%\end{center}

\subsection{Introduction}

This document describes the fourth example MITgcm experiment.
This example illustrates the use of
the MITgcm to perform sensitivity analysis in a
large scale ocean circulation simulation.

\subsection{Overview}

This example experiment demonstrates using the MITgcm to simulate
the planetary ocean circulation. The simulation is configured
with realistic geography and bathymetry on a
$4^{\circ} \times 4^{\circ}$ spherical polar grid.
Twenty vertical layers are used in the vertical, ranging in thickness
from $50\,{\rm m}$ at the surface to $815\,{\rm m}$ at depth,
giving a maximum model depth of $6\,{\rm km}$.
At this resolution, the configuration
can be integrated forward for thousands of years on a single 
processor desktop computer.
\\

The model is forced with climatological wind stress data and surface
flux data from Da Silva \cite{DaSilva94}. Climatological data
from Levitus \cite{Levitus94} is used to initialize the model hydrography.
Levitus data is also used throughout the calculation
to derive air-sea fluxes of heat at the ocean surface.
These fluxes are combined with climatological estimates of
surface heat flux and fresh water, resulting in a mixed boundary
condition of the style described in Haney \cite{Haney}.
Altogether, this yields the following forcing applied
in the model surface layer.

\begin{eqnarray}
\label{EQ:global_forcing}
\label{EQ:global_forcing_fu}
{\cal F}_{u} & = & \frac{\tau_{x}}{\rho_{0} \Delta z_{s}}
\\
\label{EQ:global_forcing_fv}
{\cal F}_{v} & = & \frac{\tau_{y}}{\rho_{0} \Delta z_{s}}
\\
\label{EQ:global_forcing_ft}
{\cal F}_{\theta} & = & - \lambda_{\theta} ( \theta - \theta^{\ast} ) 
 - \frac{1}{C_{p} \rho_{0} \Delta z_{s}}{\cal Q}
\\
\label{EQ:global_forcing_fs}
{\cal F}_{s} & = & - \lambda_{s} ( S - S^{\ast} ) 
 + \frac{S_{0}}{\Delta z_{s}}({\cal E} - {\cal P} - {\cal R})
\end{eqnarray}

\noindent where ${\cal F}_{u}$, ${\cal F}_{v}$, ${\cal F}_{\theta}$,
${\cal F}_{s}$ are the forcing terms in the zonal and meridional
momentum and in the potential temperature and salinity 
equations respectively.
The term $\Delta z_{s}$ represents the top ocean layer thickness.
It is used in conjunction with the reference density, $\rho_{0}$
(here set to $999.8\,{\rm kg\,m^{-3}}$), the
reference salinity, $S_{0}$ (here set to 35ppt),
and a specific heat capacity $C_{p}$ to convert
wind-stress fluxes given in ${\rm N}\,m^{-2}$, 
\\


The configuration is illustrated in figure \ref{simulation_config}.


\subsection{Discrete Numerical Configuration}


 The model is configured in hydrostatic form.  The domain is discretised with 
a uniform grid spacing in latitude and longitude of
 $\Delta x=\Delta y=4^{\circ}$, so 
that there are ninety grid cells in the $x$ and forty in the 
$y$ direction (Arctic polar regions are not
included in this experiment). Vertically the 
model is configured with twenty layers with the following thicknesses
$\Delta z_{1} = 50\,{\rm m},\,
 \Delta z_{2} = 50\,{\rm m},\,
 \Delta z_{3} = 55\,{\rm m},\,
 \Delta z_{4} = 60\,{\rm m},\,
 \Delta z_{5} = 65\,{\rm m},\,
$
$
 \Delta z_{6}~=~70\,{\rm m},\,
 \Delta z_{7}~=~80\,{\rm m},\,
 \Delta z_{8}~=95\,{\rm m},\,
 \Delta z_{9}=120\,{\rm m},\,
 \Delta z_{10}=155\,{\rm m},\,
$
$
 \Delta z_{11}=200\,{\rm m},\,
 \Delta z_{12}=260\,{\rm m},\,
 \Delta z_{13}=320\,{\rm m},\,
 \Delta z_{14}=400\,{\rm m},\,
 \Delta z_{15}=480\,{\rm m},\,
$
$
 \Delta z_{16}=570\,{\rm m},\,
 \Delta z_{17}=655\,{\rm m},\,
 \Delta z_{18}=725\,{\rm m},\,
 \Delta z_{19}=775\,{\rm m},\,
 \Delta z_{20}=815\,{\rm m}
$ (here the numeric subscript indicates the model level index number, ${\tt k}$).
The implicit free surface form of the pressure equation described in Marshall et. al 
\cite{marshall:97a} is employed. A Laplacian operator, $\nabla^2$, provides viscous
dissipation. Thermal and haline diffusion is also represented by a Laplacian operator.
\\

Wind-stress momentum inputs are added to the momentum equations for both
the zonal flow, $u$ and the meridional flow $v$, according to equations 
(\ref{EQ:global_forcing_fu}) and (\ref{EQ:global_forcing_fv}).
Thermodynamic forcing inputs are added to the equations for
potential temperature, $\theta$, and salinity, $S$, according to equations 
(\ref{EQ:global_forcing_ft}) and (\ref{EQ:global_forcing_fs}).
This produces a set of equations solved in this configuration as follows:
% {\fracktur}


\begin{eqnarray}
\label{EQ:model_equations}
\frac{Du}{Dt} - fv + 
  \frac{1}{\rho}\frac{\partial p^{'}}{\partial x} - 
  A_{h}\nabla_{h}^2u - A_{z}\frac{\partial^{2}u}{\partial z^{2}} 
& = &
{\cal F}_{u}
\\
\frac{Dv}{Dt} + fu + 
  \frac{1}{\rho}\frac{\partial p^{'}}{\partial y} - 
  A_{h}\nabla_{h}^2v - A_{z}\frac{\partial^{2}v}{\partial z^{2}} 
& = &
{\cal F}_{v}
\\
\frac{\partial \eta}{\partial t} + \nabla_{h}\cdot \vec{u}
&=&
0
\\
\frac{D\theta}{Dt} -
 K_{h}\nabla_{h}^2\theta  - \Gamma(K_{z})\frac{\partial^{2}\theta}{\partial z^{2}} 
& = &
{\cal F}_{\theta}
\\
\frac{D s}{Dt} -
 K_{h}\nabla_{h}^2 s  - \Gamma(K_{z})\frac{\partial^{2} s}{\partial z^{2}} 
& = &
{\cal F}_{s}
\\
g\rho_{0} \eta + \int^{0}_{-z}\rho^{'} dz & = & p^{'}
\\
\end{eqnarray}

\noindent where $u$ and $v$ are the $x$ and $y$ components of the
flow vector $\vec{u}$. The suffices ${s},{i}$ indicate surface and
interior model levels respectively. As described in
MITgcm Numerical Solution Procedure \ref{chap:discretization}, the time 
evolution of potential temperature, $\theta$, equation is solved prognostically.
The total pressure, $p$, is diagnosed by summing pressure due to surface 
elevation $\eta$ and the hydrostatic pressure.
\\

\subsubsection{Numerical Stability Criteria}

The Laplacian dissipation coefficient, $A_{h}$, is set to $400 m s^{-1}$.
This value is chosen to yield a Munk layer width \cite{adcroft:95},

\begin{eqnarray}
\label{EQ:munk_layer}
M_{w} = \pi ( \frac { A_{h} }{ \beta } )^{\frac{1}{3}}
\end{eqnarray}

\noindent  of $\approx 100$km. This is greater than the model
resolution in mid-latitudes $\Delta x$, ensuring that the frictional 
boundary layer is well resolved.
\\

\noindent The model is stepped forward with a 
time step $\delta t=1200$secs. With this time step the stability 
parameter to the horizontal Laplacian friction \cite{adcroft:95}

\begin{eqnarray}
\label{EQ:laplacian_stability}
S_{l} = 4 \frac{A_{h} \delta t}{{\Delta x}^2}
\end{eqnarray}

\noindent evaluates to 0.012, which is well below the 0.3 upper limit
for stability. 
\\

\noindent The vertical dissipation coefficient, $A_{z}$, is set to 
$1\times10^{-2} {\rm m}^2{\rm s}^{-1}$. The associated stability limit

\begin{eqnarray}
\label{EQ:laplacian_stability_z}
S_{l} = 4 \frac{A_{z} \delta t}{{\Delta z}^2}
\end{eqnarray}

\noindent evaluates to $4.8 \times 10^{-5}$ which is again well below
the upper limit.
The values of $A_{h}$ and $A_{z}$ are also used for the horizontal ($K_{h}$) 
and vertical ($K_{z}$) diffusion coefficients for temperature respectively.
\\

\noindent The numerical stability for inertial oscillations
\cite{adcroft:95} 

\begin{eqnarray}
\label{EQ:inertial_stability}
S_{i} = f^{2} {\delta t}^2
\end{eqnarray}

\noindent evaluates to $0.0144$, which is well below the $0.5$ upper 
limit for stability.
\\

\noindent The advective CFL \cite{adcroft:95} for a extreme maximum 
horizontal flow
speed of $ | \vec{u} | = 2 ms^{-1}$

\begin{eqnarray}
\label{EQ:cfl_stability}
S_{a} = \frac{| \vec{u} | \delta t}{ \Delta x}
\end{eqnarray}

\noindent evaluates to $5 \times 10^{-2}$. This is well below the stability 
limit of 0.5.
\\

\noindent The stability parameter for internal gravity waves 
\cite{adcroft:95}

\begin{eqnarray}
\label{EQ:cfl_stability}
S_{c} = \frac{c_{g} \delta t}{ \Delta x}
\end{eqnarray}

\noindent evaluates to $5 \times 10^{-2}$. This is well below the linear
stability limit of 0.25.
  
\subsection{Code Configuration}
\label{SEC:code_config}

The model configuration for this experiment resides under the 
directory {\it verification/exp1/}.  The experiment files 
\begin{itemize}
\item {\it input/data}
\item {\it input/data.pkg}
\item {\it input/eedata},
\item {\it input/windx.sin\_y},
\item {\it input/topog.box},
\item {\it code/CPP\_EEOPTIONS.h}
\item {\it code/CPP\_OPTIONS.h},
\item {\it code/SIZE.h}. 
\end{itemize}
contain the code customizations and parameter settings for this 
experiments. Below we describe the customizations
to these files associated with this experiment.

\subsubsection{File {\it input/data}}

This file, reproduced completely below, specifies the main parameters 
for the experiment. The parameters that are significant for this configuration
are

\begin{itemize}

\item Line 4, 
\begin{verbatim} tRef=20.,10.,8.,6., \end{verbatim} 
this line sets
the initial and reference values of potential temperature at each model
level in units of $^{\circ}$C.
The entries are ordered from surface to depth. For each
depth level the initial and reference profiles will be uniform in
$x$ and $y$.

\fbox{
\begin{minipage}{5.0in}
{\it S/R INI\_THETA}({\it ini\_theta.F})
\end{minipage}
}


\item Line 6, 
\begin{verbatim} viscAz=1.E-2, \end{verbatim} 
this line sets the vertical Laplacian dissipation coefficient to
$1 \times 10^{-2} {\rm m^{2}s^{-1}}$. Boundary conditions
for this operator are specified later. This variable is copied into
model general vertical coordinate variable {\bf viscAr}.

\fbox{
\begin{minipage}{5.0in}
{\it S/R CALC\_DIFFUSIVITY}({\it calc\_diffusivity.F})
\end{minipage}
}

\item Line 7, 
\begin{verbatim}
viscAh=4.E2,
\end{verbatim} 
this line sets the horizontal Laplacian frictional dissipation coefficient to
$1 \times 10^{-2} {\rm m^{2}s^{-1}}$. Boundary conditions
for this operator are specified later.

\item Lines 8,
\begin{verbatim}
no_slip_sides=.FALSE.
\end{verbatim}
this line selects a free-slip lateral boundary condition for
the horizontal Laplacian friction operator 
e.g. $\frac{\partial u}{\partial y}$=0 along boundaries in $y$ and
$\frac{\partial v}{\partial x}$=0 along boundaries in $x$.

\item Lines 9,
\begin{verbatim}
no_slip_bottom=.TRUE.
\end{verbatim}
this line selects a no-slip boundary condition for bottom
boundary condition in the vertical Laplacian friction operator 
e.g. $u=v=0$ at $z=-H$, where $H$ is the local depth of the domain.

\item Line 10,
\begin{verbatim}
diffKhT=4.E2,
\end{verbatim}
this line sets the horizontal diffusion coefficient for temperature
to $400\,{\rm m^{2}s^{-1}}$. The boundary condition on this
operator is $\frac{\partial}{\partial x}=\frac{\partial}{\partial y}=0$ at
all boundaries.

\item Line 11,
\begin{verbatim}
diffKzT=1.E-2,
\end{verbatim}
this line sets the vertical diffusion coefficient for temperature
to $10^{-2}\,{\rm m^{2}s^{-1}}$. The boundary condition on this
operator is $\frac{\partial}{\partial z}$ = 0 on all boundaries.

\item Line 13,
\begin{verbatim}
tAlpha=2.E-4,
\end{verbatim}
This line sets the thermal expansion coefficient for the fluid
to $2 \times 10^{-4}\,{\rm degrees}^{-1}$

\fbox{
\begin{minipage}{5.0in}
{\it S/R FIND\_RHO}({\it find\_rho.F})
\end{minipage}
}

\item Line 18,
\begin{verbatim}
eosType='LINEAR'
\end{verbatim}
This line selects the linear form of the equation of state.

\fbox{
\begin{minipage}{5.0in}
{\it S/R FIND\_RHO}({\it find\_rho.F})
\end{minipage}
}



\item Line 40,
\begin{verbatim}
usingSphericalPolarGrid=.TRUE.,
\end{verbatim}
This line requests that the simulation be performed in a 
spherical polar coordinate system. It affects the interpretation of
grid input parameters, for example {\bf delX} and {\bf delY} and
causes the grid generation routines to initialize an internal grid based
on spherical polar geometry.

\fbox{
\begin{minipage}{5.0in}
{\it S/R INI\_SPEHRICAL\_POLAR\_GRID}({\it ini\_spherical\_polar\_grid.F})
\end{minipage}
}

\item Line 41,
\begin{verbatim}
phiMin=0.,
\end{verbatim}
This line sets the southern boundary of the modeled
domain to $0^{\circ}$ latitude. This value affects both the
generation of the locally orthogonal grid that the model
uses internally and affects the initialization of the coriolis force.
Note - it is not required to set
a longitude boundary, since the absolute longitude does
not alter the kernel equation discretisation.

\item Line 42,
\begin{verbatim}
delX=60*1.,
\end{verbatim}
This line sets the horizontal grid spacing between each y-coordinate line
in the discrete grid to $1^{\circ}$ in longitude.

\item Line 43,
\begin{verbatim}
delY=60*1.,
\end{verbatim}
This line sets the horizontal grid spacing between each y-coordinate line
in the discrete grid to $1^{\circ}$ in latitude.

\item Line 44,
\begin{verbatim}
delZ=500.,500.,500.,500.,
\end{verbatim}
This line sets the vertical grid spacing between each z-coordinate line
in the discrete grid to $500\,{\rm m}$, so that the total model depth 
is $2\,{\rm km}$. The variable {\bf delZ} is copied into the internal
model coordinate variable {\bf delR}

\fbox{
\begin{minipage}{5.0in}
{\it S/R INI\_VERTICAL\_GRID}({\it ini\_vertical\_grid.F})
\end{minipage}
}

\item Line 47,
\begin{verbatim}
bathyFile='topog.box'
\end{verbatim}
This line specifies the name of the file from which the domain
bathymetry is read. This file is a two-dimensional ($x,y$) map of
depths. This file is assumed to contain 64-bit binary numbers 
giving the depth of the model at each grid cell, ordered with the x 
coordinate varying fastest. The points are ordered from low coordinate
to high coordinate for both axes. The units and orientation of the
depths in this file are the same as used in the MITgcm code. In this
experiment, a depth of $0m$ indicates a solid wall and a depth
of $-2000m$ indicates open ocean. The matlab program
{\it input/gendata.m} shows an example of how to generate a
bathymetry file.


\item Line 50,
\begin{verbatim}
zonalWindFile='windx.sin_y'
\end{verbatim}
This line specifies the name of the file from which the x-direction
surface wind stress is read. This file is also a two-dimensional
($x,y$) map and is enumerated and formatted in the same manner as the 
bathymetry file. The matlab program {\it input/gendata.m} includes example 
code to generate a valid 
{\bf zonalWindFile} 
file.  

\end{itemize}

\noindent other lines in the file {\it input/data} are standard values
that are described in the MITgcm Getting Started and MITgcm Parameters
notes.

\begin{small}
% % $Header: /u/gcmpack/manual/s_examples/baroclinic_gyre/input/data.tex,v 1.1.1.1 2001/08/08 16:15:46 adcroft Exp $
% $Name:  $

\begin{verbatim}
     1	# Model parameters
     2	# Continuous equation parameters
     3	 &PARM01
     4	 tRef=20.,10.,8.,6.,
     5	 sRef=10.,10.,10.,10.,
     6	 viscAz=1.E-2,
     7	 viscAh=4.E2,
     8	 no_slip_sides=.FALSE.,
     9	 no_slip_bottom=.TRUE.,
    10	 diffKhT=4.E2,
    11	 diffKzT=1.E-2,
    12	 beta=1.E-11,
    13	 tAlpha=2.E-4,
    14	 sBeta =0.,
    15	 gravity=9.81,
    16	 rigidLid=.FALSE.,
    17	 implicitFreeSurface=.TRUE.,
    18	 eosType='LINEAR',
    19	 readBinaryPrec=64,
    20	 &
    21	# Elliptic solver parameters
    22	 &PARM02
    23	 cg2dMaxIters=1000,
    24	 cg2dTargetResidual=1.E-13,
    25	 &
    26	# Time stepping parameters
    27	 &PARM03
    28	 startTime=0.,
    29	 endTime=12000., 
    30	 deltaTmom=1200.0,
    31	 deltaTtracer=1200.0,
    32	 abEps=0.1,
    33	 pChkptFreq=17000.0,
    34	 chkptFreq=0.0,
    35	 dumpFreq=2592000.0,
    36	 &
    37	# Gridding parameters
    38	 &PARM04
    39	 usingCartesianGrid=.FALSE.,
    40	 usingSphericalPolarGrid=.TRUE.,
    41	 phiMin=0.,
    42	 delX=60*1.,
    43	 delY=60*1.,
    44	 delZ=500.,500.,500.,500.,
    45	 &
    46	 &PARM05
    47	 bathyFile='topog.box',
    48	 hydrogThetaFile=,
    49	 hydrogSaltFile=,
    50	 zonalWindFile='windx.sin_y',
    51	 meridWindFile=,
    52	 &
\end{verbatim}

\end{small}

\subsubsection{File {\it input/data.pkg}}

This file uses standard default values and does not contain
customizations for this experiment.

\subsubsection{File {\it input/eedata}}

This file uses standard default values and does not contain
customizations for this experiment.

\subsubsection{File {\it input/windx.sin\_y}}

The {\it input/windx.sin\_y} file specifies a two-dimensional ($x,y$) 
map of wind stress ,$\tau_{x}$, values. The units used are $Nm^{-2}$.
Although $\tau_{x}$ is only a function of $y$n in this experiment
this file must still define a complete two-dimensional map in order
to be compatible with the standard code for loading forcing fields 
in MITgcm. The included matlab program {\it input/gendata.m} gives a complete
code for creating the {\it input/windx.sin\_y} file.

\subsubsection{File {\it input/topog.box}}


The {\it input/topog.box} file specifies a two-dimensional ($x,y$) 
map of depth values. For this experiment values are either
$0m$ or $-2000\,{\rm m}$, corresponding respectively to a wall or to deep
ocean. The file contains a raw binary stream of data that is enumerated
in the same way as standard MITgcm two-dimensional, horizontal arrays.
The included matlab program {\it input/gendata.m} gives a complete
code for creating the {\it input/topog.box} file.

\subsubsection{File {\it code/SIZE.h}}

Two lines are customized in this file for the current experiment

\begin{itemize}

\item Line 39, 
\begin{verbatim} sNx=60, \end{verbatim} this line sets
the lateral domain extent in grid points for the
axis aligned with the x-coordinate.

\item Line 40, 
\begin{verbatim} sNy=60, \end{verbatim} this line sets
the lateral domain extent in grid points for the
axis aligned with the y-coordinate.

\item Line 49, 
\begin{verbatim} Nr=4,   \end{verbatim} this line sets
the vertical domain extent in grid points.

\end{itemize}

\begin{small}
% \begin{verbatim}
     1	C     /==========================================================\
     2	C     | SIZE.h Declare size of underlying computational grid.    |
     3	C     |==========================================================|
     4	C     | The design here support a three-dimensional model grid   |
     5	C     | with indices I,J and K. The three-dimensional domain     |
     6	C     | is comprised of nPx*nSx blocks of size sNx along one axis|
     7	C     | nPy*nSy blocks of size sNy along another axis and one    |
     8	C     | block of size Nz along the final axis.                   |
     9	C     | Blocks have overlap regions of size OLx and OLy along the|
    10	C     | dimensions that are subdivided.                          |
    11  C     \==========================================================/
    12  C     Voodoo numbers controlling data layout.
    13  C     sNx - No. X points in sub-grid.
    14  C     sNy - No. Y points in sub-grid.
    15  C     OLx - Overlap extent in X.
    16  C     OLy - Overlat extent in Y.
    17  C     nSx - No. sub-grids in X.
    18  C     nSy - No. sub-grids in Y.
    19  C     nPx - No. of processes to use in X.
    20  C     nPy - No. of processes to use in Y.
    21  C     Nx  - No. points in X for the total domain.
    22  C     Ny  - No. points in Y for the total domain.
    23  C     Nr  - No. points in Z for full process domain.
    24        INTEGER sNx
    25        INTEGER sNy
    26        INTEGER OLx
    27        INTEGER OLy
    28        INTEGER nSx
    29        INTEGER nSy
    30        INTEGER nPx
    31	      INTEGER nPy
    32	      INTEGER Nx
    33	      INTEGER Ny
    34	      INTEGER Nr
    35	      PARAMETER (
    36	     &           sNx =  64,
    37	     &           sNy =  64,
    38	     &           OLx =   3,
    39	     &           OLy =   3,
    40	     &           nSx =   1,
    41	     &           nSy =   1,
    42	     &           nPx =   1,
    43	     &           nPy =   1,
    44	     &           Nx  = sNx*nSx*nPx,
    45	     &           Ny  = sNy*nSy*nPy,
    46	     &           Nr  =  20)

    47	C     MAX_OLX  - Set to the maximum overlap region size of any array
    48	C     MAX_OLY    that will be exchanged. Controls the sizing of exch
    49	C                routine buufers.
    50	      INTEGER MAX_OLX
    51	      INTEGER MAX_OLY
    52	      PARAMETER ( MAX_OLX = OLx,
    53	     &            MAX_OLY = OLy )

\end{verbatim}
\end{small}

\subsubsection{File {\it code/CPP\_OPTIONS.h}}

This file uses standard default values and does not contain
customizations for this experiment.


\subsubsection{File {\it code/CPP\_EEOPTIONS.h}}

This file uses standard default values and does not contain
customizations for this experiment.

\subsubsection{Other Files }

Other files relevant to this experiment are
\begin{itemize}
\item {\it model/src/ini\_cori.F}. This file initializes the model
coriolis variables {\bf fCorU}.
\item {\it model/src/ini\_spherical\_polar\_grid.F}
\item {\it model/src/ini\_parms.F},
\item {\it input/windx.sin\_y},
\end{itemize}
contain the code customizations and parameter settings for this 
experiments. Below we describe the customizations
to these files associated with this experiment.
