% $Header: /u/gcmpack/manual/s_examples/held_suarez_cs/Attic/hs_atmos.tex,v 1.5 2004/10/13 05:06:26 cnh Exp $
% $Name:  $

\section[Held-Suarez Atmosphere MITgcm Example]{Held-Suarez forcing atmospheric simulation on a latitude-longitude grid 2.8$^\circ$ resolution and on
a cube-sphere grid with 32 square cube faces.}
\label{www:tutorials}
\label{sect:eg-hs}

\bodytext{bgcolor="#FFFFFFFF"}

%\begin{center} 
%{\Large \bf Using MITgcm to Simulate Global Climatological Ocean Circulation
%At Four Degree Resolution with Asynchronous Time Stepping}
%
%\vspace*{4mm}
%
%\vspace*{3mm}
%{\large May 2001}
%\end{center}

This example illustrates the use of the MITgcm for large scale atmospheric 
circulation simulation. Two simulations are described
\begin{itemize}
\item global atmospheric circulation on a latitude-longitude grid and 
\item global atmospheric circulation on a cube-sphere grid
\end{itemize}
The examples show how to use the isomorphic 'p-coordinate' scheme in
MITgcm to enable atmospheric simulation.



\subsection{Overview}
\label{www:tutorials}

This example demonstrates using the MITgcm to simulate
the planetary atmospheric circulation in two ways.
In both cases the simulation is configured with flat orography.
In the first case shown a $2.8^{\circ} \times 2.8^{\circ}$ spherical polar 
horizontal grid is employed. In the second case a cube-sphere horizontal
grid is used that projects a cube with face size of $32 \times 32$ onto a
sphere.
Five pressure corrdinate levels are used in the vertical, ranging in thickness
from $100\,{\rm mb}$ at the bottom of the atmosphere to $300\,{\rm mb}$ in the middle atmosphere.
The total depth of the atmosphere is $1000{\rm mb}$.
At this resolution, the configuration can be integrated forward for many years on a 
single processor desktop computer.
\\

The model is forced by relaxation to a radiative equilibrium profile 
from Held and Suarez \cite{held-suar:94}. Initial conditions are a
statically stable thermal gradient and no motion. The atmosphere 
in these experiments is dry and the only active ``physics'' are the
terms in the Held and Suarez \cite{held-suar:94} formula. The
MITgcm intermediate atmospheric physics package (see \ref{sec:aim}) and
MITgcm high-end physics package ( see \ref{sec:pkg:fizhi}) are turned off.
Altogether, this yields the following forcing that is applied to
the fluid:

\begin{eqnarray}
\label{EQ:eg-hs-global_forcing}
\label{EQ:eg-hs-global_forcing_fu}
\vec{{\cal F}_{u}} & = & -k_{v}(\sigma)\vec{v}
\\
\label{EQ:eg-hs-global_forcing_ft}
{\cal F}_{\theta} & = & - \lambda_{\theta} ( \theta - \theta^{\ast} ) 
 - \frac{1}{C_{p} \rho_{0} \Delta z_{s}}{\cal Q}
\\
\label{EQ:eg-hs-global_forcing_fs}
{\cal F}_{s} & = & - \lambda_{s} ( S - S^{\ast} ) 
 + \frac{S_{0}}{\Delta z_{s}}({\cal E} - {\cal P} - {\cal R})
\end{eqnarray}

\noindent where ${\cal F}_{u}$, ${\cal F}_{v}$, ${\cal F}_{\theta}$,
${\cal F}_{s}$ are the forcing terms in the zonal and meridional
momentum and in the potential temperature and salinity
equations respectively.
The term $\Delta z_{s}$ represents the top ocean layer thickness in
meters.
It is used in conjunction with a reference density, $\rho_{0}$
(here set to $999.8\,{\rm kg\,m^{-3}}$), a
reference salinity, $S_{0}$ (here set to 35~ppt),
and a specific heat capacity, $C_{p}$ (here set to
$4000~{\rm J}~^{\circ}{\rm C}^{-1}~{\rm kg}^{-1}$), to convert
input dataset values into time tendencies of
potential temperature (with units of $^{\circ}{\rm C}~{\rm s}^{-1}$),
salinity (with units ${\rm ppt}~s^{-1}$) and
velocity (with units ${\rm m}~{\rm s}^{-2}$).
The externally supplied forcing fields used in this
experiment are $\tau_{x}$, $\tau_{y}$, $\theta^{\ast}$, $S^{\ast}$,
$\cal{Q}$ and $\cal{E}-\cal{P}-\cal{R}$. The wind stress fields ($\tau_x$, $\tau_y$)
have units of ${\rm N}~{\rm m}^{-2}$. The temperature forcing fields
($\theta^{\ast}$ and $Q$) have units of $^{\circ}{\rm C}$ and ${\rm W}~{\rm m}^{-2}$
respectively. The salinity forcing fields ($S^{\ast}$ and 
$\cal{E}-\cal{P}-\cal{R}$) have units of ${\rm ppt}$ and ${\rm m}~{\rm s}^{-1}$
respectively.
\\


Figures (\ref{FIG:sim_config_tclim}-\ref{FIG:sim_config_empmr}) show the
relaxation temperature ($\theta^{\ast}$) and salinity ($S^{\ast}$) fields,
the wind stress components ($\tau_x$ and $\tau_y$), the heat flux ($Q$)
and the net fresh water flux (${\cal E} - {\cal P} - {\cal R}$) used
in equations \ref{EQ:eg-hs-global_forcing_fu}-\ref{EQ:eg-hs-global_forcing_fs}. The figures
also indicate the lateral extent and coastline used in the experiment.
Figure ({\ref{FIG:model_bathymetry}) shows the depth contours of the model
domain.


\subsection{Discrete Numerical Configuration}
\label{www:tutorials}


 The model is configured in hydrostatic form.  The domain is discretised with 
a uniform grid spacing in latitude and longitude on the sphere
 $\Delta \phi=\Delta \lambda=4^{\circ}$, so 
that there are ninety grid cells in the zonal and forty in the 
meridional direction. The internal model coordinate variables
$x$ and $y$ are initialized according to
\begin{eqnarray}
x=r\cos(\phi),~\Delta x & = &r\cos(\Delta \phi) \\
y=r\lambda,~\Delta x &= &r\Delta \lambda 
\end{eqnarray}

Arctic polar regions are not
included in this experiment. Meridionally the model extends from
$80^{\circ}{\rm S}$ to $80^{\circ}{\rm N}$.
Vertically the model is configured with twenty layers with the 
following thicknesses
$\Delta z_{1} = 50\,{\rm m},\,
 \Delta z_{2} = 50\,{\rm m},\,
 \Delta z_{3} = 55\,{\rm m},\,
 \Delta z_{4} = 60\,{\rm m},\,
 \Delta z_{5} = 65\,{\rm m},\,
$
$
 \Delta z_{6}~=~70\,{\rm m},\,
 \Delta z_{7}~=~80\,{\rm m},\,
 \Delta z_{8}~=95\,{\rm m},\,
 \Delta z_{9}=120\,{\rm m},\,
 \Delta z_{10}=155\,{\rm m},\,
$
$
 \Delta z_{11}=200\,{\rm m},\,
 \Delta z_{12}=260\,{\rm m},\,
 \Delta z_{13}=320\,{\rm m},\,
 \Delta z_{14}=400\,{\rm m},\,
 \Delta z_{15}=480\,{\rm m},\,
$
$
 \Delta z_{16}=570\,{\rm m},\,
 \Delta z_{17}=655\,{\rm m},\,
 \Delta z_{18}=725\,{\rm m},\,
 \Delta z_{19}=775\,{\rm m},\,
 \Delta z_{20}=815\,{\rm m}
$ (here the numeric subscript indicates the model level index number, ${\tt k}$).
The implicit free surface form of the pressure equation described in Marshall et. al 
\cite{marshall:97a} is employed. A Laplacian operator, $\nabla^2$, provides viscous
dissipation. Thermal and haline diffusion is also represented by a Laplacian operator.

Wind-stress forcing is added to the momentum equations for both
the zonal flow, $u$ and the meridional flow $v$, according to equations 
(\ref{EQ:eg-hs-global_forcing_fu}) and (\ref{EQ:eg-hs-global_forcing_fv}).
Thermodynamic forcing inputs are added to the equations for
potential temperature, $\theta$, and salinity, $S$, according to equations 
(\ref{EQ:eg-hs-global_forcing_ft}) and (\ref{EQ:eg-hs-global_forcing_fs}).
This produces a set of equations solved in this configuration as follows:

\begin{eqnarray}
\label{EQ:eg-hs-model_equations}
\frac{Du}{Dt} - fv + 
  \frac{1}{\rho}\frac{\partial p^{'}}{\partial x} - 
  \nabla_{h}\cdot A_{h}\nabla_{h}u - 
  \frac{\partial}{\partial z}A_{z}\frac{\partial u}{\partial z} 
 & = &
\begin{cases}
{\cal F}_u & \text{(surface)} \\
0 & \text{(interior)}
\end{cases}
\\
\frac{Dv}{Dt} + fu + 
  \frac{1}{\rho}\frac{\partial p^{'}}{\partial y} - 
  \nabla_{h}\cdot A_{h}\nabla_{h}v - 
  \frac{\partial}{\partial z}A_{z}\frac{\partial v}{\partial z} 
& = &
\begin{cases}
{\cal F}_v & \text{(surface)} \\
0 & \text{(interior)}
\end{cases}
\\
\frac{\partial \eta}{\partial t} + \nabla_{h}\cdot \vec{u}
&=&
0
\\
\frac{D\theta}{Dt} -
 \nabla_{h}\cdot K_{h}\nabla_{h}\theta
 - \frac{\partial}{\partial z}\Gamma(K_{z})\frac{\partial\theta}{\partial z} 
& = &
\begin{cases}
{\cal F}_\theta & \text{(surface)} \\
0 & \text{(interior)}
\end{cases}
\\
\frac{D s}{Dt} -
 \nabla_{h}\cdot K_{h}\nabla_{h}s
 - \frac{\partial}{\partial z}\Gamma(K_{z})\frac{\partial s}{\partial z} 
& = &
\begin{cases}
{\cal F}_s & \text{(surface)} \\
0 & \text{(interior)}
\end{cases}
\\
g\rho_{0} \eta + \int^{0}_{-z}\rho^{'} dz & = & p^{'}
\end{eqnarray}

\noindent where $u=\frac{Dx}{Dt}=r \cos(\phi)\frac{D \lambda}{Dt}$ and 
$v=\frac{Dy}{Dt}=r \frac{D \phi}{Dt}$ 
are the zonal and meridional components of the
flow vector, $\vec{u}$, on the sphere. As described in
MITgcm Numerical Solution Procedure \ref{chap:discretization}, the time 
evolution of potential temperature, $\theta$, equation is solved prognostically.
The total pressure, $p$, is diagnosed by summing pressure due to surface 
elevation $\eta$ and the hydrostatic pressure.
\\

\subsubsection{Numerical Stability Criteria}
\label{www:tutorials}

The Laplacian dissipation coefficient, $A_{h}$, is set to $5 \times 10^5 m s^{-1}$.
This value is chosen to yield a Munk layer width \cite{adcroft:95},
\begin{eqnarray}
\label{EQ:eg-hs-munk_layer}
M_{w} = \pi ( \frac { A_{h} }{ \beta } )^{\frac{1}{3}}
\end{eqnarray}

\noindent  of $\approx 600$km. This is greater than the model
resolution in low-latitudes, $\Delta x \approx 400{\rm km}$, ensuring that the frictional 
boundary layer is adequately resolved.
\\

\noindent The model is stepped forward with a 
time step $\delta t_{\theta}=30~{\rm hours}$ for thermodynamic variables and
$\delta t_{v}=40~{\rm minutes}$ for momentum terms. With this time step, the stability 
parameter to the horizontal Laplacian friction \cite{adcroft:95}
\begin{eqnarray}
\label{EQ:eg-hs-laplacian_stability}
S_{l} = 4 \frac{A_{h} \delta t_{v}}{{\Delta x}^2}
\end{eqnarray}

\noindent evaluates to 0.16 at a latitude of $\phi=80^{\circ}$, which is below the 
0.3 upper limit for stability. The zonal grid spacing $\Delta x$ is smallest at
$\phi=80^{\circ}$ where $\Delta x=r\cos(\phi)\Delta \phi\approx 77{\rm km}$.
\\

\noindent The vertical dissipation coefficient, $A_{z}$, is set to 
$1\times10^{-3} {\rm m}^2{\rm s}^{-1}$. The associated stability limit
\begin{eqnarray}
\label{EQ:eg-hs-laplacian_stability_z}
S_{l} = 4 \frac{A_{z} \delta t_{v}}{{\Delta z}^2}
\end{eqnarray}

\noindent evaluates to $0.015$ for the smallest model
level spacing ($\Delta z_{1}=50{\rm m}$) which is again well below
the upper stability limit.
\\

The values of the horizontal ($K_{h}$) and vertical ($K_{z}$) diffusion coefficients 
for both temperature and salinity are set to $1 \times 10^{3}~{\rm m}^{2}{\rm s}^{-1}$ 
and $3 \times 10^{-5}~{\rm m}^{2}{\rm s}^{-1}$ respectively. The stability limit 
related to $K_{h}$ will be at $\phi=80^{\circ}$ where $\Delta x \approx 77 {\rm km}$. 
Here the stability parameter 
\begin{eqnarray} 
\label{EQ:eg-hs-laplacian_stability_xtheta}
S_{l} = \frac{4 K_{h} \delta t_{\theta}}{{\Delta x}^2} 
\end{eqnarray}
evaluates to $0.07$, well below the stability limit of $S_{l} \approx 0.5$. The 
stability parameter related to $K_{z}$
\begin{eqnarray} 
\label{EQ:eg-hs-laplacian_stability_ztheta}
S_{l} = \frac{4 K_{z} \delta t_{\theta}}{{\Delta z}^2} 
\end{eqnarray}
evaluates to $0.005$ for $\min(\Delta z)=50{\rm m}$, well below the stability limit 
of $S_{l} \approx 0.5$.
\\

\noindent The numerical stability for inertial oscillations
\cite{adcroft:95} 

\begin{eqnarray}
\label{EQ:eg-hs-inertial_stability}
S_{i} = f^{2} {\delta t_v}^2
\end{eqnarray}

\noindent evaluates to $0.24$ for $f=2\omega\sin(80^{\circ})=1.43\times10^{-4}~{\rm s}^{-1}$, which is close to 
the $S_{i} < 1$ upper limit for stability.
\\

\noindent The advective CFL \cite{adcroft:95} for a extreme maximum 
horizontal flow
speed of $ | \vec{u} | = 2 ms^{-1}$

\begin{eqnarray}
\label{EQ:eg-hs-cfl_stability}
S_{a} = \frac{| \vec{u} | \delta t_{v}}{ \Delta x}
\end{eqnarray}

\noindent evaluates to $6 \times 10^{-2}$. This is well below the stability 
limit of 0.5.
\\

\noindent The stability parameter for internal gravity waves propagating
with a maximum speed of $c_{g}=10~{\rm ms}^{-1}$
\cite{adcroft:95}

\begin{eqnarray}
\label{EQ:eg-hs-gfl_stability}
S_{c} = \frac{c_{g} \delta t_{v}}{ \Delta x}
\end{eqnarray}

\noindent evaluates to $3 \times 10^{-1}$. This is close to the linear
stability limit of 0.5.
  
\subsection{Experiment Configuration}
\label{www:tutorials}
\label{SEC:eg-hs_examp_exp_config}

The model configuration for this experiment resides under the 
directory {\it verification/hs94.128x64x5}.  The experiment files 
\begin{itemize}
\item {\it input/data}
\item {\it input/data.pkg}
\item {\it input/eedata},
\item {\it input/windx.bin},
\item {\it input/windy.bin},
\item {\it input/salt.bin},
\item {\it input/theta.bin},
\item {\it input/SSS.bin},
\item {\it input/SST.bin},
\item {\it input/topog.bin},
\item {\it code/CPP\_EEOPTIONS.h}
\item {\it code/CPP\_OPTIONS.h},
\item {\it code/SIZE.h}. 
\end{itemize}
contain the code customizations and parameter settings for these
experiments. Below we describe the customizations
to these files associated with this experiment.

\subsubsection{File {\it input/data}}
\label{www:tutorials}

This file, reproduced completely below, specifies the main parameters 
for the experiment. The parameters that are significant for this configuration
are

\begin{itemize}

\item Lines 7-10 and 11-14 
\begin{verbatim} tRef= 16.0 , 15.2 , 14.5 , 13.9 , 13.3 ,  \end{verbatim} 
$\cdots$ \\
set reference values for potential
temperature and salinity at each model level in units of $^{\circ}$C and
${\rm ppt}$. The entries are ordered from surface to depth.
Density is calculated from anomalies at each level evaluated
with respect to the reference values set here.\\
\fbox{
\begin{minipage}{5.0in}
{\it S/R INI\_THETA}({\it ini\_theta.F})
\end{minipage}
}


\item Line 15, 
\begin{verbatim} viscAz=1.E-3, \end{verbatim}
this line sets the vertical Laplacian dissipation coefficient to
$1 \times 10^{-3} {\rm m^{2}s^{-1}}$. Boundary conditions
for this operator are specified later. This variable is copied into
model general vertical coordinate variable {\bf viscAr}.

\fbox{
\begin{minipage}{5.0in}
{\it S/R CALC\_DIFFUSIVITY}({\it calc\_diffusivity.F})
\end{minipage}
}

\item Line 16, 
\begin{verbatim}
viscAh=5.E5,
\end{verbatim} 
this line sets the horizontal Laplacian frictional dissipation coefficient to
$5 \times 10^{5} {\rm m^{2}s^{-1}}$. Boundary conditions
for this operator are specified later.

\item Lines 17,
\begin{verbatim}
no_slip_sides=.FALSE.
\end{verbatim}
this line selects a free-slip lateral boundary condition for
the horizontal Laplacian friction operator 
e.g. $\frac{\partial u}{\partial y}$=0 along boundaries in $y$ and
$\frac{\partial v}{\partial x}$=0 along boundaries in $x$.

\item Lines 9,
\begin{verbatim}
no_slip_bottom=.TRUE.
\end{verbatim}
this line selects a no-slip boundary condition for bottom
boundary condition in the vertical Laplacian friction operator 
e.g. $u=v=0$ at $z=-H$, where $H$ is the local depth of the domain.

\item Line 19,
\begin{verbatim}
diffKhT=1.E3,
\end{verbatim}
this line sets the horizontal diffusion coefficient for temperature
to $1000\,{\rm m^{2}s^{-1}}$. The boundary condition on this
operator is $\frac{\partial}{\partial x}=\frac{\partial}{\partial y}=0$ on
all boundaries.

\item Line 20,
\begin{verbatim}
diffKzT=3.E-5,
\end{verbatim}
this line sets the vertical diffusion coefficient for temperature
to $3 \times 10^{-5}\,{\rm m^{2}s^{-1}}$. The boundary 
condition on this operator is $\frac{\partial}{\partial z}=0$ at both
the upper and lower boundaries.

\item Line 21,
\begin{verbatim}
diffKhS=1.E3,
\end{verbatim}
this line sets the horizontal diffusion coefficient for salinity
to $1000\,{\rm m^{2}s^{-1}}$. The boundary condition on this
operator is $\frac{\partial}{\partial x}=\frac{\partial}{\partial y}=0$ on
all boundaries.

\item Line 22,
\begin{verbatim}
diffKzS=3.E-5,
\end{verbatim}
this line sets the vertical diffusion coefficient for salinity
to $3 \times 10^{-5}\,{\rm m^{2}s^{-1}}$. The boundary 
condition on this operator is $\frac{\partial}{\partial z}=0$ at both
the upper and lower boundaries.

\item Lines 23-26
\begin{verbatim}
beta=1.E-11,
\end{verbatim}
\vspace{-5mm}$\cdots$\\
These settings do not apply for this experiment.

\item Line 27,
\begin{verbatim}
gravity=9.81,
\end{verbatim}
Sets the gravitational acceleration coefficient to $9.81{\rm m}{\rm s}^{-1}$.\\
\fbox{
\begin{minipage}{5.0in}
{\it S/R CALC\_PHI\_HYD}~({\it calc\_phi\_hyd.F})\\
{\it S/R INI\_CG2D}~({\it ini\_cg2d.F})\\
{\it S/R INI\_CG3D}~({\it ini\_cg3d.F})\\
{\it S/R INI\_PARMS}~({\it ini\_parms.F})\\
{\it S/R SOLVE\_FOR\_PRESSURE}~({\it solve\_for\_pressure.F})
\end{minipage}
}


\item Line 28-29,
\begin{verbatim}
rigidLid=.FALSE., 
implicitFreeSurface=.TRUE., 
\end{verbatim}
Selects the barotropic pressure equation to be the implicit free surface
formulation.

\item Line 30,
\begin{verbatim}
eosType='POLY3',
\end{verbatim}
Selects the third order polynomial form of the equation of state.\\
\fbox{
\begin{minipage}{5.0in}
{\it S/R FIND\_RHO}~({\it find\_rho.F})\\
{\it S/R FIND\_ALPHA}~({\it find\_alpha.F})
\end{minipage}
}

\item Line 31,
\begin{verbatim}
readBinaryPrec=32,
\end{verbatim}
Sets format for reading binary input datasets holding model fields to
use 32-bit representation for floating-point numbers.\\
\fbox{
\begin{minipage}{5.0in}
{\it S/R READ\_WRITE\_FLD}~({\it read\_write\_fld.F})\\
{\it S/R READ\_WRITE\_REC}~({\it read\_write\_rec.F})
\end{minipage}
}

\item Line 36,
\begin{verbatim}
cg2dMaxIters=1000,
\end{verbatim}
Sets maximum number of iterations the two-dimensional, conjugate
gradient solver will use, {\bf irrespective of convergence 
criteria being met}.\\
\fbox{
\begin{minipage}{5.0in}
{\it S/R CG2D}~({\it cg2d.F})
\end{minipage}
}

\item Line 37,
\begin{verbatim}
cg2dTargetResidual=1.E-13,
\end{verbatim}
Sets the tolerance which the two-dimensional, conjugate
gradient solver will use to test for convergence in equation 
\ref{EQ:eg-hs-congrad_2d_resid} to $1 \times 10^{-13}$.
Solver will iterate until 
tolerance falls below this value or until the maximum number of
solver iterations is reached.\\
\fbox{
\begin{minipage}{5.0in}
{\it S/R CG2D}~({\it cg2d.F})
\end{minipage}
}

\item Line 42,
\begin{verbatim}
startTime=0,
\end{verbatim}
Sets the starting time for the model internal time counter.
When set to non-zero this option implicitly requests a 
checkpoint file be read for initial state.
By default the checkpoint file is named according to
the integer number of time steps in the {\bf startTime} value.
The internal time counter works in seconds.

\item Line 43,
\begin{verbatim}
endTime=2808000.,
\end{verbatim}
Sets the time (in seconds) at which this simulation will terminate.
At the end of a simulation a checkpoint file is automatically
written so that a numerical experiment can consist of multiple
stages.

\item Line 44,
\begin{verbatim}
#endTime=62208000000,
\end{verbatim}
A commented out setting for endTime for a 2000 year simulation.

\item Line 45,
\begin{verbatim}
deltaTmom=2400.0,
\end{verbatim}
Sets the timestep $\delta t_{v}$ used in the momentum equations to
$20~{\rm mins}$.
See section \ref{SEC:mom_time_stepping}.

\fbox{
\begin{minipage}{5.0in}
{\it S/R TIMESTEP}({\it timestep.F})
\end{minipage}
}

\item Line 46,
\begin{verbatim}
tauCD=321428.,
\end{verbatim}
Sets the D-grid to C-grid coupling time scale $\tau_{CD}$ used in the momentum equations.
See section \ref{SEC:cd_scheme}.

\fbox{
\begin{minipage}{5.0in}
{\it S/R INI\_PARMS}({\it ini\_parms.F})\\
{\it S/R CALC\_MOM\_RHS}({\it calc\_mom\_rhs.F})
\end{minipage}
}

\item Line 47,
\begin{verbatim}
deltaTtracer=108000.,
\end{verbatim}
Sets the default timestep, $\delta t_{\theta}$, for tracer equations to
$30~{\rm hours}$.
See section \ref{SEC:tracer_time_stepping}.

\fbox{
\begin{minipage}{5.0in}
{\it S/R TIMESTEP\_TRACER}({\it timestep\_tracer.F})
\end{minipage}
}

\item Line 47,
\begin{verbatim}
bathyFile='topog.box'
\end{verbatim}
This line specifies the name of the file from which the domain
bathymetry is read. This file is a two-dimensional ($x,y$) map of
depths. This file is assumed to contain 64-bit binary numbers 
giving the depth of the model at each grid cell, ordered with the x 
coordinate varying fastest. The points are ordered from low coordinate
to high coordinate for both axes. The units and orientation of the
depths in this file are the same as used in the MITgcm code. In this
experiment, a depth of $0m$ indicates a solid wall and a depth
of $-2000m$ indicates open ocean. The matlab program
{\it input/gendata.m} shows an example of how to generate a
bathymetry file.


\item Line 50,
\begin{verbatim}
zonalWindFile='windx.sin_y'
\end{verbatim}
This line specifies the name of the file from which the x-direction
surface wind stress is read. This file is also a two-dimensional
($x,y$) map and is enumerated and formatted in the same manner as the 
bathymetry file. The matlab program {\it input/gendata.m} includes example 
code to generate a valid 
{\bf zonalWindFile} 
file.  

\end{itemize}

\noindent other lines in the file {\it input/data} are standard values
that are described in the MITgcm Getting Started and MITgcm Parameters
notes.

\begin{small}
% $Header: /u/gcmpack/manual/s_examples/baroclinic_gyre/input/data.tex,v 1.1.1.1 2001/08/08 16:15:46 adcroft Exp $
% $Name:  $

\begin{verbatim}
     1	# Model parameters
     2	# Continuous equation parameters
     3	 &PARM01
     4	 tRef=20.,10.,8.,6.,
     5	 sRef=10.,10.,10.,10.,
     6	 viscAz=1.E-2,
     7	 viscAh=4.E2,
     8	 no_slip_sides=.FALSE.,
     9	 no_slip_bottom=.TRUE.,
    10	 diffKhT=4.E2,
    11	 diffKzT=1.E-2,
    12	 beta=1.E-11,
    13	 tAlpha=2.E-4,
    14	 sBeta =0.,
    15	 gravity=9.81,
    16	 rigidLid=.FALSE.,
    17	 implicitFreeSurface=.TRUE.,
    18	 eosType='LINEAR',
    19	 readBinaryPrec=64,
    20	 &
    21	# Elliptic solver parameters
    22	 &PARM02
    23	 cg2dMaxIters=1000,
    24	 cg2dTargetResidual=1.E-13,
    25	 &
    26	# Time stepping parameters
    27	 &PARM03
    28	 startTime=0.,
    29	 endTime=12000., 
    30	 deltaTmom=1200.0,
    31	 deltaTtracer=1200.0,
    32	 abEps=0.1,
    33	 pChkptFreq=17000.0,
    34	 chkptFreq=0.0,
    35	 dumpFreq=2592000.0,
    36	 &
    37	# Gridding parameters
    38	 &PARM04
    39	 usingCartesianGrid=.FALSE.,
    40	 usingSphericalPolarGrid=.TRUE.,
    41	 phiMin=0.,
    42	 delX=60*1.,
    43	 delY=60*1.,
    44	 delZ=500.,500.,500.,500.,
    45	 &
    46	 &PARM05
    47	 bathyFile='topog.box',
    48	 hydrogThetaFile=,
    49	 hydrogSaltFile=,
    50	 zonalWindFile='windx.sin_y',
    51	 meridWindFile=,
    52	 &
\end{verbatim}

\end{small}

\subsubsection{File {\it input/data.pkg}}
\label{www:tutorials}

This file uses standard default values and does not contain
customisations for this experiment.

\subsubsection{File {\it input/eedata}}
\label{www:tutorials}

This file uses standard default values and does not contain
customisations for this experiment.

\subsubsection{File {\it input/windx.sin\_y}}
\label{www:tutorials}

The {\it input/windx.sin\_y} file specifies a two-dimensional ($x,y$) 
map of wind stress ,$\tau_{x}$, values. The units used are $Nm^{-2}$.
Although $\tau_{x}$ is only a function of $y$n in this experiment
this file must still define a complete two-dimensional map in order
to be compatible with the standard code for loading forcing fields 
in MITgcm. The included matlab program {\it input/gendata.m} gives a complete
code for creating the {\it input/windx.sin\_y} file.

\subsubsection{File {\it input/topog.box}}
\label{www:tutorials}


The {\it input/topog.box} file specifies a two-dimensional ($x,y$) 
map of depth values. For this experiment values are either
$0m$ or $-2000\,{\rm m}$, corresponding respectively to a wall or to deep
ocean. The file contains a raw binary stream of data that is enumerated
in the same way as standard MITgcm two-dimensional, horizontal arrays.
The included matlab program {\it input/gendata.m} gives a complete
code for creating the {\it input/topog.box} file.

\subsubsection{File {\it code/SIZE.h}}
\label{www:tutorials}

Two lines are customized in this file for the current experiment

\begin{itemize}

\item Line 39, 
\begin{verbatim} sNx=60, \end{verbatim} this line sets
the lateral domain extent in grid points for the
axis aligned with the x-coordinate.

\item Line 40, 
\begin{verbatim} sNy=60, \end{verbatim} this line sets
the lateral domain extent in grid points for the
axis aligned with the y-coordinate.

\item Line 49, 
\begin{verbatim} Nr=4,   \end{verbatim} this line sets
the vertical domain extent in grid points.

\end{itemize}

\begin{small}
\begin{verbatim}
     1	C     /==========================================================\
     2	C     | SIZE.h Declare size of underlying computational grid.    |
     3	C     |==========================================================|
     4	C     | The design here support a three-dimensional model grid   |
     5	C     | with indices I,J and K. The three-dimensional domain     |
     6	C     | is comprised of nPx*nSx blocks of size sNx along one axis|
     7	C     | nPy*nSy blocks of size sNy along another axis and one    |
     8	C     | block of size Nz along the final axis.                   |
     9	C     | Blocks have overlap regions of size OLx and OLy along the|
    10	C     | dimensions that are subdivided.                          |
    11  C     \==========================================================/
    12  C     Voodoo numbers controlling data layout.
    13  C     sNx - No. X points in sub-grid.
    14  C     sNy - No. Y points in sub-grid.
    15  C     OLx - Overlap extent in X.
    16  C     OLy - Overlat extent in Y.
    17  C     nSx - No. sub-grids in X.
    18  C     nSy - No. sub-grids in Y.
    19  C     nPx - No. of processes to use in X.
    20  C     nPy - No. of processes to use in Y.
    21  C     Nx  - No. points in X for the total domain.
    22  C     Ny  - No. points in Y for the total domain.
    23  C     Nr  - No. points in Z for full process domain.
    24        INTEGER sNx
    25        INTEGER sNy
    26        INTEGER OLx
    27        INTEGER OLy
    28        INTEGER nSx
    29        INTEGER nSy
    30        INTEGER nPx
    31	      INTEGER nPy
    32	      INTEGER Nx
    33	      INTEGER Ny
    34	      INTEGER Nr
    35	      PARAMETER (
    36	     &           sNx =  64,
    37	     &           sNy =  64,
    38	     &           OLx =   3,
    39	     &           OLy =   3,
    40	     &           nSx =   1,
    41	     &           nSy =   1,
    42	     &           nPx =   1,
    43	     &           nPy =   1,
    44	     &           Nx  = sNx*nSx*nPx,
    45	     &           Ny  = sNy*nSy*nPy,
    46	     &           Nr  =  20)

    47	C     MAX_OLX  - Set to the maximum overlap region size of any array
    48	C     MAX_OLY    that will be exchanged. Controls the sizing of exch
    49	C                routine buufers.
    50	      INTEGER MAX_OLX
    51	      INTEGER MAX_OLY
    52	      PARAMETER ( MAX_OLX = OLx,
    53	     &            MAX_OLY = OLy )

\end{verbatim}
\end{small}

\subsubsection{File {\it code/CPP\_OPTIONS.h}}
\label{www:tutorials}

This file uses standard default values and does not contain
customisations for this experiment.


\subsubsection{File {\it code/CPP\_EEOPTIONS.h}}
\label{www:tutorials}

This file uses standard default values and does not contain
customisations for this experiment.

\subsubsection{Other Files }
\label{www:tutorials}

Other files relevant to this experiment are
\begin{itemize}
\item {\it model/src/ini\_cori.F}. This file initializes the model
coriolis variables {\bf fCorU}.
\item {\it model/src/ini\_spherical\_polar\_grid.F}
\item {\it model/src/ini\_parms.F},
\item {\it input/windx.sin\_y},
\end{itemize}
contain the code customisations and parameter settings for this 
experiments. Below we describe the customisations
to these files associated with this experiment.
