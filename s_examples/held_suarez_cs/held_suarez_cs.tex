% $Header: /u/gcmpack/manual/s_examples/held_suarez_cs/held_suarez_cs.tex,v 1.9 2010/08/27 13:25:32 jmc Exp $
% $Name:  $

\section[Held-Suarez Atmosphere MITgcm Example]{Held-Suarez atmospheric simulation
on cube-sphere grid with 32 square cube faces.}
\label{www:tutorials}
\label{sect:eg-hs}
\begin{rawhtml}
<!-- CMIREDIR:eg-hs: -->
\end{rawhtml}
\begin{center}
(in directory: {\it verification/tutorial\_held\_suarez\_cs/})
\end{center}

\bodytext{bgcolor="#FFFFFFFF"}

%\begin{center} 
%{\Large \bf Using MITgcm to Simulate Global Climatological Ocean Circulation
%At Four Degree Resolution with Asynchronous Time Stepping}
%
%\vspace*{4mm}
%
%\vspace*{3mm}
%{\large May 2001}
%\end{center}

This example illustrates the use of the MITgcm as an Atmospheric GCM,
using simple \cite{held-suar:94} forcing 
to simulate Atmospheric Dynamics on global scale.
The set-up use the rescaled pressure coordinate ($p^*$)\cite[]{adcroft:04a}
in the vertical direction, with 20 equaly-spaced levels, and
the conformal cube-sphere grid (C32) \cite[]{adcroft:04b}.
The files for this experiment can be found in the verification directory
under tutorial\_held\_suarez\_cs.

\subsection{Overview}
\label{www:tutorials}

This example demonstrates using the MITgcm to simulate
the planetary atmospheric circulation, with flat orography
and simplified forcing.
In particular, only dry air processes are considered and 
radiation effects are represented by a simple newtownien cooling,
Thus this example does not rely on any particular atmospheric 
physics package.
This kind of simplified atmospheric simulation has been widely 
used in GFD-type experiments and in intercomparison projects of 
AGCM dynamical cores \cite[]{held-suar:94}.

The horizontal grid is obtain from the projection of a uniform gridded cube 
to the sphere. Each of the 6 faces has the same resolution, with 
$32 \times 32$ grid points. The equator line coincide with a grid line
and crosses, right in the midle, 4 of the 6 faces, leaving 2 faces 
for the Northern and Southern polar regions.
This curvilinear grid requires the use of the 2nd generation exchange
topology ({\it pkg/exch2}) to connect tile and face edges,
but without any limitation on the number of processors.

The use of the $p^*$ coordinate with 20 equally spaced levels
($20 \times 50\,{\rm mb}$, from $p^*=1000,{\rm mb}$ to $0$ at the 
top of the atmosphere) follows the choice of \cite{held-suar:94}. 
Note that without topography, the $p^*$ coordinate and the normalized 
pressure coordinate ($\sigma_p$) coincide exactly.
No viscosity and zero diffusion are used here, but
a $8^th$ order \cite{Shapiro_70} filter is applied to both momentum and
potential temperature, to remove selectively grid scale noise.
Apart from the horizontal grid, this experiment is made very similar to
the grid-point model case used in \cite{held-suar:94} study.

At this resolution, the configuration can be integrated forward 
for many years on a single processor desktop computer.
\\

\subsection{Forcing}
\label{www:tutorials}

The model is forced by relaxation to a radiative equilibrium temperature from
\cite{held-suar:94}.
A linear frictional drag (Rayleigh damping) is applied in the lower 
part of the atmosphere and account from surface friction and momentum
dissipation in the boundary layer.
Altogether, this yields the following forcing 
\cite[from][]{held-suar:94} that is applied to the fluid:

\begin{eqnarray}
\label{EQ:eg-hs-global_forcing}
\label{EQ:eg-hs-global_forcing_fv}
\vec{{\cal F}_\mathbf{v}} & = & -k_\mathbf{v}(p)\vec{\mathbf{v}}_h
\\
\label{EQ:eg-hs-global_forcing_ft}
{\cal F}_{\theta} & = & -k_{\theta}(\varphi,p)[\theta-\theta_{eq}(\varphi,p)]
\end{eqnarray}

\noindent where $\vec{\cal F}_\mathbf{v}$, ${\cal F}_{\theta}$,
are the forcing terms in the zonal and meridional
momentum and in the potential temperature equations respectively.
The term $k_\mathbf{v}$ in equation (\ref{EQ:eg-hs-global_forcing_fv}) applies a
Rayleigh damping that is active within the planetary boundary layer. 
It is defined so as to decay as pressure decreases according to
\begin{eqnarray*}
\label{EQ:eg-hs-define_kv}
k_\mathbf{v} & = & k_{f}~\max[0,~(p^*/P^{0}_{s}-\sigma_{b})/(1-\sigma_{b})]
\\
\sigma_{b} & = & 0.7 ~~{\rm and}~~
k_{f}  =  1/86400 ~{\rm s}^{-1}
\end{eqnarray*}

where $p^*$ is the pressure level of the cell center 
and $P^{0}_{s}$ is the pressure at the base of the atmospheric column,
which is constant and uniform here ($= 10^5 {\rm Pa}$), in the absence 
of topography.

The Equilibrium temperature $\theta_{eq}$ and relaxation time scale $k_{\theta}$ 
are set to:
\begin{eqnarray}
\label{EQ:eg-hs-define_Teq}
\theta_{eq}(\varphi,p^*) & = & \max \{ 200.K (P^{0}_{s}/p^*)^\kappa,\\
\nonumber
& & \hspace{8mm} 315.K - \Delta T_y~\sin^2(\varphi) 
  - \Delta \theta_z \cos^2(\varphi) \log(p^*/P^{0}_s) \}
\\
\label{EQ:eg-hs-define_kT}
k_{\theta}(\varphi,p^*) & = &
k_a + (k_s -k_a)~\cos^4(\varphi)~\max[0,(p^*/P^{0}_{s}-\sigma_{b})/(1-\sigma_{b})]
\end{eqnarray}
with:
\begin{eqnarray*}
 \Delta T_y = 60.K & k_a = 1/(40 \cdot 86400) ~{\rm s}^{-1}\\
\Delta \theta_z = 10.K & k_s = 1/(4 \cdot 86400) ~{\rm s}^{-1}
\end{eqnarray*}

Initial conditions correspond to a resting state with horizontally uniform 
stratified fluid. The initial temperature profile is simply the 
horizontally average of the radiative equilibrium temperature.

\subsection{Set-up description}
%\subsection{Discrete Numerical Configuration}
\label{www:tutorials}

The model is configured in hydrostatic form, using non-boussinesq
$p^*$ coordinate.
The vertical resolution is uniform, $\Delta p^* = 50.10^2 Pa$,
with 20 levels, from $p^*=10^5 Pa$ to $0$ at the top.
The domain is discretised using C32 cube-sphere grid \cite[]{adcroft:04b}
that cover the whole sphere with a relatively uniform grid-spacing.
The resolution at the equator or along the Greenwitch meridian
is similar to the $128 \times 64$ equaly spaced longitude-latitude grid,
but requires $25\%$ less grid points.
Grid spacing and grid-point location are not computed by the model but
read from files.

The vector-invariant form of the momentum equation (see section
\ref{sect:vect-inv_momentum_equations}) is used so that no explicit 
metrics are necessary.

Applying the vector-invariant discretization to the
atmospheric equations \ref{eq:atmos-prime}, and adding the 
forcing term 
(\ref{EQ:eg-hs-global_forcing_fv}, \ref{EQ:eg-hs-global_forcing_ft})
on the right-hand-side,
leads to the set of equations that are solved in this configuration:

%the The set of equations solved here is der
%Wind-stress forcing is added to the momentum equations for both
%the zonal flow, $u$ and the meridional flow $v$, according to equations 
%(\ref{EQ:eg-hs-global_forcing_fv}) and (\ref{EQ:eg-hs-global_forcing_fv}).
%Thermodynamic forcing inputs are added to the equations for
%potential temperature, $\theta$, and salinity, $S$, according to equations 
%(\ref{EQ:eg-hs-global_forcing_ft}) and (\ref{EQ:eg-hs-global_forcing_fs}).

\begin{eqnarray}
\label{EQ:eg-hs-model_equations}
\frac{\partial \vec{\mathbf{v}}_h}{\partial t}
+(f + \zeta)\hat{\mathbf{k}} \times \vec{\mathbf{v}}_h
%+\mathbf{\nabla }_{p} ({\rm KE})
+\mathbf{\nabla }_{p} (\mbox{\sc ke})
+ \omega \frac{\partial \vec{\mathbf{v}}_h }{\partial p}
+\mathbf{\nabla }_p \Phi ^{\prime } 
&=& 
% \vec{{\cal F}_v} = 
-k_\mathbf{v}\vec{\mathbf{v}}_h
\\
\frac{\partial \Phi ^{\prime }}{\partial p} 
+\frac{\partial \Pi }{\partial p}\theta ^{\prime } &=&0
\\
\mathbf{\nabla }_{p}\cdot \vec{\mathbf{v}}_h+\frac{\partial \omega }{
\partial p} &=&0
\\
\frac{\partial \theta }{\partial t} 
+ \mathbf{\nabla }_{p}\cdot (\theta \vec{\mathbf{v}}_h)
+ \frac{\partial (\theta \omega)}{\partial p}
%= \frac{\mathcal{Q}}{\Pi } 
&=& -k_{\theta}[\theta-\theta_{eq}]
\end{eqnarray}

%\begin{equation}
%\partial_t \vec{v} + ( 2\vec{\Omega} + \vec{\zeta}) \wedge \vec{v}
%- b \hat{r}
%+ \vec{\nabla} B = \vec{\nabla} \cdot \vec{\bf \tau}
%\end{equation}
%{\cal F}_{\theta} & = & -k_{\theta}(\varphi,p)[\theta-\theta_{eq}(\varphi,p)]

\noindent where $\vec{\mathbf{v}}_h$ and $\omega = \frac{Dp}{Dt}$ 
are the horizontal velocity vector and the vertical velocity in pressure coordinate,
$\zeta$ is the relative vorticity and $f$ the Coriolis parameter, 
$\hat{\mathbf{k}}$ is the vertical unity vector, 
{\sc ke} is the kinetic energy, $\Phi$ is the geopotential
and $\Pi$ the Exner function 
($\Pi = C_p (p/p_c)^\kappa ~{\rm with}~ p_c = 10^5 Pa$).
Variables marked with ' corresponds to anomaly from 
the resting, uniformly stratified state.

As described in MITgcm Numerical Solution Procedure \ref{chap:discretization}, 
the continuity equation is integrated vertically, to give a prognostic
equation for the surface pressure $p_s$:
\begin{equation}
\frac{\partial p_s}{\partial t} + \nabla_{h}\cdot \int_{0}^{p_s} \vec{\mathbf{v}}_h dp
= 0
\end{equation}

The implicit free surface form of the pressure equation described in 
\cite{marshall:97a} is employed to solve for $p_s$; 
Integrating vertically the hydrostatic balance 
gives the geopotential $\Phi'$ and allow to step forward the momentum equation
\ref{EQ:eg-hs-model_equations}.
The potential temperature, $\theta$, is stepped forward using the 
new velocity field ({\it staggered time-step}, section 
\ref{sect:adams-bashforth-staggered}).
\\

\subsubsection{Numerical Stability Criteria}
\label{www:tutorials}

\noindent The numerical stability for inertial oscillations
\cite{adcroft:95} 

\begin{eqnarray}
\label{EQ:eg-hs-inertial_stability}
S_{i} = f^{2} {\Delta t}^2
\end{eqnarray}

\noindent evaluates to $4.\times10^{-3}$ at the poles, 
for $f=2\Omega\sin(\pi / 2) =1.45\times10^{-4}~{\rm s}^{-1}$, 
which is well below the $S_{i} < 1$ upper limit for stability.
\\

\noindent The advective CFL \cite{adcroft:95} 
for a extreme maximum horizontal flow speed of $ | \vec{u} | = 90. {\rm m/s}$~ 
and the smallest horizontal grid spacing $ \Delta x = 1.1\times10^5 {\rm m}$~:

\begin{eqnarray}
\label{EQ:eg-hs-cfl_stability}
S_{a} = \frac{| \vec{u} | \Delta t}{ \Delta x}
\end{eqnarray}

\noindent evaluates to $0.37$, which is close to the stability 
limit of 0.5.
\\

\noindent The stability parameter for internal gravity waves propagating
with a maximum speed of $c_{g}=100~{\rm m/s}$
\cite{adcroft:95}

\begin{eqnarray}
\label{EQ:eg-hs-gfl_stability}
S_{c} = \frac{c_{g} \Delta t}{ \Delta x}
\end{eqnarray}

\noindent evaluates to $4 \times 10^{-1}$. This is close to the linear
stability limit of 0.5.
  
\subsection{Experiment Configuration}
\label{www:tutorials}
\label{SEC:eg-hs_examp_exp_config}

The model configuration for this experiment resides under the 
directory {\it verification/tutorial\_held\_suarez\_cs}.  The experiment files 
\begin{itemize}
\item {\it input/data}
\item {\it input/data.pkg}
\item {\it input/eedata},
\item {\it input/data.shap},
\item {\it code/packages.conf},
\item {\it code/CPP\_OPTIONS.h},
\item {\it code/SIZE.h},
\item {\it code/DIAGNOSTICS\_SIZE.h},
\item {\it code/external\_forcing.F},
\end{itemize}
contain the code customizations and parameter settings for these
experiments. Below we describe the customizations
to these files associated with this experiment.

\subsubsection{File {\it input/data}}
\label{www:tutorials}

\input{s_examples/held_suarez_cs/inp_data}

\subsubsection{File {\it input/data.pkg}}
\label{www:tutorials}

\input{s_examples/held_suarez_cs/inp_data.pkg}

\subsubsection{File {\it input/eedata}}
\label{www:tutorials}

This file uses standard default values except line 6:
\begin{verbatim}
 useCubedSphereExchange=.TRUE.,
\end{verbatim}
This line selects the cubed-sphere specific exchanges to
to connect tiles and faces edges.

\subsubsection{File {\it input/data.shap}}
\label{www:tutorials}

\input{s_examples/held_suarez_cs/inp_data.shap}

\subsubsection{File {\it code/SIZE.h}}
\label{www:tutorials}

Four lines are customized in this file for the current experiment

\begin{itemize}

\item Line 39, 
\begin{verbatim} sNx=32, \end{verbatim}
sets the lateral domain extent in grid points along the x-direction,
for 1 face.

\item Line 40,
\begin{verbatim} sNy=32, \end{verbatim} 
sets the lateral domain extent in grid points along the y-direction,
for 1 face.

\item Line 43,
\begin{verbatim} nSx=6, \end{verbatim} 
sets the number of tiles in the x-direction, for the model domain
decomposition. In this simple case (one processor and 1 tile per 
face), this number corresponds to the total number of faces.

\item Line 49, 
\begin{verbatim} Nr=20,   \end{verbatim} 
sets the vertical domain extent in grid points.

\end{itemize}

%\begin{small}
%\begin{verbatim}
     1	C     /==========================================================\
     2	C     | SIZE.h Declare size of underlying computational grid.    |
     3	C     |==========================================================|
     4	C     | The design here support a three-dimensional model grid   |
     5	C     | with indices I,J and K. The three-dimensional domain     |
     6	C     | is comprised of nPx*nSx blocks of size sNx along one axis|
     7	C     | nPy*nSy blocks of size sNy along another axis and one    |
     8	C     | block of size Nz along the final axis.                   |
     9	C     | Blocks have overlap regions of size OLx and OLy along the|
    10	C     | dimensions that are subdivided.                          |
    11  C     \==========================================================/
    12  C     Voodoo numbers controlling data layout.
    13  C     sNx - No. X points in sub-grid.
    14  C     sNy - No. Y points in sub-grid.
    15  C     OLx - Overlap extent in X.
    16  C     OLy - Overlat extent in Y.
    17  C     nSx - No. sub-grids in X.
    18  C     nSy - No. sub-grids in Y.
    19  C     nPx - No. of processes to use in X.
    20  C     nPy - No. of processes to use in Y.
    21  C     Nx  - No. points in X for the total domain.
    22  C     Ny  - No. points in Y for the total domain.
    23  C     Nr  - No. points in Z for full process domain.
    24        INTEGER sNx
    25        INTEGER sNy
    26        INTEGER OLx
    27        INTEGER OLy
    28        INTEGER nSx
    29        INTEGER nSy
    30        INTEGER nPx
    31	      INTEGER nPy
    32	      INTEGER Nx
    33	      INTEGER Ny
    34	      INTEGER Nr
    35	      PARAMETER (
    36	     &           sNx =  64,
    37	     &           sNy =  64,
    38	     &           OLx =   3,
    39	     &           OLy =   3,
    40	     &           nSx =   1,
    41	     &           nSy =   1,
    42	     &           nPx =   1,
    43	     &           nPy =   1,
    44	     &           Nx  = sNx*nSx*nPx,
    45	     &           Ny  = sNy*nSy*nPy,
    46	     &           Nr  =  20)

    47	C     MAX_OLX  - Set to the maximum overlap region size of any array
    48	C     MAX_OLY    that will be exchanged. Controls the sizing of exch
    49	C                routine buufers.
    50	      INTEGER MAX_OLX
    51	      INTEGER MAX_OLY
    52	      PARAMETER ( MAX_OLX = OLx,
    53	     &            MAX_OLY = OLy )

\end{verbatim}
%\end{small}

\subsubsection{File {\it code/packages.conf}}
\label{www:tutorials}

\input{s_examples/held_suarez_cs/cod_packages.conf}

\subsubsection{File {\it code/CPP\_OPTIONS.h}}
\label{www:tutorials}

This file uses standard default except for Line 40\\
({\it diff CPP\_OPTIONS.h ../../../model/inc}):
\begin{verbatim}
#define NONLIN_FRSURF 
\end{verbatim}
This line allow to use the non-linear free-surface part of the code,
which is required for the $p^*$ coordinate formulation.

\subsubsection{Other Files }
\label{www:tutorials}

Other files relevant to this experiment are
\begin{itemize}
\item {\it code/external\_forcing.F}
\item {\it input/grid\_cs32.face00[n].bin}, with $n=1,2,3,4,5,6$
\end{itemize}
contain the code customisations and binary input files for this 
experiments. Below we describe the customisations
to these files associated with this experiment.\\

The file {\it code/external\_forcing.F} contains 4 subroutines
that calculate the forcing terms (Right-Hand side term) in the
momentum equation (\ref{EQ:eg-hs-global_forcing_fv}, 
{\it S/R EXTERNAL\_FORCING\_U} and {\it EXTERNAL\_FORCING\_V})
and in the potential temperature equation 
(\ref{EQ:eg-hs-global_forcing_ft}, {\it S/R  EXTERNAL\_FORCING\_T}). 
The water-vapour forcing subroutine ({\it S/R EXTERNAL\_FORCING\_S})
is left empty for this experiment.\\

The grid-files {\it input/grid\_cs32.face00[n].bin}, with $n=1,2,3,4,5,6$,
are binary files (direct-access, big-endian 64.bits real) that 
contains all the cubed-sphere grid lengths, areas and grid-point
positions, with one file per face.
Each file contains 18 2-D arrays (dimension $33 \times 33$) that correspond
to the model variables:
{\it 
XC YC DXF DYF RA XG YG DXV DYU RAZ DXC DYC RAW RAS DXG DYG AngleCS AngleSN 
}
(see {\it GRID.h} file)

