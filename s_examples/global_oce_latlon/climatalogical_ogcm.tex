% $Header: /u/gcmpack/manual/s_examples/global_oce_latlon/climatalogical_ogcm.tex,v 1.21 2011/04/21 21:27:16 jmc Exp $
% $Name:  $

\section[Global Ocean MITgcm Example]{Global Ocean Simulation at $4^\circ$ Resolution}
%\label{www:tutorials}
\label{sec:eg-global}
\begin{rawhtml}
<!-- CMIREDIR:eg-global: -->
\end{rawhtml}
\begin{center}
(in directory: {\it verification/tutorial\_global\_oce\_latlon/})
\end{center}

\bodytext{bgcolor="#FFFFFFFF"}

\noindent {\bf WARNING: the description of this experiment is not up-to-date.
 In particular, most of the parameters description corresponds to an older
 version of {\it verification/exp2} instead of the current tutorial}\\

%\begin{center} 
%{\Large \bf Using MITgcm to Simulate Global Climatological Ocean Circulation
%At Four Degree Resolution with Asynchronous Time Stepping}
%
%\vspace*{4mm}
%
%\vspace*{3mm}
%{\large May 2001}
%\end{center}


This example experiment demonstrates using the MITgcm to simulate
the planetary ocean circulation. The simulation is configured
with realistic geography and bathymetry on a
$4^{\circ} \times 4^{\circ}$ spherical polar grid.
The files for this experiment are in the verification directory
under tutorial\_global\_oce\_latlon.
Twenty levels are used in the vertical, ranging in thickness
from $50\,{\rm m}$ at the surface to $815\,{\rm m}$ at depth,
giving a maximum model depth of $6\,{\rm km}$.
At this resolution, the configuration
can be integrated forward for thousands of years on a single 
processor desktop computer.
\\
\subsection{Overview}
%\label{www:tutorials}

The model is forced with climatological wind stress data and surface
flux data from DaSilva \cite{DaSilva94}. Climatological data
from Levitus \cite{Levitus94} is used to initialize the model hydrography.
Levitus seasonal climatology data is also used throughout the calculation
to provide additional air-sea fluxes.
These fluxes are combined with the DaSilva climatological estimates of
surface heat flux and fresh water, resulting in a mixed boundary
condition of the style described in Haney \cite{Haney}.
Altogether, this yields the following forcing applied
in the model surface layer.

\begin{eqnarray}
\label{eq:eg-global-global_forcing}
\label{eq:eg-global-global_forcing_fu}
{\cal F}_{u} & = & \frac{\tau_{x}}{\rho_{0} \Delta z_{s}}
\\
\label{eq:eg-global-global_forcing_fv}
{\cal F}_{v} & = & \frac{\tau_{y}}{\rho_{0} \Delta z_{s}}
\\
\label{eq:eg-global-global_forcing_ft}
{\cal F}_{\theta} & = & - \lambda_{\theta} ( \theta - \theta^{\ast} ) 
 - \frac{1}{C_{p} \rho_{0} \Delta z_{s}}{\cal Q}
\\
\label{eq:eg-global-global_forcing_fs}
{\cal F}_{s} & = & - \lambda_{s} ( S - S^{\ast} ) 
 + \frac{S_{0}}{\Delta z_{s}}({\cal E} - {\cal P} - {\cal R})
\end{eqnarray}

\noindent where ${\cal F}_{u}$, ${\cal F}_{v}$, ${\cal F}_{\theta}$,
${\cal F}_{s}$ are the forcing terms in the zonal and meridional
momentum and in the potential temperature and salinity
equations respectively.
The term $\Delta z_{s}$ represents the top ocean layer thickness in
meters.
It is used in conjunction with a reference density, $\rho_{0}$
(here set to $999.8\,{\rm kg\,m^{-3}}$), a
reference salinity, $S_{0}$ (here set to 35~ppt),
and a specific heat capacity, $C_{p}$ (here set to
$4000~{\rm J}~^{\circ}{\rm C}^{-1}~{\rm kg}^{-1}$), to convert
input dataset values into time tendencies of
potential temperature (with units of $^{\circ}{\rm C}~{\rm s}^{-1}$),
salinity (with units ${\rm ppt}~s^{-1}$) and
velocity (with units ${\rm m}~{\rm s}^{-2}$).
The externally supplied forcing fields used in this
experiment are $\tau_{x}$, $\tau_{y}$, $\theta^{\ast}$, $S^{\ast}$,
$\cal{Q}$ and $\cal{E}-\cal{P}-\cal{R}$. The wind stress fields ($\tau_x$, $\tau_y$)
have units of ${\rm N}~{\rm m}^{-2}$. The temperature forcing fields
($\theta^{\ast}$ and $Q$) have units of $^{\circ}{\rm C}$ and ${\rm W}~{\rm m}^{-2}$
respectively. The salinity forcing fields ($S^{\ast}$ and 
$\cal{E}-\cal{P}-\cal{R}$) have units of ${\rm ppt}$ and ${\rm m}~{\rm s}^{-1}$
respectively. The source files and procedures for ingesting this data into the
simulation are described in the experiment configuration discussion in section
\ref{sec:eg-global-clim_ocn_examp_exp_config}.


\subsection{Discrete Numerical Configuration}
%\label{www:tutorials}


 The model is configured in hydrostatic form.  The domain is discretised with 
a uniform grid spacing in latitude and longitude on the sphere
 $\Delta \phi=\Delta \lambda=4^{\circ}$, so 
that there are ninety grid cells in the zonal and forty in the 
meridional direction. The internal model coordinate variables
$x$ and $y$ are initialized according to
\begin{eqnarray}
x=r\cos(\phi),~\Delta x & = &r\cos(\Delta \phi) \\
y=r\lambda,~\Delta y &= &r\Delta \lambda 
\end{eqnarray}

Arctic polar regions are not
included in this experiment. Meridionally the model extends from
$80^{\circ}{\rm S}$ to $80^{\circ}{\rm N}$.
Vertically the model is configured with twenty layers with the 
following thicknesses
$\Delta z_{1} = 50\,{\rm m},\,
 \Delta z_{2} = 50\,{\rm m},\,
 \Delta z_{3} = 55\,{\rm m},\,
 \Delta z_{4} = 60\,{\rm m},\,
 \Delta z_{5} = 65\,{\rm m},\,
$
$
 \Delta z_{6}~=~70\,{\rm m},\,
 \Delta z_{7}~=~80\,{\rm m},\,
 \Delta z_{8}~=95\,{\rm m},\,
 \Delta z_{9}=120\,{\rm m},\,
 \Delta z_{10}=155\,{\rm m},\,
$
$
 \Delta z_{11}=200\,{\rm m},\,
 \Delta z_{12}=260\,{\rm m},\,
 \Delta z_{13}=320\,{\rm m},\,
 \Delta z_{14}=400\,{\rm m},\,
 \Delta z_{15}=480\,{\rm m},\,
$
$
 \Delta z_{16}=570\,{\rm m},\,
 \Delta z_{17}=655\,{\rm m},\,
 \Delta z_{18}=725\,{\rm m},\,
 \Delta z_{19}=775\,{\rm m},\,
 \Delta z_{20}=815\,{\rm m}
$ (here the numeric subscript indicates the model level index number, ${\tt k}$) to
give a total depth, $H$, of $-5450{\rm m}$.
The implicit free surface form of the pressure equation described in Marshall et. al 
\cite{marshall:97a} is employed. A Laplacian operator, $\nabla^2$, provides viscous
dissipation. Thermal and haline diffusion is also represented by a Laplacian operator.

Wind-stress forcing is added to the momentum equations in (\ref{eq:eg-global-model_equations}) 
for both the zonal flow, $u$ and the meridional flow $v$, according to equations 
(\ref{eq:eg-global-global_forcing_fu}) and (\ref{eq:eg-global-global_forcing_fv}).
Thermodynamic forcing inputs are added to the equations 
in (\ref{eq:eg-global-model_equations}) for
potential temperature, $\theta$, and salinity, $S$, according to equations 
(\ref{eq:eg-global-global_forcing_ft}) and (\ref{eq:eg-global-global_forcing_fs}).
This produces a set of equations solved in this configuration as follows:

\begin{eqnarray}
\label{eq:eg-global-model_equations}
\frac{Du}{Dt} - fv + 
  \frac{1}{\rho}\frac{\partial p^{'}}{\partial x} - 
  \nabla_{h}\cdot A_{h}\nabla_{h}u - 
  \frac{\partial}{\partial z}A_{z}\frac{\partial u}{\partial z} 
 & = &
\begin{cases}
{\cal F}_u & \text{(surface)} \\
0 & \text{(interior)}
\end{cases}
\\
\frac{Dv}{Dt} + fu + 
  \frac{1}{\rho}\frac{\partial p^{'}}{\partial y} - 
  \nabla_{h}\cdot A_{h}\nabla_{h}v - 
  \frac{\partial}{\partial z}A_{z}\frac{\partial v}{\partial z} 
& = &
\begin{cases}
{\cal F}_v & \text{(surface)} \\
0 & \text{(interior)}
\end{cases}
\\
\frac{\partial \eta}{\partial t} + \nabla_{h}\cdot \vec{u}
&=&
0
\\
\frac{D\theta}{Dt} -
 \nabla_{h}\cdot K_{h}\nabla_{h}\theta
 - \frac{\partial}{\partial z}\Gamma(K_{z})\frac{\partial\theta}{\partial z} 
& = &
\begin{cases}
{\cal F}_\theta & \text{(surface)} \\
0 & \text{(interior)}
\end{cases}
\\
\frac{D s}{Dt} -
 \nabla_{h}\cdot K_{h}\nabla_{h}s
 - \frac{\partial}{\partial z}\Gamma(K_{z})\frac{\partial s}{\partial z} 
& = &
\begin{cases}
{\cal F}_s & \text{(surface)} \\
0 & \text{(interior)}
\end{cases}
\\
g\rho_{0} \eta + \int^{0}_{-z}\rho^{'} dz & = & p^{'}
\end{eqnarray}

\noindent where $u=\frac{Dx}{Dt}=r \cos(\phi)\frac{D \lambda}{Dt}$ and 
$v=\frac{Dy}{Dt}=r \frac{D \phi}{Dt}$ 
are the zonal and meridional components of the
flow vector, $\vec{u}$, on the sphere. As described in
MITgcm Numerical Solution Procedure \ref{chap:discretization}, the time 
evolution of potential temperature, $\theta$, equation is solved prognostically.
The total pressure, $p$, is diagnosed by summing pressure due to surface 
elevation $\eta$ and the hydrostatic pressure.
\\

\subsubsection{Numerical Stability Criteria}
%\label{www:tutorials}

The Laplacian dissipation coefficient, $A_{h}$, is set to $5 \times 10^5 m s^{-1}$.
This value is chosen to yield a Munk layer width \cite{adcroft:95},
\begin{eqnarray}
\label{eq:eg-global-munk_layer}
&& M_{w} = \pi ( \frac { A_{h} }{ \beta } )^{\frac{1}{3}}
\end{eqnarray}

\noindent  of $\approx 600$km. This is greater than the model
resolution in low-latitudes, $\Delta x \approx 400{\rm km}$, ensuring that the frictional 
boundary layer is adequately resolved.
\\

\noindent The model is stepped forward with a 
time step $\delta t_{\theta}=30~{\rm hours}$ for thermodynamic variables and
$\delta t_{v}=40~{\rm minutes}$ for momentum terms. With this time step, the stability 
parameter to the horizontal Laplacian friction \cite{adcroft:95}
\begin{eqnarray}
\label{eq:eg-global-laplacian_stability}
&& S_{l} = 4 \frac{A_{h} \delta t_{v}}{{\Delta x}^2}
\end{eqnarray}

\noindent evaluates to 0.16 at a latitude of $\phi=80^{\circ}$, which is below the 
0.3 upper limit for stability. The zonal grid spacing $\Delta x$ is smallest at
$\phi=80^{\circ}$ where $\Delta x=r\cos(\phi)\Delta \phi\approx 77{\rm km}$.
\\

\noindent The vertical dissipation coefficient, $A_{z}$, is set to 
$1\times10^{-3} {\rm m}^2{\rm s}^{-1}$. The associated stability limit
\begin{eqnarray}
\label{eq:eg-global-laplacian_stability_z}
S_{l} = 4 \frac{A_{z} \delta t_{v}}{{\Delta z}^2}
\end{eqnarray}

\noindent evaluates to $0.015$ for the smallest model
level spacing ($\Delta z_{1}=50{\rm m}$) which is again well below
the upper stability limit.
\\

The values of the horizontal ($K_{h}$) and vertical ($K_{z}$) diffusion coefficients 
for both temperature and salinity are set to $1 \times 10^{3}~{\rm m}^{2}{\rm s}^{-1}$ 
and $3 \times 10^{-5}~{\rm m}^{2}{\rm s}^{-1}$ respectively. The stability limit 
related to $K_{h}$ will be at $\phi=80^{\circ}$ where $\Delta x \approx 77 {\rm km}$. 
Here the stability parameter 
\begin{eqnarray} 
\label{eq:eg-global-laplacian_stability_xtheta}
S_{l} = \frac{4 K_{h} \delta t_{\theta}}{{\Delta x}^2} 
\end{eqnarray}
evaluates to $0.07$, well below the stability limit of $S_{l} \approx 0.5$. The 
stability parameter related to $K_{z}$
\begin{eqnarray} 
\label{eq:eg-global-laplacian_stability_ztheta}
S_{l} = \frac{4 K_{z} \delta t_{\theta}}{{\Delta z}^2} 
\end{eqnarray}
evaluates to $0.005$ for $\min(\Delta z)=50{\rm m}$, well below the stability limit 
of $S_{l} \approx 0.5$.
\\

\noindent The numerical stability for inertial oscillations
\cite{adcroft:95} 

\begin{eqnarray}
\label{eq:eg-global-inertial_stability}
S_{i} = f^{2} {\delta t_v}^2
\end{eqnarray}

\noindent evaluates to $0.24$ for $f=2\omega\sin(80^{\circ})=1.43\times10^{-4}~{\rm s}^{-1}$, which is close to 
the $S_{i} < 1$ upper limit for stability.
\\

\noindent The advective CFL \cite{adcroft:95} for a extreme maximum 
horizontal flow
speed of $ | \vec{u} | = 2 ms^{-1}$

\begin{eqnarray}
\label{eq:eg-global-cfl_stability}
S_{a} = \frac{| \vec{u} | \delta t_{v}}{ \Delta x}
\end{eqnarray}

\noindent evaluates to $6 \times 10^{-2}$. This is well below the stability 
limit of 0.5.
\\

\noindent The stability parameter for internal gravity waves propagating
with a maximum speed of $c_{g}=10~{\rm ms}^{-1}$
\cite{adcroft:95}

\begin{eqnarray}
\label{eq:eg-global-gfl_stability}
S_{c} = \frac{c_{g} \delta t_{v}}{ \Delta x}
\end{eqnarray}

\noindent evaluates to $3 \times 10^{-1}$. This is close to the linear
stability limit of 0.5.
  
\subsection{Experiment Configuration}
%\label{www:tutorials}
\label{sec:eg-global-clim_ocn_examp_exp_config}

The model configuration for this experiment resides under the 
directory {\it tutorial\_examples/global\_ocean\_circulation/}.  
The experiment files 

\begin{itemize}
\item {\it input/data}
\item {\it input/data.pkg}
\item {\it input/eedata},
\item {\it input/windx.bin},
\item {\it input/windy.bin},
\item {\it input/salt.bin},
\item {\it input/theta.bin},
\item {\it input/SSS.bin},
\item {\it input/SST.bin},
\item {\it input/topog.bin},
\item {\it code/CPP\_EEOPTIONS.h}
\item {\it code/CPP\_OPTIONS.h},
\item {\it code/SIZE.h}. 
\end{itemize}
contain the code customizations and parameter settings for these
experiments. Below we describe the customizations
to these files associated with this experiment.

\subsubsection{Driving Datasets}
%\label{www:tutorials}

Figures ({\it --- missing figures ---}) 
%(\ref{fig:sim_config_tclim}-\ref{fig:sim_config_empmr})
show the relaxation temperature ($\theta^{\ast}$) and salinity ($S^{\ast}$) 
fields, the wind stress components ($\tau_x$ and $\tau_y$), the heat flux ($Q$)
and the net fresh water flux (${\cal E} - {\cal P} - {\cal R}$) used
in equations 
(\ref{eq:eg-global-global_forcing_fu}-\ref{eq:eg-global-global_forcing_fs}).
The figures also indicate the lateral extent and coastline used in the 
experiment. Figure ({\it --- missing figure --- }) %ref{fig:model_bathymetry}) 
shows the depth contours of the model domain.

\subsubsection{File {\it input/data}}
%\label{www:tutorials}

\input{s_examples/global_oce_latlon/inp_data}

\subsubsection{File {\it input/data.pkg}}
%\label{www:tutorials}

This file uses standard default values and does not contain
customisations for this experiment.

\subsubsection{File {\it input/eedata}}
%\label{www:tutorials}

This file uses standard default values and does not contain
customisations for this experiment.

\subsubsection{File {\it input/windx.sin\_y}}
%\label{www:tutorials}

The {\it input/windx.sin\_y} file specifies a two-dimensional ($x,y$) 
map of wind stress ,$\tau_{x}$, values. The units used are $Nm^{-2}$.
Although $\tau_{x}$ is only a function of $y$n in this experiment
this file must still define a complete two-dimensional map in order
to be compatible with the standard code for loading forcing fields 
in MITgcm. The included matlab program {\it input/gendata.m} gives a complete
code for creating the {\it input/windx.sin\_y} file.

\subsubsection{File {\it input/topog.box}}
%\label{www:tutorials}


The {\it input/topog.box} file specifies a two-dimensional ($x,y$) 
map of depth values. For this experiment values are either
$0m$ or $-2000\,{\rm m}$, corresponding respectively to a wall or to deep
ocean. The file contains a raw binary stream of data that is enumerated
in the same way as standard MITgcm two-dimensional, horizontal arrays.
The included matlab program {\it input/gendata.m} gives a complete
code for creating the {\it input/topog.box} file.

\subsubsection{File {\it code/SIZE.h}}
%\label{www:tutorials}

Two lines are customized in this file for the current experiment

\begin{itemize}

\item Line 39, 
\begin{verbatim} sNx=60, \end{verbatim} this line sets
the lateral domain extent in grid points for the
axis aligned with the x-coordinate.

\item Line 40, 
\begin{verbatim} sNy=60, \end{verbatim} this line sets
the lateral domain extent in grid points for the
axis aligned with the y-coordinate.

\item Line 49, 
\begin{verbatim} Nr=4,   \end{verbatim} this line sets
the vertical domain extent in grid points.

\end{itemize}

\begin{small}
% $Header: /u/gcmpack/manual/s_examples/global_oce_latlon/code/Attic/SIZE.h.tex,v 1.1.1.1 2001/08/08 16:16:14 adcroft Exp $
% $Name:  $

\begin{verbatim}
     1	C $Header: /u/gcmpack/manual/s_examples/global_oce_latlon/code/Attic/SIZE.h.tex,v 1.1.1.1 2001/08/08 16:16:14 adcroft Exp $
     2	C $Name:  $
     3	C
     4	C     /==========================================================\
     5	C     | SIZE.h Declare size of underlying computational grid.    |
     6	C     |==========================================================|
     7	C     | The design here support a three-dimensional model grid   |
     8	C     | with indices I,J and K. The three-dimensional domain     |
     9	C     | is comprised of nPx*nSx blocks of size sNx along one axis|
    10	C     | nPy*nSy blocks of size sNy along another axis and one    |
    11	C     | block of size Nz along the final axis.                   |
    12	C     | Blocks have overlap regions of size OLx and OLy along the|
    13	C     | dimensions that are subdivided.                          |
    14	C     \==========================================================/
    15	C     Voodoo numbers controlling data layout.
    16	C     sNx - No. X points in sub-grid.
    17	C     sNy - No. Y points in sub-grid.
    18	C     OLx - Overlap extent in X.
    19	C     OLy - Overlat extent in Y.
    20	C     nSx - No. sub-grids in X.
    21	C     nSy - No. sub-grids in Y.
    22	C     nPx - No. of processes to use in X.
    23	C     nPy - No. of processes to use in Y.
    24	C     Nx  - No. points in X for the total domain.
    25	C     Ny  - No. points in Y for the total domain.
    26	C     Nr  - No. points in Z for full process domain.
    27	      INTEGER sNx
    28	      INTEGER sNy
    29	      INTEGER OLx
    30	      INTEGER OLy
    31	      INTEGER nSx
    32	      INTEGER nSy
    33	      INTEGER nPx
    34	      INTEGER nPy
    35	      INTEGER Nx
    36	      INTEGER Ny
    37	      INTEGER Nr
    38	      PARAMETER (
    39	     &           sNx =  90,
    40	     &           sNy =  40,
    41	     &           OLx =   3,
    42	     &           OLy =   3,
    43	     &           nSx =   1,
    44	     &           nSy =   1,
    45	     &           nPx =   1,
    46	     &           nPy =   1,
    47	     &           Nx  = sNx*nSx*nPx,
    48	     &           Ny  = sNy*nSy*nPy,
    49	     &           Nr  =  20)
       
    50	C     MAX_OLX  - Set to the maximum overlap region size of any array
    51	C     MAX_OLY    that will be exchanged. Controls the sizing of exch
    52	C                routine buufers.
    53	      INTEGER MAX_OLX
    54	      INTEGER MAX_OLY
    55	      PARAMETER ( MAX_OLX = OLx,
    56	     &            MAX_OLY = OLy )
\end{verbatim}

\end{small}

\subsubsection{File {\it code/CPP\_OPTIONS.h}}
%\label{www:tutorials}

This file uses standard default values and does not contain
customisations for this experiment.


\subsubsection{File {\it code/CPP\_EEOPTIONS.h}}
%\label{www:tutorials}

This file uses standard default values and does not contain
customisations for this experiment.

\subsubsection{Other Files }
%\label{www:tutorials}

Other files relevant to this experiment are
\begin{itemize}
\item {\it model/src/ini\_cori.F}. This file initializes the model
coriolis variables {\bf fCorU}.
\item {\it model/src/ini\_spherical\_polar\_grid.F}
\item {\it model/src/ini\_parms.F},
\item {\it input/windx.sin\_y},
\end{itemize}
contain the code customisations and parameter settings for this 
experiments. Below we describe the customisations
to these files associated with this experiment.
