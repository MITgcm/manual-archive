% $Header: /u/gcmpack/manual/s_examples/advection_in_gyre/adv_gyre.tex,v 1.3 2008/01/15 17:34:49 cnh Exp $
% $Name:  $

\bodytext{bgcolor="#FFFFFFFF"}


\section[Gyre Advection Example]{Ocean Gyre Advection Schemes}
\label{sect:eg-adv-gyre}
\label{www:tutorials}
\begin{rawhtml}
<!-- CMIREDIR:eg-adv-gyre: -->
\end{rawhtml}

This set of examples is based on the barotropic and baroclinic gyre MITgcm configurations,
that are described in the tutorial sections \label{sect:eg-baro} and \label{sect:eg-fourlayer}. 
The example in this section explains how to introduce a passive tracer into the flow 
field of the barotropic and baroclinic gyre setups and looks at how the time evolution
of the passive tracer depends on the advection or transport scheme that is selected 
for the tracer. 

Passive tracers are useful in many numerical experiments. In some cases tracers are
used to track flow pathways, for example in \cite{Dutay02} a passive tracer is used
to track pathways of CFC-11 in 13 global ocean models (similar to the example
described in section \ref{sect:eg-offline-cfc}).
In other cases tracers are used as a way
to infer bulk mixing coefficients for a turbulent flow field, for example in ...... In 
biogeochemical and ecological simulations large numbers of tracers are used that carry the 
concentrations of biological nutrients and concentrations of biological species.
When using tracers for these and other purposes it is useful to have a feel for the role
that the advection scheme employed plays in determining properties of the tracer distribution.

\subsection{Advection and tracer transport}
 






