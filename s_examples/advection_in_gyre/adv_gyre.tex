% $Header: /u/gcmpack/manual/s_examples/advection_in_gyre/adv_gyre.tex,v 1.16 2010/08/30 23:09:19 jmc Exp $
% $Name:  $

\bodytext{bgcolor="#FFFFFFFF"}


\section[Gyre Advection Example]{Ocean Gyre Advection Schemes}
%\label{www:tutorials}
\label{sec:eg-adv-gyre}
\begin{rawhtml}
<!-- CMIREDIR:eg-adv-gyre: -->
\end{rawhtml}
\begin{center} 
(in directory: {\it verification/tutorial\_advection\_in\_gyre/})
\end{center}

Author: Oliver Jahn and Chris Hill



This set of examples is based on the barotropic and baroclinic gyre MITgcm configurations,
that are described in the tutorial sections \ref{sec:eg-baro} and \ref{sec:eg-fourlayer}. 
The examples in this section explain how to introduce a passive tracer into the flow 
field of the barotropic and baroclinic gyre setups and looks at how the time evolution
of the passive tracer depends on the advection or transport scheme that is selected 
for the tracer. 

Passive tracers are useful in many numerical experiments. In some cases tracers are
used to track flow pathways, for example in \cite{Dutay02} a passive tracer is used
to track pathways of CFC-11 in 13 global ocean models, using a numerical
configuration similar to the example described in section \ref{sec:eg-offline-cfc}).
In other cases tracers are used as a way
to infer bulk mixing coefficients for a turbulent flow field, for example in 
\cite{marsh06} a tracer is used to infer eddy mixing coefficients in the
Antarctic Circumpolar Current region. In biogeochemical and ecological simulations large numbers 
of tracers are used that carry the concentrations of biological nutrients and concentrations of 
biological species, for example in ....
When using tracers for these and other purposes it is useful to have a feel for the role
that the advection scheme employed plays in determining properties of the tracer distribution.
In particular, in a discrete numerical model tracer advection only approximates the 
continuum behavior in space and time and different advection schemes introduce diferent 
approximations so that the resulting tracer distributions vary. In the following 
text we illustrate how
to use the different advection schemes available in MITgcm here, and discuss which properties 
are well represented by each one. The advection schemes selections also apply to active
tracers (e.g. $T$ and $S$) and the character of the schemes also affect their distributions
and behavior.

\subsection{Advection and tracer transport}

In general, the tracer problem we want to solve can be written

\begin{equation}
\label{eq:eg-adv-gyre-generic-tracer}
\frac{\partial C}{partial t} = -U \cdot \nabla C + S
\end{equation}

where $C$ is the tracer concentration in a model cell, $U$ is the model three-dimensional
flow field ( $U=(u,v,w)$ ). In (\ref{eq:eg-adv-gyre-generic-tracer}) $S$ represents source, sink 
and tendency terms not associated with advective transport. Example of terms in $S$ include
(i) air-sea fluxes for a dissolved gas, (ii) biological grazing and growth terms (for a 
biogeochemical problem) or (iii) convective mixing and other sub-grid parameterizations of 
mixing. In this section we are primarily concerned with 
\begin{enumerate}
\item how to introduce the tracer term, $C$, into an integration
\item the different discretized forms of 
the $-U \cdot \nabla C$ term that are available
\end{enumerate}


\subsection{Introducing a tracer into the flow}

 The MITgcm ptracers package (see section \ref{sec:pkg:ptracers} for a more complete discussion
of the ptracers package and section \ref{sec:pkg:using} for a general introduction to MITgcm 
packages) provides pre-coded support for a simple passive tracer with an initial
distribution at simulation time $t=0$ of $C_0(x,y,z)$. The steps required to use this capability
are 
\begin{enumerate}
\item{\bf Activating the ptracers package.} This simply requires adding the line {\tt ptracers} to
the {\tt packages.conf} file in the {\it code/} directory for the experiment. 
\item{\bf Setting an initial tracer distribution.} 
\end{enumerate}

Once the two steps above are complete we can proceed to examine how the tracer we have created is
carried by the flow field and what properties of the tracer distribution are preserved under
different advection schemes.

\subsection{Selecting an advection scheme}

- flags in data and data.ptracers

- overlap width

- CPP GAD\_ALLOW\_SOM\_ADVECT required for SOM case

\subsection{Comparison of different advection schemes}

\begin{enumerate}
\item{Conservation}
\item{Dispersion}
\item{Diffusion}
\item{Positive definite}
\end{enumerate}

\begin{rawhtml}MITGCM_INSERT_FIGURE_BEGIN advection-in-gyre\end{rawhtml}
\begin{figure}
\begin{center}
 \includegraphics*[width=\textwidth]{s_examples/advection_in_gyre/all012.eps}
\end{center}
\caption{Dye evolving in a double gyre with different advection schemes.  The figure shows the
dye concentration one year after injection into a single grid cell near the left boundary.
Stream lines are also shown.}
\label{fig:adv-gyre-all}
\end{figure}
\begin{rawhtml}MITGCM_INSERT_FIGURE_END\end{rawhtml}


\begin{figure}
\begin{center}
 \includegraphics*[width=\textwidth]{s_examples/advection_in_gyre/stats.eps}
\end{center}
\caption{Maxima, minima and standard deviation (from left) as a function of time (in months)
for the gyre circulation experiment from figure~\ref{fig:adv-gyre-all}.}
\label{fig:adv-gyre-stats}
\end{figure}

\subsection{Code and Parameters files for this tutorial}

The code and parameters for the experiments can be found in the MITgcm example experiments 
directory {\it verification/tutorial\_advection\_in\_gyre/}.







