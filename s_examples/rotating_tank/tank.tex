% $Header: /u/gcmpack/manual/s_examples/rotating_tank/tank.tex,v 1.1 2004/06/22 15:07:37 afe Exp $
% $Name:  $

\bodytext{bgcolor="#FFFFFFFF"}

%\begin{center} 
%{\Large \bf Using MITgcm to Simulate a Rotating Tank in Cylindrical 
%Coordinates}
%
%\vspace*{4mm}
%
%\vspace*{3mm}
%{\large June 2004}
%\end{center}

This is the first in a series of tutorials describing
example MITgcm numerical experiments. The example experiments 
include both straightforward examples of idealized geophysical 
fluid simulations and more involved cases encompassing
large scale modeling and
automatic differentiation. Both hydrostatic and non-hydrostatic 
experiments are presented, as well as experiments employing
Cartesian, spherical-polar and cube-sphere coordinate systems.
These ``case study'' documents include information describing
the experimental configuration and detailed information on how to
configure the MITgcm code and input files for each experiment.

\section{Barotropic Ocean Gyre In Cartesian Coordinates}
\label{sect:eg-baro}
\label{www:tutorials}



\subsection{Equations Solved}
\label{www:tutorials}
The model is configured in hydrostatic form. The implicit free surface form of the


\subsection{Discrete Numerical Configuration}
\label{www:tutorials}

 The domain is discretised with 
a uniform grid spacing in the horizontal set to
 $\Delta x=\Delta y=20$~km, so 
that there are sixty grid cells in the $x$ and $y$ directions. Vertically the 
model is configured with a single layer with depth, $\Delta z$, of $5000$~m. 

\subsubsection{Numerical Stability Criteria}
\label{www:tutorials}


\subsection{Code Configuration}
\label{www:tutorials}
\label{SEC:eg-baro-code_config}

The model configuration for this experiment resides under the 
directory {\it verification/exp0/}.  The experiment files 
\begin{itemize}
\item {\it input/data}
\item {\it input/data.pkg}
\item {\it input/eedata},
\item {\it input/windx.sin\_y},
\item {\it input/topog.box},
\item {\it code/CPP\_EEOPTIONS.h}
\item {\it code/CPP\_OPTIONS.h},
\item {\it code/SIZE.h}. 
\end{itemize}
contain the code customizations and parameter settings for this 
experiments. Below we describe the customizations
to these files associated with this experiment.

\subsubsection{File {\it input/data}}
\label{www:tutorials}

This file, reproduced completely below, specifies the main parameters 
for the experiment. The parameters that are significant for this configuration
are

\begin{itemize}

\item Line 7, \begin{verbatim} viscAh=4.E2, \end{verbatim} this line sets
the Laplacian friction coefficient to $400 m^2s^{-1}$
\item Line 10, \begin{verbatim} beta=1.E-11, \end{verbatim} this line sets
$\beta$ (the gradient of the coriolis parameter, $f$) to $10^{-11} s^{-1}m^{-1}$

\item Lines 15 and 16
\begin{verbatim}
rigidLid=.FALSE.,
implicitFreeSurface=.TRUE.,
\end{verbatim}
these lines suppress the rigid lid formulation of the surface
pressure inverter and activate the implicit free surface form
of the pressure inverter.

\item Line 27,
\begin{verbatim}
startTime=0,
\end{verbatim}
this line indicates that the experiment should start from $t=0$
and implicitly suppresses searching for checkpoint files associated
with restarting an numerical integration from a previously saved state.

\item Line 29,
\begin{verbatim}
endTime=12000,
\end{verbatim}
this line indicates that the experiment should start finish at $t=12000s$.
A restart file will be written at this time that will enable the
simulation to be continued from this point.

\item Line 30,
\begin{verbatim}
deltaTmom=1200,
\end{verbatim}
This line sets the momentum equation timestep to $1200s$.

\item Line 39,
\begin{verbatim}
usingCartesianGrid=.TRUE.,
\end{verbatim}
This line requests that the simulation be performed in a 
Cartesian coordinate system.

\item Line 41,
\begin{verbatim}
delX=60*20E3,
\end{verbatim}
This line sets the horizontal grid spacing between each x-coordinate line
in the discrete grid. The syntax indicates that the discrete grid
should be comprise of $60$ grid lines each separated by $20 \times 10^{3}m$
($20$~km).

\item Line 42,
\begin{verbatim}
delY=60*20E3,
\end{verbatim}
This line sets the horizontal grid spacing between each y-coordinate line
in the discrete grid to $20 \times 10^{3}m$ ($20$~km).

\item Line 43,
\begin{verbatim}
delZ=5000,
\end{verbatim}
This line sets the vertical grid spacing between each z-coordinate line
in the discrete grid to $5000m$ ($5$~km).

\item Line 46,
\begin{verbatim}
bathyFile='topog.box'
\end{verbatim}
This line specifies the name of the file from which the domain
bathymetry is read. This file is a two-dimensional ($x,y$) map of
depths. This file is assumed to contain 64-bit binary numbers 
giving the depth of the model at each grid cell, ordered with the x 
coordinate varying fastest. The points are ordered from low coordinate
to high coordinate for both axes. The units and orientation of the
depths in this file are the same as used in the MITgcm code. In this
experiment, a depth of $0m$ indicates a solid wall and a depth
of $-5000m$ indicates open ocean. The matlab program
{\it input/gendata.m} shows an example of how to generate a
bathymetry file.


\item Line 49,
\begin{verbatim}
zonalWindFile='windx.sin_y'
\end{verbatim}
This line specifies the name of the file from which the x-direction
surface wind stress is read. This file is also a two-dimensional
($x,y$) map and is enumerated and formatted in the same manner as the 
bathymetry file. The matlab program {\it input/gendata.m} includes example 
code to generate a valid {\bf zonalWindFile} file.  

\end{itemize}

\noindent other lines in the file {\it input/data} are standard values
that are described in the MITgcm Getting Started and MITgcm Parameters
notes.

%%\begin{small}
%%% $Header: /u/gcmpack/manual/s_examples/baroclinic_gyre/input/data.tex,v 1.1.1.1 2001/08/08 16:15:46 adcroft Exp $
% $Name:  $

\begin{verbatim}
     1	# Model parameters
     2	# Continuous equation parameters
     3	 &PARM01
     4	 tRef=20.,10.,8.,6.,
     5	 sRef=10.,10.,10.,10.,
     6	 viscAz=1.E-2,
     7	 viscAh=4.E2,
     8	 no_slip_sides=.FALSE.,
     9	 no_slip_bottom=.TRUE.,
    10	 diffKhT=4.E2,
    11	 diffKzT=1.E-2,
    12	 beta=1.E-11,
    13	 tAlpha=2.E-4,
    14	 sBeta =0.,
    15	 gravity=9.81,
    16	 rigidLid=.FALSE.,
    17	 implicitFreeSurface=.TRUE.,
    18	 eosType='LINEAR',
    19	 readBinaryPrec=64,
    20	 &
    21	# Elliptic solver parameters
    22	 &PARM02
    23	 cg2dMaxIters=1000,
    24	 cg2dTargetResidual=1.E-13,
    25	 &
    26	# Time stepping parameters
    27	 &PARM03
    28	 startTime=0.,
    29	 endTime=12000., 
    30	 deltaTmom=1200.0,
    31	 deltaTtracer=1200.0,
    32	 abEps=0.1,
    33	 pChkptFreq=17000.0,
    34	 chkptFreq=0.0,
    35	 dumpFreq=2592000.0,
    36	 &
    37	# Gridding parameters
    38	 &PARM04
    39	 usingCartesianGrid=.FALSE.,
    40	 usingSphericalPolarGrid=.TRUE.,
    41	 phiMin=0.,
    42	 delX=60*1.,
    43	 delY=60*1.,
    44	 delZ=500.,500.,500.,500.,
    45	 &
    46	 &PARM05
    47	 bathyFile='topog.box',
    48	 hydrogThetaFile=,
    49	 hydrogSaltFile=,
    50	 zonalWindFile='windx.sin_y',
    51	 meridWindFile=,
    52	 &
\end{verbatim}

%%\end{small}

\subsubsection{File {\it input/data.pkg}}
\label{www:tutorials}

This file uses standard default values and does not contain
customizations for this experiment.

\subsubsection{File {\it input/eedata}}
\label{www:tutorials}

This file uses standard default values and does not contain
customizations for this experiment.

\subsubsection{File {\it input/windx.sin\_y}}
\label{www:tutorials}

The {\it input/windx.sin\_y} file specifies a two-dimensional ($x,y$) 
map of wind stress ,$\tau_{x}$, values. The units used are $Nm^{-2}$.
Although $\tau_{x}$ is only a function of $y$n in this experiment
this file must still define a complete two-dimensional map in order
to be compatible with the standard code for loading forcing fields 
in MITgcm. The included matlab program {\it input/gendata.m} gives a complete
code for creating the {\it input/windx.sin\_y} file.

\subsubsection{File {\it input/topog.box}}
\label{www:tutorials}


The {\it input/topog.box} file specifies a two-dimensional ($x,y$) 
map of depth values. For this experiment values are either
$0m$ or {\bf -delZ}m, corresponding respectively to a wall or to deep
ocean. The file contains a raw binary stream of data that is enumerated
in the same way as standard MITgcm two-dimensional, horizontal arrays.
The included matlab program {\it input/gendata.m} gives a complete
code for creating the {\it input/topog.box} file.

\subsubsection{File {\it code/SIZE.h}}
\label{www:tutorials}

Two lines are customized in this file for the current experiment

\begin{itemize}

\item Line 39, 
\begin{verbatim} sNx=60, \end{verbatim} this line sets
the lateral domain extent in grid points for the
axis aligned with the x-coordinate.

\item Line 40, 
\begin{verbatim} sNy=60, \end{verbatim} this line sets
the lateral domain extent in grid points for the
axis aligned with the y-coordinate.

\end{itemize}

\begin{small}
% $Header: /u/gcmpack/manual/s_examples/barotropic_gyre/code/SIZE.h.tex,v 1.1.1.1 2001/08/08 16:15:58 adcroft Exp $
% $Name:  $

\begin{verbatim}
     1	C $Header: /u/gcmpack/manual/s_examples/barotropic_gyre/code/SIZE.h.tex,v 1.1.1.1 2001/08/08 16:15:58 adcroft Exp $
     2	C $Name:  $
     3	C
     4	C     /==========================================================\
     5	C     | SIZE.h Declare size of underlying computational grid.    |
     6	C     |==========================================================|
     7	C     | The design here support a three-dimensional model grid   |
     8	C     | with indices I,J and K. The three-dimensional domain     |
     9	C     | is comprised of nPx*nSx blocks of size sNx along one axis|
    10	C     | nPy*nSy blocks of size sNy along another axis and one    |
    11	C     | block of size Nz along the final axis.                   |
    12	C     | Blocks have overlap regions of size OLx and OLy along the|
    13	C     | dimensions that are subdivided.                          |
    14	C     \==========================================================/
    15	C     Voodoo numbers controlling data layout.
    16	C     sNx - No. X points in sub-grid.
    17	C     sNy - No. Y points in sub-grid.
    18	C     OLx - Overlap extent in X.
    19	C     OLy - Overlat extent in Y.
    20	C     nSx - No. sub-grids in X.
    21	C     nSy - No. sub-grids in Y.
    22	C     nPx - No. of processes to use in X.
    23	C     nPy - No. of processes to use in Y.
    24	C     Nx  - No. points in X for the total domain.
    25	C     Ny  - No. points in Y for the total domain.
    26	C     Nr  - No. points in R for full process domain.
    27	      INTEGER sNx
    28	      INTEGER sNy
    29	      INTEGER OLx
    30	      INTEGER OLy
    31	      INTEGER nSx
    32	      INTEGER nSy
    33	      INTEGER nPx
    34	      INTEGER nPy
    35	      INTEGER Nx
    36	      INTEGER Ny
    37	      INTEGER Nr
    38	      PARAMETER (
    39	     &           sNx =  60,
    40	     &           sNy =  60,
    41	     &           OLx =   3,
    42	     &           OLy =   3,
    43	     &           nSx =   1,
    44	     &           nSy =   1,
    45	     &           nPx =   1,
    46	     &           nPy =   1,
    47	     &           Nx  = sNx*nSx*nPx,
    48	     &           Ny  = sNy*nSy*nPy,
    49	     &           Nr  =   1)
       
    50	C     MAX_OLX  - Set to the maximum overlap region size of any array
    51	C     MAX_OLY    that will be exchanged. Controls the sizing of exch
    52	C                routine buufers.
    53	      INTEGER MAX_OLX
    54	      INTEGER MAX_OLY
    55	      PARAMETER ( MAX_OLX = OLx,
    56	     &            MAX_OLY = OLy )
\end{verbatim}

\end{small}

\subsubsection{File {\it code/CPP\_OPTIONS.h}}
\label{www:tutorials}

This file uses standard default values and does not contain
customizations for this experiment.


\subsubsection{File {\it code/CPP\_EEOPTIONS.h}}
\label{www:tutorials}

This file uses standard default values and does not contain
customizations for this experiment.

