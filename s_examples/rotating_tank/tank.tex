% $Header: /u/gcmpack/manual/s_examples/rotating_tank/tank.tex,v 1.9 2004/07/27 13:40:09 afe Exp $
% $Name:  $

\bodytext{bgcolor="#FFFFFFFF"}

%\begin{center} 
%{\Large \bf Using MITgcm to Simulate a Rotating Tank in Cylindrical
%Coordinates}
%
%\vspace*{4mm}
%
%\vspace*{3mm}
%{\large May 2001}
%\end{center}

\section{A Rotating Tank in Cylindrical Coordinates}
\label{sect:eg-tank}
\label{www:tutorials}

This section illustrates an example of MITgcm simulating a laboratory
experiment on much smaller scales than those common to geophysical
fluid dynamics.

\subsection{Overview}
\label{www:tutorials}
                                                                                
                                                                                
This example experiment demonstrates using the MITgcm to simulate
a laboratory experiment with a rotating tank of water with an ice
bucket in the center. The simulation is configured for a laboratory
scale on a
$3^{\circ}$ $\times$ 20cm
cyclindrical grid with twenty-nine vertical
levels.
\\



 

\subsection{Equations Solved}
\label{www:tutorials}


\subsection{Discrete Numerical Configuration}
\label{www:tutorials}

 The domain is discretised with 
a uniform grid spacing in the horizontal set to
 $\Delta x=\Delta y=20$~km, so 
that there are sixty grid cells in the $x$ and $y$ directions. Vertically the 
model is configured with a single layer with depth, $\Delta z$, of $5000$~m. 


\subsection{Code Configuration}
\label{www:tutorials}
\label{SEC:eg-baro-code_config}

The model configuration for this experiment resides under the
directory {\it verification/rotatingi\_tank/}.  The experiment files
\begin{itemize}
\item {\it input/data}
\item {\it input/data.pkg}
\item {\it input/eedata},
\item {\it input/bathyPol.bin},
\item {\it input/thetaPol.bin},
\item {\it code/CPP\_EEOPTIONS.h}
\item {\it code/CPP\_OPTIONS.h},
\item {\it code/SIZE.h}.
\end{itemize}

contain the code customizations and parameter settings for this 
experiments. Below we describe the customizations
to these files associated with this experiment.

\subsubsection{File {\it input/data}}
\label{www:tutorials}

This file, reproduced completely below, specifies the main parameters 
for the experiment. The parameters that are significant for this configuration
are

\begin{itemize}

\item Line X, \begin{verbatim} viscAh=5.0E-6, \end{verbatim} this line sets
the Laplacian friction coefficient to $0.000006 m^2s^{-1}$, which is ususally 
low because of the small scale, presumably.... qqq

\item Line X, \begin{verbatim}f0=0.5 , \end{verbatim} this line sets the 
coriolis term, and represents a tank spinning at qqq
\item Line 10, \begin{verbatim} beta=1.E-11, \end{verbatim} this line sets
$\beta$ (the gradient of the coriolis parameter, $f$) to $10^{-11} s^{-1}m^{-1}$

\item Lines 15 and 16
\begin{verbatim}
rigidLid=.TRUE.,
implicitFreeSurface=.FALSE.,
\end{verbatim}

these lines do the opposite of the following: 
suppress the rigid lid formulation of the surface
pressure inverter and activate the implicit free surface form
of the pressure inverter.

\item Line 27,
\begin{verbatim}
startTime=0,
\end{verbatim}
this line indicates that the experiment should start from $t=0$
and implicitly suppresses searching for checkpoint files associated
with restarting an numerical integration from a previously saved state.

\item Line 30,
\begin{verbatim}
deltaT=0.1,
\end{verbatim}
This line sets the integration timestep to $0.1s$.  This is an unsually
small value among the examples due to the small physical scale of the 
experiment.

\item Line 39,
\begin{verbatim}
usingCylindricalGrid=.TRUE.,
\end{verbatim}
This line requests that the simulation be performed in a 
cylindrical coordinate system.

\item Line qqq,
\begin{verbatim}
dXspacing=3,
\end{verbatim}
This line sets the azimuthal grid spacing between each x-coordinate line
in the discrete grid. The syntax indicates that the discrete grid
should be comprise of $120$ grid lines each separated by $3^{\circ}$.
                                                                                


\item Line qqq,
\begin{verbatim}
dYspacing=0.01,
\end{verbatim}
This line sets the radial grid spacing between each $\rho$-coordinate line
in the discrete grid to $1cm$.

\item Line 43,
\begin{verbatim}
delZ=29*0.005,
\end{verbatim}
This line sets the vertical grid spacing between each z-coordinate line
in the discrete grid to $5000m$ ($5$~km).

\item Line 46,
\begin{verbatim}
bathyFile='bathyPol.bin',
\end{verbatim}
This line specifies the name of the file from which the domain
``bathymetry'' (tank depth) is read. This file is a two-dimensional 
($x,y$) map of
depths. This file is assumed to contain 64-bit binary numbers 
giving the depth of the model at each grid cell, ordered with the $x$ 
coordinate varying fastest. The points are ordered from low coordinate
to high coordinate for both axes.  The units and orientation of the
depths in this file are the same as used in the MITgcm code. In this
experiment, a depth of $0m$ indicates an area outside of the tank
and a depth
f $-0.145m$ indicates the tank itself. 

\item Line 49,
\begin{verbatim}
hydrogThetaFile='thetaPol.bin',
\end{verbatim}
This line specifies the name of the file from which the initial values 
of $\theta$ 
are read. This file is a three-dimensional
($x,y,z$) map and is enumerated and formatted in the same manner as the 
bathymetry file. 

\item Line qqq
\begin{verbatim}
 tCyl  = 0
\end{verbatim}
This line specifies the temperature in degrees Celsius of the interior
wall of the tank -- usually a bucket of ice water.


\end{itemize}

\noindent other lines in the file {\it input/data} are standard values
that are described in the MITgcm Getting Started and MITgcm Parameters
notes.

\begin{small}
% $Header: /u/gcmpack/manual/s_examples/baroclinic_gyre/input/data.tex,v 1.1.1.1 2001/08/08 16:15:46 adcroft Exp $
% $Name:  $

\begin{verbatim}
     1	# Model parameters
     2	# Continuous equation parameters
     3	 &PARM01
     4	 tRef=20.,10.,8.,6.,
     5	 sRef=10.,10.,10.,10.,
     6	 viscAz=1.E-2,
     7	 viscAh=4.E2,
     8	 no_slip_sides=.FALSE.,
     9	 no_slip_bottom=.TRUE.,
    10	 diffKhT=4.E2,
    11	 diffKzT=1.E-2,
    12	 beta=1.E-11,
    13	 tAlpha=2.E-4,
    14	 sBeta =0.,
    15	 gravity=9.81,
    16	 rigidLid=.FALSE.,
    17	 implicitFreeSurface=.TRUE.,
    18	 eosType='LINEAR',
    19	 readBinaryPrec=64,
    20	 &
    21	# Elliptic solver parameters
    22	 &PARM02
    23	 cg2dMaxIters=1000,
    24	 cg2dTargetResidual=1.E-13,
    25	 &
    26	# Time stepping parameters
    27	 &PARM03
    28	 startTime=0.,
    29	 endTime=12000., 
    30	 deltaTmom=1200.0,
    31	 deltaTtracer=1200.0,
    32	 abEps=0.1,
    33	 pChkptFreq=17000.0,
    34	 chkptFreq=0.0,
    35	 dumpFreq=2592000.0,
    36	 &
    37	# Gridding parameters
    38	 &PARM04
    39	 usingCartesianGrid=.FALSE.,
    40	 usingSphericalPolarGrid=.TRUE.,
    41	 phiMin=0.,
    42	 delX=60*1.,
    43	 delY=60*1.,
    44	 delZ=500.,500.,500.,500.,
    45	 &
    46	 &PARM05
    47	 bathyFile='topog.box',
    48	 hydrogThetaFile=,
    49	 hydrogSaltFile=,
    50	 zonalWindFile='windx.sin_y',
    51	 meridWindFile=,
    52	 &
\end{verbatim}

\end{small}

\subsubsection{File {\it input/data.pkg}}
\label{www:tutorials}

This file uses standard default values and does not contain
customizations for this experiment.

\subsubsection{File {\it input/eedata}}
\label{www:tutorials}

This file uses standard default values and does not contain
customizations for this experiment.

\subsubsection{File {\it input/thetaPol.bin}}
\label{www:tutorials}

The {\it input/thetaPol.bin} file specifies a three-dimensional ($x,y,z$) 
map of initial values of $\theta$ in degrees Celsius.

\subsubsection{File {\it input/bathyPol.bin}}
\label{www:tutorials}


The {\it input/bathyPol.bin} file specifies a two-dimensional ($x,y$) 
map of depth values. For this experiment values are either
$0m$ or {\bf -delZ}m, corresponding respectively to outside or inside of
the tank. The file contains a raw binary stream of data that is enumerated
in the same way as standard MITgcm two-dimensional, horizontal arrays.

\subsubsection{File {\it code/SIZE.h}}
\label{www:tutorials}

Two lines are customized in this file for the current experiment

\begin{itemize}

\item Line 39, 
\begin{verbatim} sNx=120, \end{verbatim} this line sets
the lateral domain extent in grid points for the
axis aligned with the x-coordinate.

\item Line 40, 
\begin{verbatim} sNy=31, \end{verbatim} this line sets
the lateral domain extent in grid points for the
axis aligned with the y-coordinate.

\end{itemize}

\begin{small}
\begin{verbatim}
     1	C     /==========================================================\
     2	C     | SIZE.h Declare size of underlying computational grid.    |
     3	C     |==========================================================|
     4	C     | The design here support a three-dimensional model grid   |
     5	C     | with indices I,J and K. The three-dimensional domain     |
     6	C     | is comprised of nPx*nSx blocks of size sNx along one axis|
     7	C     | nPy*nSy blocks of size sNy along another axis and one    |
     8	C     | block of size Nz along the final axis.                   |
     9	C     | Blocks have overlap regions of size OLx and OLy along the|
    10	C     | dimensions that are subdivided.                          |
    11  C     \==========================================================/
    12  C     Voodoo numbers controlling data layout.
    13  C     sNx - No. X points in sub-grid.
    14  C     sNy - No. Y points in sub-grid.
    15  C     OLx - Overlap extent in X.
    16  C     OLy - Overlat extent in Y.
    17  C     nSx - No. sub-grids in X.
    18  C     nSy - No. sub-grids in Y.
    19  C     nPx - No. of processes to use in X.
    20  C     nPy - No. of processes to use in Y.
    21  C     Nx  - No. points in X for the total domain.
    22  C     Ny  - No. points in Y for the total domain.
    23  C     Nr  - No. points in Z for full process domain.
    24        INTEGER sNx
    25        INTEGER sNy
    26        INTEGER OLx
    27        INTEGER OLy
    28        INTEGER nSx
    29        INTEGER nSy
    30        INTEGER nPx
    31	      INTEGER nPy
    32	      INTEGER Nx
    33	      INTEGER Ny
    34	      INTEGER Nr
    35	      PARAMETER (
    36	     &           sNx =  64,
    37	     &           sNy =  64,
    38	     &           OLx =   3,
    39	     &           OLy =   3,
    40	     &           nSx =   1,
    41	     &           nSy =   1,
    42	     &           nPx =   1,
    43	     &           nPy =   1,
    44	     &           Nx  = sNx*nSx*nPx,
    45	     &           Ny  = sNy*nSy*nPy,
    46	     &           Nr  =  20)

    47	C     MAX_OLX  - Set to the maximum overlap region size of any array
    48	C     MAX_OLY    that will be exchanged. Controls the sizing of exch
    49	C                routine buufers.
    50	      INTEGER MAX_OLX
    51	      INTEGER MAX_OLY
    52	      PARAMETER ( MAX_OLX = OLx,
    53	     &            MAX_OLY = OLy )

\end{verbatim}
\end{small}

\subsubsection{File {\it code/CPP\_OPTIONS.h}}
\label{www:tutorials}

This file uses standard default values and does not contain
customizations for this experiment.


\subsubsection{File {\it code/CPP\_EEOPTIONS.h}}
\label{www:tutorials}

This file uses standard default values and does not contain
customizations for this experiment.

