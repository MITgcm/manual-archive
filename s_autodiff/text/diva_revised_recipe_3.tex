
\section{Adjoint dump \& restart -- divided adjoint (DIVA)
\label{sec_ad_diva}}
\begin{rawhtml}
<!-- CMIREDIR:sec_ad_diva: -->
\end{rawhtml}

{\it Patrick Heimbach \& Geoffrey Gebbie, MIT/EAPS, 07-Mar-2003}

{\bf 
NOTE: \\
THIS SECTION IS SUBJECT TO CHANGE.
IT REFERS TO TAF-1.4.26.

Previous TAF versions are incomplete and have problems 
with both TAF options '-pure' and '-mpi'.

The code which is tuned to the DIVA implementation 
of this TAF version
is {\it checkpoint50} (MITgcm) and {\it ecco\_c50\_e28} (ECCO).
}

\subsection{Introduction}

Most high performance computing (HPC) centres require the use
of batch jobs for code execution. 
Limits in maximum available CPU time and memory may prevent
the adjoint code execution from fitting into any of the available
queues. This presents a serious limit for large scale /
long time adjoint ocean and climate model integrations.
The MITgcm itself enables the split of the total model 
integration into sub-intervals through standard dump/restart 
of/from the full model state.
For a similar procedure to run in reverse mode,
the adjoint model requires, in addition to the model state, 
the adjoint model state,
i.e. all variables with derivative information
which are needed in an adjoint restart.
This adjoint dump \& restart is also termed 'divided adjoint (DIVA).

For this to work in conjunction with automatic differentiation, 
an AD tool needs to perform the following tasks:
%
\begin{enumerate}
%
\item
%
identify an adjoint state, i.e. those sensitivities whose
accumulation is interrupted by a dump/restart and which influence
the outcome of the gradient.
Ideally, this state consists of 
%
\begin{itemize}
%
\item
the adjoint of the model state,
%
\item
the adjoint of other intermediate results (such as control variables,
cost function contributions, etc.)
%
\item
bookkeeping indices (such as loop indices, etc.)
%
\end{itemize}
%
\item
%
generate code for storing and reading adjoint state variables
%
\item
generate code for bookkeeping , i.e. maintaining a file
with index information
%
\item
generate a suitable adjoint loop to propagate adjoint values
for dump/restart with a minimum overhad of adjoint intermediate
values.
%
\end{enumerate}

TAF (but not TAMC!)
generates adjoint code which performs the above specified
tasks. It is closely tied to the adjoint multi-level checkpointing.
The adjoint state is dumped (and restarted) at each step of the
outermost checkpointing level and adjoint intergration is performed
over one outermost checkpointing interval.
Prior to the adjoint computations, a full foward sweep is performed to 
generate the outermost (forward state) tapes and to calculate
the cost function.
In the current implementation, the forward sweep is
immediately followed by the first adjoint leg.
Thus, in theory, the following steps are performed (automatically)
%
\begin{itemize}
%
\item {\bf 1st model call:} \\
This is the case if file {\tt costfinal} does {\it not} exist.
S/R {\tt mdthe\_main\_loop} is called.
%
\begin{enumerate}
%
\item
calculate forward trajectory and dump model state after each
outermost checkpointing interval to files {\tt tapelev3}
%
\item
calculate cost function {\tt fc} and write it to file
{\tt costfinal}
%
\end{enumerate}
%
\item{\bf 2nd and all remaining model call:} \\
This is the case if file {\tt costfinal} {\it does} exist.
S/R {\tt adthe\_main\_loop} is called.
%
\begin{enumerate}
%
\item
(forward run and cost function call is avoided
since all values are known)
%
\begin{itemize}
%
\item
if 1st adjoint leg: \\
create index file {\tt divided.ctrl} which contains
info on current checkpointing index $ilev3$
%
\item
if not $i$-th adjoint leg: \\
adjoint picks up at $ilev3 = nlev3-i+1$ and runs to $nlev3 - i$
%
\end{itemize}
%
\item
perform adjoint leg from $nlev3-i+1$ to $nlev3 - i$
%
\item
dump adjoint state to file {\tt snapshot}
%
\item
dump index file {\tt divided.ctrl} for next adjoint leg
%
\item
in the last step the gradient is written.
%
\end{enumerate}
%
\end{itemize}

A few modififications were performed in the forward code,
obvious ones such as adding the corresponding TAF-directive
at the appropriate place, and less obvious ones
(avoid some re-initializations, when in an intermediate
adjoint integration interval).

[For TAF-1.4.20 a number of hand-modifications were necessary
to compensate for TAF bugs.
Since we refer to TAF-1.4.26 onwards,
these modifications are not documented here].

\subsection{Recipe 1: single processor}


\begin{enumerate}

\item
In {\tt ECCO\_CPPOPTIONS.h} set:
%
{\footnotesize
\begin{verbatim}
      #define ALLOW_DIVIDED_ADJOINT
      #undef  ALLOW_DIVIDED_ADJOINT_MPI
\end{verbatim}
}

\item
Generate adjoint code. 
Using the TAF option '{\tt -pure}', two codes are generated:
%
\begin{itemize}
%
\item {\tt mdthe\_main\_loop}: \\
Is responsible for the forward trajectory, storing of outermost
checkpoint levels to file, computation of cost function, and
storing of cost function to file (1st step).
%
\item {\tt adthe\_main\_loop}: \\
Is responsible for computing one adjoint leg, dump adjoint state
to file and write index info to file (2nd and consecutive steps).

    for adjoint code generation, e.g. add '{\tt -pure}' to
    TAF option list
{\footnotesize
\begin{verbatim}
    make adtaf
\end{verbatim}
}
%

\item
One modification needs to be made to adjoint codes in
S/R adecco\_the\_main\_loop:

There's a remaining issue with the '{\tt -pure}' option.
The '{\tt call ad...}' 
between '{\tt call ad...}' and the read of the {\tt snapshot} file
should be called only in the firt adjoint leg between
$nlev3$ and $nlev3-1$.
In the ecco-branch, the following lines should be 
bracketed by an {\tt if (idivbeg .GE. nchklev\_3) then}, thus:

{\footnotesize
\begin{verbatim}

...
      xx_psbar_mean_dummy = onetape_xx_psbar_mean_dummy_3h(1)
      xx_tbar_mean_dummy = onetape_xx_tbar_mean_dummy_4h(1)
      xx_sbar_mean_dummy = onetape_xx_sbar_mean_dummy_5h(1)
      call barrier( mythid )
cAdd(
      if (idivbeg .GE. nchklev_3) then
cAdd)

      call adcost_final( mythid )
      call barrier( mythid )
      call adcost_sst( mythid )
      call adcost_ssh( mythid )
      call adcost_hyd( mythid )
      call adcost_averagesfields( mytime,myiter,mythid )
      call barrier( mythid )
cAdd(
      endif
cAdd)

C----------------------------------------------
C read snapshot
C----------------------------------------------
      if (idivbeg .lt. nchklev_3) then
        open(unit=77,file='snapshot',status='old',form='unformatted',
     $iostat=iers)
...

\end{verbatim}
}

For the main code, in all likelihood the block which needs to
be bracketed consists of {\tt adcost\_final} only.

\item
Now the code can be copied as usual to {\tt adjoint\_model.F}
and then be compiled:
%
{\footnotesize
\begin{verbatim}
    make adchange
    then compile
\end{verbatim}
}

\end{itemize}

\end{enumerate}

\subsection{Recipe 2: multi processor (MPI)}


\begin{enumerate}

\item
On the machine where you execute the code
(most likely not the machine where you run TAF)
find the includes directory for MPI containing {\tt mpif.h}.
Either copy {\tt mpif.h} to the machine where you generate the
{\tt .f} files before TAF-ing, or add the path to the includes
directory to you genmake {\tt platform} setup,
TAF needs some MPI parameter settings 
(essentially {\tt mpi\_comm\_world} and {\tt mpi\_integer})
to incorporate those in the adjoint code.

\item
In {\tt ECCO\_CPPOPTIONS.h} set
%
{\footnotesize
\begin{verbatim}
      #define ALLOW_DIVIDED_ADJOINT
      #define ALLOW_DIVIDED_ADJOINT_MPI
\end{verbatim}
}
%
This will include the header file {\tt mpif.h}
into the top level routine for TAF.

\item
Add the TAF option '{\tt -mpi}' to the TAF argument list in the makefile.

\item
Follow the same steps as in {\bf Recipe 1} (previous section).

\end{enumerate}

That's it. Good luck \& have fun.

