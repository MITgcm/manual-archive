
%**********************************************************************
\section{The gradient check package}
\label{sect:ad_gradient_check}
\label{sec_ad_radient_check}
\begin{rawhtml}
<!-- CMIREDIR:sec_ad_radient_check: -->
\end{rawhtml}
%**********************************************************************

An indispensable test to validate the gradient computed
via the adjoint is a comparison against finite difference
gradients.
The gradient check package {\it pkg/grdchk} enables such tests
in a straightforward and easy manner.
The driver routine {\it grdchk\_main} is called from
{\it the\_model\_main} after the gradient has been computed
via the adjoint model (cf. flow chart ???).

The gradient check proceeds as follows:
The $i-$th component of the gradient $ (\nabla _{u}{\cal J}^T)_i $
is compared with  the following finite-difference gradient:
\[
\left(\nabla _{u}{\cal J}^T  \right)_i \quad \text{ vs. } \quad
\frac{\partial {\cal J}}{\partial u_i} \, = \,
\frac{ {\cal J}(u_i + \epsilon) - {\cal J}(u_i)}{\epsilon}
\]
A gradient check at point $u_i$ may generally considered to be successful
if the deviation of the ratio between the adjoint and the 
finite difference gradient from unity is less than 1 percent,
\[
1 \, - \, 
\frac{({\rm grad}{\cal J})_i (\text{adjoint})}
{({\rm grad}{\cal J})_i (\text{finite difference})} \, < 1 \%
\]

\subsection{Code description}
~

\subsection{Code configuration}
%
The relevant CPP precompile options are set
in the following files:
%
\begin{itemize}
%
\item {\it .genmakerc} \\
option {\tt grdchk} is added to the {\bf enable list}
(alternatively, {\it genmake} may be invoked with the
option {\tt -enable=grdchk}).
%
\item {\it CPP\_OPTIONS.h} \\
Together with the flag 
{\bf ALLOW\_ADJOINT\_RUN}, define the flag
{\bf ALLOW\_GRADIENT\_CHECK}.
%
\end{itemize}

The relevant runtime flags are set in the files
%
\begin{itemize}
%
\item {\it data.pkg} \\
Set {\bf useGrdchk = .TRUE.}
%
\item {\it data.grdchk}
%
\begin{itemize}
%
\item {\bf grdchk\_eps}
~
\item {\bf nbeg}
~
\item {\bf nstep}
~
\item {\bf nend}
~
\item {\bf grdchkvarindex}
~
%
\end{itemize}
%
\end{itemize}


\begin{figure}[h!]

{\scriptsize
\begin{verbatim}

   the_model_main
   |
   |-- ctrl_unpack
   |-- adthe_main_loop            - unperturbed cost function and
   |-- ctrl_pack                    adjoint gradient are computed here
   |
   |-- grdchk_main
       |
       |-- grdchk_init
       |-- do icomp=...           - loop over control vector elements
           |
           |-- grdchk_loc         - determine location of icomp on grid
           |
           |-- grdchk_getxx       - get control vector component from file
           |                        perturb it and write back to file
           |-- grdchk_getadxx     - get gradient component calculated 
           |                        via adjoint
           |-- the_main_loop      - forward run and cost evaluation
           |                        with perturbed control vector element
           |-- calculate ratio of adj. vs. finite difference gradient
           |
           |-- grdchk_setxx       - Reset control vector element
           |
           |-- grdchk_print       - print results

\end{verbatim}
}

\caption{~}
\label{fig:grdchkflow}
\end{figure}


