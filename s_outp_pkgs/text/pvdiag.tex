\section{Potential vorticity Matlab Toolbox}
\label{sec:diag:pv}
\begin{rawhtml}
%<!-- CMIREDIR:pvdiag_matlab: -->
\end{rawhtml}

Author: Guillaume Maze

\bigskip


%%%%%%%%%%%%%%%%%%%%%%%%%%
\subsection{Introduction}

\noindent
This section of the documentation describes a Matlab package that aims to provide useful
routines to compute vorticity fields (relative, potential and planetary) and its related
components. This is an offline computation. It was developed to be used in mode water
studies, so that it comes with other related routines, in particular ones computing
surface vertical potential vorticity fluxes.

%%%%%%%%%%%%%%%%%%%%%%%%%%
\subsection{Equations}

\subsubsection{Potential Vorticity}
\noindent
The package computes the three components of the relative vorticity defined by:
\begin{eqnarray}
  \omega &= \nabla \times {\bf U} = \left( \begin{array}{c} 
      \omega_x\\
      \omega_y\\
      \zeta
  \end{array}\right)
     \simeq &\left( \begin{array}{c} 
      -\frac{\partial v}{\partial z}\\
      -\frac{\partial u}{\partial z}\\
      \frac{\partial v}{\partial x} - \frac{\partial u}{\partial y}
  \end{array}\right) \label{sec:diag:pv:eq1}
\end{eqnarray}
where we omitted (like all across the package) the vertical velocity component.

The package then computes the potential vorticity as:
\begin{eqnarray}
Q &=& -\frac{1}{\rho} \omega\cdot\nabla\sigma_\theta\\
Q &=& -\frac{1}{\rho}\left(\omega_x \frac{\partial \sigma_\theta}{\partial x} + 
\omega_y \frac{\partial \sigma_\theta}{\partial y} + 
\left(f+\zeta\right) \frac{\partial \sigma_\theta}{\partial z}\right) \label{sec:diag:pv:eq2}
\end{eqnarray}
where $\rho$ is the density, $\sigma_\theta$ is the potential density (both eventually
computed by the package) and $f$ is the Coriolis parameter.

The package is also able to compute the simpler planetary vorticity as:
\begin{eqnarray}
splQ &=& -\frac{f}{\rho}\frac{\sigma_\theta}{\partial z} \label{sec:diag:pv:eq3}
\end{eqnarray}

\subsubsection{Surface vertical potential vorticity fluxes}
These quantities are useful in mode water studies because of the impermeability theorem
which states that for a given potential density layer (embedding a mode water), the
integrated PV only changes through surface input/output.

Vertical PV fluxes due to frictional and diabatic processes are given by:
\begin{eqnarray}
  J^B_z &=& -\frac{f}{h}\left( \frac{\alpha Q_{net}}{C_w}-\rho_0 \beta S_{net}\right) \label{sec:diag:pv:eq14a}\\
  J^F_z &=& \frac{1}{\rho\delta_e} \vec{k}\times\tau\cdot\nabla\sigma_m \label{sec:diag:pv:eq15a}
\end{eqnarray}
These components can be computed with the package. Details on the variables definition and
the way these fluxes are derived can be found in section
\ref{sec:diag:pv:notes-flux-form}.

\vspace{.5cm}
\noindent
We now give some simple explanations about these fluxes and how they can reduce the PV
level of an oceanic potential density layer.

\paragraph{Diabatic process}
Let's take the PV flux due to surface buoyancy forcing from Eq.\ref{sec:diag:pv:eq14a} and
simplify it as:
\begin{eqnarray}
  J^B_z &\simeq& -\frac{\alpha f}{hC_w} Q_{net}
\end{eqnarray}
When the net surface heat flux $Q_{net}$ is upward i.e. negative and cooling the ocean
(buoyancy loss), surface density will increase, triggering mixing which reduces the
stratification and then the PV.
\begin{eqnarray}
  Q_{net} &<& 0 \,\,\,\hbox{(upward, cooling)} \nonumber \\
  J^B_z   &>& 0 \,\,\,\hbox{(upward)} \nonumber \\
  -\rho^{-1}\nabla\cdot J^B_z &<& 0 \,\,\, \hbox{(PV flux divergence)} \nonumber \\
  PV &\searrow& \hbox{where $Q_{net}<0$}\nonumber 
\end{eqnarray}


\paragraph{Frictional process: "Down-front" wind-stress}
Now let's take the PV flux due to the "wind-driven buoyancy flux" from
Eq.\ref{sec:diag:pv:eq15a} and simplify it as:
\begin{eqnarray}
  J^F_z &=& \frac{1}{\rho\delta_e} \left( \tau_x\frac{\partial \sigma}{\partial y} - \tau_y\frac{\partial \sigma}{\partial x} \right) \\
  J^F_z &\simeq& \frac{1}{\rho\delta_e} \tau_x\frac{\partial \sigma}{\partial y} \nonumber
\end{eqnarray}
When the wind is blowing from the east above the Gulf Stream (a region of high meridional
density gradient), it induces an advection of dense water from the northern side of the GS
to the southern side through Ekman currents. Then, it induces a "wind-driven" buoyancy
lost and mixing which reduces the stratification and the PV.
\begin{eqnarray}
 \vec{k}\times\tau\cdot\nabla\sigma &>& 0 \,\,\, \hbox{("Down-front" wind)} \nonumber \\
 J^F_z &>& 0 \,\,\, \hbox{(upward)} \nonumber \\
  -\rho^{-1}\nabla\cdot J^F_z &<& 0 \,\,\, \hbox{(PV flux divergence)} \nonumber \\
  PV &\searrow& \hbox{where $\vec{k}\times\tau\cdot\nabla\sigma>0$}\nonumber 
\end{eqnarray}


\paragraph{Diabatic versus frictional processes}
A recent debate in the community arose about the relative role of these processes.  Taking
the ratio of Eq.\ref{sec:diag:pv:eq14a} and Eq.\ref{sec:diag:pv:eq15a} leads to:
\begin{eqnarray}
  \frac{J^F_z}{J^B_Z} &=& \frac{ \frac{1}{\rho\delta_e} \vec{k}\times\tau\cdot\nabla\sigma }
  {-\frac{f}{h}\left( \frac{\alpha Q_{net}}{C_w}-\rho_0 \beta S_{net}\right)} \\
  &\simeq& \frac{Q_{Ek}/\delta_e}{Q_{net}/h} \nonumber
\end{eqnarray}
where appears the lateral heat flux induced by Ekman currents:
\begin{eqnarray}
  Q_{Ek} &=& -\frac{C_w}{\alpha\rho f}\vec{k}\times\tau\cdot\nabla\sigma 
  \nonumber \\
  &=& \frac{C_w}{\alpha}\delta_e\vec{u_{Ek}}\cdot\nabla\sigma
\end{eqnarray}
which can be computed with the package.  In the aim of comparing both processes, it will be
useful to plot surface net and lateral Ekman-induced heat fluxes together with PV fluxes.




%%%%%%%%%%%%%%%%%%%%%%%%%% 
\subsection{Key routines}

\begin{itemize}
\item {\bf {\ttfamily A\_compute\_potential\_density.m}}: Compute the potential density
  field. Requires the potential temperature and salinity (either total or anomalous) and
  produces one output file with the potential density field (file prefix is {\ttfamily
    SIGMATHETA}). The routine uses {\ttfamily densjmd95.m} a Matlab counterpart of the
  MITgcm built-in function to compute the density.
\item {\bf {\ttfamily B\_compute\_relative\_vorticity.m}}: Compute the three components of
  the relative vorticity defined in Eq.~(\ref{sec:diag:pv:eq1}). Requires the two
  horizontal velocity components and produces three output files with the three components
  (files prefix are {\ttfamily OMEGAX}, {\ttfamily OMEGAY} and {\ttfamily ZETA}).
\item {\bf {\ttfamily C\_compute\_potential\_vorticity.m}}: Compute the potential
  vorticity without the negative ratio by the density. Two options are possible in order
  to compute either the full component (term into parenthesis in
  Eq.~\ref{sec:diag:pv:eq2}) or the planetary component ($f\partial_z\sigma_\theta$ in
  Eq.~\ref{sec:diag:pv:eq3}). Requires the relative vorticity components and the potential
  density, and produces one output file with the potential vorticity (file prefix is
  {\ttfamily PV} for the full term and {\ttfamily splPV} for the planetary component).
\item {\bf {\ttfamily D\_compute\_potential\_vorticity.m}}: Load the field computed with
  \mbox{{\ttfamily C\_comp...}} and divide it by $-\rho$ to obtain the
  correct potential vorticity. Require the density field and after loading, overwrite the
  file with prefix {\ttfamily PV} or {\ttfamily splPV}.
\item {\bf {\ttfamily compute\_density.m}}: Compute the density $\rho$ from the potential
  temperature and the salinity fields.
\item {\bf {\ttfamily compute\_JFz.m}}: Compute the surface vertical PV flux due to
  frictional processes. Requires the wind stress components, density, potential density
  and Ekman layer depth (all of them, except the wind stress, may be computed with the
  package), and produces one output file with the PV flux $J^F_z$ (see
  Eq.~\ref{sec:diag:pv:eq15a}) and with {\ttfamily JFz} as a prefix.
\item {\bf {\ttfamily compute\_JBz.m}}: Compute the surface vertical PV flux due to
  diabatic processes as:
\begin{eqnarray}
  J^B_z &=& -\frac{f}{h}\frac{\alpha Q_{net}}{C_w} \nonumber
\end{eqnarray}
which is a simplified version of the full expression given in
Eq.~(\ref{sec:diag:pv:eq14a}).  Requires the net surface heat flux and the mixed layer depth
(of which an estimation can be computed with the package), and produces one output file
with the PV flux $J^B_z$ and with {\ttfamily JBz} as a prefix.
\item {\bf {\ttfamily compute\_QEk.m}}: Compute the horizontal heat flux due to Ekman
  currents from the PV flux induced by frictional forces as:
\begin{eqnarray}
 Q_{Ek} &=& - \frac{C_w \delta_e}{\alpha f}J^F_z\nonumber
\end{eqnarray}
Requires the PV flux due to frictional forces and the Ekman layer depth, and produces one
output with the heat flux and with {\ttfamily QEk} as a prefix.
\item {\bf {\ttfamily eg\_main\_getPV}}: A complete example of how to set up a master
  routine able to compute everything from the package.
\end{itemize}

%%%%%%%%%%%%%%%%%%%%%%%%%%
\subsection{Technical details}

\subsubsection{File name}
\noindent
A file name is formed by three parameters which need to be set up as global variables in Matlab
before running any routines. They are:
\begin{itemize}
\item the prefix, ie the variable name ({\ttfamily netcdf\_UVEL} for example). This
  parameter is specified in the help section of all diagnostic routines.
\item {\ttfamily netcdf\_domain}: the geographical domain.
\item {\ttfamily netcdf\_suff}: the netcdf extension (nc or cdf for example).
\end{itemize}
Then, for example, if the calling Matlab routine had set up:
\begin{verbatim}
global netcdf_THETA netcdf_SALTanom netcdf_domain netcdf_suff
netcdf_THETA    = 'THETA';
netcdf_SALTanom = 'SALT';
netcdf_domain   = 'north_atlantic';
netcdf_suff     = 'nc';
\end{verbatim}
the routine {\ttfamily A\_compute\_potential\_density.m} to compute the potential density
field, will look for the files:
\begin{verbatim}
THETA.north_atlantic.nc
SALT.north_atlantic.nc
\end{verbatim}
and the output file will automatically be: {\ttfamily SIGMATHETA.north\_atlantic.nc}.

Otherwise indicated, output file prefix cannot be changed.

\subsubsection{Path to file}
\noindent
All diagnostic routines look for input files in a subdirectory (relative to the Matlab
routine directory) called {\ttfamily ./netcdf-files} which in turn, is supposed to contain
subdirectories for each set of fields. For example, computing the potential density for
the timestep 12H00 02/03/2005 will require a
subdirectory with the potential temperature and salinity files like:
\begin{verbatim}
./netcdf-files/200501031200/THETA.north_atlantic.nc
./netcdf-files/200501031200/SALT.north_atlantic.nc
\end{verbatim}
\noindent
The output file {\ttfamily SIGMATHETA.north\_atlantic.nc} will be created in {\ttfamily
  ./netcdf-files/200501031200/}.
\noindent
All diagnostic routines take as argument the name of the timestep subdirectory into
{\ttfamily ./netcdf-files}.

\subsubsection{Grids}
\noindent
With MITgcm numerical outputs, velocity and tracer fields may not be defined on the same
grid. Usually, UVEL and VVEL are defined on a C-grid but when interpolated from a
cube-sphere simulation they are defined on a A-grid. When it is needed, routines allow
to set up a global variable which define the grid to use.


%%%%%%%%%%%%%%%%%%%%%%%%%% 
\subsection{Notes on the flux form of the PV equation and vertical PV fluxes}
\label{sec:diag:pv:notes-flux-form}

\subsubsection{Flux form of the PV equation}
The conservative flux form of the potential vorticity equation is:
\begin{eqnarray}
 \frac{\partial \rho Q}{\partial t} + \nabla \cdot \vec{J} &=& 0 \label{sec:diag:pv:eq4}
\end{eqnarray}
where the potential vorticity $Q$ is given by the Eq.\ref{sec:diag:pv:eq2}.

The generalized flux vector of potential vorticity is:
\begin{eqnarray}
 \vec{J} &=& \rho Q \vec{u} + \vec{N_Q}
\end{eqnarray}
which allows to rewrite Eq.\ref{sec:diag:pv:eq4} as:
\begin{eqnarray}
 \frac{DQ}{dt} &=& -\frac{1}{\rho}\nabla\cdot\vec{N_Q} \label{sec:diag:pv:eq5}
\end{eqnarray}
where the nonadvective PV flux $\vec{N_Q}$ is given by:
\begin{eqnarray}
  \vec{N_Q} &=& -\frac{\rho_0}{g}B\vec{\omega_a} + \vec{F}\times\nabla\sigma_\theta 
                \label{sec:diag:pv:eq6}
\end{eqnarray} 

Its first component is linked to the buoyancy forcing\footnote{
Note that introducing B into Eq.\ref{sec:diag:pv:eq6} yields to:
\begin{eqnarray}
  \vec{N_Q} &=& \omega_a \frac{D \sigma_\theta}{dt} + \vec{F}\times\nabla\sigma_\theta
\end{eqnarray}}:
\begin{eqnarray}
 B &=& -\frac{g}{\rho_o}\frac{D \sigma_\theta}{dt} 
\end{eqnarray}
and the second one to the nonconservative body forces per unit mass:
\begin{eqnarray}
 \vec{F} &=& \frac{D \vec{u}}{dt} + 2\Omega\times\vec{u} + \nabla p 
\end{eqnarray}

\subsubsection{Determining the PV flux at the ocean's surface}
In the context of mode water study, we're particularly interested in how the PV may be
reduced by surface PV fluxes because a mode water is characterised by a low PV level.
Considering the volume limited by two $iso-\sigma_\theta$, PV
flux is limited to surface processes and then vertical component of $\vec{N_Q}$.  It is
supposed that $B$ and $\vec{F}$ will only be nonzero in the mixed layer (of depth $h$ and
variable density $\sigma_m$) exposed to mechanical forcing by the wind and buoyancy fluxes
through the ocean's surface.

Given the assumption of a mechanical forcing confined to a thin surface Ekman layer (of
depth $\delta_e$, eventually computed by the package) and of hydrostatic and geostrophic
balances, we can write:
\begin{eqnarray}
  \vec{u_g} &=& \frac{1}{\rho f} \vec{k}\times\nabla p \\
  \frac{\partial p_m}{\partial z} &=& -\sigma_m g \\ 
  \frac{\partial \sigma_m}{\partial t} + \vec{u}_m\cdot\nabla\sigma_m &=& -\frac{\rho_0}{g}B \label{sec:diag:pv:eq7}
\end{eqnarray} 
where:
\begin{eqnarray}
  \vec{u}_m &=& \vec{u}_g + \vec{u}_{Ek} + o(R_o) \label{sec:diag:pv:eq8}
\end{eqnarray} 
is the full velocity field composed by the geostrophic current $\vec{u}_g$ and the Ekman
drift:
\begin{eqnarray}
  \vec{u}_{Ek} &=& -\frac{1}{\rho f}\vec{k}\times\frac{\partial \tau}{\partial z} \label{sec:diag:pv:eq9}
\end{eqnarray} 
(where $\tau$ is the wind stress) and last by other ageostrophic components of $o(R_o)$
which are neglected.

Partitioning the buoyancy forcing as:
\begin{eqnarray}
  B &=& B_g + B_{Ek} \label{sec:diag:pv:eq10}
\end{eqnarray} 
and using Eq.\ref{sec:diag:pv:eq8} and Eq.\ref{sec:diag:pv:eq9}, the Eq.\ref{sec:diag:pv:eq7} becomes:
\begin{eqnarray}
  \frac{\partial \sigma_m}{\partial t} + \vec{u}_g\cdot\nabla\sigma_m &=& -\frac{\rho_0}{g} B_g
\end{eqnarray}   
revealing the "wind-driven buoyancy forcing":
\begin{eqnarray}
  B_{Ek} &=& \frac{g}{\rho_0}\frac{1}{\rho f}\left(\vec{k}\times\frac{\partial \tau}{\partial z}\right)\cdot\nabla\sigma_m
\end{eqnarray} 
Note that since:
\begin{eqnarray}
  \frac{\partial B_g}{\partial z} &=& \frac{\partial}{\partial z}\left(-\frac{g}{\rho_0}\vec{u_g}\cdot\nabla\sigma_m\right)
  = -\frac{g}{\rho_0}\frac{\partial \vec{u_g}}{\partial z}\cdot\nabla\sigma_m 
  = 0
\end{eqnarray}
$B_g$ must be uniform throughout the depth of the mixed layer and then being related to
the surface buoyancy flux by integrating Eq.\ref{sec:diag:pv:eq10} through the mixed layer:
\begin{eqnarray}
  \int_{-h}^0B\,dz &=\, hB_g + \int_{-h}^0B_{Ek}\,dz  \,=& \mathcal{B}_{in} \label{sec:diag:pv:eq11}
\end{eqnarray}                   
where $\mathcal{B}_{in}$ is the vertically integrated surface buoyancy (in)flux:
\begin{eqnarray}
  \mathcal{B}_{in} &=& \frac{g}{\rho_o}\left( \frac{\alpha Q_{net}}{C_w} - \rho_0\beta S_{net}\right)  
%\label{sec:diag:pv:eq12}
\end{eqnarray}  
with $\alpha\simeq 2.5\times10^{-4}\, K^{-1}$ the thermal expansion coefficient (computed
by the package otherwise), $C_w=4187J.kg^{-1}.K^{-1}$ the specific heat of seawater,
$Q_{net}[W.m^{-2}]$ the net heat surface flux (positive downward, warming the ocean),
$\beta[PSU^{-1}]$ the saline contraction coefficient, and
$S_{net}=S*(E-P)[PSU.m.s^{-1}]$ the net freshwater surface flux with $S[PSU]$
the surface salinity and $(E-P)[m.s^{-1}]$ the fresh water flux.

Introducing the body force in the Ekman layer:
\begin{eqnarray}
  F_z &=& \frac{1}{\rho}\frac{\partial \tau}{\partial z}
\end{eqnarray}
the vertical component of Eq.\ref{sec:diag:pv:eq6} is:
\begin{eqnarray}
  \vec{N_Q}_z &=& -\frac{\rho_0}{g}(B_g+B_{Ek})\omega_z 
  + \frac{1}{\rho}
  \left( \frac{\partial \tau}{\partial z}\times\nabla\sigma_\theta \right)\cdot\vec{k}
  \nonumber \\
  &=& -\frac{\rho_0}{g}B_g\omega_z
  -\frac{\rho_0}{g}
  \left(\frac{g}{\rho_0}\frac{1}{\rho f}\vec{k}\times\frac{\partial \tau}{\partial z}
    \cdot\nabla\sigma_m\right)\omega_z
  + \frac{1}{\rho}
  \left( \frac{\partial \tau}{\partial z}\times\nabla\sigma_\theta \right)\cdot\vec{k}
  \nonumber \\
  &=& -\frac{\rho_0}{g}B_g\omega_z
  + \left(1-\frac{\omega_z}{f}\right)\left(\frac{1}{\rho}\frac{\partial \tau}{\partial z}
                \times\nabla\sigma_\theta \right)\cdot\vec{k}
\end{eqnarray}
and given the assumption that $\omega_z\simeq f$, the second term vanishes and we obtain:
\begin{eqnarray}
  \vec{N_Q}_z &=& -\frac{\rho_0}{g}f B_g %\label{sec:diag:pv:eq12}
\end{eqnarray}
Note that the wind-stress forcing does not appear explicitly here but is implicit in $B_g$
through Eq.\ref{sec:diag:pv:eq11}: the buoyancy forcing $B_g$ is determined by the
difference between the integrated surface buoyancy flux $\mathcal{B}_{in}$ and the
integrated "wind-driven buoyancy forcing":
\begin{eqnarray}
  B_g &=& \frac{1}{h}\left( \mathcal{B}_{in} - \int_{-h}^0B_{Ek}dz \right) \nonumber \\
  &=& \frac{1}{h}\frac{g}{\rho_0}\left( \frac{\alpha Q_{net}}{C_w} - \rho_0 \beta S_{net}\right)
  - \frac{1}{h}\int_{-h}^0 
  \frac{g}{\rho_0}\frac{1}{\rho f}\vec{k}\times \frac{\partial \tau}{\partial z} \cdot\nabla\sigma_m dz
  \nonumber \\
  &=& \frac{1}{h}\frac{g}{\rho_0}\left( \frac{\alpha Q_{net}}{C_w} - \rho_0 \beta S_{net}\right)  
  - \frac{g}{\rho_0}\frac{1}{\rho f \delta_e}\vec{k}\times\tau\cdot\nabla\sigma_m
\end{eqnarray}
Finally, from Eq.\ref{sec:diag:pv:eq6}, the vertical surface flux of PV may be written as:
\begin{eqnarray}
  \vec{N_Q}_z &=& J^B_z + J^F_z \label{sec:diag:pv:eq13} \\
  J^B_z &=& -\frac{f}{h}\left( \frac{\alpha Q_{net}}{C_w}-\rho_0 \beta S_{net}\right) \label{sec:diag:pv:eq14}\\
  J^F_z &=& \frac{1}{\rho\delta_e} \vec{k}\times\tau\cdot\nabla\sigma_m \label{sec:diag:pv:eq15}
\end{eqnarray}
