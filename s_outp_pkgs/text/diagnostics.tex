\section{Diagnostics--A Flexible Infrastructure}
\label{sec:pkg:diagnostics}
\begin{rawhtml}
<!-- CMIREDIR:package_diagnostics: -->
\end{rawhtml}

\subsection{Introduction}

\noindent
This section of the documentation describes the Diagnostics package
available within the GCM.  A large selection of model diagnostics is
available for output.  In addition to the diagnostic quantities
pre-defined in the GCM, there exists the option, in any experiment, to
define a new diagnostic quantity and include it as part of the
diagnostic output with the addition of a single subroutine call in the
routine where the field is computed. As a matter of philosophy, no
diagnostic is enabled as default, thus each user must specify the
exact diagnostic information required for an experiment.  This is
accomplished by enabling the specific diagnostic of interest cataloged
in the Diagnostic Menu (see Section \ref{sec:diagnostics:menu}).
Instructions for enabling diagnostic output and defining new
diagnostic quantities are found in Section
\ref{sec:diagnostics:usersguide} of this document.

\noindent
The Diagnostic Menu in this section of the manual is a listing of
diagnostic quantities available within the main (dynamics) part of the
GCM. Additional diagnostic quantities, defined within the different
GCM packages, are available and are listed in the diagnostic menu
subsection of the manual section associated with each relevant
package. Once a diagnostic is enabled, the GCM will continually
increment an array specifically allocated for that diagnostic whenever
the appropriate quantity is computed.  A counter is defined which
records how many times each diagnostic quantity has been incremented.
Several special diagnostics are included in the menu. Quantities
refered to as ``Counter Diagnostics'', are defined for selected
diagnostics which record the frequency at which a diagnostic is
incremented separately for each model grid location.  Quantitied
refered to as ``User Diagnostics'' are included in the menu to
facilitate defining new diagnostics for a particular experiment.

\subsection{Equations}
Not relevant.

\subsection{Key Subroutines and Parameters}
\label{sec:diagnostics:diagover}

\noindent
There are several utilities within the GCM available to users to
enable, disable, clear, write and retrieve model diagnostics, and may
be called from any routine.  The available utilities and the CALL
sequences are listed below.

\noindent
{\bf diagnostics\_fill}: This is the main user interface routine to
the diagnostics package. This routine will increment the specified
diagnostic quantity with a field sent through the argument list.

\begin{verbatim}
        call diagnostics_fill(
       &      arrayin, chardiag, levflg, nlevs,
       &      bibjflg, bi, bj, myThid )

     where:
        arrayin  = Field to increment diagnostics array
        chardiag = Character *8 expression for diag to fill
        levflg   = Integer flag for vertical levels:
                 = 0 indicates multiple (nlevs) levels incremented
                 = -1 indicates multiple (nlevs) levels incremented,
                     but in reverse vertical order
                     positive integer - WHICH single level to increment.
        nlevs    = indicates Number of levels to be filled (1 if levflg gt 0)
        bibjflg  = Integer flag to indicate instructions for bi bj loop
                 = 0 indicates that the bi-bj loop must be done here
                 = 1 indicates that the bi-bj loop is done OUTSIDE
                 = 2 indicates that the bi-bj loop is done OUTSIDE
                     AND that we have been sent a local array
                     AND that the array has the shadow regions
                 = 3 indicates that the bi-bj loop is done OUTSIDE
                     AND that we have been sent a local array
                     AND that the array has no shadow regions
        bi       = X-direction process(or) number - used for bibjflg=1-3
        bj       = Y-direction process(or) number - used for bibjflg=1-3
        myThid   = Current Thread number
\end{verbatim}

\noindent
{\bf diagnostics\_scale\_fill}:  This is a possible alternative routine to
diagnostics\_fill which performs the same functions and has an additional option
to scale the field before filling or raise the field to a power before filling.

\begin{verbatim}
        call diagnostics_scale_fill(
       &         arrayin, scalefactor, power, chardiag,
       &         levflg, nlevs, bibjflg, bi, bj, myThid )

     where all the arguments are the same as for diagnostics_fill with 
     the addition of:
        scalefactor = Factor to scale field
        power       = Integer power to which to raise the input field
\end{verbatim}

\noindent
{\bf diagnostics\_is\_on}: Function call to inquire whether a
diagnostic is active and can be incremented. Useful when there is a
computation that must be done locally before a call to
diagnostics\_fill. The call sequence:

\begin{verbatim}
        flag = diagnostics_is_on( diagName, myThid )

     where:
        diagName = Character *8 expression for diagnostic
        myThid   = Current Thread number
\end{verbatim}

\noindent
{\bf diagnostics\_get\_pointers}: This subroutine retrieves the value
of a the diagnostics pointers that other routines require as input -
can be useful if the diagnostics common blocks are not local to a
routine.

\begin{verbatim}
        call diagnostics_get_pointers( diagName, ipoint, jpoint, myThid )

     where:
        diagName = Character *8 expression of diagnostic
        ipoint   = Pointer into qdiag array - from idiag array in common
        jpoint   = Pointer into diagnostics menu - from jdiag array in common
        myThid   = Current Thread number
\end{verbatim}

\noindent
{\bf getdiag}: This subroutine retrieves the value of a model
diagnostic.  This routine is particulary useful when called from a
user output routine, although it can be called from any routine.  This
routine returns the time-averaged value of the diagnostic by dividing
the current accumulated diagnostic value by its corresponding counter.
This routine does not change the value of the diagnostic itself, that
is, it does not replace the diagnostic with its time-average.  The
calling sequence for this routine is givin by:

\begin{verbatim}
        call getdiag (lev, undef, qtmp, ipoint, mate, bi, bj, myThid)

     where:
        lev     = Model Level at which the diagnostic is desired
        undef   = Fill value to be used when diagnostic is undefined
        qtmp    = Time-Averaged Diagnostic Output
        ipoint  = Pointer into qdiag array - from idiag array in common
        mate    = Diagnostic mate pointer number
        bi      = X-direction process(or) number
        bj      = Y-direction process(or) number
        myThid  = Current Thread number
\end{verbatim}

\noindent 
{\bf diagnostics\_add2list}: This subroutine enables a diagnostic from
the Diagnostic Menu, meaning that space is allocated for the
diagnostic and the model routines will increment the diagnostic value
during execution.  This routine is the underlying interface routine
for defining a new permanent diagnostic in the main model or in a
package.  The calling sequence is:

\begin{verbatim}
       call diagnostics_add2list( diagNum,diagName, diagCode,
                                  diagUnits, diagTitle, myThid )

     where:
       diagNum   = Diagnostic number - Output from routine
       diagName  = character*8 diagnostic name
       diagCode  = character*16 parsing code (see description of gdiag below)
       diagUnits = Diagnostic units (character*16)
       diagTitle = Diagnostic title or long name (up to character*80)
       myThid    = Current Thread number
\end{verbatim}

\noindent
{\bf clrdiag}: This subroutine initializes the values of model
diagnostics to zero, and is particularly useful when called from user
output routines to re-initialize diagnostics during the run.  The
calling sequence is:

\begin{verbatim}
        call diagnostics_clrdiag (jpoint, ipoint, myThid)

     where:
        jpoint = Diagnostic number from menu - from jdiag array
        ipoint = Pointer number into qdiag array - from idiag array
        myThid = Current Thread number
\end{verbatim}

\noindent
The diagnostics are computed at various times and places within the
GCM. Because MITgcm may employ a staggered grid, diagnostics may be
computed at grid box centers, corners, or edges, and at the middle or
edge in the vertical. Some diagnostics are scalars, while others are
components of vectors. An internal array is defined which contains
information concerning various grid attributes of each diagnostic. The
GDIAG array (in common block {\tt diagnostics} in file {\tt
  DIAGNOSTICS.h}) is internally defined as a character*8 variable, and
is equivalenced to a character*1 "parse" array in output in order to
extract the grid-attribute information.  The GDIAG array is described
in Table \ref{tab:diagnostics:gdiag.tabl}.

\begin{table}
\caption{Diagnostic Parsing Array}
\label{tab:diagnostics:gdiag.tabl}
\begin{center}
\begin{tabular}{ |c|c|l| }
\hline
\multicolumn{3}{|c|}{\bf Diagnostic Parsing Array} \\ 
\hline
\hline
Array & Value & Description \\
\hline
  parse(1)   & $\rightarrow$ S &  Scalar Diagnostic                 \\ 
             & $\rightarrow$ U &  U-vector component Diagnostic     \\ 
             & $\rightarrow$ V &  V-vector component Diagnostic     \\ \hline
  parse(2)   & $\rightarrow$ U &  C-Grid U-Point                    \\ 
             & $\rightarrow$ V &  C-Grid V-Point                    \\ 
             & $\rightarrow$ M &  C-Grid Mass Point                 \\ 
             & $\rightarrow$ Z &  C-Grid Vorticity (Corner) Point   \\ \hline
  parse(3)   & $\rightarrow$ R &  Not Currently in Use              \\ \hline
  parse(4)   & $\rightarrow$ P &  Positive Definite Diagnostic      \\ \hline
  parse(5)   & $\rightarrow$ C &  Counter Diagnostic                \\
             & $\rightarrow$ D &  Disabled Diagnostic for output    \\ \hline
  parse(6-8) & $\rightarrow$ C &  3-digit integer corresponding to  \\
             &                 &  vector or counter component mate  \\ \hline
\end{tabular}
\addcontentsline{lot}{section}{Table 3:  Diagnostic Parsing Array}
\end{center}
\end{table}


\noindent
As an example, consider a diagnostic whose associated GDIAG parameter is equal
to ``UU  002''.  From GDIAG we can determine that this diagnostic is a 
U-vector component located at the C-grid U-point.
Its corresponding V-component diagnostic is located in Diagnostic \# 002.

\noindent
In this way, each Diagnostic in the model has its attributes (ie. vector or scalar,
C-grid location, etc.) defined internally.  The Output routines use this information 
in order to determine what type of transformations need to be performed.  Any 
interpolations are done at the time of output rather than during each model step.
In this way the User has flexibility in determining the type of gridded data which 
is output.

\subsection{Usage Notes}
\label{sec:diagnostics:usersguide}

\noindent
To use the diagnostics package, other than enabling it in packages.conf
and turning the usediagnostics flag in data.pkg to .TRUE., there are two
further steps the user must take to enable the diagnostics package for
output of quantities that are already defined in the GCM under an experiment's
configuration of packages.  A namelist must be supplied in the run directory 
called data.diagnostics, and the file DIAGNOSTICS\_SIZE.h must be included in the 
code directory.  The steps for defining a new (permanent or experiment-specific 
temporary) diagnostic quantity will be outlined later. 

\noindent The namelist will activate a user-defined list of diagnostics quantities 
to be computed, specify the frequency and type of output, the number of levels, and 
the name of all the separate output files. A sample data.diagnostics namelist file:

\begin{verbatim}
# Diagnostic Package Choices
 &diagnostics\_list
  frequency(1) = 86400., 
   levels(1,1) = 1., 
   fields(1,1) = 'RSURF   ', 
   filename(1) = 'surface', 
  frequency(2) = 86400., 
   levels(1,2) = 1.,2.,3.,4.,5., 
   fields(1,2) = 'UVEL    ','VVEL    ', 
   filename(2) = 'diagout1', 
  frequency(3) = 3600., 
   fields(1,3) = 'UVEL    ','VVEL    ','PRESSURE', 
   filename(3) = 'diagout2', 
  fileflags(3) = ' P1     ', 
 &end 
\end{verbatim}

\noindent
In this example, there are two output files that will be generated for
each tile and for each output time. The first set of output files has
the prefix diagout1, does time averaging every 86400. seconds,
(frequency is 86400.), and will write fields which are multiple-level
fields at output levels 1-5. The names of diagnostics quantities are
UVEL and VVEL.  The second set of output files has the prefix
diagout2, does time averaging every 3600. seconds, includes fields
which are multiple-level fields, levels output are 1-5, and the names
of diagnostics quantities are THETA and SALT.

\noindent
The user must assure that enough computer memory is allocated for the
diagnostics and the output streams selected for a particular
experiment.  This is acomplished by modifying the file
DIAGNOSTICS\_SIZE.h and including it in the experiment code directory.
The parameters that should be checked are called numdiags, numlists,
numperlist, and diagSt\_size.

\noindent numdiags (and diagSt\_size): \\
\noindent All GCM diagnostic quantities are stored in the single diagnostic array QDIAG 
which is located in the file \\ \filelink{pkg/diagnostics/DIAGNOSTICS.h}{pkg-diagnostics-DIAGNOSTICS.h}.\\
and has the form:\\
\begin{verbatim}
      common /diagnostics/
     &     qdiag(1-Olx,sNx+Olx,1-Olx,sNx+Olx,numdiags,Nsx,Nsy)
\end{verbatim}
\noindent
The first two-dimensions of qdiag correspond to the horizontal
dimension of a given diagnostic, and the third dimension of qdiag is
used to identify diagnostic fields and levels combined. In order to
minimize the memory requirement of the model for diagnostics, the
default GCM executable is compiled with room for only one horizontal
diagnostic array, or with numdiags set to Nr. In order for the User to
enable more than 1 three-dimensional diagnostic, the size of the
diagnostics common must be expanded to accomodate the desired
diagnostics.  This can be accomplished by manually changing the
parameter numdiags in the file
\filelink{pkg/diagnostics/DIAGNOSTICS\_SIZE.h}{pkg-diagnostics-DIAGNOSTICS\_SIZE.h}.
numdiags should be set greater than or equal to the sum of all the
diagnostics activated for output each multiplied by the number of
levels defined for that diagnostic quantity.  For the above example,
there are 4 multiple level fields, which the diagnostics menu (see
below) indicates are defined at the GCM vertical resolution, Nr. The
value of numdiag in DIAGNOSTICS\_SIZE.h would therefore be equal to
4*Nr, or, say 40 if $Nr=10$.

\noindent numlists and numperlist: \\
\noindent The parameter numlists must be set greater than or equal to
the number of separate output streams that the user specifies in the
namelist file data.diagnostics.  The parameter numperlist corresponds
to the number of diagnostics requested in each output stream.

\noindent
In order to define and include as part of the diagnostic output any
field that is desired for a particular experiment, two steps must be
taken. The first is to enable the ``User Diagnostic'' in
data.diagnostics. This is accomplished by adding one of the ``User
Diagnostic'' field names (UDIAG1 through UDIAG10, for multi-level
fields, or SDIAG1 through SDIAG10 for single level fields) to the
data.diagnostics namelist in one of the output streams. These fields
are listed in the diagnostics menu. The second step is to add a call
to diagnostics\_fill from the subroutine in which the quantity desired
for diagnostic output is computed.

\noindent
In order to add a new diagnostic to the permanent set of diagnostics
that the main model or any package contains as part of its diagnostics
menu, the subroutine diagnostics\_add2list should be called during the
initialization phase of the main model or package. For the main model,
the call should be made from subroutine diagnostics\_main\_init, and
for a package, the call should probably be made from somewhere in the
packages\_init\_fixed sequence (probaby from inside the particular
package's init\_fixed routine). A typical code sequence to set the
input arguments to diagnostics\_add2list would look like:

\begin{verbatim}
      diagName  = 'THETA   '
      diagTitle = 'Potential Temperature (degC,K)'
      diagUnits = 'Degrees K       '
      diagCode  = 'SM      MR      '
      CALL DIAGNOSTICS\_ADD2LIST( diagNum,
     I          diagName, diagCode, diagUnits, diagTitle, myThid )
\end{verbatim}

\noindent If the new diagnostic quantity is associated with either a
vector pair or a diagnostic counter, the diagCode argument must be
filled with the proper index for the ``mate''. The output argument
from diagnostics\_add2list that is called diagNum here contains a
running total of the number of diagnostics defined in the code up to
any point during the run. The sequence number for the next two
diagnostics defined (the two components of the vector pair, for
instance) will be diagNum+1 and diagNum+2. The definition of the first
component of the vector pair must fill the ``mate'' segment of the
diagCode as diagnostic number diagNum+2.  Since the subroutine
increments diagNum, the definition of the second component of the
vector fills the ``mate'' part of diagCode with diagNum. A code
sequence for this case would look like:

\begin{verbatim}
      diagName  = 'UVEL    ' 
      diagTitle = 'Zonal Velocity                ' 
      diagUnits = 'm / sec         ' 
      diagCode  = 'SM      MR      ' 
      write(diagCode,'(A,I3.3,A)') 'VV   ', diagNum+2 ,'MR      ' 
      call diagnostics\_add2list( diagNum, 
     I          diagName, diagCode, diagUnits, diagTitle, myThid ) 
      diagName  = 'VVEL    ' 
      diagTitle = 'Meridional Velocity           ' 
      diagUnits = 'm / sec         ' 
      diagCode  = 'SM      MR      ' 
      write(diagCode,'(A,I3.3,A)') 'VV   ', diagNum ,'MR      ' 
      call diagnostics\_add2list( diagNum, 
     I          diagName, diagCode, diagUnits, diagTitle, myThid ) 
\end{verbatim}

\newpage
\paragraph{MITgcm Kernel Diagnostic Menu:}
\addcontentsline{toc}{subsubsection}{Kernel Diagnostic Menu}
\label{sec:diagnostics:menu}

\begin{table}
\begin{tabular}{llll}
\hline\hline
 NAME & UNITS & LEVELS & DESCRIPTION \\
\hline

&\\
 UVEL     & $m/sec$ & Nr
         &\begin{minipage}[t]{3in}
          {U-Velocity} 
         \end{minipage}\\
 VVEL     & $m/sec$ & Nr
         &\begin{minipage}[t]{3in}
          {V-Velocity} 
         \end{minipage}\\
 UVEL\_k2  & $m/sec$ & 1
         &\begin{minipage}[t]{3in}
          {U-Velocity} 
         \end{minipage}\\
 VVEL\_k2  & $m/sec$ & 1
         &\begin{minipage}[t]{3in}
          {V-Velocity} 
         \end{minipage}\\
 WVEL     & $m/sec$ & Nr
         &\begin{minipage}[t]{3in}
          {Vertical-Velocity} 
         \end{minipage}\\
 THETASQ  & $deg^2$ & Nr
         &\begin{minipage}[t]{3in}
          {Square of Potential Temperature} 
         \end{minipage}\\
 SALTSQ   & $g^2/{kg}^2$ & Nr
         &\begin{minipage}[t]{3in}
          {Square of Salt (or Water Vapor Mixing Ratio)} 
         \end{minipage}\\
 SALTSQan & $g^2/{kg}^2$ & Nr
         &\begin{minipage}[t]{3in}
          {Square of Salt anomaly (=SALT-35)} 
         \end{minipage}\\
 UVELSQ   & $m^2/sec^2$ & Nr
         &\begin{minipage}[t]{3in}
          {Square of U-Velocity} 
         \end{minipage}\\
 VVELSQ   & $m^2/sec^2$ & Nr
         &\begin{minipage}[t]{3in}
          {Square of V-Velocity} 
         \end{minipage}\\
 WVELSQ   & $m^2/sec^2$ & Nr
         &\begin{minipage}[t]{3in}
          {Square of Vertical-Velocity} 
         \end{minipage}\\
 UV\_VEL\_C & $m^2/sec^2$ & Nr
         &\begin{minipage}[t]{3in}
          {Meridional Transport of Zonal Momentum (cell center)} 
         \end{minipage}\\
 UV\_VEL\_Z & $m^2/sec^2$ & Nr
         &\begin{minipage}[t]{3in}
          {Meridional Transport of Zonal Momentum (corner)} 
         \end{minipage}\\
 WU\_VEL   & $m^2/sec^2$ & Nr
         &\begin{minipage}[t]{3in}
          {Vertical Transport of Zonal Momentum (cell center)} 
         \end{minipage}\\
 WV\_VEL   & $m^2/sec^2$ & Nr
         &\begin{minipage}[t]{3in}
          {Vertical Transport of Meridional Momentum (cell center)} 
         \end{minipage}\\
 UVELMASS & $m/sec$ & Nr
         &\begin{minipage}[t]{3in}
          {Zonal Mass-Weighted Component of Velocity} 
         \end{minipage}\\
 VVELMASS & $m/sec$ & Nr
         &\begin{minipage}[t]{3in}
          {Meridional Mass-Weighted Component of Velocity} 
         \end{minipage}\\
 WVELMASS & $m/sec$ & Nr
         &\begin{minipage}[t]{3in}
          {Vertical Mass-Weighted Component of Velocity} 
         \end{minipage}\\
 UTHMASS  & $m-deg/sec$ & Nr
         &\begin{minipage}[t]{3in}
          {Zonal Mass-Weight Transp of Pot Temp} 
         \end{minipage}\\
 VTHMASS  & $m-deg/sec$ & Nr
         &\begin{minipage}[t]{3in}
          {Meridional Mass-Weight Transp of Pot Temp} 
         \end{minipage}\\
 WTHMASS  & $m-deg/sec$ & Nr
         &\begin{minipage}[t]{3in}
          {Vertical Mass-Weight Transp of Pot Temp} 
         \end{minipage}\\
 ETAN     & $(hPa,m)$ &    1
         &\begin{minipage}[t]{3in}
          {Perturbation of Surface (pressure, height)} 
         \end{minipage}\\
 ETANSQ   & $(hPa^2,m^2)$ & 1
         &\begin{minipage}[t]{3in}
          {Square of Perturbation of Surface (pressure, height)} 
         \end{minipage}\\
 DETADT2  & ${r-unit}^2/s^2$ & 1
         &\begin{minipage}[t]{3in}
          {Square of Eta (Surf.P,SSH) Tendency} 
         \end{minipage}\\
 THETA    & $deg K$ & Nr
         &\begin{minipage}[t]{3in}
          {Potential Temperature} 
         \end{minipage}\\
 SST      & $deg K$ & 1
         &\begin{minipage}[t]{3in}
          {Sea Surface Temperature} 
         \end{minipage}\\
 SALT     & $g/kg$ & Nr
         &\begin{minipage}[t]{3in}
          {Salt (or Water Vapor Mixing Ratio)} 
         \end{minipage}\\
 SSS      & $g/kg$ & 1
         &\begin{minipage}[t]{3in}
          {Sea Surface Salinity} 
         \end{minipage}\\
 SALTanom & $g/kg$ & Nr
         &\begin{minipage}[t]{3in}
          {Salt anomaly (=SALT-35)} 
         \end{minipage}\\
\end{tabular}
\end{table}
\vspace{1.5in}
\vfill

\newpage
\vspace*{\fill}
\begin{table}
\begin{tabular}{llll}
\hline\hline
 NAME & UNITS & LEVELS & DESCRIPTION \\
\hline

&\\
 USLTMASS & $m-kg/sec-kg$ & Nr
         &\begin{minipage}[t]{3in}
          {Zonal Mass-Weight Transp of Salt (or W.Vap Mix Rat.)} 
         \end{minipage}\\
 VSLTMASS & $m-kg/sec-kg$ & Nr
         &\begin{minipage}[t]{3in}
          {Meridional Mass-Weight Transp of Salt (or W.Vap Mix Rat.)} 
         \end{minipage}\\
 WSLTMASS & $m-kg/sec-kg$ & Nr
         &\begin{minipage}[t]{3in}
          {Vertical Mass-Weight Transp of Salt (or W.Vap Mix Rat.)} 
         \end{minipage}\\
 UVELTH   & $m-deg/sec$ & Nr
         &\begin{minipage}[t]{3in}
          {Zonal Transp of Pot Temp} 
         \end{minipage}\\
 VVELTH   & $m-deg/sec$ & Nr
         &\begin{minipage}[t]{3in}
          {Meridional Transp of Pot Temp} 
         \end{minipage}\\
 WVELTH   & $m-deg/sec$ & Nr
         &\begin{minipage}[t]{3in}
          {Vertical Transp of Pot Temp} 
         \end{minipage}\\
 UVELSLT  & $m-kg/sec-kg$ & Nr
         &\begin{minipage}[t]{3in}
          {Zonal Transp of Salt (or W.Vap Mix Rat.)} 
         \end{minipage}\\
 VVELSLT  & $m-kg/sec-kg$ & Nr
         &\begin{minipage}[t]{3in}
          {Meridional Transp of Salt (or W.Vap Mix Rat.)} 
         \end{minipage}\\
 WVELSLT  & $m-kg/sec-kg$ & Nr
         &\begin{minipage}[t]{3in}
          {Vertical Transp of Salt (or W.Vap Mix Rat.)} 
         \end{minipage}\\
 RHOAnoma & $kg/m^3  $  &  Nr  
         &\begin{minipage}[t]{3in}
          {Density Anomaly (=Rho-rhoConst)}
         \end{minipage}\\
 RHOANOSQ & $kg^2/m^6$  &  Nr  
         &\begin{minipage}[t]{3in}
          {Square of Density Anomaly (=(Rho-rhoConst))}
         \end{minipage}\\
 URHOMASS & $kg/m^2/s$  &  Nr  
         &\begin{minipage}[t]{3in}
          {Zonal Transport of Density}
         \end{minipage}\\
 VRHOMASS & $kg/m^2/s$  &  Nr  
         &\begin{minipage}[t]{3in}
          {Meridional Transport of Density}
         \end{minipage}\\
 WRHOMASS & $kg/m^2/s$  &  Nr  
         &\begin{minipage}[t]{3in}
          {Vertical Transport of Potential Density}
         \end{minipage}\\
 PHIHYD   & $m^2/s^2 $  &  Nr  
         &\begin{minipage}[t]{3in}
          {Hydrostatic (ocean) pressure / (atmos) geo-Potential}
         \end{minipage}\\
 PHIHYDSQ & $m^4/s^4 $  &  Nr  
         &\begin{minipage}[t]{3in}
          {Square of Hyd. (ocean) press / (atmos) geoPotential}
         \end{minipage}\\
 PHIBOT   & $m^2/s^2 $  &  Nr  
         &\begin{minipage}[t]{3in}
          {ocean bottom pressure / top. atmos geo-Potential}
         \end{minipage}\\
 PHIBOTSQ & $m^4/s^4 $  &  Nr  
         &\begin{minipage}[t]{3in}
          {Square of ocean bottom pressure / top. geo-Potential}
         \end{minipage}\\
 DRHODR   & $kg/m^3/{r-unit}$ & Nr
         &\begin{minipage}[t]{3in}
          {Stratification: d.Sigma/dr} 
         \end{minipage}\\
 VISCA4   & $m^4/sec$ & 1
         &\begin{minipage}[t]{3in}
          {Biharmonic Viscosity Coefficient} 
         \end{minipage}\\
 VISCAH   & $m^2/sec$ & 1
         &\begin{minipage}[t]{3in}
          {Harmonic Viscosity Coefficient} 
         \end{minipage}\\
 TAUX     & $N/m^2        $ & 1 
         &\begin{minipage}[t]{3in}
          {zonal surface wind stress, >0 increases uVel}
         \end{minipage}\\
 TAUY     & $N/m^2        $ & 1 
         &\begin{minipage}[t]{3in}
          {meridional surf. wind stress, >0 increases vVel}
         \end{minipage}\\
 TFLUX    & $W/m^2        $ & 1 
         &\begin{minipage}[t]{3in}
          {net surface heat flux, >0 increases theta}
         \end{minipage}\\
 TRELAX   & $W/m^2        $ & 1 
         &\begin{minipage}[t]{3in}
          {surface temperature relaxation, >0 increases theta}
         \end{minipage}\\
 TICE     & $W/m^2        $ & 1 
         &\begin{minipage}[t]{3in}
          {heat from melt/freeze of sea-ice, >0 increases theta}
         \end{minipage}\\
 SFLUX    & $g/m^2/s      $ & 1 
         &\begin{minipage}[t]{3in}
          {net surface salt flux, >0 increases salt}
         \end{minipage}\\
 SRELAX   & $g/m^2/s      $ & 1 
         &\begin{minipage}[t]{3in}
          {surface salinity relaxation, >0 increases salt}
         \end{minipage}\\
 PRESSURE & $Pa           $ & Nr 
         &\begin{minipage}[t]{3in}
          {Atmospheric Pressure (Pa)}
         \end{minipage}\\
\end{tabular}
\end{table}
\vspace{1.5in}
\vfill

\newpage
\vspace*{\fill}
\begin{table}
\begin{tabular}{llll}
\hline\hline
 NAME & UNITS & LEVELS & DESCRIPTION \\
\hline

&\\
 ADVr\_TH  & $K.Pa.m^2/s   $ & Nr 
         &\begin{minipage}[t]{3in}
          {Vertical   Advective Flux of Pot.Temperature}
         \end{minipage}\\
 ADVx\_TH  & $K.Pa.m^2/s   $ & Nr 
         &\begin{minipage}[t]{3in}
          {Zonal      Advective Flux of Pot.Temperature}
         \end{minipage}\\
 ADVy\_TH  & $K.Pa.m^2/s   $ & Nr 
         &\begin{minipage}[t]{3in}
          {Meridional Advective Flux of Pot.Temperature}
         \end{minipage}\\
 DFrE\_TH  & $K.Pa.m^2/s   $ & Nr 
         &\begin{minipage}[t]{3in}
          {Vertical Diffusive Flux of Pot.Temperature (Explicit part)}
         \end{minipage}\\
 DIFx\_TH  & $K.Pa.m^2/s   $ & Nr 
         &\begin{minipage}[t]{3in}
          {Zonal      Diffusive Flux of Pot.Temperature}
         \end{minipage}\\
 DIFy\_TH  & $K.Pa.m^2/s   $ & Nr 
         &\begin{minipage}[t]{3in}
          {Meridional Diffusive Flux of Pot.Temperature}
         \end{minipage}\\
 DFrI\_TH  & $K.Pa.m^2/s   $ & Nr 
         &\begin{minipage}[t]{3in}
          {Vertical Diffusive Flux of Pot.Temperature (Implicit part)}
         \end{minipage}\\
 ADVr\_SLT & $g/kg.Pa.m^2/s$ & Nr 
         &\begin{minipage}[t]{3in}
          {Vertical   Advective Flux of Water-Vapor}
         \end{minipage}\\
 ADVx\_SLT & $g/kg.Pa.m^2/s$ & Nr 
         &\begin{minipage}[t]{3in}
          {Zonal      Advective Flux of Water-Vapor}
         \end{minipage}\\
 ADVy\_SLT & $g/kg.Pa.m^2/s$ & Nr 
         &\begin{minipage}[t]{3in}
          {Meridional Advective Flux of Water-Vapor}
         \end{minipage}\\
 DFrE\_SLT & $g/kg.Pa.m^2/s$ & Nr 
         &\begin{minipage}[t]{3in}
          {Vertical Diffusive Flux of Water-Vapor (Explicit part)}
         \end{minipage}\\
 DIFx\_SLT & $g/kg.Pa.m^2/s$ & Nr 
         &\begin{minipage}[t]{3in}
          {Zonal      Diffusive Flux of Water-Vapor}
         \end{minipage}\\
 DIFy\_SLT & $g/kg.Pa.m^2/s$ & Nr 
         &\begin{minipage}[t]{3in}
          {Meridional Diffusive Flux of Water-Vapor}
         \end{minipage}\\
 DFrI\_SLT & $g/kg.Pa.m^2/s$ & Nr 
         &\begin{minipage}[t]{3in}
          {Vertical Diffusive Flux of Water-Vapor (Implicit part)}
         \end{minipage}\\
\end{tabular}
\end{table}
\vspace{1.5in}
\vfill

\newpage
\vspace*{\fill}
\begin{table}
\begin{tabular}{llll}
\hline\hline
 NAME & UNITS & LEVELS & DESCRIPTION \\
\hline

&\\
 SDIAG1   &             &    1  
         &\begin{minipage}[t]{3in}
          {User-Defined Surface Diagnostic-1} 
         \end{minipage}\\
 SDIAG2   &             &    1  
         &\begin{minipage}[t]{3in}
          {User-Defined Surface Diagnostic-2} 
         \end{minipage}\\
 SDIAG3   &             &    1  
         &\begin{minipage}[t]{3in}
          {User-Defined Surface Diagnostic-3} 
         \end{minipage}\\
 SDIAG4   &             &    1  
         &\begin{minipage}[t]{3in}
          {User-Defined Surface Diagnostic-4} 
         \end{minipage}\\
 SDIAG5   &             &    1  
         &\begin{minipage}[t]{3in}
          {User-Defined Surface Diagnostic-5} 
         \end{minipage}\\
 SDIAG6   &             &    1  
         &\begin{minipage}[t]{3in}
          {User-Defined Surface Diagnostic-6} 
         \end{minipage}\\
 SDIAG7   &             &    1  
         &\begin{minipage}[t]{3in}
          {User-Defined Surface Diagnostic-7} 
         \end{minipage}\\
 SDIAG8   &             &    1  
         &\begin{minipage}[t]{3in}
          {User-Defined Surface Diagnostic-8} 
         \end{minipage}\\
 SDIAG9   &             &    1  
         &\begin{minipage}[t]{3in}
          {User-Defined Surface Diagnostic-9} 
         \end{minipage}\\
 SDIAG10  &             &    1  
         &\begin{minipage}[t]{3in}
          {User-Defined Surface Diagnostic-1-} 
         \end{minipage}\\
 SDIAGC   &             &    1  
         &\begin{minipage}[t]{3in}
          {User-Defined Counted Surface Diagnostic} 
         \end{minipage}\\
 SDIAGCC  &             &    1  
         &\begin{minipage}[t]{3in}
          {User-Defined Counted Surface Diagnostic Counter} 
         \end{minipage}\\
 UDIAG1   &             &    Nrphys
         &\begin{minipage}[t]{3in}
          {User-Defined Upper-Air Diagnostic-1} 
         \end{minipage}\\
 UDIAG2   &             &    Nrphys
         &\begin{minipage}[t]{3in}
          {User-Defined Upper-Air Diagnostic-2} 
         \end{minipage}\\
 UDIAG3   &             &    Nrphys  
         &\begin{minipage}[t]{3in}
          {User-Defined Multi-Level Diagnostic-3} 
         \end{minipage}\\
 UDIAG4   &             &    Nrphys  
         &\begin{minipage}[t]{3in}
          {User-Defined Multi-Level Diagnostic-4} 
         \end{minipage}\\
 UDIAG5   &             &    Nrphys  
         &\begin{minipage}[t]{3in}
          {User-Defined Multi-Level Diagnostic-5} 
         \end{minipage}\\
 UDIAG6   &             &    Nrphys  
         &\begin{minipage}[t]{3in}
          {User-Defined Multi-Level Diagnostic-6} 
         \end{minipage}\\
 UDIAG7   &             &    Nrphys  
         &\begin{minipage}[t]{3in}
          {User-Defined Multi-Level Diagnostic-7} 
         \end{minipage}\\
 UDIAG8   &             &    Nrphys  
         &\begin{minipage}[t]{3in}
          {User-Defined Multi-Level Diagnostic-8} 
         \end{minipage}\\
 UDIAG9   &             &    Nrphys  
         &\begin{minipage}[t]{3in}
          {User-Defined Multi-Level Diagnostic-9} 
         \end{minipage}\\
 UDIAG10  &             &    Nrphys  
         &\begin{minipage}[t]{3in}
          {User-Defined Multi-Level Diagnostic-10} 
         \end{minipage}\\
\end{tabular}
\end{table}
\vspace{1.5in}
\vfill


\newpage
\noindent For a list of the diagnostic fields available in the
different MITgcm packages, follow the link to the diagnostics menu
in the manual section describing the package:

\filelink{part6/fizhi-diagnostics-menu.tex}{pkg-fizhi-fizhi-diagnostics-menu.tex}

\subsection{Dos and Donts}

\subsection{Diagnostics Reference}

