% $Header: /u/gcmpack/manual/s_outp_pkgs/text/mdsio.tex,v 1.7 2010/08/30 23:09:21 jmc Exp $
% $Name:  $


\section{Fortran Native I/O: MDSIO and RW}
\label{sec:mdsio_and_rw}


\subsection{MDSIO}
\label{sec:pkg:mdsio}
\begin{rawhtml}
<!-- CMIREDIR:package_mdsio: -->
\end{rawhtml}

\subsubsection{Introduction}
The \texttt{mdsio} package contains a group of Fortran routines
intended as a general interface for reading and writing direct-access
(``binary'') Fortran files.  The \texttt{mdsio} routines are used by
the \texttt{rw} package.

The \texttt{mdsio} package is currently the primary method for MITgcm
I/O, but it is not being actively extended or enhanced.  Instead, the
\texttt{mnc} netCDF package (see Section \ref{sec:pkg:mnc}) is
expected to gain all of the current \texttt{mdsio} functionality and,
eventually, replace it.  For the short term, every effort has been
made to allow \texttt{mnc} and \texttt{mdsio} to peacefully co-exist.
In may cases, the model can read one format and write to the other.
This side-by-side functionality can be used to, for instance, help
convert pickup files or other data sets between the two formats.


\subsubsection{Using MDSIO}
The \texttt{mdsio} package is geared toward the reading and writing of
floating point (Fortran \texttt{REAL*4} or \texttt{REAL*8}) arrays.
It assumes that the in-memory layout of all arrays follows the per-tile
MITgcm convention
\begin{verbatim}
C     Example of a "2D" array
      _RL anArray(1-OLx:sNx+OLx,1-OLy:sNy+OLy,nSx,nSy)

C     Example of a "3D" array
      _RL anArray(1-OLx:sNx+OLx,1-OLy:sNy+OLy,1:Nr,nSx,nSy)
\end{verbatim}
where the first two dimensions are spatial or ``horizontal'' indicies
that include a ``halo'' or exchange region (please see
Chapters \ref{chap:sarch} and \ref{sec:exch2} which describe domain
decomposition), and the remaining indicies (\texttt{Nr},\texttt{nSx},
and \texttt{nSx}) are often present but not required.

In order to write output, the \texttt{mdsio} package is called with a
function such as:
\begin{verbatim}
      CALL MDSWRITEFIELD(fn,prec,lgf,typ,Nr,arr,irec,myIter,myThid)
\end{verbatim}
where:
\begin{quote}
  \begin{description}
  \item[\texttt{fn}] is a \texttt{CHARACTER} string containing a file
    ``base'' name which will then be used to create file names that
    contain tile and/or model iteration indicies
  \item[\texttt{prec}] is an integer that contains one of two globally
    defined values (\texttt{precFloat64} or \texttt{precFloat32})
  \item[\texttt{lgf}] is a \texttt{LOGICAL} that typically contains
    the globally defined \texttt{globalFile} option which specifies
    the creation of globally (spatially) concatenated files
  \item[\texttt{typ}] is a \texttt{CHARACTER} string that specifies
    the type of the variable being written (\texttt{'RL'} or
    \texttt{'RS'})
  \item[\texttt{Nr}] is an integer that specifies the number of
    vertical levels within the variable being written
  \item[\texttt{arr}] is the variable (array) to be written
  \item[\texttt{irec}] is the starting record within the output file
    that will contain the array
  \item[\texttt{myIter,myThid}] are integers containing, respectively,
    the current model iteration count and the unique thread ID for the
    current context of execution
  \end{description}  
\end{quote}
As one can see from the above (generic) example, enough information is
made available (through both the argument list and through common blocks)
for the \texttt{mdsio} package to perform the following tasks:
\begin{enumerate}
\item open either a per-tile file such as:
  \begin{center}
    \texttt{uVel.0000302400.003.001.data}
  \end{center}
  or a ``global'' file such as
  \begin{center}
    \texttt{uVel.0000302400.data}
  \end{center}
\item byte-swap (as necessary) the input array and write its contents
  (minus any halo information) to the binary file -- or to the correct
  location within the binary file if the globalfile option is used, and 
\item create an ASCII--text metadata file (same name as the binary but
  with a \texttt{.meta} extension) describing the binary file contents
  (often, for later use with the MatLAB \texttt{rdmds()} utility).
\end{enumerate}

Reading output with \texttt{mdsio} is very similar to writing it.  A
typical function call is
\begin{verbatim}
      CALL MDSREADFIELD(fn,prec,typ,Nr,arr,irec,myThid)
\end{verbatim}
where variables are exactly the same as the \texttt{MDSWRITEFIELD}
example provided above.  It is important to note that the \texttt{lgf}
argument is missing from the \texttt{MDSREADFIELD} function.  By
default, \texttt{mdsio} will first try to read from an appropriately
named global file and, failing that, will try to read from a per-tile
file.


\subsubsection{Important Considerations}
When using \texttt{mdsio}, one should be aware of the following
package features and limitations:
\begin{description}
\item[Byte-swapping] is, for the most part, gracefully handled.  All
  files intended for reading/writing by \texttt{mdsio} should contain
  big-endian (sometimes called ``network byte order'') data.  By
  handling byte-swapping within the model, MITgcm output is more
  easily ported between different machines, architectures, compilers,
  etc.  Byteswapping can be turned on/off at compile time within
  \texttt{mdsio} using the \texttt{\_BYTESWAPIO} CPP macro which is
  usually set within a \texttt{genmake2} options file or
  ``\texttt{optfile}'' which are located in
\begin{verbatim}
      MITgcm/tools/build_options
\end{verbatim}
  Additionally, some compilers may have byte-swap options that are
  speedier or more convenient to use.

\item[Types] are currently limited to single-- or double--precision
  floating point values.  These values can be converted, on-the-fly,
  from one to the other so that any combination of either single-- or
  double--precision variables can be read from or written to files
  containing either single-- or double--precision data.

\item[Array sizes] are limited.  The \texttt{mdsio} package is very
  much geared towards the reading/writing of per-tile (that is,
  domain-decomposed and halo-ed) arrays.  Data that cannot be made to
  ``fit'' within these assumed sizes can be challenging to read or
  write with \texttt{mdsio}.

\item[Tiling] or domain decomposition is automatically handled by
  \texttt{mdsio} for logically rectangular grid topologies
  (\textit{eg.} lat-lon grids) and ``standard'' cubesphere topologies.
  More complicated topologies will probably not be supported.  The
  \texttt{mdsio} package can, without any coding changes, read and
  write to/from files that were run on the same global grid but with
  different tiling (grid decomposition) schemes.  For example,
  \texttt{mdsio} can use and/or create identical input/output files
  for a ``C32'' cube when the model is run with either 6, 12, or 24
  tiles (corresponding to 1, 2 or 4 tiles per cubesphere face).
  Currently, this is one of the primary advantages that the
  \texttt{mdsio} package has over \texttt{mnc}.

\item[Single-CPU I/O] can be specified with the flag
\begin{verbatim}
       useSingleCpuIO = .TRUE.,
\end{verbatim}
  in the \texttt{PARM01} namelist within the main \texttt{data} file.
  Single--CPU I/O mode is appropriate for computers (\textit{eg.} some
  SGI systems) where it can either speed overall I/O or solve problems
  where the operating system or file systems cannot correctly handle
  multiple threads or MPI processes simultaneously writing to the same
  file.

\item[Meta-data] is written by MITgcm on a per-file basis using a
  second file with a \texttt{.meta} extension as described above.
  MITgcm itself does not read the \texttt{*.meta} files, they are
  there primarly for convenience during post-processing.  One should
  be careful not to delete the meta-data files when using MatLAB
  post-processing scripts such as \texttt{rdmds()} since it relies
  upon them.

\item[Numerous files] can be written by \texttt{mdsio} due to its
  typically per-time-step and per-variable orientation.  The creation of
  both a binary (\texttt{*.data}) and ASCII text meta-data
  (\texttt{*.meta}) file for each output type step tends to exacerbate
  the problem.  Some (mostly, older) operating systems do not
  gracefully handle large numbers (\textit{eg.} many thousands) of
  files within one directory.  So care should be taken to split output
  into smaller groups using subdirectories.

\item[Overwriting] is the \textbf{default behavior} for
  \texttt{mdsio}.  If a model tries to write to a file name that
  already exists, the older file \textbf{will be deleted}.  For this
  reason, MITgcm users should be careful to move output that that wish
  to keep into, for instance, subdirectories before performing
  subsequent runs that may over--lap in time or otherwise produce
  files with identical names (\textit{eg.} Monte-Carlo simulations).

\item[No ``halo'' information] is written or read by \texttt{mdsio}.
  Along the horizontal dimensions, all variables are written in an
  \texttt{sNx}--by--\texttt{sNy} fashion.  So, although variables
  (arrays) may be defined at different locations on Arakawa grids [U
  (right/left horizontal edges), V (top/bottom horizontal edges), M
  (mass or cell center), or Z (vorticity or cell corner) points], they
  are all written using only interior (\texttt{1:sNx} and
  \texttt{1:sNy}) values.  For quantities defined at U, V, and M
  points, writing \texttt{1:sNx} and \texttt{1:sNy} for every tile is
  sufficient to ensure that all values are written globally for some
  grids (eg. cubesphere, re-entrant channels, and doubly-periodic
  rectangular regions).  For Z points, failing to write values at the
  \texttt{sNx+1} and \texttt{sNy+1} locations means that, for some
  tile topologies, not all values are written.  For instance, with a
  cubesphere topology at least two corner values are ``lost'' (fail to
  be written for any tile) if the \texttt{sNx+1} and \texttt{sNy+1}
  values are ignored.  To fix this problem, the \texttt{mnc} package
  writes the \texttt{sNx+1} and \texttt{sNy+1} grid values for the U,
  V, and Z locations.  Also, the \texttt{mnc} package is capable of
  reading and/or writing entire halo regions and more complicated
  array shapes which can be helpful when debugging--features that
  do not exist within \texttt{mdsio}.
\end{description}


\subsection{RW Basic binary I/O utilities}
\label{sec:pkg:rw}
\begin{rawhtml}
<!-- CMIREDIR:package_rw: -->
\end{rawhtml}

The {\tt rw} package provides a very rudimentary binary I/O capability
for quickly writing {\it single record} direct-access Fortran binary files.
It is primarily used for writing diagnostic output.

\subsubsection{Introduction}
Package {\tt rw} is an interface to the more general {\tt mdsio} package.
The {\tt rw} package can be used to write or read direct-access Fortran
binary files for two-dimensional XY and three-dimensional XYZ arrays.
The arrays are assumed to have been declared according to the standard
MITgcm two-dimensional or three-dimensional floating point array type:
\begin{verbatim}
C     Example of declaring a standard two dimensional "long"
C     floating point type array (the _RL macro is usually
C     mapped to 64-bit floats in most configurations)
      _RL anArray(1-OLx:sNx+OLx,1-OLy:sNy+OLy,nSx,nSy)
\end{verbatim}

Each call to an {\tt rw} read or write routine will read (or write) to
the first record of a file. To write direct access Fortran files with
multiple records use the package {\tt mdsio} (see section
\ref{sec:pkg:mdsio}).  To write self-describing files that contain
embedded information describing the variables being written and the
spatial and temporal locations of those variables use the package {\tt
  mnc} (see section \ref{sec:pkg:mnc}) which produces
\htlink{netCDF}{http://www.unidata.ucar.edu/packages/netcdf}
\cite{rew:97} based output.

%% \subsubsection{Key subroutines, parameters and files}
%% \label{sec:pkg:rw:implementation_synopsis}
%% The {\tt rw} package has

