\section{Monitor: Simulation state monitoring toolkit}
\label{sec:pkg:monitor}
\begin{rawhtml}
<!-- CMIREDIR:package_monitor: -->
\end{rawhtml}


\subsection{Introduction}
\label{sec:pkg:monitor:overview}


The \texttt{monitor} package is primarily intended as a convenient
method for calculating and writing the following statistics:
\begin{center}
  minimum, maximum, mean, and standard deviation
\end{center}
for spatially distributed fields.  By default, \texttt{monitor} output
is sent to the ``standard output'' channel where it appears as ASCII
text containing a \texttt{\%MON} string such as this example:
\begin{verbatim}
     (PID.TID 0000.0001) %MON time_tsnumber      =                     3
     (PID.TID 0000.0001) %MON time_secondsf      =   3.6000000000000E+03
     (PID.TID 0000.0001) %MON dynstat_eta_max    =   1.0025466645951E-03
     (PID.TID 0000.0001) %MON dynstat_eta_min    =  -1.0008899950901E-03
     (PID.TID 0000.0001) %MON dynstat_eta_mean   =   2.1037438449350E-14
     (PID.TID 0000.0001) %MON dynstat_eta_sd     =   5.0985228723396E-04
     (PID.TID 0000.0001) %MON dynstat_eta_del2   =   3.5216706549525E-07
     (PID.TID 0000.0001) %MON dynstat_uvel_max   =   3.7594045977254E-05
     (PID.TID 0000.0001) %MON dynstat_uvel_min   =  -2.8264287531564E-05
     (PID.TID 0000.0001) %MON dynstat_uvel_mean  =   9.1369201945671E-06
     (PID.TID 0000.0001) %MON dynstat_uvel_sd    =   1.6868439193567E-05
     (PID.TID 0000.0001) %MON dynstat_uvel_del2  =   8.4315445301916E-08
\end{verbatim}
The \texttt{monitor} text can be readily parsed by the
\texttt{testreport} script to determine, somewhat crudely but quickly,
how similar the output from two experiments are when run on different
platforms or before/after code changes.

The \texttt{monitor} output can also be useful for quickly diagnosing
practical problems such as CFL limitations, model progress (through
iteration counts), and behavior within some packages that use it.


\subsection{Using Monitor}
\label{sec:pkg:monitor:using}

As with most packages, \texttt{monitor} can be turned on or off at
compile and/or run times using the \texttt{packages.conf} and
\texttt{data.pkg} files.  

The monitor output can be sent to the standard output channel, to an
\texttt{mnc}--generated file, or to both simultaneously.  For
\texttt{mnc} output, the flag:
\begin{verbatim}
   monitor_mnc=.TRUE.,
\end{verbatim}
should be set within the \texttt{data.mnc} file.  For output to both
ASCII and \texttt{mnc}, the flag
\begin{verbatim}
   outputTypesInclusive=.TRUE.,
\end{verbatim}
should be set within the \texttt{PARM03} section of the main
\texttt{data} file.  It should be noted that the
\texttt{outputTypesInclusive} flag will make \textbf{ALL} kinds of
output (that is, everything written by \texttt{mdsio}, \texttt{mnc},
and \texttt{monitor}) simultaneously active so it should be used only
with caution--and perhaps only for debugging purposes.

