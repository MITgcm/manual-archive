% $Header: /u/gcmpack/manual/s_outp_pkgs/text/top_section.tex,v 1.6 2005/07/14 19:19:38 molod Exp $
% $Name:  $

\chapter{Pre-processsing and Post-processing Tools}
\begin{rawhtml}
<!-- CMIREDIR:processing_tools: -->
\end{rawhtml}

There are numerous tools for pre-processing data, converting model
output and analysis written in Matlab, fortran (f77 and f90) and perl.
As yet they remain undocumented although many are self-documenting
(Matlab routines have "help" written into them).

Here we'll summarize what is available but this is an ever growing resource
so this may not cover everything that is out there:

\section{Utilities supplied with the model}

We supply some basic scripts with the model to facilitate conversion or reading
of data into analysis software.

\subsection{utils/scripts}

In the directory {\em utils/scripts} you will find {\em joinds} and {\em joinmds}:
these are perl scripts used from joining the multi-part files created by
MITgcm. {\bf Use {\em joinmds} always}. You will only need {\em joinds} if you
are working with output older than two years (prior to c23).

\subsection{utils/matlab}

In the directory {\em utils/matlab} you will find several Matlab scripts
(.m or dot-em files). The priniciple script is {\em rdmds.m} used for reading
the multi-part model output files in to matlab. Place the scripts in your
matlab path or change the path appropriately, then at the matlab prompt type:
\begin{verbatim}
  >> help rdmds
\end{verbatim}
to get help on how to use rdmds.

Another useful script scans the terminal output file for "monitor" information.

Most other scripts are for working in the curvilinear coordinate systems which
as yet are unpublished and undocumented.

\section{Pre-processing software}

There is a suite of pre-processing software for intepolating bathymetry
and forcing data, written by Adcroft and Biastoch. At some point,
these will be made available for download. If you are in need of such
software, contact one of them.
