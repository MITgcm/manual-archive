% $Header: /u/gcmpack/manual/model_uses.tex,v 1.3 2004/03/23 16:47:04 afe Exp $
% $Name:  $

\chapter{Model Uses}
\begin{rawhtml}
<!-- CMIREDIR:model_uses_list: -->
\end{rawhtml}



To date, the MITgcm model has been used for number of purposes.  The
following papers are an incomplete yet fairly broad sample of MITgcm
applications:

\begin{itemize}

\item R.S. Gross and I. Fukumori and D. Menemenlis (2003).
Atmospheric and oceanic excitation of the Earth's wobbles during 1980-2000.
{\it Journal of Geophysical Research-Solid Earth}, vol.108.

\item M.A. Spall and R.S. Pickart (2003).
Wind-driven recirculations and exchange in the Labrador and Irminger Seas. 
{\it Journal of Physical Oceanography}, vol.33, pp.1829--1845. 

\item B. Ferron and J. Marotzke (2003).
Impact of 4D-variational assimilation of WOCE hydrography on the
meridional circulation of the Indian Ocean. {\it Deep-Sea Research Part
II-Topical Studies in Oceanography}, vol.50, pp.2005--2021.

\item G.A. McKinley and M.J. Follows and J. Marshall and S.M. Fan (2003).
Interannual variability of air-sea O-2 fluxes and the determination of
CO2 sinks using atmospheric O-2/N-2. {\it Geophysical Research Letters},
vol.30. 

\item B.Y. Huang and P.H. Stone and A.P. Sokolov and I.V. Kamenkovich
(2003). The deep-ocean heat uptake in transient climate change. {\it Journal
of Climate}, vol.16, pp.1352--1363. 

\item G. de Coetlogon and C. Frankignoul (2003). The persistence of winter
sea surface temperature in the North Atlantic. {\it Journal of Climate},
vol.16, pp.1364--1377. 

\item K.H. Nisancioglu and M.E. Raymo and P.H. Stone (2003).
Reorganization of Miocene deep water circulation in response to the
shoaling of the Central American Seaway. {\it Paleoceanography}, vol.18. 

\item T. Radko and J. Marshall (2003). Equilibration of a Warm Pumped Lens
on a beta plane. {\it Journal of Physical Oceanography}, vol.33, pp.885--899. 

\item T. Stipa (2002). The dynamics of the N/P ratio and stratification in
a large nitrogen-limited estuary as a result of upwelling: a tendency
for offshore Nodularia blooms. {\it Hydrobiologia}, vol.487, pp.219--227. 

\item D. Stammer and C. Wunsch and R. Giering and C. Eckert and P.
Heimbach and J. Marotzke and A. Adcroft and C.N. Hill and J. Marshall
(2003). Volume, heat, and freshwater transports of the global ocean
circulation 1993-2000, estimated from a general circulation model
constrained by World Ocean Circulation Experiment (WOCE) data. {\it Journal
of Geophysical Research-Oceans}, vol.108. 

\item P. Heimbach and C. Hill and R. Giering (2002). Automatic generation
of efficient adjoint code for a parallel Navier-Stokes solver.
{\it Computational Science-ICCS 2002, PT II, Proceedings}, vol.2330,
pp.1019--1028. 

\item D. Stammer and C. Wunsch and R. Giering and C. Eckert and P.
Heimbach and J. Marotzke and A. Adcroft and C.N. Hill and J. Marshall
(2002). Global ocean circulation during 1992-1997, estimated from ocean
observations and a general circulation model. {\it Journal of Geophysical
Research-Oceans}, vol.107. 

\item M. Solovev and P.H. Stone and P. Malanotte-Rizzoli (2002).
Assessment of mesoscale eddy parameterizations for a single-basin
coarse-resolution ocean model. {\it Journal of Geophysical Research-Oceans},
vol.107. 

\item C. Herbaut and J. Sirven and S. Fevrier (2002). Response of a
simplified oceanic general circulation model to idealized NAO-like
stochastic forcing. {\it Journal of Physical Oceanography}, vol.32,
pp.3182--3192. 

\item C. Wunsch (2002). Oceanic age and transient tracers: Analytical and
numerical solutions. {\it Journal of Geophysical Research-Oceans}, vol.107. 

\item J. Sirven and C. Frankignoul and D. de Coetlogon and V. Taillandier
(2002). Spectrum of wind-driven baroclinic fluctuations of the ocean in
the midlatitudes. {\it Journal of Physical Oceanography}, vol.32,
pp.2405--2417. 

\item R. Zhang and M. Follows and J. Marshall (2002). Mechanisms of
thermohaline mode switching with application to warm equable climates.
{\it Journal of Climate}, vol.15, pp.2056--2072. 

\item G. Brostrom (2002). On advection and diffusion of plankton in coarse
resolution ocean models. {\it Journal of Marine Systems}, vol.35, pp.99--110. 

\item T. Lee and I. Fukumori and D. Menemenlis and Z.F. Xing and L.L. Fu
(2002). Effects of the Indonesian Throughflow on the Pacific and Indian
oceans. {\it Journal of Physical Oceanography}, vol.32, pp.1404--1429. 


\item C. Herbaut and J. Marshall (2002). Mechanisms of buoyancy transport
through mixed layers and statistical signatures from isobaric floats.
{\it Journal of Physical Oceanography}, vol.32, pp.545--557. 

\item J. Marshall and H. Jones and R. Karsten and R. Wardle (2002). Can
eddies set ocean stratification?, {\it Journal of Physical Oceanography},
vol.32, pp.26--38. 

\item C. Herbaut and J. Sirven and A. Czaja (2001). An idealized model
study of the mass and heat transports between the subpolar and
subtropical gyres. {\it Journal of Physical Oceanography}, vol.31,
pp.2903--2916. 

\item R.H. Kase and A. Biastoch and D.B. Stammer (2001). On the Mid-Depth
Circulation in the Labrador and Irminger Seas. {\it Geophysical Research
Letters}, vol.28, pp.3433--3436. 

\item R. Zhang and M.J. Follows and J.P. Grotzinger and J. Marshall
(2001). Could the Late Permian deep ocean have been anoxic?.
{\it Paleoceanography}, vol.16, pp.317--329. 

\item A. Adcroft and J.R. Scott and J. Marotzke (2001). Impact of
geothermal heating on the global ocean circulation. {\it Geophysical Research
Letters}, vol.28, pp.1735--1738. 

\item J. Marshall and D. Jamous and J. Nilsson (2001). Entry, flux, and
exit of potential vorticity in ocean circulation. {\it Journal of Physical
Oceanography}, vol.31, pp.777--789. 

\item A. Mahadevan (2001). An analysis of bomb radiocarbon trends in the
Pacific. {\it Marine Chemistry}, vol.73, pp.273--290. 


\item G. Brostrom (2000). The role of the annual cycles for the air-sea
exchange of CO2. {\it Marine Chemistry}, vol.72, pp.151--169. 

\item Y.H. Zhou and H.Q. Wu and N.H. Yu and D.W. Zheng (2000). Excitation
of seasonal polar motion by atmospheric and oceanic angular momentums.
{\it Progress in Natural Science}, vol.10, pp.931--936. 

\item R. Wardle and J. Marshall (2000). Representation of eddies in
primitive equation models by a PV flux. {\it Journal of Physical
Oceanography}, vol.30, pp.2481--2503. 

\item P.S. Polito and O.T. Sato and W.T. Liu (2000). Characterization and
validation of the heat storage variability from TOPEX/Poseidon at four
oceanographic sites. {\it Journal of Geophysical Research-Oceans}, vol.105,
pp.16911--16921. 

\item R.M. Ponte and D. Stammer (2000). Global and regional axial ocean
angular momentum signals and length-of-day variations (1985-1996).
{\it Journal of Geophysical Research-Oceans}, vol.105, pp.17161--17171. 

\item J. Wunsch (2000). Oceanic influence on the annual polar motion.
{\it Journal of GEODYNAMICS}, vol.30, pp.389--399. 

\item D. Menemenlis and M. Chechelnitsky (2000). Error estimates for an
ocean general circulation model from altimeter and acoustic tomography
data. {\it Monthly Weather Review}, vol.128, pp.763--778. 

\item Y.H. Zhou and D.W. Zheng and N.H. Yu and H.Q. Wu (2000). Excitation
of annual polar motion by atmosphere and ocean. {\it Chinese Science
Bulletin}, vol.45, pp.139--142. 

\item J. Marotzke and R. Giering and K.Q. Zhang and D. Stammer and C. Hill
and T. Lee (1999). Construction of the adjoint MIT ocean general
circulation model and application to Atlantic heat transport
sensitivity. {\it Journal of Geophysical Research-Oceans}, vol.104,
pp.29529--29547. 

\item R.M. Ponte and D. Stammer (1999). Role of ocean currents and bottom
pressure variability on seasonal polar motion. {\it Journal of Geophysical
Research-Oceans}, vol.104, pp.23393--23409. 

\item J. Nastula and R.M. Ponte (1999). Further evidence for oceanic
excitation of polar motion. {\it Geophysical Journal International}, vol.139,
pp.123--130. 

\item K.Q. Zhang and J. Marotzke (1999). The importance of open-boundary
estimation for an Indian Ocean GCM-data synthesis. {\it Journal of Marine
Research}, vol.57, pp.305--334. 

\item J. Marshall and D. Jamous and J. Nilsson (1999). Reconciling
thermodynamic and dynamic methods of computation of water-mass
transformation rates. {\it Deep-Sea Research PART I-Oceanographic Research
Papers}, vol.46, pp.545--572. 

\item J. Marshall and F. Schott (1999). Open-ocean convection:
Observations, theory, and models. {\it Reviews of Geophysics}, vol.37,
pp.1--64. 

\item J. Marshall and H. Jones and C. Hill (1998). Efficient ocean
modeling using non-hydrostatic algorithms. {\it Journal of Marine Systems},
vol.18, pp.115--134. 

\item A.B. Baggeroer and T.G. Birdsall and C. Clark and J.A. Colosi and
B.D. Cornuelle and D. Costa and B.D. Dushaw and M. Dzieciuch and A.M.G.
Forbes and C. Hill and B.M. Howe and J. Marshall and D. Menemenlis and
J.A. Mercer and K. Metzger and W. Munk and R.C. Spindel and D. Stammer
and P.F. Worcester and C. Wunsch (1998). Ocean climate change:
Comparison of acoustic tomography, satellite altimetry, and modeling.
{\it Science}, vol.281, pp.1327--1332. 

\item C. Wunsch and D. Stammer (1998). Satellite altimetry, the marine
geoid, and the oceanic general circulation. {\it Annual Review of Earth and
Planetary Sciences}, vol.26, pp.219--253. 

\item A. Shaw and Arvind and K.C. Cho and C. Hill and R.P. Johnson and J.
Marshall (1998). A comparison of implicitly parallel multithreaded and
data-parallel implementations of an ocean model. {\it Journal of Parallel and
Distributed Computing}, vol.48, pp.1--51. 

\item T.W.N. Haine and J. Marshall (1998). Gravitational, symmetric, and
baroclinic instability of the ocean mixed layer. {\it Journal of Physical
Oceanography}, vol.28, pp.634--658. 

\item R.M. Ponte and D. Stammer and J. Marshall (1998). Oceanic signals in
observed motions of the Earth's pole of rotation. {\it Nature}, vol.391,
pp.476--479. 

\item D. Menemenlis and C. Wunsch (1997). Linearization of an oceanic
general circulation model for data assimilation and climate studies.
{\it Journal of Atmospheric and Oceanic Technology}, vol.14, pp.1420--1443. 

\item H. Jones and J. Marshall (1997). Restratification after deep
convection. {\it Journal of Physical Oceanography}, vol.27, pp.2276--2287. 

\item S.R. Jayne and R. Tokmakian (1997). Forcing and sampling of ocean
general circulation models: Impact of high-frequency motions. {\it Journal of
Physical Oceanography}, vol.27, pp.1173--1179. 

\end{itemize}
