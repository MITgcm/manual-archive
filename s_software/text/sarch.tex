
In this chapter we describe the software architecture and
implementation strategy for the MITgcm code. The first part of this
chapter discusses the MITgcm architecture at an abstract level. In the second 
part of the chapter we described practical details of the MITgcm implementation 
and of current tools and operating system features that are employed.

\section{Overall architectural goals}

Broadly, the goals of the software architecture employed in MITgcm are 
three-fold
 
\begin{itemize}

\item We wish to be able to study a very broad range
of interesting and challenging rotating fluids problems.

\item We wish the model code to be readily targeted to
a wide range of platforms

\item On any given platform we would like to be
able to achieve performance comparable to an implementation 
developed and specialized specifically for that platform.

\end{itemize}

These points are summarized in figure \ref{fig:mitgcm_architecture_goals} 
which conveys the goals of the MITgcm design. The goals lead to
a software architecture which at the high-level can be viewed as consisting 
of

\begin{enumerate}

\item A core set of numerical and support code. This is discussed in detail in
section \ref{sec:partII}.

\item A scheme for supporting optional "pluggable" {\bf packages} (containing 
for example mixed-layer schemes, biogeochemical schemes, atmospheric physics). 
These packages are used both to overlay alternate dynamics and to introduce 
specialized physical content onto the core numerical code. An overview of
the {\bf package} scheme is given at the start of part \ref{part:packages}.


\item A support framework called {\bf WRAPPER} (Wrappable Application Parallel 
Programming Environment Resource), within which the core numerics and pluggable 
packages operate.

\end{enumerate}

This chapter focuses on describing the {\bf WRAPPER} environment under which
both the core numerics and the pluggable packages function. The description
presented here is intended to be a detailed exposistion and contains significant
background material, as well as advanced details on working with the WRAPPER. 
The examples section of this manual (part \ref{part:example}) contains more
succinct, step-by-step instructions on running basic numerical
experiments both sequentially and in parallel. For many projects simply 
starting from an example code and adapting it to suit a particular situation 
will be all that is required.

\begin{figure}
\begin{center}
 \resizebox{!}{2.5in}{
  \includegraphics*[1.5in,2.4in][9.5in,6.3in]{part4/mitgcm_goals.eps}
 }
\end{center}
\caption{The MITgcm architecture is designed to allow simulation of a wide 
range of physical problems on a wide range of hardware. The computational 
resource requirements of the applications targeted range from around 
$10^7$ bytes ( $\approx 10$ megabytes ) of memory to $10^{11}$ bytes
( $\approx 100$ gigabytes). Arithmetic operation counts for the applications of 
interest range from $10^{9}$ floating point operations to more than $10^{17}$ 
floating point operations.} \label{fig:mitgcm_architecture_goals}
\end{figure}

\section{WRAPPER}

A significant element of the software architecture utilized in
MITgcm is a software superstructure and substructure collectively
called the WRAPPER (Wrappable Application Parallel Programming 
Environment Resource). All numerical and support code in MITgcm is written 
to ``fit'' within the WRAPPER infrastructure. Writing code to ``fit'' within 
the WRAPPER means that coding has to follow certain, relatively
straightforward, rules and conventions ( these are discussed further in 
section \ref{sec:specifying_a_decomposition} ).

The approach taken by the WRAPPER is illustrated in figure 
\ref{fig:fit_in_wrapper} which shows how the WRAPPER serves to insulate code 
that fits within it from architectural differences between hardware platforms 
and operating systems. This allows numerical code to be easily retargetted. 
\begin{figure}
\begin{center}
 \resizebox{6in}{4.5in}{
  \includegraphics*[0.6in,0.7in][9.0in,8.5in]{part4/fit_in_wrapper.eps}
 }
\end{center}
\caption{ Numerical code is written too fit within a software support
infrastructure called WRAPPER. The WRAPPER is portable and
can be sepcialized for a wide range of specific target hardware and 
programming environments, without impacting numerical code that fits
within the WRAPPER. Codes that fit within the WRAPPER can generally be
made to run as fast on a particular platform as codes specially 
optimized for that platform.
} \label{fig:fit_in_wrapper}
\end{figure}

\subsection{Target hardware}
\label{sec:target_hardware}

The WRAPPER is designed to target as broad as possible a range of computer
systems. The original development of the WRAPPER took place on a 
multi-processor, CRAY Y-MP system. On that system, numerical code performance 
and scaling under the WRAPPER was in excess of that of an implementation that 
was tightly bound to the CRAY systems proprietary multi-tasking and 
micro-tasking approach. Later developments have been carried out on 
uniprocessor and multi-processor Sun systems with both uniform memory access 
(UMA) and non-uniform memory access (NUMA) designs. Significant work has also 
been undertaken on x86 cluster systems, Alpha processor based clustered SMP 
systems, and on cache-coherent NUMA (CC-NUMA) systems from Silicon Graphics. 
The MITgcm code, operating within the WRAPPER, is also used routinely used on 
large scale MPP systems (for example T3E systems and IBM SP systems). In all 
cases numerical code, operating within the WRAPPER, performs and scales very 
competitively with equivalent numerical code that has been modified to contain 
native optimizations for a particular system \ref{ref hoe and hill, ecmwf}.

\subsection{Supporting hardware neutrality}

The different systems listed in section \ref{sec:target_hardware} can be 
categorized in many different ways. For example, one common distinction is 
between shared-memory parallel systems (SMP's, PVP's) and distributed memory 
parallel systems (for example x86 clusters and large MPP systems). This is one 
example of a difference between compute platforms that can impact an 
application. Another common distinction is between vector processing systems 
with highly specialized CPU's and memory subsystems and commodity 
microprocessor based systems. There are numerous other differences, especially 
in relation to how parallel execution is supported. To capture the essential 
differences between different platforms the WRAPPER uses a {\it machine model}. 

\subsection{WRAPPER machine model}

Applications using the WRAPPER are not written to target just one 
particular machine (for example an IBM SP2) or just one particular family or 
class of machines (for example Parallel Vector Processor Systems). Instead the
WRAPPER provides applications with an 
abstract {\it machine model}. The machine model is very general, however, it can
easily be specialized to fit, in a computationally effificent manner, any 
computer architecture currently available to the scientific computing community.

\subsection{Machine model parallelism}

 Codes operating under the WRAPPER target an abstract machine that is assumed to
consist of one or more logical processors that can compute concurrently.  
Computational work is divided amongst the logical
processors by allocating ``ownership'' to 
each processor of a certain set (or sets) of calculations. Each set of 
calculations owned by a particular processor is associated with a specific 
region of the physical space that is being simulated, only one processor will 
be associated with each such region (domain decomposition).  

In a strict sense the logical processors over which work is divided do not need 
to correspond to physical processors. It is perfectly possible to execute a 
configuration decomposed for multiple logical processors on a single physical 
processor. This helps ensure that numerical code that is written to fit
within the WRAPPER will parallelize with no additional effort and is
also useful when debugging codes. Generally, however,
the computational domain will be subdivided over multiple logical
processors in order to then bind those logical processors to physical 
processor resources that can compute in parallel.

\subsubsection{Tiles}

Computationally, associated with each region of physical 
space allocated to a particular logical processor, there will be data 
structures (arrays, scalar variables etc...) that hold the simulated state of 
that region. We refer to these data structures as being {\bf owned} by the 
pprocessor to which their
associated region of physical space has been allocated. Individual
regions that are allocated to processors are called {\bf tiles}. A 
processor can own more 
than one tile. Figure \ref{fig:domaindecomp} shows a physical domain being
mapped to a set of logical processors, with each processors owning a single 
region of the domain (a single tile). Except for periods of
communication and coordination, each processor computes autonomously, working 
only with data from the tile (or tiles) that the processor owns. When multiple 
tiles are alloted to a single processor, each tile is computed on 
independently of the other tiles, in a sequential fashion. 

\begin{figure}
\begin{center}
 \resizebox{7in}{3in}{
  \includegraphics*[0.5in,2.7in][12.5in,6.4in]{part4/domain_decomp.eps}
 }
\end{center}
\caption{ The WRAPPER provides support for one and two dimensional 
decompositions of grid-point domains. The figure shows a hypothetical domain of 
total size $N_{x}N_{y}N_{z}$. This hypothetical domain is decomposed in 
two-dimensions along the $N_{x}$ and $N_{y}$ directions. The resulting {\bf 
tiles} are {\bf owned} by different processors. The {\bf owning} 
processors perform the 
arithmetic operations associated with a {\bf tile}. Although not illustrated 
here, a single processor can {\bf own} several {\bf tiles}.
Whenever a processor wishes to transfer data between tiles or
communicate with other processors it calls a WRAPPER supplied
function.
} \label{fig:domaindecomp} 
\end{figure}


\subsubsection{Tile layout}

Tiles consist of an interior region and an overlap region. The overlap region 
of a tile corresponds to the interior region of an adjacent tile. 
In figure \ref{fig:tiledworld} each tile would own the region
within the black square and hold duplicate information for overlap
regions extending into the tiles to the north, south, east and west.
During 
computational phases a processor will reference data in an overlap region 
whenever it requires values that outside the domain it owns. Periodically 
processors will make calls to WRAPPER functions to communicate data between 
tiles, in order to keep the overlap regions up to date (see section 
\ref{sec:communication_primitives}). The WRAPPER functions can use a
variety of different mechanisms to communicate data between tiles.

\begin{figure}
\begin{center}
 \resizebox{7in}{3in}{
  \includegraphics*[4.5in,3.7in][12.5in,6.7in]{part4/tiled-world.eps}
 }
\end{center}
\caption{ A global grid subdivided into tiles.
Tiles contain a interior region and an overlap region.
Overlap regions are periodically updated from neighboring tiles.
} \label{fig:tiledworld} 
\end{figure}

\subsection{Communication mechanisms}

 Logical processors are assumed to be able to exchange information
between tiles and between each other using at least one of two possible 
mechanisms. 

\begin{itemize}
\item {\bf Shared memory communication}.
Under this mode of communication data transfers are assumed to be possible 
using direct addressing of regions of memory. In this case a CPU is able to read
(and write) directly to regions of memory "owned" by another CPU
using simple programming language level assignment operations of the
the sort shown in figure \ref{fig:simple_assign}. In this way one CPU
(CPU1 in the figure) can communicate information to another CPU (CPU2 in the 
figure) by assigning a particular value to a particular memory location.

\item {\bf Distributed memory communication}. 
Under this mode of communication there is no mechanism, at the application code level,
for directly addressing regions of memory owned and visible to another CPU. Instead 
a communication library must be used as illustrated in figure 
\ref{fig:comm_msg}. In this case CPU's must call a function in the API of the 
communication library to communicate data from a tile that it owns to a tile
that another CPU owns. By default the WRAPPER binds to the MPI communication
library \ref{MPI} for this style of communication.
\end{itemize}

The WRAPPER assumes that communication will use one of these two styles
of communication. The underlying hardware and operating system support
for the style used is not specified and can vary from system to system.

\begin{figure}
\begin{verbatim}

         CPU1                    |        CPU2
         ====                    |        ====
                                 |
       a(3) = 8                  |        WHILE ( a(3) .NE. 8 ) 
                                 |         WAIT
                                 |        END WHILE
                                 |
\end{verbatim}
\caption{ In the WRAPPER shared memory communication model, simple writes to an
array can be made to be visible to other CPU's at the application code level.
So that for example, if one CPU (CPU1 in the figure above) writes the value $8$ to 
element $3$ of array $a$, then other CPU's (for example CPU2 in the figure above)
will be able to see the value $8$ when they read from $a(3)$.
This provides a very low latency and high bandwidth communication 
mechanism.
} \label{fig:simple_assign}
\end{figure}

\begin{figure}
\begin{verbatim}

         CPU1                    |        CPU2
         ====                    |        ====
                                 |
       a(3) = 8                  |        WHILE ( a(3) .NE. 8 )
       CALL SEND( CPU2,a(3) )    |         CALL RECV( CPU1, a(3) )
                                 |        END WHILE
                                 |
\end{verbatim}
\caption{ In the WRAPPER distributed memory communication model
data can not be made directly visible to other CPU's.
If one CPU writes the value $8$ to element $3$ of array $a$, then
at least one of CPU1 and/or CPU2 in the figure above will need
to call a bespoke communication library in order for the updated 
value to be communicated between CPU's.
} \label{fig:comm_msg}
\end{figure}

\subsection{Shared memory communication}
\label{sec:shared_memory_communication}

Under shared communication independent CPU's are operating
on the exact same global address space at the application level.
This means that CPU 1 can directly write into global 
data structures that CPU 2 ``owns'' using a simple
assignment at the application level. 
This is the model of memory access is supported at the basic system 
design level in ``shared-memory'' systems such as PVP systems, SMP systems,
and on distributed shared memory systems (the SGI Origin).
On such systems the WRAPPER will generally use simple read and write statements 
to access directly application data structures when communicating between CPU's.

In a system where assignments statements, like the one in figure 
\ref{fig:simple_assign} map directly to 
hardware instructions that transport data between CPU and memory banks, this 
can be a very efficient mechanism for communication. In this case two CPU's,
CPU1 and CPU2, can communicate simply be reading and writing to an 
agreed location and following a few basic rules. The latency of this sort
of communication is generally not that much higher than the hardware
latency of other memory accesses on the system. The bandwidth available
between CPU's communicating in this way can be close to the bandwidth of
the systems main-memory interconnect. This can make this method of 
communication very efficient provided it is used appropriately.

\subsubsection{Memory consistency}
\label{sec:memory_consistency}

When using shared memory communication between
multiple processors the WRAPPER level shields user applications from 
certain counter-intuitive system behaviors. In particular, one issue the 
WRAPPER layer must deal with is a systems memory model. In general the order 
of reads and writes expressed by the textual order of an application code may 
not be the ordering of instructions executed by the processor performing the 
application. The processor performing the application instructions will always
operate so that, for the application instructions the processor is executing, 
any reordering is not apparent. However, in general machines are often 
designed so that reordering of instructions is not hidden from other second 
processors.  This means that, in general, even on a shared memory system two 
processors can observe inconsistent memory values. 

The issue of memory consistency between multiple processors is discussed at 
length in many computer science papers, however, from a practical point of 
view, in order to deal with this issue, shared memory machines all provide 
some mechanism to enforce memory consistency when it is needed. The exact 
mechanism employed will vary between systems. For communication using shared 
memory, the WRAPPER provides a place to invoke the appropriate mechanism to 
ensure memory consistency for a particular platform.

\subsubsection{Cache effects and false sharing}
\label{sec:cache_effects_and_false_sharing}

Shared-memory machines often have local to processor memory caches
which contain mirrored copies of main memory. Automatic cache-coherence
protocols are used to maintain consistency between caches on different
processors. These cache-coherence protocols typically enforce consistency
between regions of memory with large granularity (typically 128 or 256 byte
chunks). The coherency protocols employed can be expensive relative to other
memory accesses and so care is taken in the WRAPPER (by padding synchronization
structures appropriately) to avoid unnecessary coherence traffic.

\subsubsection{Operating system support for shared memory.}

Applications running under multiple threads within a single process can
use shared memory communication. In this case {\it all} the memory locations 
in an application are potentially visible to all the compute threads. Multiple 
threads operating within a single process is the standard mechanism for 
supporting shared memory that the WRAPPER utilizes. Configuring and launching 
code to run in multi-threaded mode on specific platforms is discussed in 
section \ref{sec:running_with_threads}.  However, on many systems, potentially 
very efficient mechanisms for using shared memory communication between 
multiple processes (in contrast to multiple threads within a single 
process) also exist. In most cases this works by making a limited region of 
memory shared between processes. The MMAP \ref{magicgarden} and 
IPC \ref{magicgarden} facilities in UNIX systems provide this capability as do 
vendor specific tools like LAPI \ref{IBMLAPI} and IMC \ref{Memorychannel}. 
Extensions exist for the WRAPPER that allow these mechanisms 
to be used for shared memory communication. However, these mechanisms are not 
distributed with the default WRAPPER sources, because of their proprietary 
nature.

\subsection{Distributed memory communication}
\label{sec:distributed_memory_communication}
Many parallel systems are not constructed in a way where it is
possible or practical for an application to use shared memory
for communication. For example cluster systems consist of individual computers
connected by a fast network. On such systems their is no notion of shared memory
at the system level. For this sort of system the WRAPPER provides support
for communication based on a bespoke communication library 
(see figure \ref{fig:comm_msg}).  The default communication library used is MPI 
\ref{mpi}. However, it is relatively straightforward to implement bindings to 
optimized platform specific communication libraries. For example the work 
described in \ref{hoe-hill:99} substituted standard MPI communication for a 
highly optimized library.

\subsection{Communication primitives}
\label{sec:communication_primitives}

\begin{figure}
\begin{center}
 \resizebox{5in}{3in}{
  \includegraphics*[1.5in,0.7in][7.9in,4.4in]{part4/comm-primm.eps}
 }
\end{center}
\caption{Three performance critical parallel primititives are provided
by the WRAPPER. These primititives are always used to communicate data
between tiles. The figure shows four tiles. The curved arrows indicate
exchange primitives which transfer data between the overlap regions at tile
edges and interior regions for nearest-neighbor tiles.
The straight arrows symbolize global sum operations which connect all tiles.
The global sum operation provides both a key arithmetic primitive and can
serve as a synchronization primitive. A third barrier primitive is also
provided, it behaves much like the global sum primitive.
} \label{fig:communication_primitives}
\end{figure}


Optimized communication support is assumed to be possibly available 
for a small number of communication operations.
It is assumed that communication performance optimizations can
be achieved by optimizing a small number of communication primitives.
Three optimizable primitives are provided by the WRAPPER 
\begin{itemize}
\item{\bf EXCHANGE} This operation is used to transfer data between interior
and overlap regions of neighboring tiles. A number of different forms of this
operation are supported. These different forms handle
\begin{itemize}
\item Data type differences. Sixty-four bit and thirty-two bit fields may be handled 
separately.
\item Bindings to different communication methods.
Exchange primitives select between using shared memory or distributed
memory communication.
\item Transformation operations required when transporting
data between different grid regions. Transferring data
between faces of a cube-sphere grid, for example, involves a rotation
of vector components.
\item Forward and reverse mode computations. Derivative calculations require 
tangent linear and adjoint forms of the exchange primitives.

\end{itemize}

\item{\bf GLOBAL SUM} The global sum operation is a central arithmetic
operation for the pressure inversion phase of the MITgcm algorithm.
For certain configurations scaling can be highly sensitive to
the performance of the global sum primitive. This operation is a collective
operation involving all tiles of the simulated domain. Different forms
of the global sum primitive exist for handling
\begin{itemize}
\item Data type differences. Sixty-four bit and thirty-two bit fields may be handled
separately.
\item Bindings to different communication methods.
Exchange primitives select between using shared memory or distributed
memory communication.
\item Forward and reverse mode computations. Derivative calculations require
tangent linear and adjoint forms of the exchange primitives.
\end{itemize}

\item{\bf BARRIER} The WRAPPER provides a global synchronization function
called barrier. This is used to synchronize computations over all tiles.
The {\bf BARRIER} and {\bf GLOBAL SUM} primitives have much in common and in
some cases use the same underlying code.
\end{itemize}


\subsection{Memory architecture}

The WRAPPER machine model is aimed to target efficiently systems with
highly pipelined memory architectures and systems with deep memory
hierarchies that favor memory reuse. This is achieved by supporting a 
flexible tiling strategy as shown in figure \ref{fig:tiling-strategy}. 
Within a CPU computations are carried out sequentially on each tile
in turn. By reshaping tiles according to the target platform it is 
possible to automatically tune code to improve memory performance.
On a vector machine a given domain might be sub-divided into a few
long, thin regions. On a commodity microprocessor based system, however,
the same region could be simulated use many more smaller
sub-domains.


\begin{figure}
\begin{center}
 \resizebox{5in}{3in}{
  \includegraphics*[0.5in,1.3in][7.9in,5.7in]{part4/tiling_detail.eps}
 }
\end{center}
\caption{The tiling strategy that the WRAPPER supports allows tiles
to be shaped to suit the underlying system memory architecture.
Compact tiles that lead to greater memory reuse can be used on cache
based systems (upper half of figure) with deep memory hierarchies, long tiles 
with large inner loops can be used to exploit vector systems having
highly pipelined memory systems.
} \label{fig:tiling-strategy}
\end{figure}

\newpage
\subsection{Summary}
Following the discussion above, the machine model that the WRAPPER
presents to an application has the following characteristics

\begin{itemize}
\item The machine consists of one or more logical processors. \vspace{-3mm}
\item Each processor operates on tiles that it owns.\vspace{-3mm}
\item A processor may own more than one tile.\vspace{-3mm}
\item Processors may compute concurrently.\vspace{-3mm}
\item Exchange of information between tiles is handled by the
machine (WRAPPER) not by the application.
\end{itemize}
Behind the scenes this allows the WRAPPER to adapt the machine model
functions to exploit hardware on which
\begin{itemize}
\item Processors may be able to communicate very efficiently with each other 
using shared memory. \vspace{-3mm}
\item An alternative communication mechanism based on a relatively
simple inter-process communication API may be required.\vspace{-3mm}
\item Shared memory may not necessarily obey sequential consistency,
however some mechanism will exist for enforcing memory consistency. 
\vspace{-3mm}
\item Memory consistency that is enforced at the hardware level
may be expensive. Unnecessary triggering of consistency protocols
should be avoided. \vspace{-3mm}
\item Memory access patterns may need to either repetitive or highly
pipelined for optimum hardware performance. \vspace{-3mm}
\end{itemize}

This generic model captures the essential hardware ingredients
of almost all successful scientific computer systems designed in the
last 50 years.

\section{Using the WRAPPER}

In order to support maximum portability the WRAPPER is implemented primarily 
in sequential Fortran 77. At a practical level the key steps provided by the 
WRAPPER are
\begin{enumerate}
\item specifying how a domain will be decomposed
\item starting a code in either sequential or parallel modes of operations
\item controlling communication between tiles and between concurrently
computing CPU's.
\end{enumerate} 
This section describes the details of each of these operations.
Section \ref{sec:specifying_a_decomposition} explains how the way in which
a domain is decomposed (or composed) is expressed. Section 
\ref{sec:starting_a_code} describes practical details of running codes
in various different parallel modes on contemporary computer systems. 
Section \ref{sec:controlling_communication} explains the internal information
that the WRAPPER uses to control how information is communicated between
tiles.

\subsection{Specifying a domain decomposition}
\label{sec:specifying_a_decomposition}

At its heart much of the WRAPPER works only in terms of a collection of tiles
which are interconnected to each other. This is also true of application
code operating within the WRAPPER. Application code is written as a series
of compute operations, each of which operates on a single tile. If
application code needs to perform operations involving data
associated with another tile, it uses a WRAPPER function to obtain
that data.
The specification of how a global domain is constructed from tiles or alternatively
how a global domain is decomposed into tiles is made in the file {\em SIZE.h}.
This file defines the following parameters \\

\fbox{ 
\begin{minipage}{4.75in}
Parameters: {\em sNx, sNy, OLx, OLy, nSx, nSy, nPx, nPy} \\
File: {\em model/inc/SIZE.h}
\end{minipage}
} \\

Together these parameters define a tiling decomposition of the style shown in 
figure \ref{fig:labelled_tile}. The parameters {\em sNx} and {\em sNy} define
the size of an individual tile. The parameters {\em OLx} and {\em OLy} define the
maximum size of the overlap extent. This must be set to the maximum width
of the computation stencil that the numerical code finite-difference operations
require between overlap region updates. The maximum overlap required
by any of the operations in the MITgcm code distributed with this release is three grid
points. This is set by the requirements of the $\nabla^4$ dissipation and 
diffusion operator. Code modifications and enhancements that involve adding wide 
finite-difference stencils may require increasing {\em OLx} and {\em OLy}.
Setting {\em OLx} and {\em OLy} to a too large value will decrease code
performance (because redundant computations will be performed), however it will
not cause any other problems.

\begin{figure}
\begin{center}
 \resizebox{5in}{7in}{
  \includegraphics*[0.5in,0.3in][7.9in,10.7in]{part4/size_h.eps}
 }
\end{center}
\caption{ The three level domain decomposition hierarchy employed by the
WRAPPER. A domain is composed of tiles. Multiple tiles can be allocated
to a single process. Multiple processes can exist, each with multiple tiles.
Tiles within a process can be spread over multiple compute threads.
} \label{fig:labelled_tile}
\end{figure}

 The parameters {\em nSx} and {\em nSy} specify the number of tiles that will
be created within a single process. Each of these tiles will have internal
dimensions of {\em sNx} and {\em sNy}. If, when the code is executed, these tiles are 
allocated to different threads of a process that are then bound to
different physical processors ( see the multi-threaded
execution discussion in section \ref{sec:starting_the_code} ) then
computation will be performed concurrently on each tile. However, it is also
possible to run the same decomposition within a process running a single thread on
a single processor. In this case the tiles will be computed over sequentially.
If the decomposition is run in a single process running multiple threads
but attached to a single physical processor, then, in general, the computation
for different tiles will be interleaved by system level software.
This too is a valid mode of operation.

 The parameters {\em sNx, sNy, OLx, OLy, nSx} and{\em nSy} are used extensively by
numerical code. The settings of {\em sNx, sNy, OLx} and {\em OLy}
are used to form the loop ranges for many numerical calculations and
to provide dimensions for arrays holding numerical state.
The {\em nSx} and{\em nSy} are used in conjunction with the thread number
parameter {\em myThid}. Much of the numerical code operating within the
WRAPPER takes the form 
\begin{verbatim}
      DO bj=myByLo(myThid),myByHi(myThid)
       DO bi=myBxLo(myThid),myBxHi(myThid)
          :
          a block of computations ranging 
          over 1,sNx +/- OLx and 1,sNy +/- OLy grid points
          :
       ENDDO
      ENDDO

      communication code to sum a number or maybe update
      tile overlap regions

      DO bj=myByLo(myThid),myByHi(myThid)
       DO bi=myBxLo(myThid),myBxHi(myThid)
          :
          another block of computations ranging 
          over 1,sNx +/- OLx and 1,sNy +/- OLy grid points
          :
       ENDDO
      ENDDO
\end{verbatim}
The variables {\em myBxLo(myThid), myBxHi(myThid), myByLo(myThid)} and {\em
myByHi(myThid)} set the bounds of the loops in {\em bi} and {\em bj} in this 
schematic. These variables specify the subset of the tiles in
the range 1,{\em nSx} and 1,{\em nSy} that the logical processor bound to
thread number {\em myThid} owns. The thread number variable {\em myThid} 
ranges from 1 to the total number of threads requested at execution time.
For each value of {\em myThid} the loop scheme above will step sequentially
through the tiles owned by that thread. However, different threads will
have different ranges of tiles assigned to them, so that separate threads can
compute iterations of the {\em bi}, {\em bj} loop concurrently.
Within a {\em bi}, {\em bj} loop
computation is performed concurrently over as many processes and threads
as there are physical processors available to compute.

The amount of computation that can be embedded
a single loop over {\em bi} and {\em bj} varies for different parts of the
MITgcm algorithm. Figure \ref{fig:bibj_extract} shows a code extract
from the two-dimensional implicit elliptic solver. This portion of the
code computes the l2Norm of a vector whose elements are held in
the array {\em cg2d\_r} writing the final result to scalar variable
{\em err}. In this case, because the l2norm requires a global reduction,
the {\em bi},{\em bj} loop only contains one statement. This computation
phase is then followed by a communication phase in which all threads and
processes must participate. However,
in other areas of the MITgcm code entries subsections of code are within
a single {\em bi},{\em bj} loop. For example the evaluation of all
the momentum equation prognostic terms ( see {\em S/R DYNAMICS()})
is within a single {\em bi},{\em bj} loop.

\begin{figure}
\begin{verbatim}
      REAL*8  cg2d_r(1-OLx:sNx+OLx,1-OLy:sNy+OLy,nSx,nSy)
      REAL*8  err
          :
          :
        other computations
          :
          :
      err = 0.
      DO bj=myByLo(myThid),myByHi(myThid)
       DO bi=myBxLo(myThid),myBxHi(myThid)
        DO J=1,sNy
         DO I=1,sNx
           err            = err            +
     &     cg2d_r(I,J,bi,bj)*cg2d_r(I,J,bi,bj)
         ENDDO
        ENDDO
       ENDDO
      ENDDO

      CALL GLOBAL_SUM_R8( err   , myThid )
      err = SQRT(err)

\end{verbatim}
\caption{Example of numerical code for calculating
the l2-norm of a vector within the WRAPPER. Notice that
under the WRAPPER arrays such as {\em cg2d\_r} have two extra trailing
dimensions. These right most indices are tile indexes. Different
threads with a single process operate on different ranges of tile
index, as controlled by the settings of
{\em myByLo, myByHi, myBxLo} and {\em myBxHi}.
} \label{fig:bibj_extract}
\end{figure}

 The final decomposition parameters are {\em nPx} and {\em nPy}. These parameters
are used to indicate to the WRAPPER level how many processes (each with
{\em nSx}$\times${\em nSy} tiles) will be used for this simulation. 
This information is needed during initialization and during I/O phases.
However, unlike the variables {\em sNx, sNy, OLx, OLy, nSx} and {\em nSy}
the values of {\em nPx} and {\em nPy} are absent
from the core numerical and support code.

\subsubsection{Examples of {\em SIZE.h} specifications}

The following different {\em SIZE.h} parameter setting illustrate how to
interpret the values of {\em sNx, sNy, OLx, OLy, nSx, nSy, nPx}
and {\em nPy}.
\begin{enumerate}
\item
\begin{verbatim}
      PARAMETER (
     &           sNx =  90,
     &           sNy =  40,
     &           OLx =   3,
     &           OLy =   3,
     &           nSx =   1,
     &           nSy =   1,
     &           nPx =   1,
     &           nPy =   1)
\end{verbatim}
This sets up a single tile with x-dimension of ninety grid points, y-dimension of
forty grid points, and x and y overlaps of three grid points each.
\item
\begin{verbatim}
      PARAMETER (
     &           sNx =  45,
     &           sNy =  20,
     &           OLx =   3,
     &           OLy =   3,
     &           nSx =   1,
     &           nSy =   1,
     &           nPx =   2,
     &           nPy =   2)
\end{verbatim}
This sets up tiles with x-dimension of forty-five grid points, y-dimension of
twenty grid points, and x and y overlaps of three grid points each. There are
four tiles allocated to four separate processes ({\em nPx=2,nPy=2}) and
arranged so that the global domain size is again ninety grid points in x and
forty grid points in y. In general the formula for global grid size (held in
model variables {\em Nx} and {\em Ny}) is
\begin{verbatim}
                 Nx  = sNx*nSx*nPx
                 Ny  = sNy*nSy*nPy
\end{verbatim}
\item
\begin{verbatim}
      PARAMETER (
     &           sNx =  90,
     &           sNy =  10,
     &           OLx =   3,
     &           OLy =   3,
     &           nSx =   1,
     &           nSy =   2,
     &           nPx =   1,
     &           nPy =   2)
\end{verbatim}
This sets up tiles with x-dimension of ninety grid points, y-dimension of
ten grid points, and x and y overlaps of three grid points each. There are
four tiles allocated to two separate processes ({\em nPy=2}) each of which
has two separate sub-domains {\em nSy=2},
The global domain size is again ninety grid points in x and
forty grid points in y. The two sub-domains in each process will be computed 
sequentially if they are given to a single thread within a single process.
Alternatively if the code is invoked with multiple threads per process
the two domains in y may be computed on concurrently.
\item
\begin{verbatim}
      PARAMETER (
     &           sNx =  32,
     &           sNy =  32,
     &           OLx =   3,
     &           OLy =   3,
     &           nSx =   6,
     &           nSy =   1,
     &           nPx =   1,
     &           nPy =   1)
\end{verbatim}
This sets up tiles with x-dimension of thirty-two grid points, y-dimension of
thirty-two grid points, and x and y overlaps of three grid points each. 
There are six tiles allocated to six separate logical processors ({\em nSx=6}).
This set of values can be used for a cube sphere calculation.
Each tile of size $32 \times 32$ represents a face of the
cube. Initialising the tile connectivity correctly ( see section
\ref{sec:cube_sphere_communication}. allows the rotations associated with
moving between the six cube faces to be embedded within the 
tile-tile communication code.
\end{enumerate}


\subsection{Starting the code}
\label{sec:starting_the_code}
When code is started under the WRAPPER, execution begins in a main routine {\em
eesupp/src/main.F} that is owned by the WRAPPER. Control is transferred 
to the application through a routine called {\em THE\_MODEL\_MAIN()}
once the WRAPPER has initialized correctly and has created the necessary variables
to support subsequent calls to communication routines
by the application code. The startup calling sequence followed by the 
WRAPPER is shown in figure \ref{fig:wrapper_startup}.


\begin{figure}
\begin{verbatim}

       MAIN  
       |
       |--EEBOOT               :: WRAPPER initialization
       |  |
       |  |-- EEBOOT_MINMAL    :: Minimal startup. Just enough to
       |  |                       allow basic I/O.
       |  |-- EEINTRO_MSG      :: Write startup greeting.
       |  |
       |  |-- EESET_PARMS      :: Set WRAPPER parameters
       |  |
       |  |-- EEWRITE_EEENV    :: Print WRAPPER parameter settings
       |  |
       |  |-- INI_PROCS        :: Associate processes with grid regions.
       |  |
       |  |-- INI_THREADING_ENVIRONMENT   :: Associate threads with grid regions.
       |       |
       |       |--INI_COMMUNICATION_PATTERNS :: Initialize between tile 
       |                                     :: communication data structures
       |
       |
       |--CHECK_THREADS    :: Validate multiple thread start up.
       |
       |--THE_MODEL_MAIN   :: Numerical code top-level driver routine


\end{verbatim}
\caption{Main stages of the WRAPPER startup procedure.
This process proceeds transfer of control to application code, which
occurs through the procedure {\em THE\_MODEL\_MAIN()}.
} \label{fig:wrapper_startup}
\end{figure}

\subsubsection{Multi-threaded execution}
Prior to transferring control to the procedure {\em THE\_MODEL\_MAIN()} the
WRAPPER may cause several coarse grain threads to be initialized. The routine
{\em THE\_MODEL\_MAIN()} is called once for each thread and is passed a single
stack argument which is the thread number, stored in the
variable {\em myThid}. In addition to specifying a decomposition with
multiple tiles per process ( see section \ref{sec:specifying_a_decomposition}) 
configuring and starting a code to run using multiple threads requires the following
steps.\\

\paragraph{Compilation}
First the code must be compiled with appropriate multi-threading directives 
active in the file {\em main.F} and with appropriate compiler flags
to request multi-threading support. The header files 
{\em MAIN\_PDIRECTIVES1.h} and {\em MAIN\_PDIRECTIVES2.h}
contain directives compatible with compilers for Sun, Compaq, SGI,
Hewlett-Packard SMP systems and CRAY PVP systems. These directives can be 
activated by using compile time
directives {\em -DTARGET\_SUN}, 
{\em -DTARGET\_DEC}, {\em -DTARGET\_SGI}, {\em -DTARGET\_HP}
or {\em -DTARGET\_CRAY\_VECTOR} respectively. Compiler options
for invoking multi-threaded compilation vary from system to system
and from compiler to compiler. The options will be described
in the individual compiler documentation. For the Fortran compiler 
from Sun the following options are needed to correctly compile
multi-threaded code
\begin{verbatim}
     -stackvar -explicitpar -vpara -noautopar
\end{verbatim}
These options are specific to the Sun compiler. Other compilers
will use different syntax that will be described in their
documentation. The effect of these options is as follows
\begin{enumerate}
\item {\bf -stackvar} Causes all local variables to be allocated in stack 
storage. This is necessary for local variables to ensure that they are private 
to their thread. Note, when using this option it may be necessary to override 
the default limit on stack-size that the operating system assigns to a process. 
This can normally be done by changing the settings of the command shells
{\em stack-size} limit variable. However, on some systems changing this limit
will require privileged administrator access to modify system parameters.

\item {\bf -explicitpar} Requests that multiple threads be spawned
in response to explicit directives in the application code. These
directives are inserted with syntax appropriate to the particular target
platform when, for example, the {\em -DTARGET\_SUN} flag is selected.

\item {\bf -vpara} This causes the compiler to describe the multi-threaded
configuration it is creating. This is not required
but it can be useful when trouble shooting.

\item {\bf -noautopar} This inhibits any automatic multi-threaded 
parallelization the compiler may otherwise generate.

\end{enumerate}


\paragraph{Environment variables}
On most systems multi-threaded execution also requires the setting
of a special environment variable. On many machines this variable
is called PARALLEL and its values should be set to the number
of parallel threads required. Generally the help pages associated
with the multi-threaded compiler on a machine will explain
how to set the required environment variables for that machines.

\paragraph{Runtime input parameters}
Finally the file {\em eedata} needs to be configured to indicate
the number of threads to be used in the x and y directions.
The variables {\em nTx} and {\em nTy} in this file are used to
specify the information required. The product of {\em nTx} and
{\em nTy} must be equal to the number of threads spawned i.e.
the setting of the environment variable PARALLEL.
The value of {\em nTx} must subdivide the number of sub-domains
in x ({\em nSx}) exactly. The value of {\em nTy} must subdivide the 
number of sub-domains in y ({\em nSy}) exactly. 

An example of valid settings for the {\em eedata} file for a
domain with two subdomains in y and running with two threads is shown 
below
\begin{verbatim}
 nTx=1,nTy=2
\end{verbatim}
This set of values will cause computations to stay within a single
thread when moving across the {\em nSx} sub-domains. In the y-direction,
however, sub-domains will be split equally between two threads.

\paragraph{Multi-threading files and parameters} The following
files and variables are used in setting up multi-threaded execution.\\

\fbox{ 
\begin{minipage}{4.75in}
File: {\em eesupp/inc/MAIN\_PDIRECTIVES1.h}\\
File: {\em eesupp/inc/MAIN\_PDIRECTIVES2.h}\\
File: {\em model/src/THE\_MODEL\_MAIN.F}\\
File: {\em eesupp/src/MAIN.F}\\
File: {\em tools/genmake}\\
File: {\em eedata}\\
CPP:  {\em TARGET\_SUN}\\
CPP:  {\em TARGET\_DEC}\\
CPP:  {\em TARGET\_HP }\\
CPP:  {\em TARGET\_SGI}\\
CPP:  {\em TARGET\_CRAY\_VECTOR}\\
Parameter:  {\em nTx}\\
Parameter:  {\em nTy}
\end{minipage}
} \\

\subsubsection{Multi-process execution}

Despite its appealing programming model, multi-threaded execution remains
less common then multi-process execution. One major reason for this
is that many system libraries are still not ``thread-safe''. This means that for
example on some systems it is not safe to call system routines to
do I/O when running in multi-threaded mode, except for in a limited set of 
circumstances. Another reason is that support for multi-threaded programming
models varies between systems. 

Multi-process execution is more ubiquitous.
In order to run code in a multi-process configuration a decomposition
specification is given ( in which the at least one of the
parameters {\em nPx} or {\em nPy} will be greater than one)
and then, as for multi-threaded operation,
appropriate compile time and run time steps must be taken.

\paragraph{Compilation} Multi-process execution under the WRAPPER 
assumes that the portable, MPI libraries are available
for controlling the start-up of multiple processes. The MPI libraries
are not required, although they are usually used, for performance
critical communication. However, in order to simplify the task
of controlling and coordinating the start up of a large number
(hundreds and possibly even thousands) of copies of the same 
program, MPI is used. The calls to the MPI multi-process startup
routines must be activated at compile time. This is done
by setting the {\em ALLOW\_USE\_MPI} and {\em ALWAYS\_USE\_MPI}
flags in the {\em CPP\_EEOPTIONS.h} file.\\

\fbox{ 
\begin{minipage}{4.75in}
File: {\em eesupp/inc/CPP\_EEOPTIONS.h}\\
CPP:  {\em ALLOW\_USE\_MPI}\\
CPP:  {\em ALWAYS\_USE\_MPI}\\
Parameter:  {\em nPx}\\
Parameter:  {\em nPy}
\end{minipage}
} \\

Additionally, compile time options are required to link in the 
MPI libraries and header files. Examples of these options 
can be found in the {\em genmake} script that creates makefiles
for compilation. When this script is executed with the {bf -mpi}
flag it will generate a makefile that includes
paths for search for MPI head files and for linking in 
MPI libraries. For example the {\bf -mpi} flag on a
 Silicon Graphics IRIX system causes a
Makefile with the compilation command
Graphics IRIX system \begin{verbatim}
mpif77 -I/usr/local/mpi/include -DALLOW_USE_MPI -DALWAYS_USE_MPI
\end{verbatim}
to be generated.
This is the correct set of options for using the MPICH open-source
version of MPI, when it has been installed under the subdirectory
/usr/local/mpi.
However, on many systems there may be several
versions of MPI installed. For example many systems have both
the open source MPICH set of libraries and a vendor specific native form
of the MPI libraries. The correct setup to use will depend on the
local configuration of your system.\\

\fbox{ 
\begin{minipage}{4.75in}
File: {\em tools/genmake}
\end{minipage}
} \\
\paragraph{\bf Execution} The mechanics of starting a program in 
multi-process mode under MPI is not standardized. Documentation 
associated with the distribution of MPI installed on a system will
describe how to start a program using that distribution.
For the free, open-source MPICH system the MITgcm program is started
using a command such as
\begin{verbatim}
mpirun -np 64 -machinefile mf ./mitgcmuv
\end{verbatim}
In this example the text {\em -np 64} specifices the number of processes 
that will be created. The numeric value {\em 64} must be equal to the
product of the processor grid settings of {\em nPx} and {\em nPy}
in the file {\em SIZE.h}. The parameter {\em mf} specifies that a text file
called ``mf'' will be read to get a list of processor names on
which the sixty-four processes will execute. The syntax of this file
is specified by the MPI distribution
\\ 

\fbox{ 
\begin{minipage}{4.75in}
File: {\em SIZE.h}\\
Parameter: {\em nPx}\\
Parameter: {\em nPy}
\end{minipage}
} \\

The multiprocess startup of the MITgcm executable {\em mitgcmuv}
is controlled by the routines {\em EEBOOT\_MINIMAL()} and
{\em INI\_PROCS()}. The first routine performs basic steps required
to make sure each process is started and has a textual output
stream associated with it. By default two output files are opened
for each process with names {\bf STDOUT.NNNN} and {\bf STDERR.NNNN}.
The {\bf NNNNN} part of the name is filled in with the process
number so that process number 0 will create output files
{\bf STDOUT.0000} and {\bf STDERR.0000}, process number 1 will create
output files {\bf STDOUT.0001} and {\bf STDERR.0001} etc... These files
are used for reporting status and configuration information and
for reporting error conditions on a process by process basis.
The {{\em EEBOOT\_MINIMAL()} procedure also sets the variables 
{\em myProcId} and {\em MPI\_COMM\_MODEL}.
These variables are related
to processor identification are are used later in the routine
{\em INI\_PROCS()} to allocate tiles to processes.

Allocation of processes to tiles in controlled by the routine
{\em INI\_PROCS()}. For each process this routine sets
the variables {\em myXGlobalLo} and {\em myYGlobalLo}.
These variables specify (in index space) the coordinate
of the southern most and western most corner of the 
southern most and western most tile owned by this process.
The variables {\em pidW}, {\em pidE}, {\em pidS} and {\em pidN}
are also set in this routine. These are used to identify
processes holding tiles to the west, east, south and north 
of this process. These values are stored in global storage
in the header file {\em EESUPPORT.h} for use by
communication routines.
\\

\fbox{ 
\begin{minipage}{4.75in}
File: {\em eesupp/src/eeboot\_minimal.F}\\
File: {\em eesupp/src/ini\_procs.F} \\
File: {\em eesupp/inc/EESUPPORT.h} \\
Parameter: {\em myProcId} \\
Parameter: {\em MPI\_COMM\_MODEL} \\
Parameter: {\em myXGlobalLo} \\
Parameter: {\em myYGlobalLo} \\
Parameter: {\em pidW       } \\
Parameter: {\em pidE       } \\
Parameter: {\em pidS       } \\
Parameter: {\em pidN       }
\end{minipage}
} \\


\subsection{Controlling communication}
The WRAPPER maintains internal information that is used for communication
operations and that can be customized for different platforms. This section 
describes the information that is held and used.
\begin{enumerate}
\item {\bf Tile-tile connectivity information} For each tile the WRAPPER
sets a flag that sets the tile number to the north, south, east and
west of that tile. This number is unique over all tiles in a 
configuration. The number is held in the variables {\em tileNo}
( this holds the tiles own number), {\em tileNoN}, {\em tileNoS},
{\em tileNoE} and {\em tileNoW}. A parameter is also stored with each tile
that specifies the type of communication that is used between tiles.
This information is held in the variables {\em tileCommModeN},
{\em tileCommModeS}, {\em tileCommModeE} and {\em tileCommModeW}.
This latter set of variables can take one of the following values
{\em COMM\_NONE}, {\em COMM\_MSG}, {\em COMM\_PUT} and {\em COMM\_GET}.
A value of {\em COMM\_NONE} is used to indicate that a tile has no
neighbor to cummnicate with on a particular face. A value
of {\em COMM\_MSG} is used to indicated that some form of distributed
memory communication is required to communicate between
these tile faces ( see section \ref{sec:distributed_memory_communication}).
A value of {\em COMM\_PUT} or {\em COMM\_GET} is used to indicate 
forms of shared memory communication ( see section 
\ref{sec:shared_memory_communication}). The {\em COMM\_PUT} value indicates 
that a CPU should communicate by writing to data structures owned by another 
CPU. A {\em COMM\_GET} value indicates that a CPU should communicate by reading
from data structures owned by another CPU. These flags affect the behavior
of the WRAPPER exchange primitive 
(see figure \ref{fig:communication_primitives}). The routine 
{\em ini\_communication\_patterns()} is responsible for setting the
communication mode values for each tile.
\\

\fbox{ 
\begin{minipage}{4.75in}
File: {\em eesupp/src/ini\_communication\_patterns.F}\\
File: {\em eesupp/inc/EESUPPORT.h} \\
Parameter: {\em tileNo} \\
Parameter: {\em tileNoE} \\
Parameter: {\em tileNoW} \\
Parameter: {\em tileNoN} \\
Parameter: {\em tileNoS} \\
Parameter: {\em tileCommModeE} \\
Parameter: {\em tileCommModeW} \\
Parameter: {\em tileCommModeN} \\
Parameter: {\em tileCommModeS} \\
\end{minipage}
} \\

\item {\bf MP directives}
The WRAPPER transfers control to numerical application code through
the routine {\em THE\_MODEL\_MAIN}. This routine is called in a way
that allows for it to be invoked by several threads. Support for this
is based on using multi-processing (MP) compiler directives.
Most commercially available Fortran compilers support the generation
of code to spawn multiple threads through some form of compiler directives.
As this is generally much more convenient than writing code to interface
to operating system libraries to explicitly spawn threads, and on some systems
this may be the only method available the WRAPPER is distributed with
template MP directives for a number of systems.

 These directives are inserted into the code just before and after the 
transfer of control to numerical algorithm code through the routine
{\em THE\_MODEL\_MAIN}. Figure \ref{fig:mp_directives} shows an example of 
the code that performs this process for a Silicon Graphics system.
This code is extracted from the files {\em main.F} and
{\em MAIN\_PDIRECTIVES1.h}. The variable {\em nThreads} specifies
how many instances of the routine {\em THE\_MODEL\_MAIN} will
be created. The value of {\em nThreads} is set in the routine
{\em INI\_THREADING\_ENVIRONMENT}. The value is set equal to the
the product of the parameters {\em nTx} and {\em nTy} that
are read from the file {\em eedata}. If the value of {\em nThreads}
is inconsistent with the number of threads requested from the
operating system (for example by using an environment
varialble as described in section \ref{sec:multi_threaded_execution})
then usually an error will be reported by the routine 
{\em CHECK\_THREADS}.\\

\fbox{ 
\begin{minipage}{4.75in}
File: {\em eesupp/src/ini\_threading\_environment.F}\\
File: {\em eesupp/src/check\_threads.F} \\
File: {\em eesupp/src/main.F} \\
File: {\em eesupp/inc/MAIN\_PDIRECTIVES1.h} \\
File: {\em eedata           } \\
Parameter: {\em nThreads} \\
Parameter: {\em nTx} \\
Parameter: {\em nTy} \\
\end{minipage}
}

\begin{figure}
\begin{verbatim}
C--
C--  Parallel directives for MIPS Pro Fortran compiler
C--
C      Parallel compiler directives for SGI with IRIX
C$PAR  PARALLEL DO
C$PAR&  CHUNK=1,MP_SCHEDTYPE=INTERLEAVE,
C$PAR&  SHARE(nThreads),LOCAL(myThid,I)
C
      DO I=1,nThreads
        myThid = I

C--     Invoke nThreads instances of the numerical model
        CALL THE_MODEL_MAIN(myThid)

      ENDDO
\end{verbatim}
\caption{Prior to transferring control to
the procedure {\em THE\_MODEL\_MAIN()} the WRAPPER may use
MP directives to spawn multiple threads.
} \label{fig:mp_directives}
\end{figure}


\item {\bf memsync flags}
As discussed in section \ref{sec:memory_consistency}, when using shared memory,
a low-level system function may be need to force memory consistency.
The routine {\em MEMSYNC()} is used for this purpose. This routine should
not need modifying and the information below is only provided for
completeness. A logical parameter {\em exchNeedsMemSync} set
in the routine {\em INI\_COMMUNICATION\_PATTERNS()} controls whether
the {\em MEMSYNC()} primitive is called. In general this
routine is only used for multi-threaded execution.
The code that goes into the {\em MEMSYNC()}
 routine is specific to the compiler and
processor being used for multi-threaded execution and in general
must be written using a short code snippet of assembly language.
For an Ultra Sparc system the following code snippet is used
\begin{verbatim}
asm("membar #LoadStore|#StoreStore");
\end{verbatim}
for an Alpha based sytem the euivalent code reads
\begin{verbatim}
asm("mb");
\end{verbatim}
while on an x86 system the following code is required
\begin{verbatim}
asm("lock; addl $0,0(%%esp)": : :"memory")
\end{verbatim}

\item {\bf Cache line size}
As discussed in section \ref{sec:cache_effects_and_false_sharing},
milti-threaded codes explicitly avoid penalties associated with excessive 
coherence traffic on an SMP system. To do this the sgared memory data structures
used by the {\em GLOBAL\_SUM}, {\em GLOBAL\_MAX} and {\em BARRIER} routines
are padded. The variables that control the padding are set in the
header file {\em EEPARAMS.h}. These variables are called
{\em cacheLineSize}, {\em lShare1}, {\em lShare4} and
{\em lShare8}. The default values should not normally need changing.
\item {\bf \_BARRIER}
This is a CPP macro that is expanded to a call to a routine
which synchronises all the logical processors running under the
WRAPPER. Using a macro here preserves flexibility to insert
a specialized call in-line into application code. By default this
resolves to calling the procedure {\em BARRIER()}. The default
setting for the \_BARRIER macro is given in the file {\em CPP\_EEMACROS.h}.

\item {\bf \_GSUM}
This is a CPP macro that is expanded to a call to a routine
which sums up a floating point numner
over all the logical processors running under the
WRAPPER. Using a macro here provides extra flexibility to insert
a specialized call in-line into application code. By default this
resolves to calling the procedure {\em GLOBAL\_SOM\_R8()} ( for
84=bit floating point operands)
or {\em GLOBAL\_SOM\_R4()} (for 32-bit floating point operands). The default
setting for the \_GSUM macro is given in the file {\em CPP\_EEMACROS.h}.
The \_GSUM macro is a performance critical operation, especially for
large processor count, small tile size configurations.
The custom communication example discussed in section \ref{sec:jam_example}
shows how the macro is used to invoke a custom global sum routine
for a specific set of hardware.

\item {\bf \_EXCH}
The \_EXCH CPP macro is used to update tile overlap regions.
It is qualified by a suffix indicating whether overlap updates are for
two-dimensional ( \_EXCH\_XY ) or three dimensional ( \_EXCH\_XYZ )
physical fields and whether fields are 32-bit floating point
( \_EXCH\_XY\_R4, \_EXCH\_XYZ\_R4 ) or 64-bit floating point
( \_EXCH\_XY\_R8, \_EXCH\_XYZ\_R8 ). The macro mappings are defined
in the header file {\em CPP\_EEMACROS.h}. As with \_GSUM, the 
\_EXCH operation plays a crucial role in scaling to small tile,
large logical and physical processor count configurations.
The example in section \ref{sec:jam_example} discusses defining an
optimised and specialized form on the \_EXCH operation.

The \_EXCH operation is also central to supporting grids such as
the cube-sphere grid. In this class of grid a rotation may be required
between tiles. Aligning the coordinate requiring rotation with the
tile decomposistion, allows the coordinate transformation to 
be embedded within a custom form of the \_EXCH primitive.

\item {\bf Reverse Mode}
The communication primitives \_EXCH and \_GSUM both employ 
hand-written adjoint forms (or reverse mode) forms. 
These reverse mode forms can be found in the
sourc code directory {\em pkg/autodiff}.
For the global sum primitive the reverse mode form
calls are to {\em GLOBAL\_ADSUM\_R4} and
{\em GLOBAL\_ADSUM\_R8}. The reverse mode form of the
exchamge primitives are found in routines
prefixed {\em ADEXCH}. The exchange routines make calls to
the same low-level communication primitives as the forward mode
operations. However, the routine argument {\em simulationMode}
is set to the value {\em REVERSE\_SIMULATION}. This signifies 
ti the low-level routines that the adjoint forms of the
appropriate communication operation should be performed.
\item {\bf MAX\_NO\_THREADS}
The variable {\em MAX\_NO\_THREADS} is used to indicate the
maximum number of OS threads that a code will use. This
value defaults to thirty-two and is set in the file {\em EEPARAMS.h}.
For single threaded execution it can be reduced to one if required.
The va;lue is largely private to the WRAPPER and application code
will nor normally reference the value, except in the following scenario.

For certain physical parametrization schemes it is necessary to have 
a substantial number of work arrays. Where these arrays are allocated
in heap storage ( for example COMMON blocks ) multi-threaded
execution will require multiple instances of the COMMON block data.
This can be achieved using a Fortran 90 module construct, however,
if this might be unavailable then the work arrays can be extended
with dimensions use the tile dimensioning scheme of {\em nSx}
and {\em nSy} ( as described in section 
\ref{sec:specifying_a_decomposition}). However, if the configuration
being specified involves many more tiles than OS threads then
it can save memory resources to reduce the variable
{\em MAX\_NO\_THREADS} to be equal to the actual number of threads that
will be used and to declare the physical parameterisation
work arrays with a sinble {\em MAX\_NO\_THREADS} extra dimension.
An example of this is given in the verification experiment
{\em aim.5l\_cs}. Here the default setting of 
{\em MAX\_NO\_THREADS} is altered to
\begin{verbatim}
      INTEGER MAX_NO_THREADS
      PARAMETER ( MAX_NO_THREADS =    6 )
\end{verbatim}
and several work arrays for storing intermediate calculations are
created with declarations of the form.
\begin{verbatim}
      common /FORCIN/ sst1(ngp,MAX_NO_THREADS)
\end{verbatim}
This declaration scheme is not used widely, becuase most global data
is used for permanent not temporary storage of state information.
In the case of permanent state information this approach cannot be used
because there has to be enough storage allocated for all tiles.
However, the technique can sometimes be a useful scheme for reducing memory 
requirements in complex physical paramterisations.

\end{enumerate}

\subsubsection{Specializing the Communication Code}

The isolation of performance critical communication primitives and the
sub-division of the simulation domain into tiles is a powerful tool.
Here we show how it can be used to improve application performance and
how it can be used to adapt to new gridding approaches.

\subsubsection{JAM example}
\label{sec:jam_example}
On some platforms a big performance boost can be obtained by
binding the communication routines {\em \_EXCH} and
{\em \_GSUM} to specialized native libraries ) fro example the
shmem library on CRAY T3E systems). The {\em LETS\_MAKE\_JAM} CPP flag 
is used as an illustration of a specialized communication configuration 
that substitutes for standard, portable forms of {\em \_EXCH} and 
{\em \_GSUM}. It affects three source files {\em eeboot.F}, 
{\em CPP\_EEMACROS.h} and {\em cg2d.F}. When the flag is defined
is has the following effects.
\begin{itemize}
\item An extra phase is included at boot time to initialize the custom 
communications library ( see {\em ini\_jam.F}).
\item The {\em \_GSUM} and {\em \_EXCH} macro definitions are replaced
with calls to custom routines ( see {\em gsum\_jam.F} and {\em exch\_jam.F})
\item a highly specialized form of the exchange operator (optimized
for overlap regions of width one) is substitued into the elliptic
solver routine {\em cg2d.F}.
\end{itemize}
Developing specialized code for other libraries follows a similar
pattern.

\subsubsection{Cube sphere communication}
\label{sec:cube_sphere_communication}
Actual {\em \_EXCH} routine code is generated automatically from 
a series of template files, for example {\em exch\_rx.template}.
This is done to allow a large number of variations on the exchange 
process to be maintained. One set of variations supports the
cube sphere grid. Support for a cube sphere gris in MITgcm is based
on having each face of the cube as a separate tile (or tiles).
The exchage routines are then able to absorb much of the
detailed rotation and reorientation required when moving around the
cube grid. The set of {\em \_EXCH} routines that contain the
word cube in their name perform these transformations.
They are invoked when the run-time logical parameter
{\em useCubedSphereExchange} is set true. To facilitate the
transformations on a staggered C-grid, exchange operations are defined 
separately for both vector and scalar quantitities and for
grid-centered and for grid-face and corner quantities.
Three sets of exchange routines are defined. Routines
with names of the form {\em exch\_rx} are used to exchange
cell centered scalar quantities. Routines with names of the form
{\em exch\_uv\_rx} are used to exchange vector quantities located at
the C-grid velocity points. The vector quantities exchanged by the 
{\em exch\_uv\_rx} routines can either be signed (for example velocity 
components) or un-signed (for example grid-cell separations).
Routines with names of the form {\em exch\_z\_rx} are used to exchange 
quantities at the C-grid vorticity point locations.




\section{MITgcm execution under WRAPPER}

Fitting together the WRAPPER elements, package elements and
MITgcm core equation elements of the source code produces calling
sequence shown in section \ref{sec:calling_sequence}

\subsection{Annotated call tree for MITgcm and WRAPPER}
\label{sec:calling_sequence}

WRAPPER layer.

\begin{verbatim}

       MAIN  
       |
       |--EEBOOT               :: WRAPPER initialization
       |  |
       |  |-- EEBOOT_MINMAL    :: Minimal startup. Just enough to
       |  |                       allow basic I/O.
       |  |-- EEINTRO_MSG      :: Write startup greeting.
       |  |
       |  |-- EESET_PARMS      :: Set WRAPPER parameters
       |  |
       |  |-- EEWRITE_EEENV    :: Print WRAPPER parameter settings
       |  |
       |  |-- INI_PROCS        :: Associate processes with grid regions.
       |  |
       |  |-- INI_THREADING_ENVIRONMENT   :: Associate threads with grid regions.
       |       |
       |       |--INI_COMMUNICATION_PATTERNS :: Initialize between tile 
       |                                     :: communication data structures
       |
       |
       |--CHECK_THREADS    :: Validate multiple thread start up.
       |
       |--THE_MODEL_MAIN   :: Numerical code top-level driver routine

\end{verbatim}

Core equations plus packages.

\begin{verbatim}
C
C
C Invocation from WRAPPER level...
C  :
C  :
C  |
C  |-THE_MODEL_MAIN :: Primary driver for the MITgcm algorithm
C    |              :: Called from WRAPPER level numerical
C    |              :: code innvocation routine. On entry
C    |              :: to THE_MODEL_MAIN separate thread and
C    |              :: separate processes will have been established.
C    |              :: Each thread and process will have a unique ID
C    |              :: but as yet it will not be associated with a
C    |              :: specific region in decomposed discrete space.
C    |
C    |-INITIALISE_FIXED :: Set fixed model arrays such as topography, 
C    | |                :: grid, solver matrices etc..
C    | |              
C    | |-INI_PARMS :: Routine to set kernel model parameters.
C    | |           :: By default kernel parameters are read from file 
C    | |           :: "data" in directory in which code executes.
C    | |
C    | |-MON_INIT :: Initialises monitor pacakge ( see pkg/monitor )
C    | |
C    | |-INI_GRID :: Control grid array (vert. and hori.) initialisation.
C    | | |        :: Grid arrays are held and described in GRID.h.
C    | | |
C    | | |-INI_VERTICAL_GRID        :: Initialise vertical grid arrays.
C    | | |
C    | | |-INI_CARTESIAN_GRID       :: Cartesian horiz. grid initialisation
C    | | |                          :: (calculate grid from kernel parameters).
C    | | |
C    | | |-INI_SPHERICAL_POLAR_GRID :: Spherical polar horiz. grid 
C    | | |                          :: initialisation (calculate grid from 
C    | | |                          :: kernel parameters).
C    | | |
C    | | |-INI_CURVILINEAR_GRID     :: General orthogonal, structured horiz.
C    | |                            :: grid initialisations. ( input from raw
C    | |                            :: grid files, LONC.bin, DXF.bin etc... )
C    | |
C    | |-INI_DEPTHS    :: Read (from "bathyFile") or set bathymetry/orgography.
C    | |
C    | |-INI_MASKS_ETC :: Derive horizontal and vertical cell fractions and
C    | |               :: land masking for solid-fluid boundaries.
C    | |
C    | |-INI_LINEAR_PHSURF :: Set ref. surface Bo_surf
C    | |
C    | |-INI_CORI          :: Set coriolis term. zero, f-plane, beta-plane,
C    | |                   :: sphere optins are coded.
C    | |
C    | |-PACAKGES_BOOT      :: Start up the optional package environment.
C    | |                    :: Runtime selection of active packages.
C    | |
C    | |-PACKAGES_READPARMS :: Call active package internal parameter load.
C    | | |
C    | | |-GMREDI_READPARMS    :: GM Package. see pkg/gmredi
C    | | |-KPP_READPARMS       :: KPP Package. see pkg/kpp
C    | | |-SHAP_FILT_READPARMS :: Shapiro filter package. see pkg/shap_filt
C    | | |-OBCS_READPARMS      :: Open bndy package. see pkg/obcs
C    | | |-AIM_READPARMS       :: Intermediate Atmos. pacakage. see pkg/aim
C    | | |-COST_READPARMS      :: Cost function package. see pkg/cost
C    | | |-CTRL_INIT           :: Control vector support package. see pkg/ctrl
C    | | |-OPTIM_READPARMS     :: Optimisation support package. see pkg/ctrl 
C    | | |-GRDCHK_READPARMS    :: Gradient check package. see pkg/grdchk
C    | | |-ECCO_READPARMS      :: ECCO Support Package. see pkg/ecco
C    | |
C    | |-PACKAGES_CHECK
C    | | |
C    | | |-KPP_CHECK           :: KPP Package. pkg/kpp
C    | | |-OBCS_CHECK          :: Open bndy Pacakge. pkg/obcs
C    | | |-GMREDI_CHECK        :: GM Package. pkg/gmredi
C    | |
C    | |-PACKAGES_INIT_FIXED
C    | | |-OBCS_INIT_FIXED     :: Open bndy Package. see pkg/obcs
C    | | |-FLT_INIT            :: Floats Package. see pkg/flt
C    | |
C    | |-ZONAL_FILT_INIT       :: FFT filter Package. see pkg/zonal_filt
C    | |
C    | |-INI_CG2D              :: 2d con. grad solver initialisation.
C    | |
C    | |-INI_CG3D              :: 3d con. grad solver initialisation.
C    | |
C    | |-CONFIG_SUMMARY        :: Provide synopsis of kernel setup.
C    |                         :: Includes annotated table of kernel 
C    |                         :: parameter settings.
C    |
C    |-CTRL_UNPACK :: Control vector support package. see pkg/ctrl
C    |
C    |-ADTHE_MAIN_LOOP :: Derivative evaluating form of main time stepping loop
C    !                 :: Auotmatically gerenrated by TAMC/TAF.
C    |
C    |-CTRL_PACK   :: Control vector support package. see pkg/ctrl
C    |
C    |-GRDCHK_MAIN :: Gradient check package. see pkg/grdchk
C    |
C    |-THE_MAIN_LOOP :: Main timestepping loop routine.
C    | |
C    | |-INITIALISE_VARIA :: Set the initial conditions for time evolving 
C    | | |                :: variables
C    | | |
C    | | |-INI_LINEAR_PHISURF :: Set ref. surface Bo_surf
C    | | |
C    | | |-INI_CORI     :: Set coriolis term. zero, f-plane, beta-plane,
C    | | |              :: sphere optins are coded.
C    | | |
C    | | |-INI_CG2D     :: 2d con. grad solver initialisation.
C    | | |-INI_CG3D     :: 3d con. grad solver initialisation.
C    | | |-INI_MIXING   :: Initialise diapycnal diffusivity.
C    | | |-INI_DYNVARS  :: Initialise to zero all DYNVARS.h arrays (dynamical
C    | | |              :: fields).
C    | | |
C    | | |-INI_FIELDS   :: Control initialising model fields to non-zero
C    | | | |-INI_VEL    :: Initialize 3D flow field.
C    | | | |-INI_THETA  :: Set model initial temperature field.
C    | | | |-INI_SALT   :: Set model initial salinity field.
C    | | | |-INI_PSURF  :: Set model initial free-surface height/pressure.
C    | | |
C    | | |-INI_TR1      :: Set initial tracer 1 distribution.
C    | | |
C    | | |-THE_CORRECTION_STEP :: Step forward to next time step.
C    | | | |                   :: Here applied to move restart conditions
C    | | | |                   :: (saved in mid timestep) to correct level in 
C    | | | |                   :: time (only used for pre-c35).
C    | | | |
C    | | | |-CALC_GRAD_PHI_SURF :: Return DDx and DDy of surface pressure
C    | | | |-CORRECTION_STEP    :: Pressure correction to momentum
C    | | | |-CYCLE_TRACER       :: Move tracers forward in time.
C    | | | |-OBCS_APPLY         :: Open bndy package. see pkg/obcs
C    | | | |-SHAP_FILT_APPLY    :: Shapiro filter package. see pkg/shap_filt
C    | | | |-ZONAL_FILT_APPLY   :: FFT filter package. see pkg/zonal_filt
C    | | | |-CONVECTIVE_ADJUSTMENT :: Control static instability mixing.
C    | | | | |-FIND_RHO  :: Find adjacent densities.
C    | | | | |-CONVECT   :: Mix static instability.
C    | | | | |-TIMEAVE_CUMULATE :: Update convection statistics.
C    | | | | 
C    | | | |-CALC_EXACT_ETA        :: Change SSH to flow divergence.     
C    | | | 
C    | | |-CONVECTIVE_ADJUSTMENT_INI :: Control static instability mixing
C    | | | |                         :: Extra time history interactions.
C    | | | |                       
C    | | | |-FIND_RHO  :: Find adjacent densities.
C    | | | |-CONVECT   :: Mix static instability.
C    | | | |-TIMEAVE_CUMULATE :: Update convection statistics.
C    | | |
C    | | |-PACKAGES_INIT_VARIABLES :: Does initialisation of time evolving 
C    | | | |                       :: package data.
C    | | | |
C    | | | |-GMREDI_INIT          :: GM package. ( see pkg/gmredi )
C    | | | |-KPP_INIT             :: KPP package. ( see pkg/kpp )
C    | | | |-KPP_OPEN_DIAGS    
C    | | | |-OBCS_INIT_VARIABLES  :: Open bndy. package. ( see pkg/obcs )
C    | | | |-AIM_INIT             :: Interm. atmos package. ( see pkg/aim )
C    | | | |-CTRL_MAP_INI         :: Control vector package.( see pkg/ctrl )
C    | | | |-COST_INIT            :: Cost function package. ( see pkg/cost )
C    | | | |-ECCO_INIT            :: ECCO support package. ( see pkg/ecco )
C    | | | |-INI_FORCING          :: Set model initial forcing fields.
C    | | |   |                    :: Either set in-line or from file as shown.
C    | | |   |-READ_FLD_XY_RS(zonalWindFile)
C    | | |   |-READ_FLD_XY_RS(meridWindFile)
C    | | |   |-READ_FLD_XY_RS(surfQFile)
C    | | |   |-READ_FLD_XY_RS(EmPmRfile)
C    | | |   |-READ_FLD_XY_RS(thetaClimFile)
C    | | |   |-READ_FLD_XY_RS(saltClimFile)
C    | | |   |-READ_FLD_XY_RS(surfQswFile)
C    | | |
C    | | |-CALC_SURF_DR   :: Calculate the new surface level thickness.
C    | | |-UPDATE_SURF_DR :: Update the surface-level thickness fraction.
C    | | |-UPDATE_CG2D    :: Update 2d conjugate grad. for Free-Surf.
C    | | |-STATE_SUMMARY    :: Summarize model prognostic variables.
C    | | |-TIMEAVE_STATVARS :: Time averaging package ( see pkg/timeave ).
C    | |
C    | |-WRITE_STATE      :: Controlling routine for IO to dump model state.
C    | | |-WRITE_REC_XYZ_RL :: Single file I/O
C    | | |-WRITE_FLD_XYZ_RL :: Multi-file I/O
C    | | 
C    | |-MONITOR          :: Monitor state ( see pkg/monitor )
C    | |-CTRL_MAP_FORCING :: Control vector support package. ( see pkg/ctrl )
C====|>| 
C====|>| ****************************
C====|>| BEGIN MAIN TIMESTEPPING LOOP
C====|>| ****************************
C====|>| 
C/\  | |-FORWARD_STEP     :: Step forward a time-step ( AT LAST !!! )
C/\  | | |
C/\  | | |-DUMMY_IN_STEPPING :: autodiff package ( pkg/autoduff ).
C/\  | | |-CALC_EXACT_ETA :: Change SSH to flow divergence.
C/\  | | |-CALC_SURF_DR   :: Calculate the new surface level thickness.
C/\  | | |-EXF_GETFORCING :: External forcing package. ( pkg/exf )
C/\  | | |-EXTERNAL_FIELDS_LOAD :: Control loading time dep. external data.
C/\  | | | |                    :: Simple interpolcation between end-points 
C/\  | | | |                    :: for forcing datasets.
C/\  | | | |                  
C/\  | | | |-EXCH :: Sync forcing. in overlap regions.
C/\  | | |
C/\  | | |-THERMODYNAMICS :: theta, salt + tracer equations driver.
C/\  | | | |
C/\  | | | |-INTEGRATE_FOR_W :: Integrate for vertical velocity.
C/\  | | | |-OBCS_APPLY_W    :: Open bndy. package ( see pkg/obcs ).
C/\  | | | |-FIND_RHO        :: Calculates [rho(S,T,z)-Rhonil] of a slice
C/\  | | | |-GRAD_SIGMA      :: Calculate isoneutral gradients
C/\  | | | |-CALC_IVDC       :: Set Implicit Vertical Diffusivity for Convection
C/\  | | | |
C/\  | | | |-OBCS_CALC            :: Open bndy. package ( see pkg/obcs ).
C/\  | | | |-EXTERNAL_FORCING_SURF:: Accumulates appropriately dimensioned 
C/\  | | | |                      :: forcing terms.
C/\  | | | |
C/\  | | | |-GMREDI_CALC_TENSOR   :: GM package ( see pkg/gmredi ).
C/\  | | | |-GMREDI_CALC_TENSOR_DUMMY :: GM package ( see pkg/gmredi ). 
C/\  | | | |-KPP_CALC             :: KPP package ( see pkg/kpp ).
C/\  | | | |-KPP_CALC_DUMMY       :: KPP package ( see pkg/kpp ).
C/\  | | | |-AIM_DO_ATMOS_PHYSICS :: Intermed. atmos package ( see pkg/aim ).
C/\  | | | |-GAD_ADVECTION        :: Generalised advection driver (multi-dim
C/\  | | | |                         advection case) (see pkg/gad).
C/\  | | | |-CALC_COMMON_FACTORS  :: Calculate common data (such as volume flux)
C/\  | | | |-CALC_DIFFUSIVITY     :: Calculate net vertical diffusivity
C/\  | | | | |
C/\  | | | | |-GMREDI_CALC_DIFF   :: GM package ( see pkg/gmredi ).
C/\  | | | | |-KPP_CALC_DIFF      :: KPP package ( see pkg/kpp ).
C/\  | | | |
C/\  | | | |-CALC_GT              :: Calculate the temperature tendency terms
C/\  | | | | |
C/\  | | | | |-GAD_CALC_RHS       :: Generalised advection package 
C/\  | | | | |                    :: ( see pkg/gad )
C/\  | | | | |-EXTERNAL_FORCING_T :: Problem specific forcing for temperature.
C/\  | | | | |-ADAMS_BASHFORTH2   :: Extrapolate tendencies forward in time.
C/\  | | | | |-FREESURF_RESCALE_G :: Re-scale Gt for free-surface height.
C/\  | | | |
C/\  | | | |-TIMESTEP_TRACER      :: Step tracer field forward in time
C/\  | | | |
C/\  | | | |-CALC_GS              :: Calculate the salinity tendency terms
C/\  | | | | |
C/\  | | | | |-GAD_CALC_RHS       :: Generalised advection package 
C/\  | | | | |                    :: ( see pkg/gad )
C/\  | | | | |-EXTERNAL_FORCING_S :: Problem specific forcing for salt.
C/\  | | | | |-ADAMS_BASHFORTH2   :: Extrapolate tendencies forward in time.
C/\  | | | | |-FREESURF_RESCALE_G :: Re-scale Gs for free-surface height.
C/\  | | | |
C/\  | | | |-TIMESTEP_TRACER      :: Step tracer field forward in time
C/\  | | | |
C/\  | | | |-CALC_GTR1            :: Calculate other tracer(s) tendency terms
C/\  | | | | |
C/\  | | | | |-GAD_CALC_RHS       :: Generalised advection package 
C/\  | | | | |                    :: ( see pkg/gad )
C/\  | | | | |-EXTERNAL_FORCING_TR:: Problem specific forcing for tracer.
C/\  | | | | |-ADAMS_BASHFORTH2   :: Extrapolate tendencies forward in time.
C/\  | | | | |-FREESURF_RESCALE_G :: Re-scale Gs for free-surface height.
C/\  | | | |
C/\  | | | |-TIMESTEP_TRACER      :: Step tracer field forward in time
C/\  | | | |-OBCS_APPLY_TS        :: Open bndy. package (see pkg/obcs ).
C/\  | | | |-FREEZE               :: Limit range of temperature.
C/\  | | | |
C/\  | | | |-IMPLDIFF             :: Solve vertical implicit diffusion equation.
C/\  | | | |-OBCS_APPLY_TS        :: Open bndy. package (see pkg/obcs ). 
C/\  | | | |
C/\  | | | |-AIM_AIM2DYN_EXCHANGES :: Inetermed. atmos (see pkg/aim).
C/\  | | | |-EXCH                 :: Update overlaps
C/\  | | |
C/\  | | |-DYNAMICS       :: Momentum equations driver.
C/\  | | | |
C/\  | | | |-CALC_GRAD_PHI_SURF :: Calculate the gradient of the surface 
C/\  | | | |                       Potential anomaly.
C/\  | | | |-CALC_VISCOSITY     :: Calculate net vertical viscosity
C/\  | | | | |-KPP_CALC_VISC    :: KPP package ( see pkg/kpp ).
C/\  | | | |                                                      
C/\  | | | |-CALC_PHI_HYD       :: Integrate the hydrostatic relation.
C/\  | | | |-MOM_FLUXFORM       :: Flux form mom eqn. package ( see
C/\  | | | |                       pkg/mom_fluxform ).
C/\  | | | |-MOM_VECINV         :: Vector invariant form mom eqn. package ( see
C/\  | | | |                       pkg/mom_vecinv   ).
C/\  | | | |-TIMESTEP           :: Step momentum fields forward in time
C/\  | | | |-OBCS_APPLY_UV      :: Open bndy. package (see pkg/obcs ).
C/\  | | | |
C/\  | | | |-IMPLDIFF           :: Solve vertical implicit diffusion equation.
C/\  | | | |-OBCS_APPLY_UV      :: Open bndy. package (see pkg/obcs ).
C/\  | | | |
C/\  | | | |-TIMEAVE_CUMUL_1T   :: Time averaging package ( see pkg/timeave ).
C/\  | | | |-TIMEAVE_CUMUATE    :: Time averaging package ( see pkg/timeave ).
C/\  | | | |-DEBUG_STATS_RL     :: Quick debug package ( see pkg/debug ).
C/\  | | |
C/\  | | |-CALC_GW        :: vert. momentum tendency terms ( NH, QH only ).
C/\  | | |
C/\  | | |-UPDATE_SURF_DR :: Update the surface-level thickness fraction.
C/\  | | |
C/\  | | |-UPDATE_CG2D    :: Update 2d conjugate grad. for Free-Surf.
C/\  | | |
C/\  | | |-SOLVE_FOR_PRESSURE           :: Find surface pressure.
C/\  | | | |-CALC_DIV_GHAT     :: Form the RHS of the surface pressure eqn.
C/\  | | | |-CG2D              :: Two-dim pre-con. conjugate-gradient.
C/\  | | | |-CG3D              :: Three-dim pre-con. conjugate-gradient solver.
C/\  | | |
C/\  | | |-THE_CORRECTION_STEP          :: Step forward to next time step.
C/\  | | | |
C/\  | | | |-CALC_GRAD_PHI_SURF :: Return DDx and DDy of surface pressure
C/\  | | | |-CORRECTION_STEP    :: Pressure correction to momentum
C/\  | | | |-CYCLE_TRACER       :: Move tracers forward in time.
C/\  | | | |-OBCS_APPLY         :: Open bndy package. see pkg/obcs
C/\  | | | |-SHAP_FILT_APPLY    :: Shapiro filter package. see pkg/shap_filt
C/\  | | | |-ZONAL_FILT_APPLY   :: FFT filter package. see pkg/zonal_filt
C/\  | | | |-CONVECTIVE_ADJUSTMENT :: Control static instability mixing.
C/\  | | | | |-FIND_RHO  :: Find adjacent densities.
C/\  | | | | |-CONVECT   :: Mix static instability.
C/\  | | | | |-TIMEAVE_CUMULATE :: Update convection statistics.
C/\  | | | | 
C/\  | | | |-CALC_EXACT_ETA        :: Change SSH to flow divergence.     
C/\  | | |
C/\  | | |-DO_FIELDS_BLOCKING_EXCHANGES :: Sync up overlap regions.
C/\  | | | |-EXCH                                                   
C/\  | | |
C/\  | | |-FLT_MAIN         :: Float package ( pkg/flt ).
C/\  | | |
C/\  | | |-MONITOR          :: Monitor package ( pkg/monitor ).
C/\  | | |
C/\  | | |-DO_THE_MODEL_IO  :: Standard diagnostic I/O.
C/\  | | | |-WRITE_STATE    :: Core state I/O
C/\  | | | |-TIMEAVE_STATV_WRITE :: Time averages. see pkg/timeave
C/\  | | | |-AIM_WRITE_DIAGS     :: Intermed. atmos diags. see pkg/aim
C/\  | | | |-GMREDI_DIAGS        :: GM diags. see pkg/gmredi
C/\  | | | |-KPP_DO_DIAGS        :: KPP diags. see pkg/kpp
C/\  | | |
C/\  | | |-WRITE_CHECKPOINT :: Do I/O for restart files.
C/\  | |
C/\  | |-COST_TILE        :: Cost function package. ( see pkg/cost )
C<===|=|
C<===|=| **************************
C<===|=| END MAIN TIMESTEPPING LOOP
C<===|=| **************************
C<===|=|
C    | |-COST_FINAL       :: Cost function package. ( see pkg/cost )
C    |
C    |-WRITE_CHECKPOINT :: Final state storage, for restart.
C    |
C    |-TIMER_PRINTALL :: Computational timing summary
C    |
C    |-COMM_STATS     :: Summarise inter-proc and inter-thread communication
C                     :: events.
C 
\end{verbatim}

\subsection{Measuring and Characterizing Performance}

TO BE DONE (CNH)

\subsection{Estimating Resource Requirements}

TO BE DONE (CNH) 

\subsubsection{Atlantic 1/6 degree example}
\subsubsection{Dry Run testing}
\subsubsection{Adjoint Resource Requirements}
\subsubsection{State Estimation Environment Resources}

