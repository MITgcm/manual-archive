% $Header: /u/gcmpack/manual/s_overview/Attic/illustration.tex,v 1.1.1.1 2001/08/08 16:16:16 adcroft Exp $
% $Name:  $

\section{Illustrations of the model in action}

The MITgcm has been designed and used to model a vast range of phenomena,
from convection on the scale of meters in the ocean to the global pattern of
atmospheric winds - see fig.2\ref{fig:all-scales}. To give a flavor of the
kinds of problems the model has been used to study, we briefly describe some
of them here. A more detailed description of the underlying formulation,
numerical algorithm and implementation that lie behind these calculations is
given later. Indeed it is easy to reproduce the results shown here: simply
download the model (the minimum you need is a PC running linux, together
with a FORTRAN\ 77 compiler) and follow the examples.

\subsection{Global atmosphere: `Held-Suarez' benchmark}

Fig.E1a.\ref{fig:Held-Suarez} is an instaneous plot of the 500$mb$ height
field obtained using a 5-level version of the atmospheric pressure isomorph
run at 300$km$ resolution. We see fully developed baroclinic eddies along
the northern hemisphere storm track. There are no mountains (but you can
easily put them in) or land-sea contrast. The model is driven by relaxation
to a radiative-convective equilibrium profile, following the description set
out in Held and Suarez; 1994 designed to test atmospheric hydrodynamical
cores - there are no mountains or land-sea contrast. As decribed in Adcroft
(2001), `cubed sphere' is used to descretize the globe permitting a uniform
gridding and obviated the need to fourier filter.

Fig.E1b shows the 5-year mean, zonally averaged potential temperature, zonal
wind and meridional overturning streamfunction from the 5-level model.

A regular spherical lat-lon grid can also be used.

Results from this integration, together with a 20-level calculation, can be
found here - link through to channel.

This calculation takes 1 day of computation per year of intergation on a
pentium IV PC.

\subsection{Ocean gyres}

\subsection{Global ocean circulation}

Fig.E2a shows the pattern of ocean currents at the surface of a 4$^{\circ }$
global ocean model run with 15 vertical levels. The model is driven using
monthly-mean winds with mixed boundary conditions on temperature and
salinity at the surface. Fig.E2b shows the overturning (thermohaline)
circulation. Lopped cells are used to represent topography on a regular $%
lat-lon$ grid extending from 70$^{\circ }N$ to 70$^{\circ }S$.

\subsection{Flow over topography}

\subsection{Ocean convection}

Fig.E3 shows convection over a slope using the non-hydrostatic ocean
isomorph and lopped cells to respresent topography. .....The grid resolution
is

\subsection{Boundary forced internal waves}

\subsection{Carbon outgassing sensitivity}

Fig.E4 shows....
