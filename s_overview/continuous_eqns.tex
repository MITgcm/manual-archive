% $Header: /u/gcmpack/manual/s_overview/Attic/continuous_eqns.tex,v 1.1.1.1 2001/08/08 16:16:18 adcroft Exp $
% $Name:  $

\section{Continuous equations in `r' coordinates}

To render atmosphere and ocean models from one dynamical core we exploit
`isomorphisms' between equation sets that govern the evolution of the
respective fluids - see fig.4%
\marginpar{
Fig.4. Isomorphisms}. One system of hydrodynamical equations is written down
and encoded. The model variables have different interpretations depending on
whether the atmosphere or ocean is being studied. Thus, for example, the
vertical coordinate `$r$' is interpreted as pressure, $p$, if we are
modeling the atmosphere and height, $z$, if we are modeling the ocean. A
complete list of the isomorphisms is given in table 1.%
\marginpar{
Table 1. Isomorphisms}

The state of the fluid at any time is characterized by the distribution of
velocity $\vec{\mathbf{v}}$, active tracers $\theta $ and $S$, a
`geopotential' $\phi $ and density $\rho =\rho (\theta ,S,p)$ which may
depend on $\theta $, $S$, and $p$. The equations that govern the evolution
of these fields, obtained by applying the laws of classical mechanics and
thermodynamics to a Boussinesq, Navier-Stokes fluid are, written in terms of
a generic vertical coordinate, $r$, see fig.5%
\marginpar{
Fig.5 The vertical coordinate of model}:

\[
\frac{D\vec{\mathbf{v}_{h}}}{Dt}+\left( 2\vec{\Omega}\times \vec{\mathbf{v}}%
\right) _{h}+\mathbf{\nabla }_{h}\phi =\left( \mathcal{F}_{\vec{\mathbf{v}}}%
\mathcal{+D}_{\vec{\mathbf{v}}}\right) _{h}\text{horizontal mtm} 
\]

\[
\frac{D\dot{r}}{Dt}+\widehat{k}\cdot \left( 2\vec{\Omega}\times \vec{\mathbf{%
v}}\right) +\frac{\partial \phi }{\partial r}+b=\left( \mathcal{F}_{\vec{%
\mathbf{v}}}\mathcal{+D}_{\vec{\mathbf{v}}}\right) _{r}\text{vertical mtm} 
\]

\begin{equation}
\mathbf{\nabla }_{h}\cdot \vec{\mathbf{v}}_{h}+\frac{\partial \dot{r}}{%
\partial r}=0\text{ continuity}  \label{incompressible}
\end{equation}

\[
b=b(\theta ,S,r)\text{ equation of state} 
\]

\[
\frac{D\theta }{Dt}=\mathcal{F}_{\theta }\text{ }\mathcal{+D}_{\theta }\text{
potential temperature} 
\]

\[
\frac{DS}{Dt}=\mathcal{F}_{S}\text{ }\mathcal{+D}_{S}\text{ humidity/salinity%
} 
\]

Here:

\[
r\text{ is the vertical coordinate} 
\]

\[
\frac{D}{Dt}=\frac{\partial }{\partial t}+\vec{\mathbf{v}}\cdot \nabla \text{
is the total derivative} 
\]

\[
\mathbf{\nabla }=\mathbf{\nabla }_{h}+\widehat{k}\frac{\partial }{\partial r}%
\text{ is the `grad' operator} 
\]
with $\mathbf{\nabla }_{h}$ operating in the horizontal and $\widehat{k}%
\frac{\partial }{\partial r}$ operating in the vertical, where $\widehat{k}$
is a unit vector in the vertical

\[
t\text{ is time} 
\]

\[
\vec{\mathbf{v}}=(u,v,\dot{r})=(\vec{\mathbf{v}}_{h},\dot{r})\text{ is the
velocity} 
\]

\[
\phi \text{ is the `pressure'/`geopotential'} 
\]

\[
\vec{\Omega}\text{ is the Earth's rotation} 
\]

\[
b\text{ is the `buoyancy'} 
\]

\[
\theta \text{ is potential temperature} 
\]

\[
S\text{ is specific humidity in the atmosphere; salinity in the ocean} 
\]

\[
\mathcal{F}_{\vec{\mathbf{v}}}\text{ and }\mathcal{D}_{\vec{\mathbf{v}}}%
\text{ are forcing and dissipation of }\vec{\mathbf{v}} 
\]

\[
\mathcal{F}_{\theta }\mathcal{\ }\text{and }\mathcal{D}_{\theta }\text{ are
forcing and dissipation of }\theta 
\]

\[
\mathcal{F}_{S}\mathcal{\ }\text{and }\mathcal{D}_{S}\text{ are forcing and
dissipation of }S 
\]

The $\mathcal{F}^{\prime }s$ and $\mathcal{D}^{\prime }s$ are provided by
extensive `physics' packages for atmosphere and ocean described in section
?.?.

\subsection{Kinematic Boundary conditions}

\subsubsection{vertical}

at fixed and moving $r$ surfaces we set (see fig.4):

\begin{eqnarray*}
\dot{r} &=&0\text{ at }r=R_{fixed}(x,y):\text{(ocean bottom, top of the
atmosphere)} \\
\dot{r} &=&\frac{Dr}{Dt}\text{ at }r=R_{moving}\text{ (ocean surface, bottom
of the atmosphere)}
\end{eqnarray*}
Here

\[
R_{moving}=R_{o}+\eta 
\]
where $R_{o}(x,y)$ is the `$r-$value' (height or pressure, depending on
whether we are in the atmosphere or ocean) of the `moving surface' in the
resting fluid and $\eta $ is the departure from $R_{o}(x,y)$ in the presence
of motion.

\subsubsection{horizontal}

\[
\vec{\mathbf{v}}\cdot \vec{\mathbf{n}}=0 
\]
where $\vec{\mathbf{n}}$ is the normal to a solid boundary.

\subsection{Atmosphere}

In the atmosphere, see fig. we interpret: 
\begin{eqnarray}
r &=&p\text{ is the pressure} \\
\dot{r} &=&\frac{Dp}{Dt}=\omega \text{ is the vertical velocity in }p\text{
coordinates} \\
\phi  &=&g\,z\text{ is the geopotential height} \\
b &=&\frac{\partial \Pi }{\partial p}\theta \text{ is the buoyancy} \\
\theta  &=&T(\frac{p_{c}}{p})^{\kappa }\text{ is potential temperature} \\
S &=&q\text{, the specific humidity}
\end{eqnarray}
where

\[
T\text{is absolute temperature}
\]
\[
p\text{ is the pressure}
\]
\begin{eqnarray*}
&&z\text{ is the height of the pressure surface} \\
&&g\text{ is the acceleration due to gravity}
\end{eqnarray*}

In the above the ideal gas law, $p=\rho RT$, has been expressed in terms of
the Exner function $\Pi (p)$ given by (see Appendix Atmosphere) 
\[
\Pi (p)=c_{p}(\frac{p}{p_{c}})^{\kappa } 
\]
where $p_{c}$ is a reference pressure and $\kappa =R/c_{p}$ with $R$ the gas
constant and $c_{p}$ the specific heat of air at constant pressure.

At the top of the atmosphere (which is `fixed' in our $r$ coordinate):

\[
R_{fixed}=p_{top}=0 
\]
In a resting atmosphere the elevation of the mountains at the bottom is
given by 
\[
R_{moving}=R_{o}(x,y)=p_{o}(x,y) 
\]
i.e. the (hydrostatic) pressure at the top of the mountains in a resting
atmosphere.

The boundary conditions at top and bottom are given by:

\begin{eqnarray*}
&&\omega =0~\text{at }r=R_{fixed}\text{ (top of the atmosphere)} \\
\omega &=&\frac{Dp_{s}}{Dt}\text{; at }r=R_{moving}\text{ (bottom of the
atmosphere)}
\end{eqnarray*}

Then the (hydrostatic form of) eq(\ref{incompressible}) yields a consistent
set of atmospheric equations which, for convenience, are written out in $p$
coordinates in Appendix Atmosphere - see eqs(\ref{eq-p-hmom}) to (\ref
{eq-p-heat}).

\subsection{Ocean}

In the ocean we interpret: 
\begin{eqnarray}
r &=&z\text{ is the height} \\
\dot{r} &=&\frac{Dz}{Dt}=w\text{ is the vertical velocity} \\
\phi &=&\frac{p}{\rho _{c}}\text{ is the pressure} \\
b(\theta ,S,r) &=&\frac{g}{\rho _{c}}\left( \rho (\theta ,S,r)-\rho
_{c}\right) \text{ is the buoyancy}
\end{eqnarray}
where $\rho _{c}$ is a fixed reference density of water and $g$ is the
acceleration due to gravity.\noindent

In the above

At the bottom of the ocean: $R_{fixed}(x,y)=-H(x,y)$.

The surface of the ocean is given by: $R_{moving}=\eta $

The position of the resting free surface of the ocean is given by $%
R_{o}=Z_{o}=0$.

Boundary conditions are:

\begin{eqnarray*}
w &=&0~\text{at }r=R_{fixed}\text{ (ocean bottom)} \\
w &=&\frac{D\eta }{Dt}\text{ at }r=R_{moving}=\eta \text{ (ocean surface)}
\end{eqnarray*}
where $\eta $ is the elevation of the free surface.

Then eq(\ref{incompressible}) yields a consistent set of oceanic equations
which, for convenience, are written out in $z$ coordinates in Appendix Ocean.

\subsection{Hydrostatic, Quasi-hydrostatic, Quasi-nonhydrostatic and
Non-hydrostatic forms}

Let us separate $\phi $ in to surface, hydrostatic and non-hydrostatic terms:

\begin{equation}
\phi (x,y,r)=\phi _{s}(x,y)+\phi _{hyd}(x,y,r)+\phi _{nh}(x,y,r)
\label{pressuresplit}
\end{equation}
and write eq(\ref{incompressible}a) in the form:

\begin{equation}
\frac{\partial \vec{\mathbf{v}_{h}}}{\partial t}+\mathbf{\nabla }_{h}\phi
_{s}+\mathbf{\nabla }_{h}\phi _{hyd}+\epsilon _{nh}\mathbf{\nabla }_{h}\phi
_{nh}=\vec{\mathbf{G}}_{\vec{v}_{h}}  \label{hor-mtm}
\end{equation}

\begin{equation}
\frac{\partial \phi _{hyd}}{\partial r}=-b  \label{hydro}
\end{equation}

\begin{equation}
\epsilon _{nh}\frac{\partial \dot{r}}{\partial t}+\frac{\partial \phi _{nh}}{%
\partial r}=G_{\dot{r}}  \label{vertmtm}
\end{equation}
Here $\epsilon _{nh}$ is a non-hydrostatic parameter.

The $\left( \vec{\mathbf{G}}_{\vec{v}},G_{\dot{r}}\right) $ in eq(\ref
{hor-mtm}) and (\ref{vertmtm}) represent advective, metric and Coriolis
terms in the momentum equations. In spherical coordinates they take the form%
\footnote{%
In the hydrostatic primitive equations (\textbf{HPE}) all underlined terms
in (\ref{Gu}), (\ref{Gv}) and (\ref{Gw}) are omitted; the singly-underlined
terms are included in the quasi-hydrostatic model (\textbf{QH}). The fully
non-hydrostatic model (\textbf{NH}) includes all terms.}:

\begin{equation}
\left. 
\begin{tabular}{l}
$G_{u}=-\vec{\mathbf{v}}.\nabla u$ \\ 
$-\left\{ \underline{\frac{u\dot{r}}{{r}}}-\frac{uv\tan lat}{{r}}%
\right\} $ \\ 
$-\left\{ -2\Omega v\sin lat+\underline{\underline{2\Omega \dot{r}\cos lat}}%
\right\} $ \\ 
$+\mathcal{F}_{u}\mathcal{+D}_{u}$%
\end{tabular}
\right\} \left\{ 
\begin{tabular}{l}
\textit{advection} \\ 
\textit{metric} \\ 
\textit{Coriolis} \\ 
\textit{\ Forcing/Dissipation}
\end{tabular}
\right. \qquad  \label{Gu}
\end{equation}

\begin{equation}
\left. 
\begin{tabular}{l}
$G_{v}=-\vec{\mathbf{v}}.\nabla v$ \\ 
$-\left\{ \underline{\frac{v\dot{r}}{{r}}}-\frac{u^{2}\tan lat}{{r}%
}\right\} $ \\ 
$-\left\{ -2\Omega u\sin lat\right\} $ \\ 
$+\mathcal{F}_{v}\mathcal{+D}_{v}$%
\end{tabular}
\right\} \left\{ 
\begin{tabular}{l}
\textit{advection} \\ 
\textit{metric} \\ 
\textit{Coriolis} \\ 
\textit{\ Forcing/Dissipation}
\end{tabular}
\right. \qquad  \label{Gv}
\end{equation}
\qquad \qquad \qquad \qquad \qquad

\begin{equation}
\left. 
\begin{tabular}{l}
$G_{\dot{r}}=-\vec{\mathbf{v}}.\nabla \dot{r}$ \\ 
$+\left\{ \frac{u^{_{^{2}}}+v^{2}}{{{r}}}%
\right\} $ \\ 
${+2\Omega u\cos lat}$ \\ 
$\mathcal{F}_{\dot{r}}\mathcal{+D}_{\dot{r}}$%
\end{tabular}
\right\} \left\{ 
\begin{tabular}{l}
\textit{advection} \\ 
\textit{metric} \\ 
\textit{Coriolis} \\ 
\textit{\ Forcing/Dissipation}
\end{tabular}
\right.  \label{Gw}
\end{equation}
\qquad \qquad \qquad \qquad \qquad

In the above `${r}$' is the distance from the center of the earth and `$%
lat$' is latitude.

Grad and div operators in spherical coordinates are defined in appendix
OPERATORS.%
\marginpar{
Fig.6 Spherical polar coordinate system.}

\subsubsection{Shallow atmosphere approximation}

............................

\subsubsection{Hydrostatic and quasi-hydrostatic forms}

These are discussed at length in Marshall et al (1997a).

In the `hydrostatic primitive equations' (\textbf{HPE)} all the underlined
terms in Eqs. (\ref{Gu} $\rightarrow $\ \ref{Gw}) are neglected and `${r%
}$' is replaced by `$a$', the mean radius of the earth. Once the pressure is
found at one level - e.g. by inverting a 2-d Elliptic equation for $\phi
_{s} $ at $r=R_{moving}$ - the pressure can be computed at all other levels
by integration of the hydrostatic relation, eq(\ref{hydro}).

In the `quasi-hydrostatic' equations (\textbf{QH)} strict balance between
gravity and vertical pressure gradients is not imposed. The $2\Omega u\cos
\phi $ Coriolis term are not neglected and are balanced by a non-hydrostatic
contribution to the pressure field: only the terms underlined twice in Eqs. (%
\ref{Gu} $\rightarrow $\ \ref{Gw}) are set to zero and, simultaneously, the
shallow atmosphere approximation is relaxed. In \textbf{QH}\ \textit{all}
the metric terms are retained and the full variation of the radial position
of a particle monitored. The \textbf{QH}\ vertical momentum equation (\ref
{vertmtm}) becomes:

\[
\frac{\partial \phi _{nh}}{\partial r}=2\Omega u\cos lat
\]
making a small correction to the hydrostatic pressure.

\textbf{QH} has good energetic credentials - they are the same as for 
\textbf{HPE}. Importantly, however, it has the same angular momentum
principle as the full non-hydrostatic model (\textbf{NH)} - see Marshall
et.al., 1997a. As in \textbf{HPE }only a 2-d elliptic problem need be solved.

\subsubsection{Non-hydrostatic and quasi-nonhydrostatic forms}

The MIT model presently supports a full non-hydrostatic ocean isomorph, but
only a quasi-non-hydrostatic atmospheric isomorph.

\paragraph{Non-hydrostatic Ocean}

In the non-hydrostatic ocean model all terms in equations (\ref{Gu} $%
\rightarrow $\ \ref{Gw}) are retained. A three dimensional elliptic equation
must be solved subject to Neumann boundary conditions (see below). It is
important to note that use of the full \textbf{NH} does not admit any new
`fast' waves in to the system - the incompressible condition (\ref
{incompressible}) has already filtered out acoustic modes. It does, however,
ensure that the gravity waves are treated accurately with an exact
dispersion relation. The \textbf{NH} set has a complete angular momentum
principle and consistent energetics - see White and Bromley, 1995; Marshall
et.al.\ 1997a.

\paragraph{Quasi-nonhydrostatic Atmosphere}

In the non-hydrostatic version of our atmospheric model we approximate $\dot{%
r}$ in the vertical momentum eqs(\ref{vertmtm}) and (\ref{Gw}) (but only
here) by:

\begin{equation}
\dot{r}=\frac{Dp}{Dt}=\frac{Dp_{hyd}}{Dt}  \label{quasinonhydro}
\end{equation}
where $p_{hy}$ is the hydrostatic pressure.

........................................

\subsubsection{Summary of equation sets supported by model}

The key equation sets and isomorphisms are summarised in fig.4.

\paragraph{Atmosphere}

\subparagraph{Hydrostatic and quasi-hydrostatic}

Hydrostatic, and quasi-hydrostatic forms of the compressible non-Boussinesq
equations in $p-$coordinates are supported\ref{eq-p} - see appendix
Atmosphere, where they are written out in $p-$coordinates.

\subparagraph{Quasi-nonhydrostatic}

A quasi-nonhydrostatic form is also supported - see appendix Ocean.

\paragraph{Ocean}

\subparagraph{Hydrostatic and quasi-hydrostatic}

Hydrostatic, and quasi-hydrostatic forms of the incompressible Boussinesq
equations in $z-$coordinates are supported

\subparagraph{Non-hydrostatic }

Non-hydrostatic forms of the incompressible Boussinesq equations in $z-$%
coordinates are supported.

\subsection{Solution strategy}

The method of solution employed in the \textbf{HPE}, \textbf{QH} and \textbf{%
NH} models are summarized in Fig.7.%
\marginpar{
Fig.7 Solution strategy}

Overview paragraph......

There is no penalty in implementing \textbf{QH} over \textbf{HPE} except, of
course, some complication that goes with the inclusion of $\cos \phi \ $%
Coriolis terms and the relaxation of the shallow atmosphere approximation.
But this leads to negligible increase in computation. In \textbf{NH}, in
contrast, one additional elliptic equation - a three-dimensional one - must
be inverted for $p_{nh}$. However the `overhead' of the \textbf{NH} model is
essentially negligible in the hydrostatic limit (see detailed discussion in
Marshall et al, 1997) resulting in a non-hydrostatic algorithm that, in the
hydrostatic limit, is as computationally economic as the \textbf{HPEs}.

\subsection{Finding the pressure field}

Unlike the prognostic variables $u$, $v$, $w$, $\theta $ and $S$, the
pressure field must be obtained diagnostically. We proceed, as before, by
dividing the total (pressure/geo) potential in to three parts, a surface
part, $\phi _{s}(x,y)$, a hydrostatic part $\phi _{hyd}(x,y,r)$ and a
non-hydrostatic part $\phi _{nh}(x,y,r)$, as in (\ref{pressuresplit}), and
writing the momentum equation in the form 
\begin{equation}
\frac{\partial }{\partial t}\vec{\mathbf{v}_{h}}+\mathbf{\nabla }_{h}\phi
_{s}+\mathbf{\nabla }_{h}\phi _{hyd}+\epsilon _{nh}\mathbf{\nabla }\phi
_{nh}=\vec{\mathbf{G}}_{\vec{v}}  \label{mtm-split}
\end{equation}
as in (\ref{hor-mtm}).

\subsubsection{Hydrostatic pressure}

Hydrostatic pressure is obtained by integrating (\ref{hydro}) vertically
from $r=R_{o}$ where $\phi _{hyd}(r=R_{o})=0$, to yield:

\[
\int_{r}^{R_{o}}\frac{\partial \phi _{hyd}}{\partial r}dr=\left[ \phi
_{hyd}\right] _{r}^{R_{o}}=\int_{r}^{R_{o}}-bdr 
\]
and so

\[
\phi _{hyd}(x,y,r)=\int_{r}^{R_{o}}bdr 
\]

\subsubsection{Surface pressure}

The surface pressure equation can be obtained by integrating continuity, (%
\ref{incompressible})c, vertically from $r=R_{fixed}$ to $r=R_{moving}$

\[
\int_{R_{fixed}}^{R_{moving}}\left( \mathbf{\nabla }_{h}\cdot \vec{\mathbf{v}%
}_{h}+\partial _{r}\dot{r}\right) dr=0 
\]

Thus:

\[
\frac{\partial \eta }{\partial t}+\vec{\mathbf{v}}.\nabla \eta
+\int_{R_{fixed}}^{R_{moving}}\mathbf{\nabla }_{h}\cdot \vec{\mathbf{v}}%
_{h}dr=0 
\]
where $\eta =R_{moving}-R_{o}$ is the free-surface $r$-anomaly in units of $%
r $. The above can be rearranged to yield, using Leibnitz's theorem:

\begin{equation}
\frac{\partial \eta }{\partial t}+\mathbf{\nabla }_{h}\cdot
\int_{R_{fixed}}^{R_{moving}}\vec{\mathbf{v}}_{h}dr=0
\label{integralcontinuity}
\end{equation}

Whether $\phi $ is pressure (ocean model, $p/\rho _{c}$) or geopotential
(atmospheric model), in (\ref{mtm-split}), the horizontal gradient term can
be written 
\begin{equation}
\mathbf{\nabla }_{h}\phi _{s}=\mathbf{\nabla }_{h}b\eta   \label{link}
\end{equation}
where $b$ is the buoyancy.

In the hydrostatic limit ($\epsilon _{nh}=0$), Eqs(\ref{mtm-split}), (\ref
{integralcontinuity}) and (\ref{link}) can be solved by inverting a 2-d
elliptic equation for $\phi _{s}$ as described in section ?.?. Both `free
surface' and `rigid lid' approaches are available.

\subsubsection{Non-hydrostatic pressure}

Taking the horizontal divergence of (\ref{hor-mtm}) and adding $\frac{%
\partial }{\partial r}$ of (\ref{vertmtm}), invoking the continuity equation
(\ref{incompressible}), we deduce that:

\begin{equation}
\nabla _{3}^{2}\phi _{nh}=\nabla .\vec{\mathbf{G}}_{\vec{v}}-\left( \mathbf{%
\nabla }_{h}^{2}\phi _{s}+\mathbf{\nabla }^{2}\phi _{hyd}\right) =\nabla .%
\vec{\mathbf{F}}  \label{3dinvert}
\end{equation}

For a given rhs this 3-d elliptic equation must be inverted for $\phi _{nh}$
subject to appropriate choice of boundary conditions. This method is usually
called \textit{The Pressure Method} [Harlow and Welch, 1965; Williams, 1969;
Potter, 1976]. In the hydrostatic primitive equations case (\textbf{HPE}),
the 3-d problem does not need to be solved.

\paragraph{Boundary Conditions}

We apply the condition of no normal flow through all solid boundaries - the
coasts (in the ocean) and the bottom:

\begin{equation}
\vec{\mathbf{v}}.\widehat{n}=0  \label{nonormalflow}
\end{equation}
where $\widehat{n}$ is a vector of unit length normal to the boundary. The
kinematic condition (\ref{nonormalflow}) is also applied to the vertical
velocity at $r=R_{moving}$. No-slip $\left( v_{T}=0\right) \ $or slip $%
\left( \partial v_{T}/\partial n=0\right) \ $conditions are employed on the
tangential component of velocity, $v_{T}$, at all solid boundaries,
depending on the form chosen for the dissipative terms in the momentum
equations - see below.

Eq.(\ref{nonormalflow}) implies, making use of (\ref{mtm-split}), that:

\begin{equation}
\widehat{n}.\nabla \phi _{nh}=\widehat{n}.\vec{\mathbf{F}}
\label{inhomneumann}
\end{equation}
where

\[
\vec{\mathbf{F}}=\vec{\mathbf{G}}_{\vec{v}}-\left( \mathbf{\nabla }_{h}\phi
_{s}+\mathbf{\nabla }\phi _{hyd}\right) 
\]
presenting inhomogeneous Neumann boundary conditions to the Elliptic problem
(\ref{3dinvert}). As shown, for example, by Williams (1969), one can exploit
classical 3D potential theory and, by introducing an appropriately chosen $%
\delta $-function sheet of `source-charge', replace the inhomogenous
boundary condition on pressure by a homogeneous one. The source term $rhs$
in (\ref{3dinvert}) is the divergence of the vector $\vec{\mathbf{F}}.$ By
simultaneously setting $
\begin{array}{l}
\widehat{n}.\vec{\mathbf{F}}
\end{array}
=0$\ and $\widehat{n}.\nabla \phi _{nh}=0\ $on the boundary the following
self-consistent but simpler homogenised Elliptic problem is obtained:

\[
\nabla ^{2}\phi _{nh}=\nabla .\widetilde{\vec{\mathbf{F}}}\qquad 
\]
where $\widetilde{\vec{\mathbf{F}}}$ is a modified $\vec{\mathbf{F}}$ such
that $\widetilde{\vec{\mathbf{F}}}.\widehat{n}=0$. As is implied by (\ref
{inhomneumann}) the modified boundary condition becomes:

\begin{equation}
\widehat{n}.\nabla \phi _{nh}=0  \label{homneuman}
\end{equation}

If the flow is `close' to hydrostatic balance then the 3-d inversion
converges rapidly because $\phi _{nh}\ $is then only a small correction to
the hydrostatic pressure field (see the discussion in Marshall et al, a,b).

The solution $\phi _{nh}\ $to (\ref{3dinvert}) and (\ref{homneuman}) does
not vanish at $r=R_{moving}$, and so refines the pressure there.

\subsection{Forcing/dissipation}

\subsubsection{Forcing}

The forcing terms $\mathcal{F}$ on the rhs of the equations are provided by
`physics packages' described in detail in section ?.?.

\subsubsection{Dissipation}

\paragraph{Momentum}

Many forms of momentum dissipation are available in the model. Laplacian and
biharmonic frictions are commonly used:

\[
D_{V}=A_{h}\nabla _{h}^{2}v+A_{v}\frac{\partial ^{2}v}{\partial z^{2}}%
+A_{4}\nabla _{h}^{4}v
\]
where $A_{h}$ and $A_{v}\ $are (constant) horizontal and vertical viscosity
coefficients and $A_{4}\ $is the horizontal coefficient for biharmonic
friction. These coefficients are the same for all velocity components.

\paragraph{Tracers}

The mixing terms for the temperature and salinity equations have a similar
form to that of momentum except that the diffusion tensor can be
non-diagonal and have varying coefficients. $\qquad $%
\[
D_{T,S}=\nabla .[\underline{\underline{K}}\nabla (T,S)]+K_{4}\nabla
_{h}^{4}(T,S)
\]
where $\underline{\underline{K}}\ $is the diffusion tensor and the $K_{4}\ $%
horizontal coefficient for biharmonic diffusion. In the simplest case where
the subgrid-scale fluxes of heat and salt are parameterized with constant
horizontal and vertical diffusion coefficients, $\underline{\underline{K}}$,
reduces to a diagonal matrix with constant coefficients:

\[
\qquad \qquad \qquad \qquad K=\left( 
\begin{array}{ccc}
K_{h} & 0 & 0 \\ 
0 & K_{h} & 0 \\ 
0 & 0 & K_{v}
\end{array}
\right) \qquad \qquad \qquad 
\]
where $K_{h}\ $and $K_{v}\ $are the horizontal and vertical diffusion
coefficients. These coefficients are the same for all tracers (temperature,
salinity ... ).

\subsection{Vector invariant form}

For some purposes it is advantageous to write momentum advection in eq(\ref
{hor-mtm}) and (\ref{vertmtm}) in the (so-called) `vector invariant' form:

\begin{equation}
\frac{D\vec{\mathbf{v}}}{Dt}=\frac{\partial \vec{\mathbf{v}}}{\partial t}%
+\left( \nabla \times \vec{\mathbf{v}}\right) \times \vec{\mathbf{v}}+\nabla
\left[ \frac{1}{2}(\vec{\mathbf{v}}\cdot \vec{\mathbf{v}})\right] 
\label{vecinvariant}
\end{equation}
This permits alternative numerical treatments of the non-linear terms based
on their representation as a vorticity flux. Because gradients of coordinate
vectors no longer appear on the rhs of (\ref{vecinvariant}) (???), explicit
representation of the metric terms in (\ref{Gu}), (\ref{Gv}) and (\ref{Gw}),
can be avoided: information about the geometry is contained in the areas and
lengths of the volumes used to discretize the model.

\subsection{Adjoint}

......
