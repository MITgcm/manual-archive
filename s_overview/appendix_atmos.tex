% $Header: /u/gcmpack/manual/s_overview/Attic/appendix_atmos.tex,v 1.4 2001/09/26 17:00:34 adcroft Exp $
% $Name:  $

\section{Appendix ATMOSPHERE}

\subsection{Hydrostatic Primitive Equations for the Atmosphere in pressure
coordinates}

\label{sect-hpe-p}

The hydrostatic primitive equations (HPEs) in p-coordinates are: 
\begin{eqnarray}
\frac{D\vec{\mathbf{v}}_{h}}{Dt}+f\hat{\mathbf{k}}\times \vec{\mathbf{v}}%
_{h}+\mathbf{\nabla }_{p}\phi &=&\vec{\mathbf{\mathcal{F}}}
\label{eq:atmos-mom} \\
\frac{\partial \phi }{\partial p}+\alpha &=&0  \label{eq-p-hydro-start} \\
\mathbf{\nabla }_{p}\cdot \vec{\mathbf{v}}_{h}+\frac{\partial \omega }{%
\partial p} &=&0  \label{eq:atmos-cont} \\
p\alpha &=&RT  \label{eq:atmos-eos} \\
c_{v}\frac{DT}{Dt}+p\frac{D\alpha }{Dt} &=&\mathcal{Q}  
\label{eq:atmos-heat}
\end{eqnarray}
where $\vec{\mathbf{v}}_{h}=(u,v,0)$ is the `horizontal' (on pressure
surfaces) component of velocity,$\frac{D}{Dt}=\vec{\mathbf{v}}_{h}\cdot 
\mathbf{\nabla }_{p}+\omega \frac{\partial }{\partial p}$ is the total
derivative, $f=2\Omega \sin lat$ is the Coriolis parameter, $\phi =gz$ is
the geopotential, $\alpha =1/\rho $ is the specific volume, $\omega =\frac{Dp%
}{Dt}$ is the vertical velocity in the $p-$coordinate. Equation \ref
{eq-p-firstlaw} is the first law of thermodynamics where internal energy $%
e=c_{v}T$, $T$ is temperature, $Q$ is the rate of heating per unit mass and $%
p\frac{D\alpha }{Dt}$ is the work done by the fluid in compressing.

It is convenient to cast the heat equation in terms of potential temperature 
$\theta $ so that it looks more like a generic conservation law.
Differentiating \ref{eq-p-eos-start} we get: 
\[
p\frac{D\alpha }{Dt}+\alpha \frac{Dp}{Dt}=R\frac{DT}{Dt}
\]
which, when added to the heat equation \ref{eq-p-firstlaw} and using $%
c_{p}=c_{v}+R$, gives: 
\begin{equation}
c_{p}\frac{DT}{Dt}-\alpha \frac{Dp}{Dt}=\mathcal{Q}
\label{eq-p-heat-interim}
\end{equation}
Potential temperature is defined: 
\begin{equation}
\theta =T(\frac{p_{c}}{p})^{\kappa }  \label{eq:potential-temp}
\end{equation}
where $p_{c}$ is a reference pressure and $\kappa =R/c_{p}$. For convenience
we will make use of the Exner function $\Pi (p)$ which defined by: 
\begin{equation}
\Pi (p)=c_{p}(\frac{p}{p_{c}})^{\kappa }  \label{Exner}
\end{equation}
The following relations will be useful and are easily expressed in terms of
the Exner function: 
\[
c_{p}T=\Pi \theta \;\;;\;\;\frac{\partial \Pi }{\partial p}=\frac{\kappa \Pi 
}{p}\;\;;\;\;\alpha =\frac{\kappa \Pi \theta }{p}=\frac{\partial \ \Pi }{%
\partial p}\theta \;\;;\;\;\frac{D\Pi }{Dt}=\frac{\partial \Pi }{\partial p}%
\frac{Dp}{Dt}
\]
where $b=\frac{\partial \ \Pi }{\partial p}\theta $ is the buoyancy.

The heat equation is obtained by noting that 
\[
c_{p}\frac{DT}{Dt}=\frac{D(\Pi \theta )}{Dt}=\Pi \frac{D\theta }{Dt}+\theta 
\frac{D\Pi }{Dt}=\Pi \frac{D\theta }{Dt}+\alpha \frac{Dp}{Dt}
\]
and on substituting into (\ref{eq-p-heat-interim}) gives: 
\begin{equation}
\Pi \frac{D\theta }{Dt}=\mathcal{Q}  \label{eq:potential-temperature-equation}
\end{equation}
which is in conservative form.

For convenience in the model we prefer to step forward (\ref{theta-equation}%
) rather than (\ref{eq-p-firstlaw}).

\subsubsection{Boundary conditions}

The upper and lower boundary conditions are : 
\begin{eqnarray}
\mbox{at the top:}\;\;p=0 &&\text{, }\omega =\frac{Dp}{Dt}=0 \\
\mbox{at the surface:}\;\;p=p_{s} &&\text{, }\phi =\phi _{topo}=g~Z_{topo}
\label{eq:boundary-condition-atmosphere}
\end{eqnarray}
In $p$-coordinates, the upper boundary acts like a solid boundary ($\omega
=0 $); in $z$-coordinates and the lower boundary is analogous to a free
surface ($\phi $ is imposed and $\omega \neq 0$).

\subsubsection{Splitting the geo-potential}

For the purposes of initialization and reducing round-off errors, the model
deals with perturbations from reference (or ``standard'') profiles. For
example, the hydrostatic geopotential associated with the resting atmosphere
is not dynamically relevant and can therefore be subtracted from the
equations. The equations written in terms of perturbations are obtained by
substituting the following definitions into the previous model equations: 
\begin{eqnarray}
\theta &=&\theta _{o}+\theta ^{\prime } \label{eq:atmos-ref-prof-theta} \\
\alpha &=&\alpha _{o}+\alpha ^{\prime }  \label{eq:atmos-ref-prof-alpha}\\
\phi &=&\phi _{o}+\phi ^{\prime } \label{eq:atmos-ref-prof-phi}
\end{eqnarray}
The reference state (indicated by subscript ``0'') corresponds to
horizontally homogeneous atmosphere at rest ($\theta _{o},\alpha _{o},\phi
_{o}$) with surface pressure $p_{o}(x,y)$ that satisfies $\phi
_{o}(p_{o})=g~Z_{topo}$, defined: 
\begin{eqnarray*}
\theta _{o}(p) &=&f^{n}(p) \\
\alpha _{o}(p) &=&\Pi _{p}\theta _{o} \\
\phi _{o}(p) &=&\phi _{topo}-\int_{p_{0}}^{p}\alpha _{o}dp
\end{eqnarray*}
%\begin{eqnarray*}
%\phi'_\alpha & = & \int^p_{p_o} (\alpha_o -\alpha) dp \\
%\phi'_s(x,y,t) & = & \int_{p_o}^{p_s} \alpha dp
%\end{eqnarray*}

The final form of the HPE's in p coordinates is then: 
\begin{eqnarray}
\frac{D\vec{\mathbf{v}}_{h}}{Dt}+f\hat{\mathbf{k}}\times \vec{\mathbf{v}}%
_{h}+\mathbf{\nabla }_{p}\phi ^{\prime } &=&\vec{\mathbf{\mathcal{F}}}
                  \\
\frac{\partial \phi ^{\prime }}{\partial p}+\alpha ^{\prime } &=&0
                   \\
\mathbf{\nabla }_{p}\cdot \vec{\mathbf{v}}_{h}+\frac{\partial \omega }{%
\partial p} &=&0                    \\
\frac{\partial \Pi }{\partial p}\theta ^{\prime } &=&\alpha ^{\prime }
                  \\
\frac{D\theta }{Dt} &=&\frac{\mathcal{Q}}{\Pi }  \label{eq:atmos-prime}
\end{eqnarray}
