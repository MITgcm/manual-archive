% $Header: /u/gcmpack/manual/s_overview/Attic/introduction.tex,v 1.1.1.1 2001/08/08 16:16:16 adcroft Exp $
% $Name:  $

\section{Introduction}

This documentation provides the reader with the information necessary to
carry out numerical experiments using MITgcm. It gives a comprehensive
description of the continuous equations on which the model is based, the
numerical algorithms the model employs and a description of the associated
program code. Along with the hydrodynamical kernel, physical and
biogeochemical parameterizations of key atmospheric and oceanic processes
are available. A number of examples illustrating the use of the model in
both process and general circulation studies of the atmosphere and ocean are
also presented.

MITgcm has a number of novel aspects:

\begin{itemize}
\item  it can be used to study both atmospheric and oceanic phenomena; one
hydrodynamical kernel is used to drive forward both atmospheric and oceanic
models - see fig.1%
\marginpar{
Fig.1 One model}\ref{fig:onemodel}

\item  it has a non-hydrostatic capability and so can be used to study both
small-scale and large scale processes - see fig.2%
\marginpar{
Fig.2 All scales}\ref{fig:all-scales}

\item  finite volume techniques are employed yielding an intuitive
discretization and support for the treatment of irregular geometries using
orthogonal curvilinear grids and shaved cells - see fig.3%
\marginpar{
Fig.3 Finite volumes}\ref{fig:Finite volumes}

\item  tangent linear and adjoint counterparts are automatically maintained
along with the forward model, permitting sensitivity and optimization
studies.

\item  the model is developed to perform efficiently on a wide variety of
computational platforms.
\end{itemize}

Key publications reporting on and charting the development of the model are
listed in Appendix Refs.

We begin by briefly showing some of the results of the model in action to
give a feel for the wide range of problems that can be addressed using it.
