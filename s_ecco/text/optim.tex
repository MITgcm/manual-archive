\section{The line search optimisation algorithm
\label{sectionoptim}}

\subsection{General features}

The line search algorithm is based on a quasi-Newton
variable storage method which was implemented by
\cite{gil-lem:89}.

TO BE CONTINUED...

\subsection{The online vs. offline version}

\begin{itemize}
%
\item {\bf Online version} \\
Every call to {\it simul} refers to an execution of the 
forward and adjoint model.
Several iterations of optimization may thus be performed within
a single run of the main program (lsopt\_top).
The following cases may occur:
%
\begin{itemize}
\item
cold start only (no optimization)
\item
cold start, followed by one or several iterations of optimization
\item
warm start from previous cold start with one or several iterations
\item
warm start from previous warm start with one or several iterations
\end{itemize}
%
\item {\bf Offline version} \\
Every call to simul refers to a read procedure which
reads the result of a forward and adjoint run
Therefore, only one call to simul is allowed,
                     {\tt itmax = 0}, for cold start
                     {\tt itmax = 1}, for warm start
Also, at the end, {\bf x(i+1)} needs to be computed and saved
to be available for the offline model and adjoint run
\end{itemize}

In order to achieve minimum difference between the online and offline code
{\bf xdiff(i+1)} is stored to file at the end of an (offline) iteration,
but recomputed identically at the beginning of the next iteration.

\subsection{Number of iterations vs. number of simulations}

{\tt - itmax:} controls the max. number of iterations \\
{\tt - nfunc:} controls the max. number of simulations 
within one iteration

\paragraph{Summary} ~ \\
From one iteration to the next the descent direction changes.
Within one iteration more than one forward and adjoint 
run may be performed.
The updated control used as input for these simulations uses the same
descent direction, but different step sizes.

\paragraph{Description} ~ \\
From one iteration to the next the descent direction dd changes using
the result for the adjoint vector gg of the previous iteration.
In lsline the updated control
\[
\tt
xdiff(i,1) = xx(i-1) + tact(i-1,1)*dd(i-1)
\]
serves as input for
a forward and adjoint model run yielding a new {\tt gg(i,1)}.
In general, the new solution passes the 1st and 2nd Wolfe tests 
so {\tt xdiff(i,1)} represents the solution sought:
\[ 
{\tt xx(i) = xdiff(i,1)}
\]
If one of the two tests fails, 
an inter- or extrapolation is invoked to determine
a new step size {\tt tact(i-1,2)}.
If more than one function call is permitted, 
the new step size is used together
with the "old" descent direction {\tt dd(i-1)}
(i.e. dd is not updated using the new {\tt gg(i)}),
to compute a new 
\[
{\tt xdiff(i,2) = xx(i-1) + tact(i-1,2)*dd(i-1)} 
\]
that serves as input
in a new forward and adjoint run, yielding {\tt gg(i,2)}.
If now, both Wolfe tests are successful, 
the updated solution is given by
\[
\tt
xx(i) = xdiff(i,2) = xx(i-1) + tact(i-1,2)*dd(i-1)
\]

In order to save memory both the fields dd and xdiff 
have a double usage.
%
\begin{itemize}
%
\item [{\tt xdiff}] ~\\
- in {\it lsopt\_top}: used as {\tt x(i) - x(i-1)} for Hessian update \\
- in {\it lsline}:    intermediate result for control update 
{\tt x = x + tact*dd}
%
\item [{\tt dd}] ~\\
- in {\it lsopt\_top, lsline}: descent vector, {\tt dd = -gg} 
and {\tt hessupd} \\
- in {\it dgscale}: intermediate result to compute new preconditioner
%
\end{itemize}

\paragraph{The parameter file lsopt.par}

%
\begin{itemize}
%
\item {\bf NUPDATE}
max. no. of update pairs {\tt (gg(i)-gg(i-1), xx(i)-xx(i-1))}
to be stored in {\tt OPWARMD} to estimate Hessian
[pair of current iter. is stored in 
{\tt (2*jmax+2, 2*jmax+3)}
jmax must be > 0 to access these entries]
Presently {\tt NUPDATE} must be > 0 
(i.e. iteration without reference to previous
 iterations through {\tt OPWARMD} has not been tested)
%
\item {\bf EPSX}
relative precision on xx bellow which xx should not be improved
%
\item {\bf EPSG}
relative precision on gg below which optimization is 
considered successful
%
\item {\bf IPRINT}
controls verbose (>=1) or non-verbose output
%
\item {\bf NUMITER}
max. number of iterations of optimisation;
NUMTER = 0: cold start only, no optimization
%
\item {\bf ITER\_NUM}
index of new restart file to be created (not necessarily = NUMITER!)
%
\item {\bf NFUNC}
max. no. of simulations per iteration
(must be > 0);
is used if step size {\tt tact} is inter-/extrapolated;
in this case, if NFUNC > 1, a new simulation is performed with
same gradient but "improved" step size
%
\item {\bf FMIN}
first guess cost function value
(only used as long as first iteration not completed,
i.e. for jmax <= 0)
%
\end{itemize}

\paragraph{OPWARMI, OPWARMD files}
Two files retain values of previous iterations which are
used in latest iteration to update Hessian:
\begin{itemize}
%
\item {\bf OPWARMI}: contains index settings and scalar variables

{\footnotesize
\begin{tabular}{ll}
{\tt n  = nn}      & no. of control variables \\
{\tt fc = ff}      & cost value of last iteration \\
{\tt isize}        & no. of bytes per record in OPWARMD \\
{\tt m = nupdate}  & max. no. of updates for Hessian \\
{\tt jmin, jmax}   & pointer indices for OPWARMD file (cf. below) \\
{\tt gnorm0}       & norm of first (cold start) gradient gg \\
{\tt iabsiter}     & total number of iterations with respect to cold start
\end{tabular}
}
%
\item {\bf OPWARMD}: contains vectors (control and gradient)

{\scriptsize
\begin{tabular}{cll}
entry & name & description \\
\hline
     1    & {\tt xx(i)}         & control vector of latest iteration \\
     2    & {\tt gg(i)}         & gradient of latest iteration \\
     3    & {\tt xdiff(i),diag} & preconditioning vector; (1,...,1)
for cold start \\
 2*jmax+2 & {\tt gold=g(i)-g(i-1)} & for last update (jmax) \\
 2*jmax+3 & {\tt xdiff=tact*d=xx(i)-xx(i-1)} & for last update (jmax)
\end{tabular}
}
%
\end{itemize}
%
\begin{figure}[b!]
{\footnotesize
\begin{verbatim}

Example 1: jmin = 1, jmax = 3, mupd = 5

  1   2   3   |   4   5     6   7     8   9     empty     empty
|___|___|___| | |___|___| |___|___| |___|___| |___|___| |___|___|
      0       |     1         2         3

Example 2: jmin = 3, jmax = 7, mupd = 5   ---> jmax = 2

  1   2   3   |  
|___|___|___| | |___|___| |___|___| |___|___| |___|___| |___|___|
              |     6         7         3         4         5

\end{verbatim}
}
\caption{Examples of OPWARM file handling}
\label{fig:opwarm}
\end{figure}

\paragraph{Error handling}



\newpage

\begin{figure}
%{\scriptsize
\begin{verbatim}
  lsopt_top
      |
      |---- check arguments
      |---- CALL INSTORE
      |       |
      |       |---- determine whether OPWARMI available:
      |                * if no:  cold start: create OPWARMI
      |                * if yes: warm start: read from OPWARMI
      |             create or open OPWARMD
      |
      |---- check consistency between OPWARMI and model parameters
      | 
      |---- >>> if COLD start: <<<
      |      |  first simulation with f.g. xx_0; output: first ff_0, gg_0
      |      |  set first preconditioner value xdiff_0 to 1
      |      |  store xx(0), gg(0), xdiff(0) to OPWARMD (first 3 entries)
      |      |
      |     >>> else: WARM start: <<<
      |         read xx(i), gg(i) from OPWARMD (first 2 entries)
      |         for first warm start after cold start, i=0
      |
      |
      |
      |---- /// if ITMAX > 0: perform optimization (increment loop index i)
      |      (
      |      )---- save current values of gg(i-1) -> gold(i-1), ff -> fold(i-1)
      |      (---- CALL LSUPDXX
      |      )       |
      |      (       |---- >>> if jmax=0 <<<
      |      )       |      |  first optimization after cold start:
      |      (       |      |  preconditioner estimated via ff_0 - ff_(first guess)
      |      )       |      |  dd(i-1) = -gg(i-1)*preco
      |      (       |      |  
      |      )       |     >>> if jmax > 0 <<<
      |      (       |         dd(i-1) = -gg(i-1)
      |      )       |         CALL HESSUPD
      |      (       |           |
      |      )       |           |---- dd(i-1) modified via Hessian approx.
      |      (       |
      |      )       |---- >>> if <dd,gg> >= 0 <<<
      |      (       |         ifail = 4
      |      )       |
      |      (       |---- compute step size: tact(i-1)
      |      )       |---- compute update: xdiff(i) = xx(i-1) + tact(i-1)*dd(i-1)
      |      (
      |      )---- >>> if ifail = 4 <<<
      |      (         goto 1000
      |      )
      |      (---- CALL OPTLINE / LSLINE
      |      )       |
     ...    ...     ...
\end{verbatim}
}

{\scriptsize
\begin{verbatim}
  lsopt_top
      |
      |---- check arguments
      |---- CALL INSTORE
      |       |
      |       |---- determine whether OPWARMI available:
      |                * if no:  cold start: create OPWARMI
      |                * if yes: warm start: read from OPWARMI
      |             create or open OPWARMD
      |
      |---- check consistency between OPWARMI and model parameters
      | 
      |---- >>> if COLD start: <<<
      |      |  first simulation with f.g. xx_0; output: first ff_0, gg_0
      |      |  set first preconditioner value xdiff_0 to 1
      |      |  store xx(0), gg(0), xdiff(0) to OPWARMD (first 3 entries)
      |      |
      |     >>> else: WARM start: <<<
      |         read xx(i), gg(i) from OPWARMD (first 2 entries)
      |         for first warm start after cold start, i=0
      |
      |
      |
      |---- /// if ITMAX > 0: perform optimization (increment loop index i)
      |      (
      |      )---- save current values of gg(i-1) -> gold(i-1), ff -> fold(i-1)
      |      (---- CALL LSUPDXX
      |      )       |
      |      (       |---- >>> if jmax=0 <<<
      |      )       |      |  first optimization after cold start:
      |      (       |      |  preconditioner estimated via ff_0 - ff_(first guess)
      |      )       |      |  dd(i-1) = -gg(i-1)*preco
      |      (       |      |  
      |      )       |     >>> if jmax > 0 <<<
      |      (       |         dd(i-1) = -gg(i-1)
      |      )       |         CALL HESSUPD
      |      (       |           |
      |      )       |           |---- dd(i-1) modified via Hessian approx.
      |      (       |
      |      )       |---- >>> if <dd,gg> >= 0 <<<
      |      (       |         ifail = 4
      |      )       |
      |      (       |---- compute step size: tact(i-1)
      |      )       |---- compute update: xdiff(i) = xx(i-1) + tact(i-1)*dd(i-1)
      |      (
      |      )---- >>> if ifail = 4 <<<
      |      (         goto 1000
      |      )
      |      (---- CALL OPTLINE / LSLINE
      |      )       |
     ...    ...     ...
\end{verbatim}
}
\caption{Flow chart (part 1 of 3)}
\label{fig:lsoptflow1}
\end{figure}

\begin{figure}
%{\scriptsize
\begin{verbatim}
     ...    ...
      |      )
      |      (---- CALL OPTLINE / LSLINE
      |      )       |
      |      (       |---- /// loop over simulations
      |      )              (  
      |      (              )---- CALL SIMUL
      |      )              (       |
      |      (              )       |----  input: xdiff(i)
      |      )              (       |---- output: ff(i), gg(i)
      |      (              )       |---- >>> if ONLINE <<<
      |      )              (                 runs model and adjoint
      |      (              )             >>> if OFFLINE <<<
      |      )              (                 reads those values from file
      |      (              )
      |      )              (---- 1st Wolfe test:
      |      (              )     ff(i) <= tact*xpara1*<gg(i-1),dd(i-1)>
      |      )              (
      |      (              )---- 2nd Wolfe test:
      |      )              (     <gg(i),dd(i-1)> >= xpara2*<gg(i-1),dd(i-1)>
      |      (              )
      |      )              (---- >>> if 1st and 2nd Wolfe tests ok <<<
      |      (              )      |  320: update xx: xx(i) = xdiff(i)
      |      )              (      |
      |      (              )     >>> else if 1st Wolfe test not ok <<<
      |      )              (      |  500: INTERpolate new tact:
      |      (              )      |  barr*tact < tact < (1-barr)*tact
      |      )              (      |  CALL CUBIC
      |      (              )      |
      |      )              (     >>> else if 2nd Wolfe test not ok <<<
      |      (              )         350: EXTRApolate new tact:
      |      )              (         (1+barmin)*tact < tact < 10*tact
      |      (              )         CALL CUBIC
      |      )              (
      |      (              )---- >>> if new tact > tmax <<<
      |      )              (      |  ifail = 7
      |      (              )      |
      |      )              (---- >>> if new tact < tmin OR tact*dd < machine precision <<<
      |      (              )      |  ifail = 8
      |      )              (      |
      |      (              )---- >>> else <<<
      |      )              (         update xdiff for new simulation
      |      (              )
      |      )             \\\ if nfunc > 1: use inter-/extrapolated tact and xdiff
      |      (                               for new simulation
      |      )                               N.B.: new xx is thus not based on new gg, but
      |      (                                     rather on new step size tact
      |      )        
      |      (---- store new values xx(i), gg(i) to OPWARMD (first 2 entries)
      |      )---- >>> if ifail = 7,8,9 <<<
      |      (         goto 1000
      |      )
     ...    ...
\end{verbatim}
}

{\scriptsize
\begin{verbatim}
     ...    ...
      |      )
      |      (---- CALL OPTLINE / LSLINE
      |      )       |
      |      (       |---- /// loop over simulations
      |      )              (  
      |      (              )---- CALL SIMUL
      |      )              (       |
      |      (              )       |----  input: xdiff(i)
      |      )              (       |---- output: ff(i), gg(i)
      |      (              )       |---- >>> if ONLINE <<<
      |      )              (                 runs model and adjoint
      |      (              )             >>> if OFFLINE <<<
      |      )              (                 reads those values from file
      |      (              )
      |      )              (---- 1st Wolfe test:
      |      (              )     ff(i) <= tact*xpara1*<gg(i-1),dd(i-1)>
      |      )              (
      |      (              )---- 2nd Wolfe test:
      |      )              (     <gg(i),dd(i-1)> >= xpara2*<gg(i-1),dd(i-1)>
      |      (              )
      |      )              (---- >>> if 1st and 2nd Wolfe tests ok <<<
      |      (              )      |  320: update xx: xx(i) = xdiff(i)
      |      )              (      |
      |      (              )     >>> else if 1st Wolfe test not ok <<<
      |      )              (      |  500: INTERpolate new tact:
      |      (              )      |  barr*tact < tact < (1-barr)*tact
      |      )              (      |  CALL CUBIC
      |      (              )      |
      |      )              (     >>> else if 2nd Wolfe test not ok <<<
      |      (              )         350: EXTRApolate new tact:
      |      )              (         (1+barmin)*tact < tact < 10*tact
      |      (              )         CALL CUBIC
      |      )              (
      |      (              )---- >>> if new tact > tmax <<<
      |      )              (      |  ifail = 7
      |      (              )      |
      |      )              (---- >>> if new tact < tmin OR tact*dd < machine precision <<<
      |      (              )      |  ifail = 8
      |      )              (      |
      |      (              )---- >>> else <<<
      |      )              (         update xdiff for new simulation
      |      (              )
      |      )             \\\ if nfunc > 1: use inter-/extrapolated tact and xdiff
      |      (                               for new simulation
      |      )                               N.B.: new xx is thus not based on new gg, but
      |      (                                     rather on new step size tact
      |      )        
      |      (---- store new values xx(i), gg(i) to OPWARMD (first 2 entries)
      |      )---- >>> if ifail = 7,8,9 <<<
      |      (         goto 1000
      |      )
     ...    ...
\end{verbatim}
}
\caption{Flow chart (part 2 of 3)}
\label{fig:lsoptflow2}
\end{figure}

\begin{figure}
%{\scriptsize
\begin{verbatim}
     ...    ...
      |      )        
      |      (---- store new values xx(i), gg(i) to OPWARMD (first 2 entries)
      |      )---- >>> if ifail = 7,8,9 <<<
      |      (         goto 1000
      |      )
      |      (---- compute new pointers jmin, jmax to include latest values
      |      )     gg(i)-gg(i-1), xx(i)-xx(i-1) to Hessian matrix estimate
      |      (---- store gg(i)-gg(i-1), xx(i)-xx(i-1) to OPWARMD
      |      )     (entries 2*jmax+2, 2*jmax+3)
      |      (
      |      )---- CALL DGSCALE
      |      (       |
      |      )       |---- call dostore
      |      (       |       |
      |      )       |       |---- read preconditioner of previous iteration diag(i-1)
      |      (       |             from OPWARMD (3rd entry)
      |      )       |
      |      (       |---- compute new preconditioner diag(i), based upon diag(i-1),
      |      )       |     gg(i)-gg(i-1), xx(i)-xx(i-1)
      |      (       |
      |      )       |---- call dostore
      |      (               |
      |      )               |---- write new preconditioner diag(i) to OPWARMD (3rd entry)
      |      (
      |---- \\\ end of optimization iteration loop
      |
      |
      |
      |---- CALL OUTSTORE
      |       |
      |       |---- store gnorm0, ff(i), current pointers jmin, jmax, iterabs to OPWARMI
      |
      |---- >>> if OFFLINE version <<<
      |         xx(i+1) needs to be computed as input for offline optimization
      |          |
      |          |---- CALL LSUPDXX
      |          |       |
      |          |       |---- compute dd(i), tact(i) -> xdiff(i+1) = x(i) + tact(i)*dd(i)
      |          |
      |          |---- CALL WRITE_CONTROL
      |          |       |
      |          |       |---- write xdiff(i+1) to special file for offline optim.
      |
      |---- print final information
      |
      O
\end{verbatim}
}

{\scriptsize
\begin{verbatim}
     ...    ...
      |      )        
      |      (---- store new values xx(i), gg(i) to OPWARMD (first 2 entries)
      |      )---- >>> if ifail = 7,8,9 <<<
      |      (         goto 1000
      |      )
      |      (---- compute new pointers jmin, jmax to include latest values
      |      )     gg(i)-gg(i-1), xx(i)-xx(i-1) to Hessian matrix estimate
      |      (---- store gg(i)-gg(i-1), xx(i)-xx(i-1) to OPWARMD
      |      )     (entries 2*jmax+2, 2*jmax+3)
      |      (
      |      )---- CALL DGSCALE
      |      (       |
      |      )       |---- call dostore
      |      (       |       |
      |      )       |       |---- read preconditioner of previous iteration diag(i-1)
      |      (       |             from OPWARMD (3rd entry)
      |      )       |
      |      (       |---- compute new preconditioner diag(i), based upon diag(i-1),
      |      )       |     gg(i)-gg(i-1), xx(i)-xx(i-1)
      |      (       |
      |      )       |---- call dostore
      |      (               |
      |      )               |---- write new preconditioner diag(i) to OPWARMD (3rd entry)
      |      (
      |---- \\\ end of optimization iteration loop
      |
      |
      |
      |---- CALL OUTSTORE
      |       |
      |       |---- store gnorm0, ff(i), current pointers jmin, jmax, iterabs to OPWARMI
      |
      |---- >>> if OFFLINE version <<<
      |         xx(i+1) needs to be computed as input for offline optimization
      |          |
      |          |---- CALL LSUPDXX
      |          |       |
      |          |       |---- compute dd(i), tact(i) -> xdiff(i+1) = x(i) + tact(i)*dd(i)
      |          |
      |          |---- CALL WRITE_CONTROL
      |          |       |
      |          |       |---- write xdiff(i+1) to special file for offline optim.
      |
      |---- print final information
      |
      O
\end{verbatim}
}
\caption{Flow chart (part 3 of 3)}
\label{fig:lsoptflow3}
\end{figure}

