\section{ECCO: model-data comparisons using gridded data sets}
\label{sec:pkg:ecco}
\begin{rawhtml}
<!-- CMIREDIR:package_ecco: -->
\end{rawhtml}

\def\mitgcmCheckpointVersion{65x}

The functionalities implemented in \texttt{pkg/ecco} are: (1) output time-averaged model fields to compare with gridded data sets; (2) compute normalized model-data distances (i.e., cost functions); (3) compute averages and transports (i.e., integrals). The former is achieved as the model runs forwards in time whereas the others occur after time-integration has completed. Following \cite{for-eta:15} the total cost function is formulated generically as
\begin{align} 
	\mathcal{J}(\vec{u}) &= \sum_i \alpha_i \left(\vec{d}_i^T R_i^{-1} \vec{d}_i\right) + \sum_j \beta_j \vec{u}^T\vec{u}, \label{eq:Jtotal} \\
	\vec{d}_i &= \mathcal{P}(\vec{m}_i - \vec{o}_i), \label{eq:Jposproc} \\
	\vec{m}_i &= \mathcal{S}\mathcal{D}\mathcal{M}(\vec{v}), \label{eq:Jpreproc} \\
	\vec{v}	  &= \mathcal{Q}(\vec{u}), \label{eq:Upreproc} \\
	\vec{u}	  &= \mathcal{R}(\vec{u}') \label{eq:Uprecond}
\end{align}
using symbols defined in table~\ref{tbl:gencost_symbols}. Per Eq.~\eqref{eq:Jpreproc} model counterparts ($\vec{m}_i$) to observational data ($\vec{o}_i$) derive from adjustable model parameters ($\vec{v}$) through model dynamics integration ($\mathcal{M}$), diagnostic calculations ($\mathcal{D}$), and averaging in space and time ($\mathcal{S}$). Alternatively $\mathcal{S}$ stands for subsampling in space and time (section~\ref{sec:pkg:profiles}). Plain model-data misfits ($\vec{m}_i-\vec{o}_i$) can be penalized directly in Eq.~\eqref{eq:Jtotal} but penalized misfits ($\vec{d}_i$) more generally derive from $\vec{m}_i-\vec{o}_i$ through the generic $\mathcal{P}$ post-processor (Eq. \eqref{eq:Jposproc}). Eqs.~\eqref{eq:Upreproc}-\eqref{eq:Uprecond} pertain to model control parameter adjustment capabilities described in section~\ref{sec:pkg:ctrl}.

\begin{table}[!ht]
\centering
\begin{tabular}{rl}
symbol			&	definition	\\ \hline
$\vec{u}$			&	vector of nondimensional control variables \\
$\vec{v}$			&	vector of dimensional control variables \\
$\alpha_i, \beta_j$	&	misfit and control cost function multipliers (1 by default) \\
$R_i$ 			&	data error covariance matrix ($R_i^{-1}$ are weights) \\
$\vec{d}_i$		&	a set of model-data differences \\
$\vec{o}_i$		&	observational data vector \\
$\vec{m}_i$		&	model counterpart to $\vec{o}_i$ \\
$\mathcal{P}$		&	post-processing operator (e.g., a smoother) \\
$\mathcal{M}$		&	forward model dynamics operator \\
$\mathcal{D}$		&	diagnostic computation operator \\
$\mathcal{S}$		&	averaging/subsampling operator \\
$\mathcal{Q}$		&	Pre-processing operator \\
$\mathcal{R}$		&	Pre-conditioning operator
\end{tabular}
\caption{Symbol definitions for pkg/ecco and pkg/ctrl generic cost functions.}
\label{tbl:gencost_symbols}
\end{table}

\subsection{Generic Cost Function} \label{costgen}

The parameters available for configuring generic cost function terms in \texttt{data.ecco} are given in table~\ref{tbl:gencost_ecco_params} and examples of possible specifications are available in:
\begin{itemize}
\itemsep0em
\item MITgcm\_contrib/verification\_other/global\_oce\_cs32/input/data.ecco
\item MITgcm\_contrib/verification\_other/global\_oce\_cs32/input\_ad.sens/data.ecco
\item MITgcm\_contrib/gael/verification/global\_oce\_llc90/input.ecco\_v4/data.ecco
\end{itemize}
 
 \noindent
The gridded observation file name is specified by \texttt{gencost\_datafile}. Observational time series may be provided as on big file or split into yearly files finishing in `\_1992', `\_1993', etc. The corresponding $\vec{m}_i$ physical variable is specified via the \texttt{gencost\_barfile} root (see table~\ref{tbl:gencost_ecco_barfile}). A file named as specified by \texttt{gencost\_barfile} gets created where averaged fields are written progressively as the model steps forward in time. After the final time step this file is re-read by \texttt{cost\_generic.F} to compute the corresponding cost function term. If \texttt{gencost\_outputlevel} = 1 and \texttt{gencost\_name}=`foo' then \texttt{cost\_generic.F} outputs model-data misfit fields (i.e., $\vec{d}_i$) to a file named `misfit\_foo.data' for offline analysis and visualization.

In the current implementation, model-data error covariance matrices $R_i$ omit non-diagonal terms. Specifying $R_i$ thus boils down to providing uncertainty fields ($\sigma_i$ such that $R_i=\sigma_i^2$) in a file specified via \texttt{gencost\_errfile}. By default $\sigma_i$ is assumed to be time-invariant but a $\sigma_i$ time series of the same length as the $\vec{o}_i$ time series can be provided using the \texttt{variaweight} option (table~\ref{tbl:gencost_ecco_preproc}). By default cost functions are quadratic but $\vec{d}_i^T R_i^{-1} \vec{d}_i$ can be replaced with $R_i^{-1/2} \vec{d}_i$ using the \texttt{nosumsq} option (table~\ref{tbl:gencost_ecco_preproc}). 

In principle, any averaging frequency should be possible, but only {`day'}, {`month'}, {`step'}, and {`const'} are implemented for \texttt{gencost\_avgperiod}. If two different averaging frequencies are needed for a variable used in multiple cost function terms (e.g., daily and monthly) then an extension starting with `\_' should be added to \texttt{gencost\_barfile} (such as `\_day' and `\_mon'). \footnote{ecco\_check may be missing a test for conflicting names...} If two cost function terms use the same variable and frequency, however, then using a common \texttt{gencost\_barfile} saves disk space. 

Climatologies of $\vec{m}_i$ can be formed from the time series of model averages in order to compare with climatologies of $\vec{o}_i$ by activating the `clim' option via \texttt{gencost\_preproc} and setting the corresponding \texttt{gencost\_preproc\_i}  integer parameter to the number of records (i.e., a \# of months, days, or time steps) per climatological cycle. The generic post-processor ($\mathcal{P}$ in Eq.~\eqref{eq:Jposproc}) also allows model-data misfits to be, for example, smoothed in space by setting \texttt{gencost\_posproc} to {`smooth'} and specifying the smoother parameters via \texttt{gencost\_posproc\_c} and \texttt{gencost\_posproc\_i} (see table~\ref{tbl:gencost_ecco_preproc}). Other options associated with the computation of Eq.~\eqref{eq:Jtotal} are summarized in table~\ref{tbl:gencost_ecco_preproc} and further discussed below. Multiple \texttt{gencost\_preproc} / \texttt{gencost\_posproc} options may be specified per cost term. 

In general the specification of \texttt{gencost\_name} is optional, has no impact on the end-result, and only serves to distinguish between cost function terms amongst the model output (STDOUT.0000, STDERR.0000, costfunction000, misfit*.data). Exceptions listed in table~\ref{tbl:gencost_ecco_name} however activate alternative cost function codes (in place of \texttt{cost\_generic.F}) described in section~\ref{v4custom}. The specification of \texttt{gencost\_mask}\footnote{This should be renamed \texttt{gencost\_loc} or \texttt{gencost\_point}...} allows the user to specify whether the gridded data input and model counterparts are located at tracer points (`c'; the default) or velocity points (`w' or `s'). However the `c' option (not `w' or `s') should be used when gridded velocity data is provided as zonal/meridional components at tracer points (e.g., for all vector cases listed in table~\ref{tbl:gencost_ecco_barfile}). 

\begin{table}[!ht]
\centering
\begin{tabular}{lll}
parameter					&	type			&	function \\ \hline
%\texttt{using\_gencost} 		&	logical			&	Turns specified generic cost term on. \\
\texttt{gencost\_name} 			&	character(*) 	&	Name of cost term \\
\texttt{gencost\_barfile} 		&	character(*)	&	File to receive model counterpart $\vec{m}_i$ (see table~\ref{tbl:gencost_ecco_barfile}) \\
\texttt{gencost\_datafile} 		&	character(*)	&	File containing observational data $\vec{o}_i$ \\
\texttt{gencost\_avgperiod}	&	character(5)	&	Averaging period for $\vec{o}_i$ and $\vec{m}_i$ (see text) \\
\texttt{gencost\_outputlevel} 	&	integer 		&	Greater than 0 will output misfit fields\\
\texttt{gencost\_errfile} 		& 	character(*)	&	File containing diagonal of error matrix $R_i$\\ 
\texttt{mult\_gencost} 			&	real			&	Multiplier $\alpha_i$ (default: 1) \\ 
\hline
\texttt{gencost\_preproc} 		&	character(*)	&	Preprocessor names \\
\texttt{gencost\_preproc\_c} 	&	character(*)	&	Preprocessor character arguments 	\\
\texttt{gencost\_preproc\_i} 	&	integer(*)		&	Preprocessor integer arguments 		\\
\texttt{gencost\_preproc\_r} 	&	real(*)			&	Preprocessor real arguments 	\\
\texttt{gencost\_posproc} 		&	character(*)	&	Post-processor names \\
\texttt{gencost\_posproc\_c} 	&	character(*)	&	Post-processor character arguments 	\\
\texttt{gencost\_posproc\_i} 	&	integer(*) 		&	Post-processor integer arguments 	\\
\texttt{gencost\_posproc\_r} 	&	real(*)			&	Post-processor real arguments 	\\
\hline
\texttt{gencost\_mask}		&	character(1)	&	Location of $\vec{m}_i$\\
\texttt{gencost\_spmin}		&	real			&	Data less than this value will be omitted \\
\texttt{gencost\_spmax}		&	real			&	Data greater than this value will be omitted \\
\texttt{gencost\_spzero}		&	real			&	Data points equal to this value will be omitted \\
\texttt{gencost\_startdate1} 	&	integer			&	Start date of observations (YYYMMDD)	\\
\texttt{gencost\_startdate2} 	&	integer			&	Start date of observations (HHMMSS)				\\
\texttt{gencost\_is3d}		&	logical 		&	Needs to be true for 3D fields \\
\hline
\texttt{gencost\_enddate1} 		&	integer	&	Not fully implemented (sec.~\ref{v4custom} only)\\
\texttt{gencost\_enddate2} 		&	integer	&	Not fully implemented (sec.~\ref{v4custom} only)\\
\end{tabular}
\caption{Parameters in \texttt{ecco\_gencost\_nml} namelist in \texttt{data.ecco}. All parameters are vectors of length \texttt{NGENCOST} (the \# of available cost terms) except for \texttt{gencost\_*proc*} are arrays of size \texttt{NGENPPROC}$\times$\texttt{NGENCOST}. Notes: \texttt{gencost\_is3d} will automatically be reset to true in cases listed as 3D in table~\ref{tbl:gencost_ecco_barfile}; the last group of parameters should be disregarded except for the section~\ref{v4custom} special cases.}
\label{tbl:gencost_ecco_params}
\end{table}

\begin{table}[!ht]
\centering
\begin{tabular}{lll}
variable name				&	description				& remarks \\ \hline\hline
\texttt{m\_eta}				&	sea surface height			& free surface + ice + global steric correction \\
\texttt{m\_sst}				&	sea surface temperature		& first level potential temperature \\
\texttt{m\_sss}				&	sea surface salinity			& first level salinity \\ 
\texttt{m\_bp}				&	bottom pressure			& phiHydLow\\ 
\texttt{m\_siarea}			&	sea-ice area				& from pkg/seaice \\
\texttt{m\_siheff}			&	sea-ice effective thickness		& from pkg/seaice \\
\texttt{m\_sihsnow}			&	snow effective thickness		& from pkg/seaice \\ \hline
\texttt{m\_theta}				&	potential temperature		& three-dimensional \\
\texttt{m\_salt}				&	salinity					& three-dimensional \\
\texttt{m\_UE}				&	zonal velocity				& three-dimensional \\
\texttt{m\_VN}				&	meridional velocity			& three-dimensional \\ \hline
\texttt{m\_ustress}			&	zonal wind stress			& from pkg/exf \\
\texttt{m\_vstress}			&	meridional wind stress		& from pkg/exf\\
\texttt{m\_uwind}			&	zonal wind 				& from pkg/exf\\
\texttt{m\_vwind}			&	meridional wind 			& from pkg/exf\\
\texttt{m\_atemp}			&	atmospheric temperature		& from pkg/exf\\
\texttt{m\_aqh}				&	atmospheric specific humidity	& from pkg/exf\\
\texttt{m\_precip}			&	precipitation				& from pkg/exf\\
\texttt{m\_swdown}			&	downward shortwave		& from pkg/exf\\
\texttt{m\_lwdown}			&	downward longwave			& from pkg/exf\\
\texttt{m\_wspeed}			&	wind speed				& from pkg/exf\\ \hline
\texttt{m\_diffkr}				&	vertical/diapycnal diffusivity	& three-dimensional, constant \\ 
\texttt{m\_kapgm}			&	GM diffusivity				& three-dimensional, constant \\ 
\texttt{m\_kapredi}			&	isopycnal diffusivity			& three-dimensional, constant \\ 
\texttt{m\_geothermalflux}		&	geothermal heat flux			& constant \\ 
\texttt{m\_bottomdrag}		&	bottom drag				& constant \\
\end{tabular}
\caption{Implemented \texttt{gencost\_barfile} options (as of checkpoint \mitgcmCheckpointVersion) that can be used via \texttt{cost\_generic.F} (section~\ref{costgen}). An extension starting with `\_' can be appended at the end of the variable name to distinguish between separate cost function terms. Notes: here `zonal' / `meridional' are to be taken literally (unlike in other parts of the manual) and these components are centered (i.e., not at the staggered C-grid velocity points); the `m\_eta' formula depends on the \texttt{ATMOSPHERIC\_LOADING} and \texttt{ALLOW\_PSBAR\_STERIC} compile time options and `useRealFreshWaterFlux' run time parameter.}
\label{tbl:gencost_ecco_barfile}
\end{table}

\begin{table}[!ht]
\centering
\begin{tabular}{lll}
name					&	description					&	specs needed via \texttt{gencost\_preproc\_i}, \texttt{\_r}, or \texttt{\_c} \\ \hline\hline
\texttt{gencost\_preproc} \\ \hline
\texttt{clim} 				&	Use climatological misfits	&	integer: no.\ of records per climatological cycle \\
\texttt{mean} 				&	Use time mean of misfits 	&	--- \\
\texttt{anom} 				&	Use anomalies from time mean &	--- \\
\texttt{variaweight}		&	Use time-varying weight $W_i$&	--- \\
\texttt{nosumsq} 			&	Use linear misfits 			&	--- \\
\texttt{factor} 			&	Multiply $\vec{m}_i$ by a scaling factor	&	real: the scaling factor \\ \hline \hline
\texttt{gencost\_posproc} \\ \hline
\texttt{smooth} 			&	Smooth misfits				&	character: smoothing scale file\\ 
						&								&	integer: smoother \# of time steps \\
\end{tabular} 
\caption{\texttt{gencost\_preproc} and \texttt{gencost\_posproc} options implemented as of checkpoint \mitgcmCheckpointVersion. Note: the distinction between \texttt{gencost\_preproc} and \texttt{gencost\_posproc} may be revisited in the future.}
\label{tbl:gencost_ecco_preproc}
\end{table}

\clearpage

\subsection{Generic Integral Function} \label{intgen}

The functionality described in this section is operated by \texttt{cost\_gencost\_boxmean.F}. It is primarily aimed at obtaining a mechanistic understanding of a chosen physical variable via adjoint sensitivity computations (see Chapter~\ref{chap:autodiff}) as done for example in \cite{maro-eta:99,heim-eta:11,fuku-etal:14}. Thus the quadratic term in Eq.~\ref{eq:Jtotal} ($\vec{d}_i^T R_i^{-1} \vec{d}_i$) is by default replaced with a $d_i$ scalar\footnote{The quadratic option in fact does not yet exist in \texttt{cost\_gencost\_boxmean.F}...} that derives from model fields through a generic integral formula (Eq.~\ref{eq:Jpreproc}). The specification of \texttt{gencost\_barfile} again selects the physical variable type. Current valid options to use \texttt{cost\_gencost\_boxmean.F} are reported in table~\ref{tbl:genint_ecco_barfile}. A suffix starting with \texttt{`\_'} can again be appended to \texttt{gencost\_barfile}.
% and the basic averaging  frequency is specified via \texttt{gencost\_avgperiod}. 

The integral formula is defined by masks provided via binary files which names are specified via \texttt{gencost\_errfile}. There are two cases: (1) if \texttt{gencost\_errfile = `foo\_mask'} and \texttt{gencost\_barfile} is of the `m\_boxmean*' type then the model will search for horizontal, vertical, and temporal mask files  named \texttt{foo\_maskC}, \texttt{foo\_maskK}, and \texttt{foo\_maskT}; (2) if instead \texttt{gencost\_barfile} is of the `m\_horflux\_*' type then the model will search for \texttt{foo\_maskW}, \texttt{foo\_maskS}, \texttt{foo\_maskK}, and \texttt{foo\_maskT}. 

The `C' mask or the `W' / `S' masks are expected to be two-dimensional fields. The `K' and T masks (both optional; all 1 by default) are expected to be one-dimensional vectors. The`K' vector length should match Nr. The `T' vector length should match the \# of records that the specification of \texttt{gencost\_avgperiod} implies but there is no restriction on its values. In case \#1 (`m\_boxmean*') the `C' and `K' masks should consists of +1 and 0 values and a volume average will be computed accordingly. In case \#2 (`m\_horflux*') the `W', `S', and `K' masks should consists of +1, -1, and 0 values and an integrated horizontal transport (or overturn) will be computed accordingly. 

\begin{table}[!ht]
\centering
\begin{tabular}{lll}
variable name				&	description						&	remarks \\ \hline\hline
\texttt{m\_boxmean\_theta}	&	mean of theta over box			& specify box \\
\texttt{m\_boxmean\_salt}		&	mean of salt over box			& specify box \\
\texttt{m\_boxmean\_eta}		&	mean of SSH over box			& specify box \\
\hline
\texttt{m\_horflux\_vol}		&	volume transport through section	& specify transect \\ 
\end{tabular}
\caption{Implemented \texttt{gencost\_barfile} options (as of checkpoint \mitgcmCheckpointVersion) that can be used via \texttt{cost\_gencost\_boxmean.F} (section~\ref{intgen}).}
\label{tbl:genint_ecco_barfile}
\end{table}

\subsection{Custom Cost Functions} \label{v4custom}

This section (very much a work in progress...) pertains to the special cases of \texttt{cost\_gencost\_bpv4.F}, \texttt{cost\_gencost\_seaicev4.F}, \texttt{cost\_gencost\_sshv4.F}, \texttt{cost\_gencost\_sstv4.F}, and \texttt{cost\_gencost\_transp.F}. The cost\_gencost\_transp.F function can be used to compute a transport of volume, heat, or salt through a specified section (non quadratic cost function). To this end one sets \texttt{gencost\_name = `transp*'}, where \texttt{*} is an optional suffix starting with \texttt{`\_'}, and set \texttt{gencost\_barfile} to one of \texttt{m\_trVol}, \texttt{m\_trHeat}, and \texttt{m\_trSalt}. 

\begin{table}[!ht]
\centering
\begin{tabular}{lll}
name					&	description				&	remarks \\ \hline\hline
\texttt{sshv4-mdt}			&	sea surface height			&	mean dynamic topography (SSH - geod) \\
\texttt{sshv4-tp}				&	sea surface height			&	Along-Track Topex/Jason SLA (level 3) \\
\texttt{sshv4-ers}			&	sea surface height			&	Along-Track ERS/Envisat SLA (level 3)\\
\texttt{sshv4-gfo}			&	sea surface height			&	Along-Track GFO class SLA (level 3)\\
\texttt{sshv4-lsc}			&	sea surface height			&	Large-Scale SLA (from the above)\\
\texttt{sshv4-gmsl}			&	sea surface height			&	Global-Mean SLA (from the above)\\ \hline
\texttt{bpv4-grace}			&	bottom pressure			&	GRACE maps (level 4) \\ \hline
\texttt{sstv4-amsre}			&	sea surface temperature		&	Along-Swath SST (level 3)\\
\texttt{sstv4-amsre-lsc}		&	sea surface temperature		&	Large-Scale SST (from the above)\\ \hline
\texttt{si4-cons}				&	sea ice concentration		& 	needs sea-ice adjoint (level 4)\\
\texttt{si4-deconc}			&	model sea ice deficiency		& 	proxy penalty (from the above)\\
\texttt{si4-exconc}			&	model sea ice excess		& 	proxy penalty (from the above)\\ \hline
\texttt{transp\_trVol}			&	volume transport			& specify section as in section~\ref{intgen}\\ 
\texttt{transp\_trHeat}		&	heat transport				& specify section as in section~\ref{intgen} \\ 
\texttt{transp\_trSalt}			&	salt transport				& specify section as in section~\ref{intgen} \\ 
\end{tabular}
\caption{Pre-defined \texttt{gencost\_name} special cases (as of checkpoint \mitgcmCheckpointVersion; section~\ref{v4custom}).}
\label{tbl:gencost_ecco_name}
\end{table}

\subsection{Key Routines}

TBA... \texttt{cost\_generic.F}, \texttt{cost\_gencost\_boxmean.F}, \texttt{ecco\_phys.F}, \texttt{cost\_gencost\_customize.F}, \texttt{cost\_averagesfields.F} ... \texttt{ecco\_readparms.F}, \texttt{ecco\_check.F}, \texttt{ecco\_summary.F}, ...

\subsection{Compile Options}

TBA... ALLOW\_GENCOST3D, ALLOW\_PSBAR\_STERIC, ECCO\_CTRL\_DEPRECATED, ...
