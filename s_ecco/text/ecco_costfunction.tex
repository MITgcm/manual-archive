\section{The ECCO state estimation cost function DRAFT!!!
\label{sectioneccocost}}
\begin{rawhtml}
<!-- CMIREDIR:ecco_cost: -->
\end{rawhtml}

The current ECCO state estimation covers an $nYears = 11$ year
model trajectory.
A variety of data sets enter a least squares cost function,
in addition to penalty terms which constrain deviations
of control variables beyound their a priori errors.

\subsection{Sea surface height from TOPEX/Poseidon and ERS-1/2 altimetry}

Altimetric SSH contributions from T/P and ERS-1/2 are four-fold:
%
\begin{enumerate}
%
\item 
an $nYears$ time mean SSH misfit between
model and T/P
%
\item
daily SSH anomaly misfits between T/P and model
%
\item
daily SSH anomaly misfits between ERS-1/2 and model
%
\item
daily absolute SSH misfit between T/P and model,
weighted by the full geoid error covariance.
%
\end{enumerate}

\subsubsection{Input fields}
~

\begin{table}[h!]
\begin{center}
\begin{tabular}{lllc}
\hline \hline
~&~&~&~\\
field & file name & deccription & unit \\
~&~&~&~\\
\hline
~&~&~&~\\
{\it psbar} & {\tt psbarfile} & daily model mean SSH fields & [m] \\
{\it tpmean} & {\tt topexmeanfile} & $nYears$ T/P mean & [cm] \\
{\it tpobs}  & {\tt topexfile} & daily T/P SSH anomalies & [cm] \\
{\it erspobs}  & {\tt ersfile} & daily ERS-1/2 SSH anomalies & [cm] \\
{\it wp} & {\tt geoid\_errfile} & diagonal of geoid error covariance & [m] \\
{\it wtp, wers} & {\tt ssh\_errfile} & rms of SSH anomalies & [cm] \\
~&~&~&~\\
\hline \hline
\end{tabular}
\end{center}
\end{table}


\subsubsection{\textit{\textbf{nYears}} time mean SSH misfit}

\begin{enumerate}
%
\item
Compute $nYears$ model mean spatial distribution
%
\begin{equation}
psmean(i,j)\, =\, 
\frac{1}{nDaysRec} \sum_{i=1}^{nDaysRec}
psbar(i,j)
\end{equation}
%
\item
Compute global offset between $nYears$ model and T/P mean:
%
\begin{equation}
\begin{split}
offset & = \, \overline{tpmean} \, - \, \overline{psmean} \\
~ & = \, \frac{1}{normaliz.} \sum_{i,j}
\left\{ tpmean(i,j) \, - \, psmean(i,j) \right\} 
\cdot cosphi(i,j) \cdot tpmeanmask(i,j)
\end{split}
\end{equation}
%
\item
Misfits are computed w.r.t. global $offset$. 
\\
First spatial distribution:
%
\begin{equation}
\begin{split}
cost\_ssh\_mean(i,j) & = \,
\frac{1}{wp^2} \left\{ \,
\left[ \, psmean(i,j) - \overline{psmean} \, \right] \, - \,
\left[ \, tpmean(i,j) - \overline{tpmean} \, \right] \, \right\}^2 \\ 
~ & = \, \frac{1}{wp^2} \left\{ \,
psmean(i,j) \, - \, tpmean(i,j) \, + \, offset \, \right\}^2
\end{split}
\end{equation}

%
Finally, sum over all spatial entries:
\begin{equation}
\overline{cost\_ssh\_mean} \, = \,
\sum_{i,j} cost\_ssh\_mean(i,j)
\end{equation}



\end{enumerate}

\subsubsection{Misfit of daily SSH anomalies}

Computation is same for T/P and ERS-1/2.
Here we write out computation for T/P.

\begin{enumerate}
%
\item
Compute difference in anomalies:

\begin{equation}
\begin{split}
cost\_ssh\_anom(i,j,t) & = \, \frac{1}{wtp^2} \left\{ \, 
\left[ \, psbar(i,j,t) - psmean(i,j) \, \right] \, - \, 
\left[ \, tpobs(i,j,t) \, \right] \,
\right\}^2
\end{split}
\end{equation}
%
where $t$ denotes time (day) index, and
where it is assumed that $ nYears$ mean T/P spatial distribution
$tpmean(i,j)$ has already been removed from data $tpobs(i,j)$!

\item
Sum over all spatial points and all times

\begin{equation}
\begin{split}
\overline{cost\_ssh\_anom} & = \, \sum_{t} \sum_{i,j}
cost\_ssh\_anom(i,j,t)
\end{split}
\end{equation}

\end{enumerate}

\subsubsection{Flow chart}

\begin{verbatim}

cost_ssh
|
|- < compute nYears model mean >
|
|- < read nYears T/P mean >
|  CALL COST_READTOPEXMEAN
|
|- < compute global T/P vs. model offset >
|
|- < compute cost_hmean >
|  CALL COST_SSH_MEAN
|
|- < ... >

\end{verbatim}

\subsubsection{Weights and notes}

\begin{itemize}
%
\item
All data are currently masked to zero where less than 13 depth levels,
mimicing no contribution for depth less than 1000m.
%
\item
$cosphi$ term in weights is set to 1.
%
\item
bad T/P and ERS-1/2 values are flagged $ \le \, -9990. $
%
\item
T/P and ERS-1/2 data $ \le \, 1.\exp^{-8}$ cm are flagged as bad values
%
\item
$wp$ is read from {\tt geoid\_errfile}
and $1/wp^2$ is pre-computed in {\tt ecco\_cost\_weights}
%
\end{itemize}

\paragraph{$wp$ for SSH mean misfit} ~

$1/wp^2$ is pre-computed in {\tt ecco\_cost\_weights}; \\
$wp$ is read from {\tt geoid\_errfile};

\paragraph{$wtp$ and $wers$ for SSH anomaly misfit} ~

$1/wtp^2$, $1/wers^2$ are pre-computed in {\tt ecco\_cost\_weights}; \\
%
\begin{itemize}
%
\item
$wtp$, $wers$ are read from single {\tt ssh\_errfile}
%
\item
both are converted to meters and halved \\
$ wtp \, \longrightarrow \, wtp \cdot 0.01 \cdot 0.5 $
%
\item
ERS error is set to T/P error + 5cm \\
$ wers \, = \, wtp \, + 0.5cm $
%
\end{itemize}

\subsubsection{Cost diagnostics}

\begin{itemize}
%
\item
Map out $ cost\_ssh\_mean(i,j) $
%
\item
Map out $ cost\_ssh\_anom(i,j,t) $ averaged over 1 month, i.e.
\[
\frac{1}{\text{monthly entries}} \sum_{t}^{monthly} cost\_ssh\_anom(i,j,t)
\]
%
\item
sum over daily entries and plot daily average as function of time. i.e.
\[
\frac{1}{\text{daily entries}} \sum_{i,j} cost\_ssh\_anom(i,j,t)
\]
\end{itemize}

\subsection{Hydrographic constraints}

Observation of temperature and salinity from various sources are
used to constrain the model. These are:
%
\begin{enumerate}
%
\item
CTD obs. for $T$, $S$ from various WOCE sections
%
\item
XBT obs. for $T$
%
\item
Sea surface temperature (SST) and salinity (SSS) from
Reynolds et al. (???)
%
\item
$T$, $S$ from ARGO floats
%
\item
$T$, $S$ from fields from Levitus (???)
%
\end{enumerate}

\subsubsection{Input fields}
~

\begin{table}[h!]
\begin{center}
\begin{tabular}{lllc}
\hline \hline
~&~&~&~\\
field & file name & deccription & unit \\
~&~&~&~\\
\hline
~&~&~&~\\
{\it tbar} & {\tt tbarfile} & monthly model mean pot. temperature & 
[$^{\circ}$C] \\
{\it sbar} & {\tt sbarfile} & monthly model mean salinity & 
[ppt] \\
{\it tdat} & {\tt tdatfile} & monthly mean Levitus pot. temperature & 
[$^{\circ}$C] \\
{\it sdat} & {\tt sdatfile} & monthly mean Levitus salinity & 
[ppt] \\
{\it ctdtobs}  & {\tt ctdtfile} & monthly WOCE CTD pot. temperature & 
[$^{\circ}$C] \\
{\it ctdsobs}  & {\tt ctdsfile} & monthly WOCE CTD salinity & 
[ppt] \\
{\it xbtobs} & {\tt xbtfile} & monthly XBT in-situ(!) temperature & 
[$^{\circ}$C] \\
{\it sstdat}  & {\tt sstdatfile} & monthly Reynolds pot. SST & 
[$^{\circ}$C] \\
{\it sssdat}  & {\tt sssdatfile} & monthly Reynolds SSS & 
[ppt] \\
{\it argotobs}  & {\tt argotfile} & monthly ARGO in-situ(!) temperature & 
[$^{\circ}$C] \\
{\it argosobs}  & {\tt argosfile} & monthly ARGO salinity & 
[ppt] \\
{\it wti, wsi} & {\tt data\_errfile} & vert. stdev. profile for $T$, $S$ &
~ \\
{\it wtvar} & {\tt temperrfile} & spatially varying stdev. & [$^{\circ}$C] \\
{\it wsvar} & {\tt salterrfile} & spatially varying stdev. & [ppt] \\
~&~&~&~\\
\hline \hline
\end{tabular}
\end{center}
\end{table}

\subsubsection{XBT data}

\begin{equation}
\begin{split}
cost\_xbt\_t(i,j,k) & = \,
\left[ \, \frac{fac \cdot ratio}{wti^2 + wtvar^2} \sum_{\tau=1}^{nMonsRec}
\left\{ Tbar(\tau) \, - \, T2\theta[xbtobs(\tau)] \right\}^2 \, \right](i,j,k)
 \\
\end{split}
\end{equation}

\subsubsection{WOCE CTD data}

\begin{equation}
\begin{split}
cost\_ctd\_t(i,j,k) & = \,
\left[ \, \frac{fac \cdot ratio}{wti^2 + wtvar^2} \sum_{\tau=1}^{nMonsRec}
\left\{ Tbar(\tau) \, - \, ctdTobs(\tau) \right\}^2 \, \right](i,j,k)
 \\
cost\_ctd\_s(i,j,k) & = \,
\left[ \, \frac{fac \cdot ratio}{wsi^2 + wsvar^2} \sum_{\tau=1}^{nMonsRec}
\left\{ Sbar(\tau) \, - \, ctdSobs(\tau) \right\}^2 \, \right](i,j,k)
 \\
\end{split}
\end{equation}

\subsubsection{ARGO float data}

\begin{equation}
\begin{split}
cost\_argo\_t(i,j,k) & = \,
\left[ \, \frac{fac \cdot ratio}{wti^2 + wvar^2} \sum_{\tau=1}^{nMonsRec}
\left\{ Tbar(\tau) \, - \, T2\theta[argoTobs(\tau)] \right\}^2 \, \right](i,j,k)
 \\
cost\_argo\_s(i,j,k) & = \,
\left[ \, \frac{fac \cdot ratio}{wsi^2 + wsvar^2} \sum_{\tau=1}^{nMonsRec}
\left\{ Sbar(\tau) \, - \, argoSobs(\tau) \right\}^2 \, \right](i,j,k)
 \\
\end{split}
\end{equation}

\subsubsection{Reynolds sea surface T, S data}

\begin{equation}
\begin{split}
cost\_sst(i,j) & = \,
\left[ \, wsst \sum_{\tau=1}^{nMonsRec}
\left\{ Tbar(\tau) \, - \, sstDat(\tau) \right\}^2 \, \right](i,j)
 \\
cost\_sss(i,j) & = \,
\left[ \, wsss \sum_{\tau=1}^{nMonsRec}
\left\{ Sbar(\tau) \, - \, sssDat(\tau) \right\}^2 \, \right](i,j)
 \\
\end{split}
\end{equation}

\subsubsection{Levitus montly T, S climatological data}

Model vs. data misfits are taken from $nYears$ monthly model means
vs. Levitus monthly data.
The description below is for potential temperature.
Procedure for salinity is fully analogous.
Spatial indices $(i,j,k)$ are omitted throughout.
%
\begin{enumerate}
%
\item
Compute $nYears$ monthly model means for each month $imon$:
\[
\overline{Tbar}(imon) \, = \, \frac{1}{nYears} 
\sum_{iyear=1}^{nYears} Tbar(iyear,imon)
\]
%
\item
Compute misfit:
\[
cost\_theta(i,j,k) \, = \, \left[ 
\frac{fac \cdot ratio}{wti^2} \sum_{imon=1}^{12}
\left\{ \overline{Tbar}(imon) \, - \, Tdat(imon) \right\}^2  \right] (i,j,k)
\]

\end{enumerate}


\subsubsection{Weights and notes}

\begin{itemize}
%
\item
$T2\theta$ is an operator mapping in-situ to potential temperatures
%
\item
Latitudinal weight not used:
\[
cosphi(i,j) \, = \, 1
\]
%
\item
$ fac \, = \, cosphi \cdot mask $
%
\item
Spatially {\it constant} weights:
%
\begin{enumerate}
%
\item
Read standard deviation vertical profiles for $T$, $S$ \\
$ {\tt data\_errfile} \, \longrightarrow \, 
wti(k), \,\, wsi(k) $ \\
$ {\tt data\_errfile} \, \longrightarrow \, 
ratio = 0.25 = \left( \frac{1}{2} \right)^2 $
%
\item
Take inverse squares:
\[
\begin{split}
wtheta(k) & = \, \frac{ratio}{wti(k)^2} \\
wsalt(k) & = \, \frac{ratio}{wsi(k)^2} \\
\end{split}
\]
%
\end{enumerate}
%
\item
Spatially {\it varying} weights:
%
\begin{enumerate}
%
\item
Read standard deviation fields \\
$ {\tt temperrfile} \, \longrightarrow \, wtvar(i,j,k) $ \\
$ {\tt salterrfile} \, \longrightarrow \, wsvar(i,j,k) $ \\
%
\item
Weights are combination of spatially constant and varying parts:
\[
\begin{split}
wtheta2(i,j,k) & = \, \frac{ratio}
{wti(k)^2 \, + \,wtvar(i,j,k)^2 } \\
wsalt2(i,j,k) & = \, 
\frac{ratio}
{wsi(k)^2 \, + \,wsvar(i,j,k)^2 } \\
\end{split}
\]
%
\end{enumerate}
%
\item
Sea surface $T$, $S$ weights:
\begin{itemize}
\item
SST: $ wsst \, = \, fac \cdot wtheta(1)$: horizontally constant
\item
SSS: $ wsss \, = \, fac \cdot wsalt2(i,j,1)$: horizontally varying
\end{itemize}
(Why this difference? I don't know.)
%
\end{itemize}


\subsubsection{Diagnostics}

\begin{itemize}
%
\item
Map out $wtheta2(i,j,k)$, $wsalt2(i,j,k)$.

%
\end{itemize}

