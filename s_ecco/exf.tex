\section{The external forcing package
\label{sectionexf}}

\subsection{Summary}

{\footnotesize
\begin{verbatim}
c     Field definitions, units, and sign conventions:
c     ===============================================
c
c     ustress   :: Zonal surface wind stress in N/m^2
c                  > 0 for increase in uVel, which is west to
c                      east for cartesian and spherical polar grids
c                  Typical range: -0.5 < ustress < 0.5
c                  Southwest C-grid U point
c                  Input field
c
c     vstress   :: Meridional surface wind stress in N/m^2
c                  > 0 for increase in vVel, which is south to
c                      north for cartesian and spherical polar grids
c                  Typical range: -0.5 < vstress < 0.5
c                  Southwest C-grid V point
c                  Input field
c
c     hflux     :: Net upward surface heat flux excluding shortwave in W/m^2
c                  hflux = latent + sensible + lwflux
c                  > 0 for decrease in theta (ocean cooling)
c                  Typical range: -250 < hflux < 600
c                  Southwest C-grid tracer point
c                  Input field
c
c     sflux     :: Net upward freshwater flux in m/s
c                  sflux = evap - precip - runoff
c                  > 0 for increase in salt (ocean salinity)
c                  Typical range: -1e-7 < sflux < 1e-7
c                  Southwest C-grid tracer point
c                  Input field
c
c     swflux    :: Net upward shortwave radiation in W/m^2
c                  swflux = - ( swdown - ice and snow absorption - reflected )
c                  > 0 for decrease in theta (ocean cooling)
c                  Typical range: -350 < swflux < 0
c                  Southwest C-grid tracer point
c                  Input field
c
c     uwind     :: Surface (10-m) zonal wind velocity in m/s
c                  > 0 for increase in uVel, which is west to
c                      east for cartesian and spherical polar grids
c                  Typical range: -10 < uwind < 10
c                  Southwest C-grid U point
c                  Input or input/output field
c
c     vwind     :: Surface (10-m) meridional wind velocity in m/s
c                  > 0 for increase in vVel, which is south to
c                      north for cartesian and spherical polar grids
c                  Typical range: -10 < vwind < 10
c                  Southwest C-grid V point
c                  Input or input/output field
c
c     atemp     :: Surface (2-m) air temperature in deg K
c                  Typical range: 200 < atemp < 300
c                  Southwest C-grid tracer point
c                  Input or input/output field
c
c     aqh       :: Surface (2m) specific humidity in kg/kg
c                  Typical range: 0 < aqh < 0.02
c                  Southwest C-grid tracer point
c                  Input or input/output field
c
c     lwflux    :: Net upward longwave radiation in W/m^2
c                  lwflux = - ( lwdown - ice and snow absorption - emitted )
c                  > 0 for decrease in theta (ocean cooling)
c                  Typical range: -20 < lwflux < 170
c                  Southwest C-grid tracer point
c                  Input field
c
c     evap      :: Evaporation in m/s
c                  > 0 for increase in salt (ocean salinity)
c                  Typical range: 0 < evap < 2.5e-7
c                  Southwest C-grid tracer point
c                  Input, input/output, or output field
c
c     precip    :: Precipitation in m/s
c                  > 0 for decrease in salt (ocean salinity)
c                  Typical range: 0 < precip < 5e-7
c                  Southwest C-grid tracer point
c                  Input or input/output field
c
c     runoff    :: River and glacier runoff in m/s
c                  > 0 for decrease in salt (ocean salinity)
c                  Typical range: 0 < runoff < ????
c                  Southwest C-grid tracer point
c                  Input or input/output field
c                  !!! WATCH OUT: Default exf_inscal_runoff !!!
c                  !!! in exf_readparms.F is not 1.0        !!!
c
c     swdown    :: Downward shortwave radiation in W/m^2
c                  > 0 for increase in theta (ocean warming)
c                  Typical range: 0 < swdown < 450
c                  Southwest C-grid tracer point
c                  Input/output field
c
c     lwdown    :: Downward longwave radiation in W/m^2
c                  > 0 for increase in theta (ocean warming)
c                  Typical range: 50 < lwdown < 450
c                  Southwest C-grid tracer point
c                  Input/output field
c
c     apressure :: Atmospheric pressure field in N/m^2
c                  > 0 for ????
c                  Typical range: ???? < apressure < ????
c                  Southwest C-grid tracer point
c                  Input field
C
C
c     NOTES:
c     ======
c
c     Input and output units and sign conventions can be customized
c     using variables exf_inscal_* and exf_outscal_*, which are set
c     by exf_readparms.F
c
c     Output fields fu, fv, Qnet, Qsw, and EmPmR are
c     defined in FFIELDS.h
c
c     #ifndef SHORTWAVE_HEATING, hflux includes shortwave,
c     that is, hflux = latent + sensible + lwflux +swflux
c
c     If (EXFwindOnBgrid .EQ. .TRUE.), uwind and vwind are
c     defined on northeast B-grid U and V points, respectively.
c
c     Arrays *0 and *1 below are used for temporal interpolation.
\end{verbatim}
}

